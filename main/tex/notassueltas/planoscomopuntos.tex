\theoremstyle{plain}
\newtheorem{teoEquivalenciasTensoresElementalesRCuatro}{Teorema}[section]
\newtheorem{coroTensoresElementalesRTres}%
	[teoEquivalenciasTensoresElementalesRCuatro]{Corolario}

\theoremstyle{remark}

%-------------

Un tensor elemental en $\exterior[2]{\bb{R}^{4}}$ es, simplemente, un
elemento de la forma $f_{1}\wedge f_{2}$, donde $f_{1}$ y $f_{2}$ son
vectores en $\bb{R}^{4}$. Todo tensor elemental no nulo determina
un\'{\i}vocamente un plano en $\bb{R}^{4}$: si $f_{1}\wedge f_{2}\not =0$
entonces $\{f_{1},f_{2}\}$ es un conjunto linealmente independiente y
consituye una base de un subespacio de dimensi\'{o}n $2$ en $\bb{R}^{4}$,
es decir, un plano por el origen. Esto permitir\'{a} pensar al conjunto
de tales planos como una variedad y a cada plano por el origen como
un punto de dicha variedad. Pero para poder establecer esta correspondencia
entre planos y puntos es necesario, en primera instancia, dar una
caracterizaci\'{o}n de los tensores elementales en $\exterior[2]{\bb{R}^{4}}$.

\subsection{Tensores elementales en el producto exterior %
	$\exterior[2]{\bb{R}^{4}}$}
Sea $\phi\in\dual{(\bb{R}^{4})}$. Se define una transformaci\'{o}n
$\exterior[2]{\bb{R}^{4}}\rightarrow\exterior[1]{\bb{R}^{4}}$ que en
tensores elementales est\'{a} dada por
\begin{align*}
	\phi\convol (f_{1}\wedge f_{2}) & \,=\,
		\phi(f_{1})\,f_{2}-\phi(f_{2})\,f_{1}
\end{align*}
%
denominada \emph{convoluci\'{o}n por $\phi$}. Dados vectores $f_{1}$ y
$f_{2}$ en $\bb{R}^{4}$, o bien $\phi|_{\generado{f_{1},f_{2}}}=0$, o bien
$\phi(f_{1})\not =0$ o $\phi(f_{2})\not=0$. En todo caso,
\begin{align*}
	\big(\phi\convol(f_{1}\wedge f_{2})\big)\wedge (f_{1}\wedge f_{2}) &
		\,=\,\big(\phi(f_{1})\,f_{2}-\phi(f_{2})\,f_{1}\big)\wedge
			(f_{1}\wedge f_{2}) \,=\,0
\end{align*}
%
en $\exterior[2]{\bb{R}^{4}}$. Dicho de otra manera, el vector de $\bb{R}^{4}$
dado por $\phi\convol(f_{1}\wedge f_{2})$ pertenece al subespacio generado
por $f_{1}$ y por $f_{2}$.

Sea $x\in\exterior[2]{\bb{R}^{4}}$ un elemento arbitrario del producto
exterior. Si $\{e_{1},\,e_{2},\,e_{3},\,e_{4}\}$ es una base de
$\bb{R}^{4}$, entonces $\{e_{i}\wedge e_{j}\,:\,i<j\}$ es una base de
$\exterior[2]{\bb{R}^{4}}$. En t\'{e}rminos de esta base,
\begin{align*}
	x & \,=\,\sum_{i<j}\,a^{ij}\,e_{i}\wedge e_{j}
\end{align*}
%
para ciertos coeficientes $a^{ij}\in\bb{R}$. Si $\phi\in\dual{(\bb{R}^{4})}$,
entonces, en t\'{e}rminos de esta descomposici\'{o}n, vale que
\begin{align*}
	(\phi\convol x)\wedge x & \,=\,\Big(\sum_{i<j}\,a^{ij}\,
		\big(\phi(e_{i})\,e_{j}-\phi(e_{j})\,e_{i}\big)\Big)\wedge
		\Big(\sum_{k<l}\,a^{kl}\,e_{k}\wedge e_{l}\Big) \\
	& \,=\,\sum_{i<j}\,\sum_{k<l}\,a^{ij}a^{kl}\,
		\big(\phi(e_{i})\,e_{j}-\phi(e_{j})\,e_{i}\big)\wedge
			(e_{k}\wedge e_{l}) \\
	& \,=\,
	\sbox0{$\begin{smallmatrix} k<l \\k,l\not =j\end{smallmatrix}$}
		\sum_{i<j}\,\sum_{\usebox{0}}\,a^{ij}a^{kl}\phi(e_{i})\,
			e_{j}\wedge e_{k}\wedge e_{l} \\
	& \qquad\qquad \,-\,
	\sbox1{$\begin{smallmatrix} k<l \\k,l\not =i\end{smallmatrix}$}
		\sum_{i<j}\,\sum_{\usebox{1}}\,a^{ij}a^{kl}\phi(e_{j})\,
			e_{i}\wedge e_{k}\wedge e_{l}
\end{align*}
%
En particular, si $\{\varepsilon^{1},\,\varepsilon^{2},\,%
\varepsilon^{3},\,\varepsilon^{4}\}$ es la base de
$\dual{(\bb{R}^{4})}$ dual de $\{e_{1},\,e_{2},\,e_{3},\,e_{4}\}$,
entonces, tomando $\phi=\varepsilon^{1}$, por ejemplo, se tiene que
\begin{align*}
	(\varepsilon^{1}\convol x)\wedge x
	& \,=\,
	\sbox0{$\begin{smallmatrix} k<l \\k,l\not =j\end{smallmatrix}$}
		\sum_{1<j}\,\sum_{\usebox{0}}\,a^{1j}a^{kl}\cdot 1\,
			e_{j}\wedge e_{k}\wedge e_{l} \\
	& \qquad\qquad \,-\,
	\sbox1{$\begin{smallmatrix} k<l \\k,l\not =i\end{smallmatrix}$}
		\sum_{i<1}\,\sum_{\usebox{1}}\,a^{i1}a^{kl}\cdot 1\,
			e_{i}\wedge e_{k}\wedge e_{l}
	\text{ .}
\end{align*}
%
Las sumatorias que aparecen restando son vac\'{\i}as pues se suma sobre
$i<1$, con lo cual
\begin{align*}
	(\varepsilon^{1}\convol x)\wedge x & \,=\,
		a^{12}\,\big( a^{13}\,e_{2}\wedge e_{1}\wedge e_{3} +
			a^{14}\,e_{2}\wedge e_{1}\wedge e_{4} +
			a^{34}\,e_{2}\wedge e_{3}\wedge e_{4}\big) \\
	& \quad\,+\,a^{13}\,\big( a^{12}\,e_{3}\wedge e_{1}\wedge e_{2} +
			a^{14}\,e_{3}\wedge e_{1}\wedge e_{4} +
			a^{24}\,e_{3}\wedge e_{2}\wedge e_{4}\big) \\
	& \quad\,+\,a^{14}\,\big( a^{12}\,e_{4}\wedge e_{1}\wedge e_{2} +
			a^{13}\,e_{4}\wedge e_{1}\wedge e_{3} +
			a^{23}\,e_{4}\wedge e_{2}\wedge e_{3}\big) \\
	& \,=\, (-a^{12}a^{13}+a^{12}a^{13})\,e_{1}\wedge e_{2}\wedge e_{3} +
		(-a^{12}a^{14}+a^{12}a^{14})\,e_{1}\wedge e_{2}\wedge e_{4} \\
	& \quad\,+\,
		(-a^{13}a^{14}+a^{13}a^{14})\,e_{1}\wedge e_{3}\wedge e_{4} +
		(a^{12}a^{34}-a^{13}a^{24}+a^{14}a^{23})\,
			e_{2}\wedge e_{3}\wedge e_{4} \\
	& \,=\,	(a^{12}a^{34}-a^{13}a^{24}+a^{14}a^{23})\,
			e_{2}\wedge e_{3}\wedge e_{4}
\end{align*}
%
Se deduce de esto que, si $(\phi\convol x)\wedge x=0$ para toda funcional
$\phi\in\dual{(\bb{R}^{4})}$ y $x=\sum_{i<j}\,a^{ij}\,e_{i}\wedge e_{j}$,
entonces se debe cumplir que
\begin{equation}
	\label{eq:pluckerparaplanos}
	a^{12}a^{34}-a^{13}a^{24}+a^{14}a^{23} \,=\,0
	\text{ .}
\end{equation}
%
Las otras funcionales de la base, $\varepsilon^{2},\,\varepsilon^{3},\,%
\varepsilon^{4}$, no determinan nuevas relaciones entre los coeficientes
de $x$. De hecho, asumiendo se cumple la igualdad anterior, se
puede deducir que $x$ es un tensor elemental y que por lo tanto
$(\phi\convol x)\wedge x=0$ para toda funcional $\phi$.

Los tensores elementales $f_{1}\wedge f_{2}\in\exterior[2]{\bb{R}^{4}}$
tienen otra propiedad que no es cierta en general para un elemento
arbitrario del producto exterior:
\begin{align*}
	(f_{1}\wedge f_{2})\wedge (f_{1}\wedge f_{2}) & \,=\,0
	\text{ .}
\end{align*}
%
Si $x\in\exterior[2]{\bb{R}^{4}}$ es un elemento arbitrario y
$x=\sum_{i<j}\,a^{ij}\,e_{i}\wedge e_{j}$, entonces
\begin{align*}
	x\wedge x & \,=\,
		a^{12}a^{34}\,e_{1}\wedge e_{2}\wedge e_{3}\wedge e_{4} +
		a^{13}a^{24}\,e_{1}\wedge e_{3}\wedge e_{2}\wedge e_{4} \\
	& \quad\,+\,
		a^{14}a^{23}\,e_{1}\wedge e_{4}\wedge e_{2}\wedge e_{3} +
		a^{23}a^{14}\,e_{2}\wedge e_{3}\wedge e_{1}\wedge e_{4} \\
	& \quad\,+\,
		a^{24}a^{13}\,e_{2}\wedge e_{4}\wedge e_{1}\wedge e_{3} +
		a^{34}a^{12}\,e_{3}\wedge e_{4}\wedge e_{1}\wedge e_{2} \\
	& \,=\,2\cdot(a^{12}a^{34}-a^{13}a^{24}+a^{14}a^{23})\,
		e_{1}\wedge e_{2}\wedge e_{3}\wedge e_{4}
	\text{ .}
\end{align*}
%
De esto se deduce, como la caracter\'{\i}stica de $\bb{R}$ es $0$, que
$x\wedge x=0$, si y s\'{o}lo si se cumple \eqref{eq:pluckerparaplanos}.

Sea $x\in\exterior[2]{\bb{R}^{4}}$ con $x=\sum_{i<j}\,%
a^{ij}\,e_{i}\wedge e_{j}$ y $x\not =0$. Alguno de los coeficientes
$a^{ij}$ debe ser distinto de cero. Asumiendo que es $a^{12}\not =0$,
sean $f_{1},f_{2}\in\bb{R}^{4}$ los vectores dados por
\begin{align*}
	f_{1} & \,=\, e_{1}+p^{13}\,e_{3}+p^{14}\,e_{4} \\
	f_{2} & \,=\, e_{2}+p^{23}\,e_{3}+p^{24}\,e_{4}
	\text{ .}
\end{align*}
%
El producto de estos dos vectores est\'{a} dado por
\begin{align*}
	f_{1}\wedge f_{2} & \,=\, e_{1}\wedge e_{2}+p^{23}\,e_{1}\wedge e_{3}
		+p^{24}\,e_{1}\wedge e_{4}+p^{13}\,e_{3}\wedge e_{2} \\
	& \quad +p^{13}p^{24}\,e_{3}\wedge e_{4}+p^{14}\,e_{4}\wedge e_{2}
		+p^{14}p^{23}\,e_{4}\wedge e_{3}
\end{align*}
%
El elemento $x$ y $a^{12}\cdot f_{1}\wedge f_{2}$ son iguales en el
coeficiente de $e_{1}\wedge e_{2}$. La igualdad
\begin{align*}
	x & \,=\, a^{12}\cdot f_{1}\wedge f_{2}
\end{align*}
%
se cumple, si y s\'{o}lo si se verifica el siguiente sistema de ecuaciones:
\begin{equation}
	\label{eq:tensorelemental}
	\begin{aligned}
		a^{13} & \,=\, a^{12}p^{23} \\
		a^{14} & \,=\, a^{12}p^{24} \\
		a^{23} & \,=\, -a^{12}p^{13} \\
		a^{24} & \,=\, -a^{12}p^{14} \\
		a^{34} & \,=\,a^{12}\,(p^{13}p^{24}-p^{14}p^{23})
	\end{aligned}
\end{equation}
%
Las primeras cuatro ecuaciones determinan los valores de los coeficientes
de $f_{1}$ y de $f_{2}$:
\begin{align*}
	p^{23} \,=\,\frac{a^{13}}{a^{12}} & \quad\text{,}\quad
		p^{24} \,=\,\frac{a^{14}}{a^{12}} \text{ ,} \\
	p^{13} \,=\,\frac{-a^{23}}{a^{12}} & \quad\text{,}\quad
		p^{14} \,=\,\frac{-a^{24}}{a^{12}}
	\text{ .}
\end{align*}
%
Reemplazando en la \'{u}ltima ecuaci\'{o}n, se deduce,
asumiendo que $a^{12}\not =0$, que $x=a^{12}\cdot f_{1}\wedge f_{2}$, si y
s\'{o}lo si se verifica \eqref{eq:pluckerparaplanos}. Pero esto es equivalente
a $x\wedge x=0$. En definitiva, vale el siguiente resultado.

\begin{teoEquivalenciasTensoresElementalesRCuatro}%
	\label{thm:equivtensoreselementalesrcuatro}
	Sea $x\in\exterior[2]{\bb{R}^{4}}$ con $x=\sum_{i<j}\,
	a^{ij}\,e_{i}\wedge e_{j}$ y $x\not =0$. Entonces $x$ es un
	tensor elemental, si y s\'{o}lo si se cumple cualquiera de las
	siguientes condiciones equivalentes:
	\begin{itemize}
		\item[(a)] $x\wedge x=0$, o bien,
		\item[(b)] $(\phi\convol x)\wedge x=0$ para toda funcional
			$\phi\in\dual{(\bb{R}^{4})}$.
	\end{itemize}
	%
	En coordenadas, estas condiciones se traducen en
	\begin{align*}
		a^{12}a^{34}-a^{24}a^{13}+a^{14}a^{23} & \,=\,0
		\text{ .}
	\end{align*}
	%
	Tambi\'{e}n en coordenadas, si se verifica esta igualdad y
	$a^{12}\not =0$, entonces $x=a^{12}\cdot f_{1}\wedge f_{2}$, donde
	\begin{align*}
		f_{1} & \,=\,e_{1}+\frac{-a^{23}}{a^{12}}\,e_{3}+
			\frac{-a^{24}}{a^{12}}\,e_{4}
		\quad\text{y} \\
		f_{2} & \,=\,e_{2}+\frac{a^{13}}{a^{12}}\,e_{3}+
			\frac{a^{14}}{a^{12}}\,e_{4}
		\text{ .}
	\end{align*}
	%
\end{teoEquivalenciasTensoresElementalesRCuatro}

\begin{coroTensoresElementalesRTres}%
	\label{thm:tensoreselementalesrtres}
	En $\exterior[2]{\bb{R}^{3}}$, todo tensor es elemental.
\end{coroTensoresElementalesRTres}

\begin{proof}
	Sea $\{e_{1},\,e_{2},\,e_{3}\}$ la base can\'{o}nica de
	$\bb{R}^{3}$. La base inducida en $\exterior[2]{\bb{R}^{3}}$ es
	\begin{align*}
		 & \{e_{1}\wedge e_{2},
		 	\,e_{1}\wedge e_{3},\,e_{2}\wedge e_{3}\}
		\text{ .}
	\end{align*}
	%
	Sea $x=x^{12}\,e_{1}\wedge e_{2}+x^{13}\,e_{1}\wedge e_{3}+%
	x^{23}\,e_{2}\wedge e_{3}\in\exterior[2]{\bb{R}^{3}}$ un
	elemento arbitrario y sea
	$\{\varepsilon^{1},\,\varepsilon^{2},\,\varepsilon^{3}\}$ la base
	dual de $\bb{R}^{3}$ respecto de $\{e_{1},\,e_{2},\,e_{3}\}$.
	La convoluci\'{o}n de $x$ por cada una funcional $\phi$ es igual a
	\begin{align*}
		\phi\convol x & \,=\,
			x^{12}\,\big(\phi(e_{1})\,e_{2}-\phi(e_{2})\,e_{1}\big)
			+
			x^{13}\,\big(\phi(e_{1})\,e_{3}-\phi(e_{3})\,e_{1}\big)
			\\
		& \quad +
			x^{23}\,\big(\phi(e_{2})\,e_{3}-\phi(e_{3})\,e_{2}\big)
		\text{ .}
	\end{align*}
	%
	En particular, tomando como $\phi$ las funcionales de la base dual,
	\begin{align*}
		\varepsilon^{1}\convol x & \,=\,x^{12}\,e_{2}+x^{13}\,e_{3}
		\text{ ,} \\
		\varepsilon^{2}\convol x & \,=\,x^{12}\,e_{1}+x^{23}\,e_{3}
		\quad\text{y} \\
		\varepsilon^{3}\convol x & \,=\,x^{13}\,e_{1}+(-x^{23})\,e_{2}
		\text{ .}
	\end{align*}
	%
	Tomando el producto contra $x$, se deduce que
	\begin{align*}
		(\varepsilon^{1}\convol x)\wedge x & \,=\,
			(x^{12}\,e_{2}+x^{13}\,e_{3})\wedge
			\big(x^{12}\,e_{1}\wedge e_{2}+
			x^{13}\,e_{1}\wedge e_{3}+
			x^{23}\,e_{2}\wedge e_{3}\big) \\
		& \,=\,x^{12}x^{13}\,e_{2}\wedge e_{1}\wedge e_{3}+
			x^{13}x^{12}\,e_{3}\wedge e_{1}\wedge e_{2}
			\,=\,0
	\end{align*}
	%
	y, de manera an\'{a}loga, que
	\begin{align*}
		(\varepsilon^{2}\convol x)\wedge x & \,=\,0
		\quad\text{y} \\
		(\varepsilon^{3}\convol x)\wedge x & \,=\,0
		\text{ ,}
	\end{align*}
	%
	tambi\'{e}n. Esto implica que $(\phi\convol x)\wedge x=0$
	para toda funcional $\phi:\,\bb{R}^{3}\rightarrow\bb{R}$.

	Ahora bien, si $\bb{R}^{3}$ se incluye en $\bb{R}^{4}$
	v\'{\i}a $e_{1}\mapsto e_{1}$, $e_{2}\mapsto e_{2}$ y
	$e_{3}\mapsto e_{3}$, entonces $\exterior[2]{\bb{R}^{3}}$ tambi\'{e}n
	se incluye en $\exterior[2]{\bb{R}^{4}}$ como el subespacio
	\begin{align*}
		\exterior[2]{\bb{R}^{3}} & \,=\,\bigg\{
			\sum_{i<j}\,a^{ij}\,e_{i}\wedge e_{j}\,:\,
			a^{14}=a^{24}=a^{34}=0\bigg\}
		\text{ .}
	\end{align*}
	%
	Por otro lado, toda funcional $\phi:\,\bb{R}^{3}\rightarrow\bb{R}$ se
	extiende a $\bb{R}^{4}$ tomando el valor $0$ en el elemento $e_{4}$
	de la base, de manera que $(\varepsilon^{i}\convol x)\wedge x=0$ para
	$i\in\{1,2,3\}$, si $x\in\exterior[2]{\bb{R}^{3}}$. Pero tambi\'{e}n
	vale que $\varepsilon^{4}\convol x=0$ y, en particular,
	\begin{align*}
		(\varepsilon^{4}\convol x)\wedge x & \,=\,0
		\text{ .}
	\end{align*}
	%
	Por lo tanto, $(\phi\convol x)\wedge x=0$ para toda funcional de
	$\bb{R}^{4}$ y $x=f\wedge g$ para ciertos vectores $f,g\in\bb{R}^{4}$.
	Dado que $\varepsilon^{4}\convol x=0$, debe valer
	\begin{align*}
		0 & \,=\varepsilon^{4}\convol(f\wedge g) \,=\,
			\varepsilon^{4}(f)\,g-\varepsilon^{4}(g)\,f
		\text{ .}
	\end{align*}
	%
	Pero, entonces, si $\varepsilon^{4}(f)\not=0$ o
	$\varepsilon^{4}(g)\not =0$, los vectores $f$ y $g$ no son linealmente
	independientes y $x=f\wedge g=0$. Es decir, si $x\not =0$, entonces
	$\varepsilon^{4}(f)=\varepsilon^{4}(g)=0$ y $f$ y $g$ pertenecen
	a $\bb{R}^{3}$. Con lo cual, $x=f\wedge g$ en
	$\exterior[2]{\bb{R}^{3}}$.
\end{proof}

Una manera m\'{a}s simple de demostrar que todo elemento de
$\exterior[2]{\bb{R}^{3}}$ es elemental es notando que, si
$x=x^{12}\,e_{1}\wedge e_{2}+x^{13}\,e_{1}\wedge e_{3}+%
x^{23}\,e_{2}\wedge e_{3}$ y $x^{12}\not =0$, entonces, mediante un
argumento similar al del caso $\bb{R}^{4}$, tomando los vectores
\begin{align*}
	f_{1} & \,=\,e_{1}+p^{13}\,e_{3}\quad\text{y} \\
	f_{2} & \,=\,e_{2}+p^{23}\,e_{3}\text{ ,}
\end{align*}
%
entonces $x=x^{12}\cdot f_{1}\wedge f_{2}$, si y s\'{o}lo si
\begin{align*}
	p^{23} & \,=\,\frac{x^{13}}{x^{12}}\quad\text{y} \\
	p^{13} & \,=\,\frac{-x^{23}}{x^{12}}\text{ .}
\end{align*}
%
Si alg\'{u}n otro coeficiente es distinto de cero, se deber\'{a} modificar
la definici\'{o}n de estos vectores, pero tambi\'{e}n es posible escribir
a $x$ como tensor elemental.

