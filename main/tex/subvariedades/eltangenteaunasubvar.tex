\theoremstyle{plain}
\newtheorem{teoRestringirElDominio}{Teorema}[section]
\newtheorem{teoCorrestringirElCodominio}[teoRestringirElDominio]{Teorema}
\newtheorem{teoUnicidadDeLaEstructuraRegular}[teoRestringirElDominio]{Teorema}
\newtheorem{teoUnicidadDeLaEstructuraInmersa}[teoRestringirElDominio]{Teorema}
\newtheorem{teoUnicidadDebilmenteRegular}[teoRestringirElDominio]{Teorema}
\newtheorem{lemaDeExtensiones}[teoRestringirElDominio]{Lema}
\newtheorem{propoDeNoExtensiones}[teoRestringirElDominio]{Proposici\'{o}n}
\newtheorem{propoTangenteSubvarCoordenadas}[teoRestringirElDominio]%
	{Proposici\'{o}n}
\newtheorem{propoTangenteSubvarRegularI}[teoRestringirElDominio]%
	{Proposici\'{o}n}
\newtheorem{propoTangenteSubvarRegularII}[teoRestringirElDominio]%
	{Proposici\'{o}n}
\newtheorem{propoTangenteBordeCurvas}[teoRestringirElDominio]{Proposici\'{o}n}
\newtheorem{propoBordeDeNivel}[teoRestringirElDominio]{Proposici\'{o}n}

\theoremstyle{remark}
\newtheorem{obsUnicidadDeLaEstructuraRegular}{Observaci\'{o}n}[section]
\newtheorem{obsTangenteValorRegular}[obsUnicidadDeLaEstructuraRegular]%
	{Observaci\'{o}n}
\newtheorem{obsTangenteConjuntoDeNivel}[obsUnicidadDeLaEstructuraRegular]%
	{Observaci\'{o}n}
\newtheorem{obsBordeDeNivel}[obsUnicidadDeLaEstructuraRegular]%
	{Observaci\'{o}n}
\newtheorem{obsDiferencialComoFuncional}[obsUnicidadDeLaEstructuraRegular]%
	{Observaci\'{o}n}

%-------------

\subsection{Restricci\'{o}n y correstricci\'{o}n de transformaciones}
Ha surgido, en algunas ocasiones, la necesidad de restringir o de
correstringir una transformaci\'{o}n suave. Empecemos, pues, esta secci\'{o}n
estudiando el comportamiento de una transformaci\'{o}n suave al restringir
su dominio o su codominio. En particular, veamos algunos casos en los que
la suavidad de la transformaci\'{o}n se preserva.

\begin{teoRestringirElDominio}[restricci\'{o}n de dominio]%
	\label{thm:restringireldominio}
	Sea $F:\,M\rightarrow N$ una transformaci\'{o}n suave. Sea $S\subset M$
	una subvariedad (inmersa o regular). Entonces $F|_{S}:\,S\rightarrow N$
	es suave.
\end{teoRestringirElDominio}

\begin{proof}
	La restricci\'{o}n de $F$ a la subvariedad $S$ es una composici\'{o}n
	de funciones suaves: $F|_{S}=F\circ\inc[S]$.
\end{proof}

\begin{teoCorrestringirElCodominio}[correstricci\'{o}n del codominio]%
	\label{thm:correstringirelcodominio}
	Sea $M$ una variedad diferencial \emph{sin} borde y sea
	$F:\,N\rightarrow M$ una transformaci\'{o}n suave. Sea $S\subset M$
	una subvariedad (inmersa o regular) tal que $F(N)\subset S$.
	Si $F|^{S}:\,N\rightarrow S$ es continua, entonces es suave.
\end{teoCorrestringirElCodominio}

\begin{proof}
	Sean $p\in N$ y $q=F(p)\in S$. Sea $V\subset S$ un abierto tal que
	$q\in V$ y tal que $\inc[V]:\,V\rightarrow M$ sea embedding. Como
	$V\subset M$ es una subvariedad regular, existe una carta preferencial
	$(W,\psi)$ para $V$ en $M$ en $q$. A partir de esta carta preferencial,
	definimos una carta para $V$: sea $V_{0}=V\cap W$ y sea
	$\psi_{0}=\pi\circ\psi|_{V}=\pi\circ\psi\circ\inc[V]$, donde
	$\pi:\,\bb{R}^{n}\rightarrow\bb{R}^{k}$ es la proyecci\'{o}n en
	las primeras $k$ coordenadas. El par $(V_{0},\psi_{0})$ es una
	carta para $V$ en $q$. Como $\inc[V]$ es continua,
	$V_{0}=\inc[V]^{-1}(W)$ es abierta en $V$. Como $V$ es abierto en $S$,
	$V_{0}$ es abierto en $S$. Asumiendo que la correstricci\'{o}n
	$F|:\,N\rightarrow S$ es continua, la preimagen $U=F^{-1}(V_{0})$ es
	abierta en $N$. Adem\'{a}s, $p\in U$. Sea $(U_{0},\varphi)$ una
	carta para $N$ en $p$ con $U_{0}\subset U$. Entonces
	$\psi_{0}\circ F|\circ\varphi^{-1}:%
	\,\varphi(U_{0})\rightarrow\psi_{0}(v_{0})$ es igual a
	\begin{align*}
		\psi_{0}\circ F|\circ\varphi^{-1} & \,=\,
			\pi\circ (\psi\circ F\circ\varphi^{-1})
		\text{ .}
	\end{align*}
	%
	Como $F:\,N\rightarrow M$ es suave, $\psi\circ F\circ\varphi^{-1}$ es
	suave y, por lo tanto, $\psi_{0}\circ F|\circ\varphi^{-1}$ es suave.
	En definitiva, la correstricci\'{o}n $F|:\,N\rightarrow S$ es suave.
\end{proof}

Una subvariedad inmersa $S\subset M$ se dice \emph{d\'{e}bilmente regular}
(?), si, para toda transformaci\'{o}n suave $F:\,N\rightarrow M$ tal que
$F(N)\subset S$, la correstricci\'{o}n $F|^{S}:\,N\rightarrow S$ es suave.
Vale que

\begin{center}
\begin{tikzcd}
	\left\lbrace
		\begin{array}{c}
			\text{subvariedades} \\
			\text{regulares}
		\end{array}
	\right\rbrace \arrow[r, phantom,"\supsetneq"] &
	\left\lbrace
		\begin{array}{c}
			\text{subvariedades} \\
			\text{d\'{e}bilmente regulares}
		\end{array}
	\right\rbrace \arrow[r, phantom, "\supsetneq"] &
	\left\lbrace
		\begin{array}{c}
			\text{subvariedades} \\
			\text{inmersas}
		\end{array}
	\right\rbrace
\end{tikzcd}
\end{center}
Por ejemplo, la lemniscata es inmersa pero no d\'{e}bilmente regular y
la curva irracional en el toro es d\'{e}bilmente regular pero no
regular.

\begin{teoUnicidadDeLaEstructuraRegular}[unicidad de la estructura en una %
	subvariedad regular]\label{thm:unicidaddelaestructuraregular}
	Sea $M$ una variedad diferencial y sea $S\subset M$ una subvariedad
	regular. La topolog\'{\i}a de subespacio y la estructura diferencial
	dad son las \'{u}nica con respecto a las cuales $S$ es una
	subvariedad regular de $M$.
\end{teoUnicidadDeLaEstructuraRegular}

\begin{proof}
	Sea $\tilde{S}$ una variedad diferencial obtenida a partir del
	conjunto $S$ imponiendo una topolog\'{\i}a y una estructura diferencial
	posiblemente distintas. Sean $\iota=\inc[S]:\,S\rightarrow M$ e
	$\tilde{\iota}=\inc[\tilde{S}]:\,\tilde{S}\rightarrow M$ las
	respectivas inclusiones. Supongamos, adem\'{a}s, que $\tilde{\iota}$
	es una inmersi\'{o}n. Como $\iota$ es un embedding e
	$\tilde{\iota}(\tilde{S})\subset S$, podemos concluir que la
	correstricci\'{o}n $\tilde{\iota}|:\,\tilde{S}\rightarrow S$ es
	suave (porque es continua). Entonces,
	\begin{align*}
		\iota\circ(\tilde{\iota}|) & \,=\,\tilde{\iota}
			\quad\text{y} \\
		\diferencial[p]{\iota}\cdot\diferencial[p]{(\tilde{\iota}|)}
			& \,=\,\diferencial[p]{\tilde{\iota}}
			\,:\,\tangente[p]{\tilde{S}}\rightarrow
				\tangente[p]{S}\rightarrow\tangente[p]{M}
		\text{ .}
	\end{align*}
	%
	Como $\diferencial[p]{\tilde{\iota}}$ es, por hip\'{o}tesis, inyectivo,
	el diferencial $\diferencial[p]{(\tilde{\iota}|)}$ debe serlo,
	tambi\'{e}n. Pero esto es cierto para todo punto $p\in\tilde{S}$,
	con lo que $\tilde{\iota}|:\,\tilde{S}\rightarrow S$ es de rango
	constante. Como, adem\'{a}s, $\tilde{\iota}|$ es biyectiva, el
	teorema global del rango \ref{thm:} implica que $\tilde{\iota}|$
	es difeomorfismo. Pero $\tilde{\iota}|=\id[S]$ como funci\'{o}n
	de conjuntos, con lo cual, concluimos que $\tilde{S}$ tiene
	exactamente \emph{la misma} topolog\'{\i}a y \emph{la misma}
	estructura diferencial que $S$.
\end{proof}

\begin{obsUnicidadDeLaEstructuraRegular}%
	\label{obs:unicidaddelaestructuraregular}
	Por el teorema \ref{thm:delasfetas} toda subvariedad regular verifica
	localmente la propiedad de ser $k$-feta para alg\'{u}n entero
	$k\leq n$. A su vez, esta \emph{estructura de $k$-feta} en el
	\emph{subconjunto} $S$ determina una estructura de subvariedad
	regular de $M$. Podemos concluir, por el teorema anterior, que estas
	dos estructuras coinciden. Adem\'{a}s, si un subconjunto verifica la
	propiedad de ser $k$-feta localmente, entonces tiene estructura de
	subvariedad regular y dicha estructura es \'{u}nica.
\end{obsUnicidadDeLaEstructuraRegular}

\begin{teoUnicidadDeLaEstructuraInmersa}%
	\label{thm:unicidaddelaestructurainmersa}
	Sea $M$ una variedad diferencial y sea $S\subset M$ una
	subvariedad (inmersa). Fijada la topolog\'{\i}a de $S$, existe una
	\'{u}nica estructura diferencial con respecto a la cual
	$S\subset M$ es subvariedad.
\end{teoUnicidadDeLaEstructuraInmersa}

\begin{proof}
	El \'{u}nico obst\'{a}culo en la demostraci\'{o}n en el caso
	en que $S$ fuere una subvariedad meramente inmersa es que no queda
	garantizado que la correstricci\'{o}n
	$\tilde{\iota}|:\,\tilde{S}\rightarrow S$ sea suave. Pero, si
	asumimos que las topolog\'{\i}as coinciden, como
	$\tilde{\iota}|=\id[S]$, concluimos que es continua y, por lo tanto,
	suave.
\end{proof}

\begin{teoUnicidadDebilmenteRegular}\label{thm:unicidaddebilmenteregular}
	Sea $M$ una variedad diferencial y sea $S\subset M$ una
	subvariedad d\'{e}bilmente regular. Entonces la topolog\'{\i}a y la
	estructura suave en $S$ dadas son las \'{u}nicas respecto de las
	cuales $S\subset M$ es una subvariedad (inmersa).
\end{teoUnicidadDebilmenteRegular}

\begin{proof}
	La demostraci\'{o}n de este resultado es an\'{a}loga a la de
	\ref{thm:unicidaddelaestructuraregular}, la \'{u}nica diferencia
	est\'{a} en la justificaci\'{o}n de que la correstricci\'{o}n
	$\tilde{\iota}|:\,\tilde{S}\rightarrow S$ es suave. En aquel caso,
	esto se deb\'{\i}a a que $S$ era regular e $\tilde{\iota}|$ era la
	correstricci\'{o}n a $S$, un subespacio, de una transformaci\'{o}n
	suave que resulta continua y, en consecuencia, suave. En este caso,
	esto queda garantizado porque $S$ se supone d\'{e}bilmente regular.
\end{proof}

\subsection{Extensi\'{o}n de funciones}
Dada una subvariedad $S\subset M$, hemos dado dos definiciones de lo que
significa que una funci\'{o}n definida en $S$ sea suave: por un lado,
podemos decir que una funci\'{o}n $f:\,S\rightarrow\bb{R}$ es suave, si es
suave respecto de la estructura diferencial en $S$, es decir, si es suave en
sentido usual tomando coordenadas; por otro lado, podemos decir que $f$ es
suave, si para cada punto $p\in S$ existe un abierto $U$ de $M$ tal que
$p\in U$ y una extensi\'{o}n suave $\tilde{f}:\,U\rightarrow\bb{R}$, es decir,
si $f$ es suave en $S$ como subconjunto de $M$. Usaremos la notaci\'{o}n
$C^{\infty}(S)$ para referirnos a las funciones suaves en $S$ en tanto
variedad diferencial. En el caso de subvariedades regulares, las dos
nociones coinciden. Notemos que, en el caso de las subvariedades que son
meramente inmersas, tambi\'{e}n es posible hacer coincidir ambas nociones,
si nos restringimos a un entorno \emph{de la subvariedad} suficientemente
peque\~{n}o alrededor de cada punto.

\begin{lemaDeExtensiones}\label{thm:deextensiones}
	Sea $M$ una variedad diferencial. Sea $S\subset M$ una subvariedad
	y sea $f\in C^{\infty}(S)$. \emph{(a)} Si $S$ es subvariedad regular,
	entonces existen un entorno $S\subset U\subset M$ y una funci\'{o}n
	suave $\tilde{f}\in C^{\infty}(U)$ tal que $\tilde{f}|_{S}=f$.
	\emph{(b)} Si, adem\'{a}s, $S$ es cerrada (regular propia), entonces
	se puede tomar $U=M$.
\end{lemaDeExtensiones}

\begin{proof}
	Supongamos que $\dim\,S=k$. Para cada $p\in S$ existe una carta
	$(U_{p},\varphi_{p})$ para $M$ centrada en $p$ tal que
	$S\cap U_{p}=\{x^{k+1}=0,\,\dots,\,x^{n}=0\}$. Sea
	$U=\bigcup_{p}\,U_{p}$. Entonces $S\subset U$ y $U$ es abierto en
	$M$. Para cada punto $p\in S$, definimos una funci\'{o}n
	$f_{p}:\,U_{p}\rightarrow\bb{R}$:
	\begin{align*}
		f_{p}(\lista*{x}{k},\,x^{k+1},\,\dots,\,x^{n}) & \,=\,
			f(\lista*{x}{k})
		\text{ .}
	\end{align*}
	%
	Expl\'{\i}citamente, $f_{p}(q)=f(\varphi^{-1}(\pi\circ\varphi(q)))$,
	donde $\pi:\,\bb{R}^{n}\rightarrow\bb{R}^{k}$ es la proyecci\'{o}n en
	las primeras $k$ coordenadas. Cada funci\'{o}n $f_{p}$ es suave y,
	si $q\in S\cap U_{p}$, entonces $f_{p}(q)=f(q)$.
	Para pegar estas funciones y as\'{\i} definir una funci\'{o}n
	en $U$, usamos particiones de la unidad. El abierto $U$ es una
	variedad (diferencial) y $\{U_{p}\}_{p\in S}$ es un cubrimiento por
	abiertos. Entonces existe una partici\'{o}n suave de la unidad
	subordinada al cubrimiento, $\{\psi_{p}\}_{p}$. Para cada punto
	$p\in S$, el producto $\psi_{p}f_{p}$ est\'{a} definido \'{u}nicamente
	en $U_{p}$, pero $V_{p}=\{\psi_{p}\not =0\}$ es abierto
	y $\clos{V_{p}}\subset U_{p}$. No es cierto \emph{a priori} que
	$\psi_{p}(p)\not =0$, pero podr\'{\i}amos tener un poco m\'{a}s
	de cuidado en la definici\'{o}n del cubrimiento como para que
	as\'{\i} sea (por ejemplo, tomando bolas coordenadas regulares
	dentro de cada $U_{p}$ y una partici\'{o}n subordinada al cubrimiento
	formado por las bolas coordenadas que contienen a las bolas
	coordenadas regulares). De todas maneras, podemos extender el
	producto $\psi_{p}f_{p}$ a todo $U$ por cero fuera de
	$\soporte{\psi_{p}}$. Esta funci\'{o}n sigue siendo suave. Sea,
	entonces
	\begin{align*}
		\tilde{f} & \,=\,\sum_{p\in S}\,\psi_{p}f_{p}
		\text{ .}
	\end{align*}
	%
	La funci\'{o}n $\tilde{f}$ est\'{a} definida en todo el abierto
	$U$ y es suave. Si $q\in S$,
	\begin{align*}
		\tilde{f}(q) & \,=\,\sum_{p\in S}\,\psi_{p}(q)f(q) \,=\,f(q)
		\text{ .}
	\end{align*}
	%
	Entonces $\tilde{f}\in C^{\infty}(U)$ y $\tilde{f}|_{S}=f$.

	Si asumimos que $S$ es subvariedad regular propia, entonces
	$U_{0}=S\subset M$ es un subconjunto cerrado y
	$\{U_{p}\}_{p\in S}\cup\{U_{0}\}$ es un cubrimiento de $M$ por
	abiertos. Si $\{\psi_{p}\}_{p}\cup\{\psi_{0}\}$ es una partici\'{o}n
	suave subordinada a este cubrimiento, entonces, definiendo
	$\tilde{f}=\sum_{p\in S}\psi_{p}f_{p}$, como antes, pero sin
	incluir la funci\'{o}n $\psi_{0}$, obtenemos una funci\'{o}n
	suave definida en $U=U_{0}\cup\bigcup_{p\in S}\,U_{p}=M$ que
	extiende a $f$.
\end{proof}

En general, si $S$ no es cerrada en $M$, no es posible extender una
funci\'{o}n arbitraria $f\in C^{\infty}(S)$ a toda la variedad: por ejemplo,
si $S=(-1,0)\cup (0,1)$ y $M=\bb{R}^{1}$, entonces la funci\'{o}n
$f$ que toma el valor $-1$ en $(-1,0)$ y el valor $1$ en $(0,1)$, es suave,
pero no se puede extender de manera suave a $\bb{R}^{1}$. De hecho, no se
puede extender de manera continua. Si existe una extensi\'{o}n continua,
?`existe una extensi\'{o}n suave? Por otro lado, si $S$ es una subvariedad
inmersa, las funciones suaves en $S$, en tanto variedad, no son, en general,
suaves como funciones definidas en el subconjunto $S\subset M$.
Si $f\in C^{\infty}(S)$ y $f:\,S\rightarrow\bb{R}$ es continua en $S$ como
subespacio de $M$, ?`es $f$ suave en $S$ como subconjunto de $M$?

Dejamos de lado estas preguntas. Las propiedades enunciadas en el lema
\ref{thm:deextensiones} son, en realidad, equivalencias.

\begin{propoDeNoExtensiones}\label{thm:denoextensiones}
	Sea $M$ una variedad diferencial y sea $S\subset M$ una subvariedad.
	\emph{(a)} Si toda funci\'{o}n $f\in C^{\infty}(S)$ se puede extender
	a una funci\'{o}n suave definida en alg\'{u}n abierto
	$U\subset M$ tal que $S\subset U$, entonces $S$ es subvariedad
	regular. \emph{(b)} Si toda funci\'{o}n $f\in C^{\infty}(S)$ se puede
	extender de manera suave a toda la variedad $M$, entonces $S$
	es una subvariedad propia.
\end{propoDeNoExtensiones}

\begin{proof}
	Supongamos que $S$ no es regular. Sea $p\in S$ un punto para
	el cual no existe una carta preferencial en $M$. Aun as\'{\i},
	existe un abierto $V\subset S$ tal que $V$ es una subvariedad
	regular de $M$. Como $S$ es una variedad diferencial, existe
	una partici\'{o}n de la unidad $\{\psi_{0},\psi_{1}\}$ para
	$S$ subordinada al cubrimiento $\{S\setmin\{p\},V\}$. En particular,
	$f=\psi_{1}|_{V}$ es suave en $V$ y $f(p)=1$. Como $V$ es una
	subvariedad regular de $M$, existe un abierto $U\subset M$ tal que
	$V\subset U$ y existe una funci\'{o}n $\tilde{f}:\,U\rightarrow\bb{R}$
	suave tal que $\tilde{f}|_{V}=f$. Pero, en $U\cap S$, no es cierto
	que $\tilde{f}$ coincida con $\psi_{1}$: como $p$ no admite una
	carta preferencial, si $\{U_{n}\}_{n\geq 1}$ es una sucesi\'{o}n
	de entornos de $p$ en $M$ con $U_{n+1}\subset U_{n}\subset U$ y
	$\bigcap_{n}\,U_{n}=\{p\}$, entonces existe
	\begin{align*}
		y_{n} & \,\in\, (S\cap U_{n})\setmin V
		\text{ .}
	\end{align*}
	%
	Evaluando $\tilde{f}$ en estos puntos, por continuidad de $\tilde{f}$
	en $M$, debe valer $\tilde{f}(y_{n})\to \tilde{f}(p)=f(p)=1$, pero
	$\psi_{1}(y_{n})=0$ para todo $n\geq 1$. En definitiva, la funci\'{o}n
	$\psi_{1}$ no se puede extender de manera suave (continua) a un
	abierto que contenga a $S$, pues, si se pudiese, dicha extensi\'{o}n
	deber\'{\i}a tomar valor $1$ en $p$ y $0$ en puntos arbitrariamente
	cerca (en $M$) de $p$.

	Supongamos que $S\subset M$ es una subvariedad (inmersa o regular),
	pero que no es cerrada como subconjunto de $M$. Entonces existe una
	sucesi\'{o}n de puntos $y_{n}\in S$ y un punto $y\in M\setmin S$
	tales que $y_{n}\to y$ en la topolog\'{\i}a de $M$ y tales que
	$\{y_{n}\}_{n}$ es un subconjunto cerrado en $S$. Para cada
	entero $n\geq 1$ sea $f(y_{n})=n$. Como $\{y_{n}\}_{n}$ es cerrado
	en $S$, podemos extender $f$ de manera suave a $S$ (por ejemplo,
	tomando entornos $V_{n}$ de cada punto $y_{n}$ de manera que
	$y_{m}\not\in V_{n}$, si $m\not =n$, y una partici\'{o}n
	de la unidad subordinada a estos conjuntos y a
	$S\setmin\{y_{n}\}_{n}$). Pero no hay manera de extender $f$ a una
	funci\'{o}n suave en $y$ (ni siquiera de manera continua).
\end{proof}

\subsection{El espacio tangente a una subvariedad}
Sea $M$ una variedad diferencial y sea $S\subset M$ una subvariedad.
Dado que la inclusi\'{o}n $\inc[S]:\,S\rightarrow M$ es una inmersi\'{o}n,
en cada punto $p\in S$, el diferencial $\diferencial[p]{\inc}:%
\,\tangente[p]{S}\rightarrow\tangente[p]{M}$ es una transformaci\'{o}n
lineal inyectiva. Ya vimos en alguna ocasi\'{o}n c\'{o}mo el diferencial
permite identificar el espacio tangente a un abierto en un punto dado.
Aquella identificaci\'{o}n funciona de manera general en el contexto de
subvariedades. Sea $v\in\tangente[p]{S}$ un vector tangente a $S$ en $p$ y
sea $\tilde{v}=\diferencial[p]{\inc}$ su imagen en $\tangente[p]{M}$.
En tanto derivaci\'{o}n, dada $f\in C^{\infty}(M)$,
\begin{align*}
	\tilde{v}\,f & \,=\,\diferencial[p]{\inc}(v)\,f \,=\,
		v(f\circ\inc) \,=\,v(f|_{S})
	\text{ .}
\end{align*}
%

Una forma alternativa de caracterizar el espacio tangente $\tangente[p]{S}$
cuando $S$ es una subvariedad arbitraria como subespacio de
$\tangente[p]{M}$ es en t\'{e}rminos de curvas suaves. Sabemos que
todo vector tangente a una variedad es igual a la velocidad de alguna curva.
Supongamos, entonces, en primer lugar, que $\gamma:\,J\rightarrow M$ es una
curva suave con origen en $p$ tal que $\gamma(J)\subset S$ y
$\eta=\gamma|:\,J\rightarrow S$ es suave, tambi\'{e}n. Entonces, tomando
diferenciales,
\begin{align*}
	\dot{\gamma}(0) & \,=\,\diferencial[p]{\inc}(\dot{\eta}(0))
	\text{ ,}
\end{align*}
%
de lo que se deduce que el vector tangente $\dot{\gamma}(0)\in\tangente[p]{M}$
pertenece a la imagen del diferencial $\diferencial[p]{\inc}$.
Rec\'{\i}procamente, si $v\in\tangente[p]{S}$ y $\eta:\,J\rightarrow S$ es
una curva suave con origen en $p$ y velocidad $\dot{\eta}(0)=v$, entonces
$\gamma=\inc\circ\eta:\,J\rightarrow M$ es suave y
$\dot{\gamma}(0)=\diferencial[p]{\inc}(v)=\tilde{v}$. En definitiva,
en t\'{e}rminos de curvas, un vector $v$ tangente a $M$ en $p$
\emph{es tangente a la subvariedad $S$}, si y s\'{o}lo si existe una curva
suave en $M$ cuya imagen est\'{e} contenida en $S$, que sea suave como
curva en $S$ y con velocidad $v$.

\begin{propoTangenteSubvarCoordenadas}\label{thm:tangentesubvarcoordenadas}
	Sea $M$ una variedad diferencial y sea $S\subset M$ una subvariedad.
	Sea $p\in S$ y sea $V\subset S$ un entorno de $p$ que es subvariedad
	regular de $M$. Sea $(U,\varphi)$ una carta para $M$ en $p$ tal que
	$V\cap U=\{x^{k+1}=0,\,\dots,\,x^{n}=0\}$. Entonces
	\begin{align*}
		\tangente[p]{S} & \,=\,\tangente[p]{V} \,=\,
		\generado{\gancho[p]{x^{1}},\,\dots,\,\gancho[p]{x^{k}}}
		\,\subset\,\tangente[p]{M}
		\text{ .}
	\end{align*}
	%
\end{propoTangenteSubvarCoordenadas}

\begin{proof}
	Sea $v\in\tangente[p]{S}$. Entonces
	\begin{align*}
		v & \,\equiv\,\diferencial[p]{\inc}(v) \,=\,
			\diferencial[p]{v}(x^{i})\gancho[p]{x^{i}} \\
		& \,=\, v(x^{i}|_{V\cap U})\gancho[p]{x^{i}} \,=\,
			\sum_{i=1}^{k}\,v(x^{i}|_{V\cap U})\gancho[p]{x^{i}}
		\text{ .}
	\end{align*}
	%
	Con lo cual
	$\tangente[p]{S}\subset\generado{\gancho[p]{x^{1}},\,\dots,\,%
	\gancho[p]{x^{k}}}$. Podemos concluir sabiendo que
	$\dim\,\tangente[p]{S}=k$. Otra manera de concluir que los
	subespacios coinciden es usando curvas: si $w=w^{i}\gancho[p]{x^{i}}$
	con $w^{k+1}=\,\cdot\,w^{n}=0$, entonces, tomamos
	$\gamma:(-\epsilon,\epsilon)\rightarrow M$ tal que
	$\gamma(0)=p$, $\dot{\gamma}(0)=w$ y $\epsilon>0$ suficientemente
	peque\~{n}o de manera que $\gamma(J)\subset U$. Precisamente,
	elegimos
	$\gamma(t)=\varphi^{-1}(w^{1}t,\,\dots,\,w^{k}t,\,0,\,\dots,\,0)$.
	En particular, $\gamma$ tiene imagen en $V$ y, como $V$ es una
	subvariedad regular de $M$, $\gamma$ es suave como curva en $V$.
	En definitiva, $w\in\tangente[p]{S}$.
\end{proof}

En el caso en que $S\subset M$ sea una subvariedad regular, el subespacio
$\tangente[p]{S}\subset\tangente[p]{M}$ est\'{a} caracterizado como el
conjunto de ceros de ciertas ecuaciones. Hay, al menos, dos maneras de
interpretar esta afirmaci\'{o}n.

\begin{propoTangenteSubvarRegularI}\label{thm:tangentesubvarregulari}
	Se $M$ una variedad diferencial (sin borde) y sea $S\subset M$ una
	subvariedad regular. Sea $p\in S$. En tanto subespacio de
	$\tangente[p]{M}$,
	\begin{align*}
		\tangente[p]{S} & \,=\,\left\lbrace v\in\tangente[p]{M}\,:\,
			v\,f=0\,\forall f\in C^{\infty}(M),\, f|_{S}=0
			\right\rbrace
		\text{ .}
	\end{align*}
	%
\end{propoTangenteSubvarRegularI}

\begin{propoTangenteSubvarRegularII}\label{thm:tangentesubvarregularii}
	Sea $M$ una variedad diferencial (sin borde) y sea $S\subset M$ una
	subvariedad regular. Si $\Phi:\,U\rightarrow N$ es una
	transformaci\'{o}n suave definida en un abierto $U$ de $M$ tal que
	$S\cap U=\Phi^{-1}(c)$, para cierto valor regular $c\in N$, entonces,
	para todo punto $p\in S\cap U$,
	\begin{align*}
		\tangente[p]{S} & \,=\, \ker\,\big(\diferencial[p]{\Phi}:\,
			\tangente[p]{M}\rightarrow\tangente[\Phi(p)]{N}\big)
		\text{ .}
	\end{align*}
	%
\end{propoTangenteSubvarRegularII}

\begin{proof}[Demostraci\'{o}n de \ref{thm:tangentesubvarregulari}]
	Sea $v\in\tangente[p]{S}$. Si $f\in C^{\infty}(M)$ es una funci\'{o}n
	suave que se anula en $S$, entonces
	\begin{align*}
		v\,f & \,\equiv\,v(f|_{S}) \,=\, 0
		\text{ .}
	\end{align*}
	%
	Rec\'{\i}procamente, dado un vector tangente $w\in\tangente[p]{M}$
	que verifica $w\,f=0$ para toda funci\'{o}n suave $f\in C^{\infty}(M)$
	tal que $f|_{S}=0$, entonces podemos demostrar que
	$w\in\tangente[p]{S}$. Para ver que esto es as\'{\i}, elegimos
	una carta $(U,\varphi)$ para $M$ en $p$ tal que
	$S\cap U=\{x^{k+1}=0,\,\dots,\,x^{n}=0\}$. (Aqu\'{i} podr\'{\i}amos
	tomar, en lugar de $S$, un entorno $V$ de $p$ en $S$ que sea
	subvariedad regular de $M$ y una carta $(U,\varphi)$ tal que
	$V\cap U$ sea el conjunto en donde las \'{u}ltimas coordenadas se
	anulan. Esto es v\'{a}lido incluso en el caso de una subvariedad
	meramente inmersa, pero m\'{a}s adelante se requerir\'{a} que $S$
	sea subvariedad regular). Por \ref{thm:tangentesubvarcoordenadas},
	sabemos que $w\in\tangente[p]{S}$, si y s\'{o}lo si, con
	respecto a las funciones coordenadas,
	$w(x^{k+1})=\,\cdots\,=w(x^{n})=0$. Vamos a ver que esta \'{u}ltima
	condici\'{o}n se cumple. Sea $\psi$ una funci\'{o}n chich\'{o}n
	con soporte en $U$ y que vale $1$ en un entorno (de $M$) de $p$
	(por ejemplo, una de las dos funciones de alguna partici\'{o}n
	suave de la unidad subordinada al cubrimiento
	$\{U,M\setmin\clos{U'}\}$, donde $U'$ es un entorno de $p$ en $M$
	cuya clausura est\'{a} contenida en $U$). Para cada $j>k$, la
	funci\'{o}n $\psi(x)x^{j}\equiv\psi\cdot\varphi^{j}$ es suave en
	$U$ y tiene soporte contenido en $U$. Si llamamos $f$ a la 
	extensi\'{o}n de esta funci\'{o}n por cero fuera de $U$, entonces
	$f\in C^{\infty}(M)$ y $f|_{S}=0$ (esto no es cierto, en general,
	si $S$ no es regular). As\'{\i},
	\begin{align*}
		0 & \,=\, w\,f \,=\,w^{i}\derivada{f}{x^{i}} \,=\, v^{j}
	\end{align*}
	%
	y concluimos que $w\in\tangente[p]{S}$.
\end{proof}

\begin{proof}[Demostraci\'{o}n de \ref{thm:tangentesubvarregularii}]
	Si $S\cap U=\Phi^{-1}(c)$ para alg\'{u}n punto $c\in N$, entonces
	$\Phi|_{S\cap U}$ toma constantemente el valor $c$. As\'{\i},
	\begin{align*}
		\diferencial[p]{\Phi}\cdot\diferencial[p]{\inc} & \,=\,0
	\end{align*}
	%
	en tanto transformaci\'{o}n lineal de $\tangente[p]{S}$ en
	$\tangente[\Phi(p)]{N}$. Es decir,
	\begin{align*}
		\img\,\diferencial[p]{\inc} & \,\subset\,
		\ker\,\diferencial[p]{\Phi}
		\text{ .}
	\end{align*}
	%
	Pero, como $\diferencial[p]{\Phi}$ es suryectivo,
	\begin{align*}
		\dim\big(\ker\,\diferencial[p]{\Phi}\big) & \,=\,
			\dim\,\tangente[p]{M}\,-\,\tangente[\Phi(p)]{N}
			\,=\,\dim\,M\,-\,\dim\,N \\
		& \,=\,\dim\,S \,=\, \dim\,\tangente[p]{S}
		\,=\,\dim\big(\img\,\diferencial[p]{\inc}\big)
		\text{ .}
	\end{align*}
	%
	En definitiva,
	$\ker\,\diferencial[p]{\Phi}=\img\,\diferencial[p]{\inc}$.
\end{proof}

\begin{obsTangenteValorRegular}\label{obs:tangentevalorregular}
Si la funci\'{o}n que define localmente a $S$ como conjunto de nivel regular
es de la forma $\Phi=(\lista*{\Phi}{k}):\,U\rightarrow\bb{R}^{k}$, entonces
$w\in\tangente[p]{M}$ pertenece al n\'{u}cleo $\ker\,\diferencial[p]{\Phi}$,
si y s\'{o}lo si $w\,\Phi^{j}=0$ para cada una de las funionces que
componen a $\Phi$: si $\diferencial[p]{\Phi}(w)=0$, entonces
\begin{align*}
	w\,\Phi^{j} & \,=\,w(y^{j}\circ\Phi) \,=\,
		\diferencial[p]{\Phi}(w)\,y^{j} \,=\, 0
\end{align*}
%
para todo $j\in [\![1,k]\!]$ y, si $w$ se anula en las funciones
$\Phi^{j}=y^{j}\circ\Phi$, entonces
\begin{align*}
	\diferencial[p]{\Phi}(w) & \,=\,
		\diferencial[p]{\Phi}(w)(y^{j})\gancho[\Phi(p)]{y^{j}}
		\,=\,w(y^{j}\circ\Phi)\gancho[\Phi(p)]{y^{j}}
		\,=\, 0
	\text{ .}
\end{align*}
%
\end{obsTangenteValorRegular}

\begin{obsTangenteConjuntoDeNivel}\label{obs:tangeteconjuntodenivel}
	Supongamos que $S\subset M$ est\'{a} dada como el conjunto de nivel de
	una transformaci\'{o}n $\Phi:\,M\rightarrow N$ de rango constante,
	no necesariamente una submersi\'{o}n. Aun as\'{\i}, sabemos, por el
	teorema del reango \ref{thm:delrango}, que
	$\codim\,S=\rango{\Phi}$. Pero $\codim(\ker\,\diferencial[p]{\Phi})$
	es igual al rango de $\diferencial[p]{\Phi}$. Entonces
	$\tangente[p]{S}$ y $\ker\,\diferencial[p]{\Phi}$ tienen la misma
	dimensi\'{o}n. Por otro lado, sigue siendo cierto, en este caso
	tambi\'{e}n, que $\tangente[p]{S}\subset\ker\,\diferencial[p]{\Phi}$.
	En definitiva, deben coincidir.
\end{obsTangenteConjuntoDeNivel}

Ya hemos mencionado que el borde de una variedad con borde es una
subvariedad regular. Sea $M$ una variedad con borde
$\borde[M]\not=\varnothing$. De acuerdo con las observaciones realizadas al
comienzo de esta parte, podemos, en esta situaci\'{o}n tambi\'{e}n,
identificar el tangente a $\borde[M]$ en un punto $p$ con un subespacio
de $\tangente[p]{M}$ v\'{\i}a el diferencial de la inclusi\'{o}n
$\inc[{\borde[M]}]:\,\borde[M]\rightarrow M$ o en t\'{e}rminos de curvas
suaves. Las identificaciones que hacen uso de coordenadas, por otro lado,
funcionaron porque ten\'{\i}amos a nuestra disposici\'{o}n las cartas
preferenciales, que, en el caso de una variedad con borde no est\'{a}n
definidas. En lugar de cartas preferenciales, contamos con las cartas de
borde. Usando estas cartas, podemos intentar repetir los argumentos
de las demostraciones de los resultados de esta parte.

Sea $M$ una variedad con borde $\borde[M]\not =\varnothing$ y sea
$p\in\borde[M]$. Sea $\inc=\inc[{\borde[M]}]$ la includi\'{o}n del borde
en la variedad. Si $v\in\tangente[p]{\borde[M]}$, entonces, como antes,
si llamamos $\tilde{v}=\diferencial[p]{\inc}(v)$, vale que
\begin{align*}
	\tilde{v}\,f & \,=\,v(f\circ\inc) \,=\,v(f|_{\borde[M]})
\end{align*}
%
para toda funci\'{o}n $f\in C^{\infty}(M)$. En coordenadas,
\begin{align*}
	\tilde{v} & \,=\,v^{i}\gancho[p]{x^{i}}
\end{align*}
%
donde $v^{i}=\tilde{v}(x^{i})=v(x^{i}|_{\borde[M]})$. En particular,
como $x^{n}|_{\borde[M]}=0$, debe valer que $v^{n}=0$. Entonces
\begin{align*}
	\tangente[p]{\borde[M]} & \,\subset\,
		\generado{\gancho[p]{x^{1}},\,\dots,\,\gancho[p]{x^{n-1}}}
		\,=\,\left\lbrace w=w^{i}\gancho[p]{x^{i}}\in\tangente[p]{M}
		\,:\,w^{n}=0\right\rbrace
		\text{ .}
\end{align*}
%
Ahora, como en la demostraci\'{o}n de \ref{thm:tangentesubvarcoordenadas},
hay varias maneras de concluir que estos subespacios son iguales. Podemos
usar la dimensi\'{o}n: $\dim(\tangente[p]{\borde[M]})=n-1$; podemos
usar la curva $\eta(t)=(w^{1}t,\,\dots,\,w^{n-1}t)$ en $\borde[M]$ con
origen en $p$, que es suave y, al componer con la inclusi\'{o}n,
$\gamma=\inc\circ\eta$ es suave en $M$, tiene origen en $p$ y velocidad
$\dot{\gamma}(0)=w$. Tambi\'{e}n podemos recurrir al hecho de que,
como $\varphi(\borde[M]\cap U)=\borde[{\hemi[n]}]\cap\varphi(U)$, localmente
contamos con una descripci\'{o}n de $\borde[M]$ como conjunto de nivel
regular de una funci\'{o}n suave: $x^{n}:\,U\rightarrow\bb{R}$ y
$\borde[M]\cap U=\{x^{n}=0\}$.

Esta descripci\'{o}n del tangente al borde de una variedad nos permite,
tambi\'{e}n, descomponer el tangente a la variedad misma. Sea $M$ una
variedad diferencial con borde $\borde[M]\not=\varnothing$ y sea $p$
un punto del borde. Sea $(U,\varphi)$ una carta de borde centrada en $p$.
En coordenadas, $w\in\tangente[p]{M}$ se escribe como
$w=w^{i}\gancho[p]{x^{i}}$ y $w\in\tangente[p]{\borde[M]}$, si y s\'{o}lo
si $w^{n}=0$. El resto de los vectores tangentes, aquellos con
$w^{n}\not =0$, se dividen en dos clases: aquellos con $w^{n}>0$ y
aquellos con $w^{n}<0$. Para ver que esta divisi\'{o}n no depende de la
carta de borde elegida, usamos la descripci\'{o}n de los vectores
tangentes como velocidades de curvas contenidas en la variedad. Supongamos
que $w\in\tangente[p]{M}$ es tal que $w^{n}<0$, entonces la curva
$\gamma(t)=(w^{1}t,\,\dots,\,w^{n-1}t,\,w^{n}t)$ definida en $(-\epsilon,0]$
es una curva suave en $M$, $\gamma(0)=p$ y $\dot{\gamma}(0)=w$.
Rec\'{\i}procamente, si $w\in\tangente[p]{M}$ es tal que existe una
curva suave $\gamma:\,(-\epsilon,0]\rightarrow M$ con origen en $p$ y
velocidad $w$, entonces en las coordenadas dadas, como en cualquier otra
carta,
\begin{align*}
	w^{n} & \,=\,\dot{\gamma}(0)(x^{n}) \,=\,
		\gancho[0]{t}(x^{n}\circ\gamma) \\
	& \,=\,\lim_{t\to0,\,t<0}\,\frac{\gamma^{n}(t)-\gamma^{n}(0)}{t}
		\,<\,0
	\text{ ,}
\end{align*}
%
pues $\gamma^{n}(0)=0$, $t<0$ y $\gamma^{n}(t)>0$. An\'{a}logamente,
$w^{n}>0$, si y s\'{o}lo si existe una curva suave
$\gamma:\,[0,\epsilon)\rightarrow M$ con origen en $p$ y velocidad $w$.

\begin{propoTangenteBordeCurvas}\label{thm:tangentebordecurvas}
	Sea $M$ una variedad diferencial y sea $p\in\borde[M]$. Sea
	$(U,\varphi)$ una carta de borde para $M$ centrada en $p$ y
	sea $w\in\tangente[p]{M}$ un vector tangente a $M$ en $p$.
	Entonces \emph{(i)} $w$ es la velocidad de una curva suave
	$\gamma:\,(-\epsilon,0]\rightarrow M$ con origen en $p$, si y
	s\'{o}lo si, en coordenadas, $w^{n}=w(x^{n})<0$; \emph{(ii)} $w$
	es la velocidad de una curva suave
	$\gamma:\,[0,\epsilon)\rightarrow M$ con origen en $p$, si y
	s\'{o}lo si $w^{n}>0$; \emph{(iii)} $w\in\tangente[p]{\borde[M]}$,
	si y s\'{o}lo si $w^{n}=0$, si y s\'{o}lo si $w$ es la velocidad
	de una curva suave con origen en $p$ e imagen contenida en
	$\borde[M]$.
\end{propoTangenteBordeCurvas}

Notemos que los tres conjuntos de la proposici\'{o}n anterior son disjuntos
y que, adem\'{a}s, si $w$ es tal que $w^{n}>0$ para alguna --y, por lo
tanto, para todas-- carta, entones $-w$ verifica la desigualdad opuesta.
Aquellos vectores tangentes con $w^{n}<0$ se dice que \emph{apuntan hacia %
afuera} y aquellos con $w^{n}>0$ que \emph{apuntan hacia adentro}. Si
$w$ apunta hacia adentro, entonces $-w$ apunta hacia afuera.

Hemos mencionado, m\'{a}s arriba que $\borde[M]$ se puede ver localmente
como el conjunto de ceros de la funci\'{o}n coordenada $x^{n}$ (dada una
carta de borde para la variedad). En general, decimos que una funci\'{o}n
$f:\,M\rightarrow [0,\infty)$ es una \emph{funci\'{o}n de definici\'{o}n %
del borde}, si $f^{-1}(0)=\borde[M]$ y $\diferencial[p]{f}\not=0$ para
todo $p\in\borde[M]$. Estas funciones son an\'{a}logas a las funciones de
definici\'{o}n de un conjunto de nivel.

\begin{propoBordeDeNivel}\label{thm:bordedenivel}
	Sea $M$ una variedad diferencial con $\borde[M]\not =\varnothing$.
	Existe una funci\'{o}n de definici\'{o}n del borde de $M$.
\end{propoBordeDeNivel}

\begin{proof}
	Sea $\{(U_{\alpha},\varphi_{\alpha})\}_{\alpha}$ un atlas compatible
	para $M$. Sea $\{\psi_{\alpha}\}_{\alpha}$ una partici\'{o}n
	de la unidad subordinada al cubrimiento $\{U_{\alpha}\}_{\alpha}$.
	Para definir la funci\'{o}n de definici\'{o}n de borde, definimos
	primero funciones en cada abierto $U_{\alpha}$ del cubrimiento.
	Si $U_{\alpha}$ es el dominio de una carta del interior,
	es decir, si $U_{\alpha}\cap\borde[M]=\varnothing$, entonces
	definimos $f_{\alpha}:\,U_{\alpha}\rightarrow\bb{R}$ como la
	funci\'{o}n suave que toma constantemente el valor $1$. Si, en cambio,
	$U_{\alpha}\cap\borde[M]\not=\varnothing$, entonces
	$(U_{\alpha},\varphi_{\alpha})$ es una carta de borde. En este caso,
	definimos $f_{\alpha}=x^{n}$. En particular,
	\begin{align*}
		f_{\alpha}^{-1}(0) & \,=\,\{x^{n}=0\} \,=\,
		\borde[M]\cap U_{\alpha}
		\text{ .}
	\end{align*}
	%
	La funci\'{o}n $\psi_{\alpha}$, por otro lado, tiene soporte en
	un cerrado contenido en $U_{\alpha}$, entonces, si extendemos
	$f_{\alpha}\psi_{\alpha}$ por cero fuera de $U_{\alpha}$, queda
	definida una funci\'{o}n suave en $M$ que se anula fuera
	de $\soporte{\psi_{\alpha}}$. Sea
	$f=\sum_{\alpha}\,\psi_{\alpha}f_{\alpha}$. Entonces $f$ est\'{a}
	bien definida y es suave en $M$, porque los sumandos lo son y porque
	los soportes de los sumandos forman una familia localmente
	finita.

	Si $p\in\borde[M]$, entonces $\psi_{\alpha}(p)\not=0$ s\'{o}lo si
	$U_{\alpha}$ interseca el borde, y, en ese caso, $f_{\alpha}(p)=0$.
	Entonces $f(p)=0$, si $p$ es un punto del borde. Rec\'{\i}procamente,
	si $p\in\interior{M}$, entonces $f_{\alpha}(p)>0$ para todo $\alpha$.
	Como $\sum_{\alpha}\,\psi_{\alpha}=1$, existe $\alpha$ tal que
	$\psi_{\alpha}(p)\not =0$. Entonces $f(p)>0$, si $p$ es un punto
	del interior.

	Resta verificar que el diferencial $\diferencial[p]{f}$ no es la
	transformaci\'{o}n lineal (funcional lineal) nula para ning\'{u}n
	punto $p\in\borde[M]$. Sea $w\in\tangente[p]{M}$ un vector tangente
	que apunta hacia afuera. Como $p$ es un punto del borde de $M$,
	si $\alpha$ es tal que $p\in U_{\alpha}$ y
	$\varphi_{\alpha}=(\lista*{x}{n})$, entonces, por definici\'{o}n,
	$f_{\alpha}(p)=0$ y
	\begin{align*}
		\diferencial[p]{f_{\alpha}}(w) & \,=\,
			\diferencial[p]{x^{n}}(w) \,=\,w(x^{n})\,<\,0
		\text{ .}
	\end{align*}
	%
	En particular, como $p$ pertenece a alg\'{u}n $U_{\alpha}$,
	vale que $\diferencial[p]{f_{\alpha}}(w)\leq 0$ para todo
	$\alpha$ y que $\diferencial[p]{f_{\alpha}}(W)<0$ estrictamente
	para, al menos, un $\alpha$. En definitiva,
	\begin{align*}
		\diferencial[p]{f}(w) & \,=\,\sum_{\alpha}\,\big(
			f_{\alpha}(p)\diferencial[p]{\psi_{\alpha}}(w) +
			\psi_{\alpha}(p)\diferencial[p]{f_{\alpha}}(w)
			\big)
			\,<\,0
		\text{ ,}
	\end{align*}
	%
	de lo que se deduce que $\diferencial[p]{f}$ no es la
	transformaci\'{o}n cero.
\end{proof}

\begin{obsBordeDeNivel}\label{obs:bordedenivel}
	De la \'{u}ltima parte de la demostraci\'{o}n anterior, se ve
	que, si, en lugar de tomar un vector que apunta hacia afuera,
	hubi\'{e}semos elegido un vector $w$ que apunta hacia adentro,
	entonces $\diferencial[p]{f}(w)>0$ estrictamente, y, si
	$w\in\tangente[p]{\borde[M]}$, entonces $\diferencial[p]{f}(w)=0$.
	En general, si $f$ es cualquier funci\'{o}n de definici\'{o}n de
	borde para $M$, entonces estas afirmaciones siguen siendo ciertas.
	Por ejemplo, si $w\in\tangente[p]{M}$ apunta hacia afuera, entonces,
	dada una carta de borde $(U,\varphi)$ centrada en $p$,
	\begin{align*}
		\diferencial[p]{f}(w) & \,=\,w^{i}\derivada{f}{x^{i}}
			\,=\, w^{n}\derivada{f}{x^{n}}
		\text{ ,}
	\end{align*}
	%
	pues $f$ es constantemente $0$ en $\{x^{n}=0\}=\borde[M]\cap U$.
	Pero $\derivada{f}{x^{n}}>0$ y $w^{n}<0$. El mismo argumento,
	cambiando $w^{n}<0$ por $w^{n}>0$ o por $w^{n}=0$ sigue valiendo,
	si, en lugar de tomar un vector que apunta hacia afuera, elegimos
	$w$ apuntando hacia adentro o tangente al borde de la variedad.
\end{obsBordeDeNivel}

\begin{obsDiferencialComoFuncional}\label{obs:diferencialcomofuncional}
	Tanto en la demostraci\'{o}n de \ref{thm:bordedenivel}, como en la
	observaci\'{o}n \ref{obs:bordedenivel} abusamos de la siguiente
	identificaci\'{o}n entre elementos de $\bb{R}$ y elementos del
	tangente $\bb{R}$. Esta identificaci\'{o}n consiste en identificar,
	paralelamente, el diferencial de una funci\'{o}n
	$f:\,M\rightarrow\bb{R}$ con un elemento del dual de
	$\tangente[p]{M}$. Concretamente, si $f:\,M\rightarrow\bb{R}$ es una
	funci\'{o}n suave y $w\in\tangente[p]{M}$ es un vector tangente,
	entonces $w\,f\in\bb{R}$ y
	$\diferencial[p]{f}(w)\in\tangente[f(p)]{\bb{R}}$. La relaci\'{o}n
	entre ambos est\'{a} dada por evaluar el vector tangente a $\bb{R}$
	en la funci\'{o}n identidad $\id[\bb{R}]:\,\bb{R}\rightarrow\bb{R}$.
	Es decir, identificamos el vector $\diferencial[p]{f}(w)$ con el
	n\'{u}mero real
	\begin{align*}
		\diferencial[p]{f}(w)(\id[\bb{R}]) & \,=\,
			w(\id[\bb{R}]\circ f) \,=\, w\,f
		\text{ .}
	\end{align*}
	%
	Con esta identificaci\'{o}n, tiene sentido decir que
	$\diferencial[p]{f}(w)$ es mayor, menor o igual a cero.
\end{obsDiferencialComoFuncional}
