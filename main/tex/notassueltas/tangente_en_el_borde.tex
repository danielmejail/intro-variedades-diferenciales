\theoremstyle{plain}
\newtheorem{propoTangenteEnElBorde}{Proposici\'{o}n}[section]

\theoremstyle{remark}

%-------------

\begin{propoTangenteEnElBorde}\label{propo:tangenteenelborde}
	La inclusi\'{o}n $\inc:\,\hemi[n]\rightarrow\bb R^n$ induce un iso
	\begin{math}
		\diferencial[x]{\inc}:\,\tangente[x]{\hemi[n]}\rightarrow
			\tangente[x]{\bb R^n}
	\end{math} para todo $x\in\hemi[n]$
\end{propoTangenteEnElBorde}

\begin{proof}
	Sea $x\in\hemi[n]$ y sea $v\in\tangente[x]{\hemi[n]}$ una
	derivaci\'{o}n en $\suaves{x,\hemi[n]}$. Si $f\in\suaves{\hemi[n]}$,
	existe un abierto $U\subset\bb R^n$ que contiene a $x$ y una
	funci\'{o}n suave $\tilde f:\,U\rightarrow\bb R$ tales que
	\begin{align*}
		\tilde f|_{U\cap\hemi[n]} & \,=\,f|_{U\cap\hemi[n]}
		\text{ .}
	\end{align*}
	%
	En particular, $\tilde f|_{U\cap\hemi[n]}$ y $f$ tiene el mismo germen
	en $x$. Entonces, llamando $j$ a la inclusi\'{o}n
	$U\cap\hemi[n]\hookrightarrow\hemi[n]$, se verifica que
	\begin{align*}
		v\,f & \,=\,v(\tilde f\circ j)
		\text{ .}
	\end{align*}
	%
	Por definici\'{o}n de la diferencial,
	\begin{align*}
		v\,f & \,=\,\big(
			\diferencial[x]{(\inc\circ j)}\,v\big)\,\tilde f \,=\,
			\big((\diferencial[x]{\inc})\cdot
				(\diferencial[x]{j})\,v\big)\,\tilde f
		\text{ .}
	\end{align*}
	%
	Como $j$ es la inclusi\'{o}n de un abierto,
	$\diferencial[x]{j}=\id[{\tangente[x]{\hemi[n]}}]$, identificando
	naturalmente $\tangente[x]{(U\cap\hemi[n])}$ con
	$\tangente[x]{\hemi[n]}$, v\'{\i}a
	$\suaves{x,U\cap\hemi[n]}=\suaves{x,\hemi[n]}$. En definitiva, si
	$(\diferencial[x]{\inc})\,v=0$, debe ser $v\,f=0$ para toda
	$f\in\suaves{x,\hemi[n]}$. Esto demuestra que $\diferencial[x]{\inc}$
	es monomorfismo.

	Para ver que es epimorfismo, sea $w\in\tangente[x]{\bb R^n}$ y sea
	\begin{align*}
		w & \,=\,w^i\,\gancho[x]{x^i}
	\end{align*}
	%
	su escritura en la base can\'{o}nica. Supongamos, m\'{a}s aun, que
	$x\in\borde{\hemi[n]}$. Sean $g_1:\,U_1\rightarrow\bb R$ y
	$g_2:\,U_2\rightarrow\bb R$ dos funciones suaves definidas en
	abiertos $U_1,U_2\subset\bb R^n$ que contienen a $x$. Si
	$g_1|_{U_1\cap\hemi[n]}=g_2|_{U_2\cap\hemi[n]}$ --si coinciden en
	el semiespacio superior-- entonces
	\begin{align*}
		\derivada{g_1}{x^i}(x) & \,=\,\lim_{t\to0}\,
			\frac{g_1(x+e_it)-g_1(x)}{t} \,=\,
			\lim_{t\to0^+}\,\frac{g_1(x+e_it)-g_1(x)}{t} \\
		& \,=\,\lim_{t\to0^+}\,\frac{g_2(x+e_it)-g_2(x)}{t} \,=\,
			\lim_{t\to0}\,\frac{g_2(x+e_it)-g_2(x)}{t} \\
		& \,=\,\derivada{g_2}{x^i}(x)
		\text{ .}
	\end{align*}
	%
	Es decir, toda derivaci\'{o}n $w\in\tangente[x]{\bb R^n}$ queda
	determinada por los valores en $\hemi[n]$:
	\begin{align*}
		w\,g_1 & \,=\,w^i\,\derivada{g_1}{x^i}(x)\,=\,
			w^i\,\derivada{g_2}{x^i}(x) \,=\,w\,g_2
		\text{ .}
	\end{align*}
	%
	Si llamamos $v$ a la derivaci\'{o}n $v\in\tangente[x]{\hemi[n]}$ dada
	por $v\,f=w\,\tilde f$, donde $\tilde f$ es alguna extensi\'{o}n de $f$
	a un entrno de $x$ en $\bb R^n$, por lo anterior, $v$ queda bien
	definida y es derivaci\'{o}n, porque $w$ lo es. Si ahora
	$g\in\suaves{x,\bb R^n}$, entonces
	\begin{align*}
		\big(\diferencial[x]{\inc}\,v\big)\,g & \,=\,
			v\big(g|_{\hemi[n]}\big) \,=\,w\,g
	\end{align*}
	%
	y $\diferencial[x]{\inc}\,v=w$. En conclusi\'{o}n,
	$\diferencial[x]{\inc}$ es epimorfismo, tambi\'{e}n.
\end{proof}
