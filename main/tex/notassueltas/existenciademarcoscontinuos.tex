
Sea $M$ una variedad diferencial de dimensi\'{o}n $n$ y sea
$\gamma:\,I\rightarrow M$ una curva diferenciable definida en el intervalo
$I=[0,1]$. Un \emph{campo continuo sobre $\gamma$} es una aplicaci\'{o}n
continua $X:\,I\rightarrow\tangente{M}$ tal que $X=\pi\circ\gamma$, es decir,
$X(t)\in\tangente[\gamma(t)]{M}$ es un vector tangente a $M$ en el punto
$\gamma(t)$ para cada instante $t$. Un \emph{marco continuo sobre $\gamma$}
es un conjunto ordenado $\{\lista{E}{n}\}$ de campos continuos sobre $\gamma$
tal que, para cada $t\in I$, el conjunto ordenado
$\{E_{1}(t),\,\dots,\,E_{n}(t)\}$ es una base del espacio tangente
$\tangente[\gamma(t)]{M}$. En esta secci\'{o}n se demostrar\'{a} que siempre
existen marcos continuos sobre curvas diferenciales.

Sea entonces $M$ una variedad diferencial y sea $\gamma:\,I\rightarrow M$
una curva diferenciable. Dado $p\in\gamma(I)$, se puede elegir una carta
$(U,\varphi)$ centrada en $p$, compatible con la estructura diferencial
de $M$. Sea $\cal{U}=\left\lbrace (U_{t},\varphi_{t})\,:\,%
t\in I\right\rbrace$ la colecci\'{o}n dada por una posible elecci\'{o}n
de cartas compatibles tales que $(U_{t},\varphi_{t})$ est\'{e} centrada en
$\gamma(t)$. En particular, $\cal{U}$ es un cubrimiento de la imagen
$\gamma(I)$ por abiertos de $M$. Dado que $\gamma:\,I\rightarrow M$ es
continua, las preim\'{a}genes $\gamma^{-1}(U_{t})\subset I$ son abiertas
y $t\in\gamma^{-1}(U_{t})$ para cada instante $t\in I$. En particlar,
$\left\lbrace\gamma^{-1}(U_{t})\,:\,t\in I\right\rbrace$ es un cubrimiento
por abiertos del inervalo $I$. Dado que $I$ es un espacio m\'{e}trico
compacto, por el \emph{lema del n\'{u}mero de Lebesgue}, existen finitos
instantes $a_{0}=0<a_{1}<\,\cdots\,<a_{r}=1$ tales que, para
$i\in[\![0,r-1]\!]$, el subintervalo $[a_{i},a_{i+1}]$ est\'{e} contenido
en alg\'{u}n abierto $V_{i}=\gamma^{-1}(U_{t_{i}})$ del cubrimiento
del intervalo.

Sea $\gamma_{i}=\gamma|_{[a_{i},a_{i+1}]}$ la restricci\'{o}n de la curva
$\gamma$ a alguno de los subintervalos $[a_{i},a_{i+1}]$. Cada una de las
curvas $\gamma_{i}:\,[a_{i},a_{i+1}]\rightarrow M$ es diferenciable y
admite un marco continuo: sea $E^{i}_{k}:\,%
[a_{i},a_{i+1}]\rightarrow\tangente{M}$ la aplicaci\'{o}n
\begin{align*}
	E^{i}_{k}(t) & \,=\,\gancho[\gamma(t)]{x_{i}^{k}}
\end{align*}
%
donde $\gancho[\gamma(t)]{x_{i}^{k}}$ denota la derivaci\'{o}n en
$\tangente[\gamma(t)]{M}$ determinada por las funciones coordenadas
$\varphi_{t_{i}}=(x_{i}^{l})$ correspondiente a la funci\'{o}n
$x_{i}^{k}:\,U_{t_{i}}\rightarrow\bb{R}$. La cuesti\'{o}n es c\'{o}mo
pegar estos marcos de manera obtener un marco continuo sobre $\gamma$.
Para cada \'{\i}ndice $i\in[\![0,r-1]\!]$, los campos
$\Big\{\gancho{x_{i+1}^{k}}\Big\}_{k}$ se pueden escribir en t\'{e}rminos
de los campos $\Big\{\gancho{x_{i}^{k}}\Big\}_{k}$: para cada punto
$p\in U_{t_{i}}\cap U_{t_{i+1}}$, existen coeficientes
$A^{i,k}_{l}(p)\in\bb{R}$ tales que
\begin{align*}
	\gancho[p]{x_{i+1}^{l}} & \,=\, A^{i,k}_{l}(p)\,
		\gancho[p]{x_{i}^{k}}
\end{align*}
%
(sin sumar sobre $i$). De esta manera, dado que
$\Big\{\gancho[p]{x_{i+1}^{k}}\Big\}_{k}$ y que
$\Big\{\gancho[p]{x_{i}^{k}}\Big\}_{k}$ forman bases del espacio tangente
en $p$, se obtiene un elemento, una matriz, $A^{i}(p)\in\GL{n,\bb{R}}$,
cuyo coeficiente $(k,l)$ es $A^{i,k}_{l}(p)$. Dado que las cartas
$(U_{t_{i}},\varphi_{t_{i}})$ y $(U_{t_{i+1}},\varphi_{t_{i+1}})$ son
compatibles, las funciones
\begin{align*}
	A^{i,k}_{l} & \,:\,U_{t_{i}}\cap U_{t_{i+1}} \,\rightarrow\,\bb{R}
\end{align*}
%
son suaves y la aplicaci\'{o}n $A^{i}:\,%
U_{t_{i}}\cap U_{t_{i+1}}\rightarrow\GL{n,\bb{R}}$ es suave, tambi\'{e}n.
En t\'{e}rminos de los campos, vale que
\begin{align*}
	\gancho{x_{i+1}^{l}} & \,=\,A^{i,k}_{l}\,\gancho{x_{i}^{k}}
\end{align*}
%
en $U_{t_{i}}\cap U_{t_{i+1}}$. En particular, componiendo con la curva
$\gamma$, siempre que $\gamma(t)$ pertenezca a la intersecci\'{o}n,
se cumple que
\begin{align*}
	E^{i+1}_{l}(t) & \,=\,A^{i,k}_{l}(\gamma(t))\,E^{i}_{k}(t)
	\text{ .}
\end{align*}
%
Esta igualdad vale, por ejemplo, cuando $t=a_{i+1}$.

Para cada $i\in[\![0,r-1]\!]$, sea $p_{i}=\gamma(a_{i})$ y sea
$p_{r}=\gamma(a_{r})=\gamma(1)$. Cada punto de la forma $p_{i+1}$ de la
variedad pertenece a la intersecci\'{o}n $U_{t_{i}}\cap U_{t_{i+1}}$ de
abiertos coordenados. Sea $\epsilon_{i}>0$ tal que
$a_{i}<a_{i+1}-\epsilon_{i}$ y que
$\gamma([a_{i+1}-\epsilon_{i},a_{i+1}])\subset U_{t_{i}}\cap U_{t_{i+1}}$.
Sea $B(a_{i+1}-\epsilon)=I_{n}\in\GL{n,\bb{R}}$ la matriz identidad y sea
$B(a_{i+1})=A^{i}(p_{i+1})=A^{i}(\gamma(a_{i+1}))\in\GL{n,\bb{R}}$ la matriz
de coeficientes de los campos coordenados en $U_{t_{i+1}}$ en t\'{e}rminos
de los campos coordenados en $U_{t_{i}}$. Si existiese una curva
(suave) $B:\,[a_{i+1}-\epsilon_{i},a_{i+1}]\rightarrow\GL{n,\bb{R}}$
tal que $B(t)=I_{n}$ cerca de $t=a_{i+1}-\epsilon_{i}$ y tal que
$B(t)=A^{i}(p_{i+1})$ cerca de $t=a_{i+1}$, entonces se podr\'{\i}an definir
campos $F_{1},\,\dots,\,F_{n}:\,[a_{i+1}-\epsilon_{i},a_{i+1}]\rightarrow%
\tangente{M}$ dados por
\begin{align*}
	F_{l}(t) & \,=\, B(t)^{k}_{l}\,E^{i}_{k}(t) \,=\,
		B(t)^{k}_{l}\,\gancho[\gamma(t)]{x_{i}^{k}}
	\text{ .}
\end{align*}
%
Dado que $B(t)\in\GL{n,\bb{R}}$ para todo $t$ y que
$\{E^{i}_{1}(t),\,\dots,\,E^{i}_{n}(t)\}$ es base de $\tangente[\gamma(t)]{M}$,
el conjunto ordenado $\{F_{1}(t),\,\dots,\,F_{n}(t)\}$ tambi\'{e}n es base
del espacio tangente para cada instante $t$ en donde $B$ est\'{e} definida.
Adem\'{a}s, como cada una de las funciones $t\mapsto B(t)^{k}_{l}$ son
continuas (suaves), los campos $\lista{F}{n}$ sobre la restricci\'{o}n
$\gamma|_{[a_{i+1}-\epsilon_{i},a_{i+1}]}$ son continuos (suaves). En
definitiva, asumiendo que la curva $B$ existe, se obtiene un marco continuo
$\{\lista{F}{n}\}$ en $[a_{i+1}-\epsilon_{i},a_{i+1}]$ tal que
\begin{align*}
	F_{l}(a_{i+1}-\epsilon_{i}) & \,=\,
		(I_{n})^{k}_{l}\,E^{i}_{k}(a_{i+1}-\epsilon_{i}) \,=\,
		E^{i}_{l}(a_{i+1}-\epsilon_{i})
	\quad\text{y} \\
	F_{l}(a_{i+1}) & \,=\,
		A^{i,k}_{l}(\gamma(a_{i+1}))\,E^{i}_{k}(a_{i+1}) \,=\,
		E^{i+1}_{l}(a_{i+1})
\end{align*}
%
para cada $l\in[\![1,n]\!]$ (sin sumar sobre $i$). De esta manera,
concatenando con el marco $\{E^{i}_{1},\,\dots,\,E^{i}_{n}\}$ en
$[a_{i},a_{i+1}-\epsilon_{i}]$, se obtiene un marco continuo (?`suave?)
sobre $\gamma_{i}$. Pero, si $\{F^{i}_{1},\,\dots,\,F^{i}_{n}\}$ denota
el marco continuo definido en $[a_{i+1}-\epsilon_{i},a_{i+1}]$, entonces,
para cada $i$, este marco se pega bien con el marco
$\{E^{i+1}_{1},\,\dots,\,E^{i+1}_{n}\}$ en $[a_{i+1},a_{i+2}-\epsilon_{i+1}]$,
dando como resultado un marco continuo en
$[a_{i+1}-\epsilon_{i},a_{i+2}-\epsilon_{i+1}]$. Inductivamente, asumiendo
que las curvas $B:\,[a_{i+1}-\epsilon_{i},a_{i+1}]\rightarrow\GL{n,\bb{R}}$
existen para cada $i\in[\![0,r-1]\!]$, se deduce que existe un marco
continuo sobre $\gamma$.

Para concluir la demostraci\'{o}n de que existe un marco continuo sobrela curva
(diferenciable) $\gamma$, resta demostrar que existen las curvas continuas
(suaves) $B:\,[a_{i+1}-\epsilon_{i},a_{i+1}]\rightarrow\GL{n,\bb{R}}$ tales
que $B(a_{i+1}-\epsilon_{i})=I_{n}$ y $B(a_{i+1})=A^{i}(p_{i+1})$. Pero
una curva con estas propiedades existe, si y s\'{o}lo si $A^{i}(p_{i+1})$
pertenece a la componente conexa de la identidad $I_{n}$ en $\GL{n,\bb{R}}$,
es decir, si y s\'{o}lo si $\det\big(A^{i}(p_{i+1})\big)>0$. Pero esta
condici\'{o}n no se tiene por qu\'{e} cumplir, \textit{a priori}. Aun
as\'{\i}, esto se puede corregir progresivamente, a medida que se va
avanzando sobre la curva. Si $A^{0}(p_{1})$ tiene determinante negativo,
entonces, cambiando la coordenada
$\varphi_{t_{1}}=(x_{1}^{1},\,\dots,\,x_{1}^{n})$ por
$\tilde{\varphi}_{t_{1}}=(-x_{1}^{1},\,x_{1}^{2},\,\dots,\,x_{1}^{n})$,
los campos coordenados correspondientes en $U_{t_{1}}$ (el dominio no cambia)
est\'{a}n dados, en relaci\'{o}n con los anteriores, por
\begin{align*}
	\gancho{\tilde{x}_{1}^{1}} & \,=\,-\gancho{x_{1}^{1}}
	\quad\text{y} \\
	\gancho{\tilde{x}_{1}^{k}} & \,=\,\gancho{x_{1}^{k}}
	\text{ ,}
\end{align*}
%
si $k\geq 2$. Definiendo $E^{1}_{k}(t)=\gancho[\gamma(t)]{\tilde{x}^{k}_{1}}$
y $\tilde{A}^{0}:\,U_{t_{0}}\cap U_{t_{1}}\rightarrow\GL{n,\bb{R}}$ como
la aplicaci\'{o}n que a un punto $p$ le asigna la matriz cuyos coeficientes
est\'{a}n determinados por
\begin{align*}
	\gancho[p]{\tilde{x}^{l}_{1}} & \,=\,
		\tilde{A}^{0,k}_{l}(p)\,\gancho[p]{x_{0}^{k}}
	\text{ ,}
\end{align*}
%
se deduce que $\det\big(\tilde{A}^{0}(p_{1})\big)>0$. Si, ahora,
$\epsilon_{0}>0$ es tal que $a_{0}=0<a_{1}-\epsilon_{0}$ y que
$\gamma([a_{1}-\epsilon_{0},a_{1}])$, se puede elegir una
curva (posiblemente componiendo con alguno reparametrizaci\'{o}n)
$B:\,[a_{1}-\epsilon_{0},a_{1}]\rightarrow\GL{n,\bb{R}}$ tal que
$B(a_{1}-\epsilon_{0})=I_{n}$ y que $B(a_{1})=\tilde{A}^{0}(\gamma(a_{1}))$.
La existencia de esta curva permite concatenar el marco continuo
$\{E^{0}_{1},\,\dots,\,E^{0}_{n}\}$ en $[a_{0},a_{1}-\epsilon_{0}]$ con
un marco continuo $\{F^{0}_{1},\,\dots,\,F^{0}_{n}\}$ definido en
$[a_{1}-\epsilon_{0},a_{1}]$ tal que $F^{0}_{l}(a_{1}-\epsilon_{0})=E^{0}_{l}(a_{1}-\epsilon_{0})$ y que $F^{0}_{l}(a_{1})=E^{1}_{l}(a_{1})$. Concatenando
el marco $\{E^{0}_{l}\}_{l}$ en $[a_{0},a_{1}-\epsilon_{0}]$ con
$\{F^{0}_{l}\}_{l}$ definido en $[a_{1}-\epsilon_{0},a_{1}]$, seguido luego
del marco $\{E^{1}_{l}\}_{l}$ en $[a_{1},a_{2}]$, se obtiene un marco
continuo en $[a_{0},a_{2}]$. Inductivamente, si se tiene definido un
marco continuo sobre $\gamma$ en el subintervalo $[a_{0},a_{i}]$, para
extenderlo hasta $a_{i+1}$, se repite el procedimiento anterior, obteniendo
un marco continuo hasta $a_{i+1}-\epsilon_{i}$ usando el
marco dado por los campos coordenados $\gancho{x_{i+1}^{l}}$ (o bien por
los campos $\gancho{\tilde{x}_{i+1}^{l}}$ correspondientes a modificar
la funci\'{o}n coordenada $x_{i+1}^{1}$ por $\tilde{x}_{i+1}^{1}=-x_{i+1}^{1}$,
en caso de que $\det\big(A^{i}(p_{i+1})\big)<0$), seguido de un marco
de transici\'{o}n $\{F^{i+1}_{l}\}_{l}$ hasta $a_{i+2}$, si $i+1<r-1$, o
sin modificaciones, si $i+1=r-1$.

