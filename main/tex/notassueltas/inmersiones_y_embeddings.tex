\theoremstyle{plain}
\newtheorem{propoEntornoTajadaEmbedding}{Proposici\'{o}n}[section]
\newtheorem{propoInmersionEsLocalmenteEmbedding}[propoEntornoTajadaEmbedding]%
	{Proposici\'{o}n}
\newtheorem{propoEmbeddingEntornoTajada}[propoEntornoTajadaEmbedding]%
	{Proposici\'{o}n}

\theoremstyle{remark}

%-------------

\begin{propoInmersionEsLocalmenteEmbedding}%
	\label{propo:inmersioneslocalmenteembedding}
	Toda inmersi\'{o}n es localmente embedding.
\end{propoInmersionEsLocalmenteEmbedding}

\begin{proof}
	Si $f:\,N\rightarrow M$ es una inmersi\'{o}n y $p\in N$, existen cartas
	en $p$ y en $q=f(p)$ tales que
	$\encoordenadas{f}=(\lista*{x}{n},0,\,\dots,\,0)$, donde $n=\dim\,N$.
	En particular, $f$ es inyectiva en un entorno $U_1$ de $p$ en $N$. Sea
	$U\subset U_1$ otro entorno del punto tal que $\clos U\subset U_1$ y
	$\clos U$ sea compacta. Entonces
	$f|_{\clos U}:\,\clos U\rightarrow f(\clos U)$ es biyectiva (y
	continua) entre compactos (Hausdorff) y, por lo tanto, subespacio.
	Restringiendo $f$ al entorno $U$, $f|_U:\,U\rightarrow M$ es embedding.
\end{proof}

\begin{proof}[Otra demostraci\'{o}n]
	Como $f$ tiene rango constante, si $p\in N$, existen cartas
	$(U,\varphi)$ y $(V,\psi)$ para $N$ en $p$ y para $M$ en $f(p)$,
	respectivamente, tales que $f(U)\subset V$ y
	\begin{align*}
		\encoordenadas{f}(\lista*{x}{n}) & \,=\,
			(\lista*{x}{k},0,\,\dots,\,0)
		\text{ .}
	\end{align*}
	%
	Sea $\epsilon'>0$ tal que $\cubo[m]{\epsilon'}{0}\subset\psi(V)$
	($\psi(f(p))=0$) y sea
	$V_0=\psi^{-1}(\cubo[m]{\epsilon'}{0})\subset V$. Como $V_0$ es
	abierto, $f|_U^{-1}(V_0)\subset U$ es abierto, por continuidad. Sea
	$W=U\cap f^{-1}(V_0)=f|_U^{-1}(V_0)$. Notamos que $p\in W$, por
	definici\'{o}n. Sea $\epsilon>0$ tal que
	$\cubo[n]{\epsilon}{0}\subset\varphi(W)$ ($\varphi(p)=0$) y sea
	$U_0=\varphi^{-1}(\cubo[n]{\epsilon}{0})\subset W$. Entonces
	\begin{align*}
		\psi\circ f\circ\varphi^{-1}(\cubo[n]{\epsilon}{0}) & \,=\,
			\psi\circ f(U_0) \,\subset\,\psi\circ f(W)\,\subset\,
			\psi(V_0)\,=\,\cubo[m]{\epsilon'}{0}
		\text{ .}
	\end{align*}
	%
	Con respecto a la expresi\'{o}n en coordenadas $\encoordenadas{f}$, se
	deduce que $\epsilon\leq\epsilon'$ y que
	\begin{align*}
		\encoordenadas{f}(\cubo[n]{\epsilon}{0}) & \,=\,
			\big\{(\lista*{x}{m})\,:\,
				|x^1|,\,\dots,\,|x^k|<\epsilon,\,
				x^{k+1}=\cdots=x^m=0
				\big\}
		\text{ .}
	\end{align*}
	%
	Expresado de otra manera,
	\begin{align*}
		f(U_0) & \,=\,V_1\cap \big\{x^{k+1}=\cdots=x^m=0\big\}
		\text{ ,}
	\end{align*}
	%
	donde $V_1=\psi^{-1}(\cubo[m]{\epsilon}{0})$. Vale observar que no
	es necesariamente cierto que $U_0$ sea exactamente igual a la preimagen
	de esta intersecci\'{o}n por la transformaci\'{o}n $f$.

	En definitiva, si $f$ tiene rango constante $k$ (cerca de $p$), existen
	cartas $(U_0,\varphi_0)$ y $(V_0,\psi_0)$ centradas en $p$ y en $f(p)$,
	respectivamente, tales que
	\begin{align*}
		\varphi_0(U_0) & \,=\,\cubo[n]{\epsilon}{0} \text{ ,} \\
		\psi_0(V_0) & \,=\,\cubo[m]{\epsilon}{0}
			\quad\text{(mismo radio),} \\
		f(U_0) & \,=\,V_0\cap\big\{\psi^{k+1}=\cdots=\psi^m=0\big\}
		\text{ .}
	\end{align*}
	%
\end{proof}

\begin{propoEmbeddingEntornoTajada}\label{propo:embeddingentornotajada}
	Dado un embedding $f:\,N\rightarrow M$ y un punto $p\in N$, existe un
	entorno $V_1\subset M$ de $q=f(p)$ tal que $f(N)\cap V_1$ sea una $n$-%
	tajada de $V_1$.
\end{propoEmbeddingEntornoTajada}

\begin{proof}
	Sean $(U_0,\varphi_0)$ y $(V_0,\psi_0)$ como en la demostraci\'{o}n
	anterior. Dado que $f$ es subespacio, existe un abierto $W\subset M$
	tal que $f(U_0)=f(N)\cap W$. Tomamos $V_1:=V_0\cap W$ y
	$\psi_1=\psi_0|_{V_1}$. Entonces $(V_1,\psi_1)$ es una carta para $M$
	centrada en $q$ y vale que $f(U_0)\subset V_1$. Adem\'{a}s,
	\begin{align*}
		\psi_1\circ f\circ\varphi_0^{-1}(\lista*{x}{n}) & \,=\,
			(\lista*{x}{n},\,0,\,\dots,\,0)
	\end{align*}
	%
	($k=n$). Pero, entonces,
	\begin{align*}
		V_1\,\cap\,\big\{\psi^{n+1}=\cdots=\psi^m=0\big\} & \,=\,
			f(U_0) \,=\,f(N)\,\cap\,V_1
		\text{ ,}
	\end{align*}
	%
	lo que quiere decir que $f(N)\cap V_1$ es una $n$-tajada de $V_1$.
\end{proof}

\begin{propoEntornoTajadaEmbedding}\label{propo:entornotajadaembedding}
	Si $f:\,N\rightarrow M$ verifica que, para todo punto $p\in N$, existe
	una carta $(V,\psi)$ para $M$ en $f(p)$ tal que $f(N)\cap V$ sea una
	$n$-tajada de $V$, entonces $f(N)$ tiene estructura de subvariedad
	regular.
\end{propoEntornoTajadaEmbedding}

\begin{proof}
	A $S=f(N)\subset M$ le damos la topolog\'{\i}a subespacio. Con respecto
	a esta topolog\'{\i}a, $S$ es $T_2$ y $N_2$. Sea
	\begin{align*}
		\cal A & \,:=\,\big\{(V,\psi)\text{ carta para } M\,:\,
			U\cap S\text{ es } n\text{-tajada }\big\}
		\text{ .}
	\end{align*}
	%
	Es decir, si $(V,\psi)\in\cal A$, entonces
	\begin{align*}
		\psi(V)\,=\,\cubo[m]{\epsilon}{0} & \quad\text{y}\quad
			\psi(V\cap S) \,=\,\cubo[m]{\epsilon}{0}\,\cap\,
				\big\{x^{n+1}=\cdots=x^m\big\}
		\text{ .}
	\end{align*}
	%
	Sea $\pi:\,\psi(V)\rightarrow\bb R^n$ la proyecci\'{o}n en las
	primeras $n$ coordenadas y sea $j:\,\bb R^n\rightarrow\bb R^m$ la
	inserci\'{o}n en las primeras $n$ coordenadas. En t\'{e}rminos de esta
	notaci\'{o}n, observamos que $V\cap S$ es abierto en $S$, la
	restricci\'{o}n
	\begin{align*}
		\pi_V|_{\psi(V\cap S)} & \,:\,\psi(V\cap S)\,\rightarrow\,
			\cubo[n]{\epsilon}{0}
	\end{align*}
	%
	es un homeo con inversa la restricci\'{o}n correstricci\'{o}n
	\begin{math}
		j|:\,\cubo[n]{\epsilon}{0}\rightarrow\psi(V\cap S)
	\end{math} y el par $(V\cap S,\pi_V\circ\psi)$ es una carta para $S$.

	Ahora bien, como todo punto $p\in S$ pertenece al dominio de alguno de
	las cartas en $\cal A$, la familia
	\begin{align*}
		\cal A_S & \,:=\,\big\{(V\cap S,\pi_V\circ\psi)\,:\,
			(V,\psi)\in\cal A\big\}
	\end{align*}
	%
	es un atlas para $S$. En definitiva, $S$ es localmente euclidea de
	dimensi\'{o}n $n$ y, por lo tanto, una variedad topol\'{o}gica.

	Para darle estructura diferencial, corroboramos que el atlas $\cal A_S$
	sea un atlas suavemente compatible. Sean $(V,\psi)$ y $(V',\psi')$
	cartas en $\cal A$. El cambio de coordenadas
	\begin{align*}
		(\pi_V\circ\psi)\circ(\pi_{V'}\circ\psi')^{-1} & \,:\,
		\pi_{V'}\circ\psi'\big((V\cap S)\cap(V'\cap S)\big)
			\,\rightarrow\,
		\pi_V\circ\psi\big((V\cap S)\cap(V'\cap S)\big)
	\end{align*}
	%
	est\'{a} dado por
	\begin{align*}
		(\pi_V\circ\psi)\circ(\pi_{V'}\circ\psi')^{-1}
			(\lista*{x}{n}) & \,=\,
			\pi_V\circ\psi\big(
				{\psi'}^{-1}(\lista*{x}{n},\,0,\,\dots,\,0)
				\big) \\
		& \,=\,\pi_V(\lista*{y}{m}) \,=\,(\lista*{y}{n})
		\text{ .}
	\end{align*}
	%
	Como ${\psi'}^{-1}(\lista*{x}{n},\,0,\,\dots,\,0)$ est\'{a} en $S$,
	debe ser $y^{n+1}(=\psi^{n+1})=\cdots=y^m=0$. De todos modos, como la
	composici\'{o}n $\psi\circ{\psi'}^{-1}$ es suave y $\pi_V$ y $j$
	tambi\'{e}n lo son, las cartas son compatibles. En definitiva, la
	variedad $S$ admite una estructura de variedad diferencial de
	dimensi\'{o}n $n$, sobre la topolog\'{\i}a de subespacio de $M$.

	La inclusi\'{o}n $\inc[S]:\,S\hookrightarrow M$ es un embedding: en
	coordenadas,
	\begin{align*}
		\psi\circ\inc[S]\circ(\pi_V\circ\psi)^{-1}(\lista*{x}{n})
			& \,=\,(\lista*{x}{n},\,0,\,\dots,\,0)
		\text{ .}
	\end{align*}
	%
	En particular, $\inc[S]$ es suave y tiene rango constante $n=\dim\,S$.

	Si consideramos ahora $f|:\,N\rightarrow S$, dado $p\in N$ existen
	cartas preferenciales $(U,\varphi)$ para $N$ en $p$ y $(V,\psi)$ para
	$M$ en $q=f(p)$. Como $f(N)=S$ tiene la propiedad de $n$-subvariedad,
	vale que
	\begin{math}
		f(N)\cap V=
			\cubo[m]{\epsilon}{0}\cap\{\psi^{n+1}=\cdots\psi^m=0\}
	\end{math}, con lo cual, $(V,\psi)\in\cal A$ y
	$(V\cap S,\pi_V\circ\psi)\in\cal A_S$. Con respecto a estas cartas,
	\begin{align*}
		(\pi_V\circ\psi)\circ f\circ\varphi^{-1}(\lista*{x}{n}) & \,=\,
			\pi_V\circ\encoordenadas{f}(\lista*{x}{n}) \,=\,
			\pi_V(\lista*{x}{n},\,0,\,\dots,\,0) \\
		& \,=\, (\lista*{x}{n})
		\text{ .}
	\end{align*}
	%
	Es decir, la representaci\'{o}n en coordenadas es
	\begin{math}
		(\pi_V\circ\psi)\circ f\circ\varphi^{-1}=
			\id[{\cubo[n]{\epsilon}{0}}]
	\end{math}, suave y, m\'{a}s aun, un difeo. Como $f|:\,N\rightarrow S$
	es biyectiva y difeomorfismo local, debe ser difeomorfismo. En
	particular, $f:\,N\rightarrow M$ tiene rango constante (el rango de
	$\inc[S]$) y es subespacio.
\end{proof}
