\theoremstyle{plain}
\newtheorem{lemaNotasSueltasGermenes}{Lema}[section]
\newtheorem{propoSuavidadCampos}[lemaNotasSueltasGermenes]{Proposici\'{o}n}
\newtheorem{propoEquivalenciaSuavidadCampos}[lemaNotasSueltasGermenes]%
	{Proposici\'{o}n}
\newtheorem{propoSuavidadFormas}[lemaNotasSueltasGermenes]{Proposici\'{o}n}
\newtheorem{propoEquivalenciaSuavidadFormas}[lemaNotasSueltasGermenes]%
	{Proposici\'{o}n}

\theoremstyle{remark}
\newtheorem{obsBaseEnElCotangente}[lemaNotasSueltasGermenes]{Observaci\'{o}n}

%-------------

\begin{lemaNotasSueltasGermenes}\label{lema:notassueltasgermenes}
	Sea $M$ una variedad diferencial y sea $p\in M$. Sean
	$v\in\tangente[p]{M}$, $U\subset M$ un entorno de $p$ y
	$f,g\in C^\infty(U)$. Si $f|_V=g|_V$ para cierto abierto tal que
	$V\subset U$ y $p\in V$, entonces $v\,f=v\,g$.
\end{lemaNotasSueltasGermenes}

\begin{proof}
	Sea $h=f-g$ y sea $\psi:\,M\rightarrow\bb{R}$ una funci\'{o}n
	suave, $\psi\geq 0$, tal que existe una bola abierta $B$ que verifica
	$p\in B\subset\soporte{\psi}\subset V$ y $\psi|_B=1$. En particular,
	$\psi(p)=1$. Entonces $\psi\,h=0$ en $M$ y
	\begin{align*}
		0 & \,=\,v(\psi\,h)\,=\,(v\,\psi)\,h(p)\,+\,
			\psi(p)\,(v\,h)\,=\,v\,h
		\text{ ,}
	\end{align*}
	%
	pues $v(\psi\,h)=v(0)=0$ y $\psi(p)=1$ (no estamos usando que $\psi$
	es constante en un entorno de $p$). As\'{\i},
	\begin{align*}
		\psi\,h & \,=\,
			\begin{cases}
				0 & \text{ en } U\setmin\soporte{\psi} \\
				0 & \text{ en } V
			\end{cases}
		\text{ .}
	\end{align*}
	%
\end{proof}

\begin{propoSuavidadCampos}\label{propo:suavidadcampos}
	Sea $X:\,M\rightarrow\tangente{M}$ una secci\'{o}n del fibrado
	tangente. Entonces $X$ es suave (como secci\'{o}n), si y s\'{o}lo si,
	para todo punto $p\in M$, existe una carta $(U,\psi)$, compatible con
	la estructura de $M$, tal que $p\in U$ y con respecto a la cual las
	funciones $X^i:\,U\rightarrow\bb R$ determinadas por
	\begin{equation}\label{eq:suavidadcampos}
		X|_U \,=\,X^i\,\gancho{x^i}
	\end{equation}
	%
	son suaves.
\end{propoSuavidadCampos}

\begin{proof}
	En primer lugar, $X^i=X(x^i)$, evaluar el campo $X$ en la funci\'{o}n
	coordenada $x^i$. Que $X$ sea suave quiere decir que, para toda carta
	compatible $(U,\varphi)$, la expresi\'{o}n en coordenadas
	\begin{math}
		\encoordenadas{X}=
			\delfibrado{\varphi}\circ X\circ\varphi^{-1}:\,
			\varphi(U)\rightarrow\bb{R}^{2n}
	\end{math} --donde
	\begin{math}
		\delfibrado{\varphi}:\,\delfibrado{U}=\pi^{-1}(U)\rightarrow
			\delfibrado{\varphi}(\delfibrado{U})=U\times\bb{R}^{n}
	\end{math} es la carta correspondiente
	$(\delfibrado{U},\delfibrado{\varphi})$ en $\tangente{M}$-- es suave,
	en sentido usual. Pero $\encoordenadas{X}$ est\'{a} dada por
	\begin{align*}
		\encoordenadas{X}(x) & \,=\,\encoordenadas{X}(\lista{x}{n}) \\
		& \,=\,
			\big(X^1(\varphi^{-1}(x)),\,\dots,\,
				X^n(\varphi^{-1}(x)),\,
				x^1(\varphi^{-1}(x)),\,\dots,\,
				x^n(\varphi^{-1}(x))\big) \\
		& \,=\,\big(\encoordenadas{X}^1(x),\,\dots,\,
			\encoordenadas{X}^n(x),\lista{x}{n}\big)
		\text{ .}
	\end{align*}
	%
	Esta expresi\'{o}n es v\'{a}lida para toda carta $(U,\varphi)$ y todo
	campo $X:\,M\rightarrow\tangente{M}$, independientemente de su
	suavidad. En particular, $X$ es suave (continua) en $U$, si y s\'{o}lo
	si las funciones $\encoordenadas{X}^i=X^i\circ\varphi^{-1}$ los son en
	$\varphi(U)$. Como $\varphi:\,U\rightarrow\varphi(U)$ es difeomorfismo,
	esto equivale a que las funciones $X^i:\,U\rightarrow\bb R$ sean
	suaves.
\end{proof}

\begin{propoEquivalenciaSuavidadCampos}\label{propo:equivalenciasuavidadcampos}
	Sea $X:\,M\rightarrow\tangente{M}$ una secci\'{o}n. Las siguientes
	afirmaciones son equivalentes.
	\begin{enumerate}
		\item\label{propo:equivalenciasuavidadcampos:seccion}
			$X$ es suave;
		\item\label{propo:equivalenciasuavidadcampos:global}
			$Xf\in C^\infty(M)$ para toda $f\in C^\infty(M)$;
		\item\label{propo:equivalenciasuavidadcampos:entornos}
			$X|_Uf\in C^\infty(U)$ para toda $f\in C^\infty(U)$.
	\end{enumerate}
\end{propoEquivalenciaSuavidadCampos}

Antes de demostrar esto, vale la pena recordar que una funci\'{o}n en $M$ es
suave en un punto respecto de una carta compatible, si y s\'{o}lo si es suave,
en el mismo punto, respecto de cualquier otra carta compatible.

\begin{proof}
	Si $X\in\champs M$ y $f\in C^\infty(M)$, dados $p\in M$ y una carta
	$(U,\varphi)$ en $p$, en un entorno del punto (en $U$),
	\begin{align*}
		X\,f & \,=\,X^i\,\derivada{f}{x^i}
		\text{ ,}
	\end{align*}
	%
	que es suave. Expl\'{\i}citamente, en coordenadas, si $x\in\varphi(U)$,
	\begin{align*}
		\big(X\,f\big)(\varphi^{-1}(x)) & \,=\,
			X^i\circ\varphi^{-1}(x)\,
				\derivada{(f\circ\varphi^{-1})}{x^i}
					(\varphi(\varphi^{-1}(x))) \,=\,
			\encoordenadas{X}^i(x)\,
				\derivada{\encoordenadas{f}}{x^i}(x)
		\text{ ,}
	\end{align*}
	%
	donde $\encoordenadas{X}^i$ y $\encoordenadas{f}$ son las
	representaciones en coordenadas de las funciones componentes $X^i$ y
	$f$, definidas en $U$ (en el caso de $X^i$) y en $M$. Entonces $X\,f$
	es suave en un entorno de $p$. Como el punto fue elegido de forma
	arbitraria, $X\,f\in\suaves M$.

	Si vale \ref{propo:equivalenciasuavidadcampos:global}, y
	$f\in\suaves U$ para cierto abierto $U\subset M$, dado $p\in U$, existe
	una funci\'{o}n $\psi:\,M\rightarrow\bb R$ suave tal que $\psi\geq 0$ y
	un abierto $V$ tal que $p\in V$, $\clos{V}\subset U$, $\psi|_V=1$ y
	$\soporte{\psi}\subset U$. Sea $\tilde{f}:\,M\rightarrow\bb R$ la
	funci\'{o}n
	\begin{align*}
		\tilde{f} & \,:=\,
			\begin{cases}
				f\cdot\psi & \text{ en } U \\
				0 & \text{ en } M\setmin\soporte{\psi}
			\end{cases}
		\text{ .}
	\end{align*}
	%
	Como el soporte de la funci\'{o}n $\psi$ est\'{a} contenido en $U$,
	esta funci\'{o}n est\'{a} bien definida y es suave. Adem\'{a}s,
	\begin{align*}
		\big(\tilde f|_U\big)|_V & \,=\,\tilde f|_V \,=\,f|_V
		\text{ ,}
	\end{align*}
	%
	con lo cual, para $q\in V$,
	\begin{align*}
		\big(X|_Uf\big)(q) & \,=\,\big(X|_U\tilde f|_U\big)(g) \,=\,
			\big(X\,\tilde f\big)(q)
		\text{ .}
	\end{align*}
	%
	Pero, por hip\'{o}tesis, $X\,\tilde f$ es suave en $M$, con lo que
	$X|_Uf$ es suave en $V$. Como $p\in U$ fue elegido de manera
	arbitraria, $X|_Uf$ es suave en todo el abierto $U$.

	Finalmente, si se cumple
	\ref{propo:equivalenciasuavidadcampos:entornos} y $(U,\varphi)$ es una
	carta compatible, las componentes de $X$ en la carta est\'{a}n dadas
	por
	\begin{align*}
		X|_U & \,=\, X^i\,\gancho{x^i}
		\text{ ,}
	\end{align*}
	%
	donde $X^i=X|_U(x^i)$. Por hip\'{o}tesis, cada una de estas funciones
	es suave en $U$ y, por la Proposici\'{o}n~\ref{propo:suavidadcampos},
	$X\in\champs M$.
\end{proof}

\begin{propoSuavidadFormas}\label{propo:suavidadformas}
	Sea $\omega:\,M\rightarrow\tangente*{M}$ una secci\'{o}n del fibrado
	cotangente. Entonces $\omega$ es suave, si y s\'{o}lo si, para todo
	$p\in M$, existe una carta $(U,\varphi)$ compatible, cuyo dominio
	contiene a $p$ y tal que la representaci\'{o}n en coordenadas
	\begin{equation}\label{eq:suavidadformas}
		\omega|_U \,=\,\omega_i\,\lambda^i
	\end{equation}
	%
	sea suave, es decir, de forma que las funciones
	$\omega_i:\,U\rightarrow\bb R$ sean suaves, donde
	$\big\{\lambda^i|_p\big\}_i$ es la base dual de
	$\big\{\gancho[p]{x^i}\big\}_i$ en $\tangente*[p]{M}$ para $p\in U$.
\end{propoSuavidadFormas}

\begin{proof}
	Sea $\pi:\,\tangente*{M}\rightarrow M$ la proyecci\'{o}n can\'{o}nica
	del fibrado. Sea $(U,\varphi)$ una carta compatible con la estructura
	de $M$ y sean $\delfibrado{U}=\pi^{-1}(U)\subset\tangente*{M}$ y
	$\delfibrado{\varphi}:\,\delfibrado{U}\rightarrow U\times\bb R^n$ las
	coordenadas correspondientes. La representaci\'{o}n en coordenadas de
	la forma $\omega$ es la funci\'{o}n
	\begin{math}
		\encoordenadas{\omega}=
			\delfibrado{\varphi}\circ\omega\circ\varphi^{-1}:\,
			\varphi(U)\rightarrow\bb R^{2n}
	\end{math} dada por
	\begin{align*}
		\encoordenadas{\omega}(x) & \,=\,\big(
			\encoordenadas{\omega}_1(x),\,\dots,\,
			\encoordenadas{\omega}_n(x),\,\lista{x}{n}\big)
		\text{ ,}
	\end{align*}
	%
	donde $\encoordenadas{\omega}_i=\omega_i\circ\varphi^{-1}$ y cada
	funci\'{o}n $\omega_i:\,U\rightarrow\bb R$ est\'{a} determinada por la
	representaci\'{o}n \eqref{eq:suavidadformas}, es decir,
	$\omega_i=\omega|_U\big(\gancho{x^i}\big)$. As\'{\i},
	$\omega:\,M\rightarrow\tangente*{M}$ es, por definici\'{o}n, suave
	(continua) en $U$, si y s\'{o}lo si las funciones
	$\omega_i:\,U\rightarrow\bb R$ lo son.
\end{proof}

\begin{obsBaseEnElCotangente}\label{obs:baseenelcotangente}
	Sea $(U,\varphi)$ una carta, sea $\big\{\gancho[p]{x^i}\big\}$ la base
	dada por los campos coordenados en $\tangente[p]{M}$ y sea
	$\big\{\lambda^i|_p\big\}_i$ la base dual. Sean $\{x^i\}_i$ las
	funciones coordenadas en $U$ y sean $\big\{\de{x^i}\big\}_i$ las
	$1$-formas $(\de{x^i})\,X=X(x^i)$. Entonces, si $p\in U$ y
	$v\in\tangente[p]{M}$,
	\begin{align*}
		\de[p]{x^i}(v) & \,=\,v(x^i)
		\text{ .}
	\end{align*}
	%
	Pero $v=v^i\,\gancho[p]{x^i}$, donde $v^i=v(x^i)$. Como
	$\big\{\lambda^i|_p\big\}_i$ es la base dual a la base can\'{o}nica,
	\begin{align*}
		\lambda^i|_p\,v & \,=\,v^i
		\text{ ,}
	\end{align*}
	%
	es decir, para cada \'{\i}ndice $i$,
	\begin{equation}\label{eq:baseenelcotangente}
		\lambda^i|_p \,=\,\de[p]{x^i}
		\text{ .}
	\end{equation}
	%
\end{obsBaseEnElCotangente}

\begin{propoEquivalenciaSuavidadFormas}\label{propo:equivalenciasuavidadformas}
	Sea $\omega:\,M\rightarrow\tangente*{M}$ una secci\'{o}n. Las
	siguientes afirmaciones son equivalentes.
	\begin{enumerate}
		\item\label{propo:equivalenciasuavidadformas:seccion}
			$\omega$ es suave;
		\item\label{propo:equivalenciasuavidadformas:global}
			$\omega\,X\in\suaves M$ para todo $X\in\champs M$;
		\item\label{propo:equivalenciasuavidadformas:entornos}
			$\omega|_UX\in\suaves U$ para todo $X\in\champs U$.
	\end{enumerate}
\end{propoEquivalenciaSuavidadFormas}

\begin{proof}
	Asumiendo que $\omega:\,M\rightarrow\tangente*{M}$ es suave y que
	$X\in\champs M$, para verificar la suavidad de
	\begin{math}
		\omega\,X:\,p\mapsto\omega_p\,X_p
	\end{math}, verificamos que sea suave en coordenadas: sea $(U,\varphi)$
	una carta compatible. La expresi\'{o}n de la funci\'{o}n en coordenadas
	est\'{a} determinada por
	\begin{equation}\label{eq:formacampoencoordenadas}
		\omega\,X \,=\,\omega|_U\Big(X^i\,\gancho{x^i}\Big) \,=\,
			X^i\Big(\omega|_U\gancho{x^i}\Big) \,=\,
			X^i\,\omega_i
		\text{ .}
	\end{equation}
	%
	Por un lado, como $X$ es suave en $M$, las funciones $X^i$ son suaves
	en $U$. Por otro, como $\omega\in\formes[1]{M}$, vale que
	$\omega_i\in\suaves U$. En definitiva, la funci\'{o}n determinada por
	evaluar $\omega\,X$ es suave en $U$. Como la carta era arbitraria, la
	funci\'{o}n es suave en $M$.

	Asumiendo \ref{propo:equivalenciasuavidadformas:global}, sea
	$U\subset M$ un abierto y sea $X\in\champs U$. Sea $p\in U$ y sea
	$\psi:\,M\rightarrow\bb R$ una funci\'{o}n suave tal que
	$\psi\geq 0$, $\psi|_V=1$ en cierto abierto $V\subset U$,
	$p\in V$, $\clos V\subset U$ y $\soporte{\psi}\subset U$. Sea
	$\tilde X:\,M\rightarrow\tangente M$ el \emph{campo} dado por
	\begin{align*}
		\tilde X & \,:=\,
			\begin{cases}
				\psi\cdot X & \text{ en } U \\
				0 & \text{ en } M\setmin\soporte{\psi}
			\end{cases}
		\text{ .}
	\end{align*}
	%
	Como el soporte de la funci\'{o}n chich\'{o}n $\psi$ est\'{a} contenido
	en $U$, $\tilde X$ est\'{a} bien definido y es suave en $M$. Pero,
	adem\'{a}s, $\tilde X|_V=X|_V$. Por un lado,
	$\omega\,\tilde X\in\suaves M$, por hip\'{o}tesis. Por otro lado, si
	$q\in V$,
	\begin{align*}
		\big(\omega\,\tilde X\big)(q) & \,=\,
			\omega_q\,\tilde X|_q \,=\,
			\omega_q\,X_q \,=\,\big(\omega|_UX\big)(q)
		\text{ .}
	\end{align*}
	%
	Entonces $\omega|_UX$ es suave en $V$. Como $p\in U$ fue elegido de
	manera arbitraria, $\omega|_UX\in\suaves U$.

	Finalmente, asumiendo que vale
	\ref{propo:equivalenciasuavidadformas:entornos} y que $(U,\varphi)$ es
	una carta compatible, por hip\'{o}tesis, las componentes
	$\omega_i=\omega|_U\gancho{x^i}$ de $\omega$ en $U$ son suave.
	As\'{\i}, por la Proposici\'{o}n~\ref{propo:suavidadformas}, se ve que
	$\omega$ es suave como secci\'{o}n, ya que la carta $(U,\varphi)$
	hab\'{\i}a sido elegida de manera arbitraria.
\end{proof}
