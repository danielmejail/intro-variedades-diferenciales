\theoremstyle{plain}
\newtheorem{teoFunInvVar}{Teorema}[section]
\newtheorem{propoDifeoLocal}[teoFunInvVar]{Proposici\'{o}n}
\newtheorem{coroDifeoLocal}[teoFunInvVar]{Corolario}

\theoremstyle{remark}
\newtheorem{obsDifeoLocal}{Observaci\'{o}n}[section]

%-------------

Sea $F:\,M\rightarrow N$ una transformaci\'{o}n suave. Dado un punto $p\in M$,
el \emph{rango de $F$ en $p$} se define como el rango de la transformaci\'{o}n
lineal asociada $\diferencial[p]{F}:\,T_{p}M\rightarrow T_{F(p)}N$. Este
n\'{u}mero es igual al rango de la matriz jacobiana de $F$ en cualquiera de
sus representaciones en coordenadas $\jacobiana[\widehat{p}]{\widehat{F}}$,
siendo \'{e}sta la matriz de la transformaci\'{o}n lineal $\diferencial[p]{F}$
con respecto a las bases determinadas por tomar cartas compatibles en $p$ y en
$F(p)$, por lo que, equivalentemente, podr\'{\i}amos definir el rango de
esta manera, sin hacer referencia directamente a los tangentes y al
diferencial. Denotamos el rango de $F$ en $p$ por $\rango[p]{F}$ y,
por definici\'{o}n $\rango[p]{F}=\rango{\diferencial[p]{F}}$.

Si existe $r\geq 0$ tal que $\rango{\diferencial[p]{F}}=r$, para todo punto
$p$, se dice que $F$ \emph{tiene rango constante $r$ en $M$} o que
\emph{es de rango constante}. En todo caso, el rango de $F$ est\'{a}
acotado:
\begin{align*}
	0 & \,\leq\,\rango{F}\,\leq\,\min\{\dim\,M,\dim\,N\}
	\text{ .}
\end{align*}
%
Si la cota superior es alcanzada en un punto $p\in M$, se dice que $F$
\emph{tiene rango m\'{a}ximo en $p$}. Esto se puede deber a cualquiera de
dos cosas: o bien $\diferencial[p]{F}:\,T_{p}M\rightarrow T_{F(p)}N$ es
inyectivo ($\rango{\diferencial[p]{F}}=\dim\,M$), o bien es sobreyectivo
($\rango{\diferencial[p]{F}}=\dim\,N$).

El rango de una transformaci\'{o}n suave $F$ verifica que
\begin{align*}
	\{\rango{F}>k\} & \,=\,\{\rango{F}\geq k+1\}
\end{align*}
%
es abierto para todo $k\geq 0$. En particular, si $F$ tiene rango m\'{a}ximo
en $p$, entonces tendr\'{a} rango m\'{a}ximo en todo un entorno del punto,
en particular, el rango de $F$ ser\'{a} constante en el entorno. Si
$\diferencial[p]{F}$ es sobreyectivo para todo punto $p$ en el dominio de
$F$, decimos que $F$ es una \emph{submersi\'{o}n}; si es inyectivo, decimos
que es una \emph{inmersi\'{o}n}. Con estas definiciones, podemos afirmar
que, si $\diferencial[p]{F}$ es inyectivo, entonces la restricci\'{o}n
$F|_{U}:\,U\rightarrow N$ es una inmersi\'{o}n en alg\'{u}n entorno $U$ de
$p$. An\'{a}logamente, si $\diferencial[p]{F}$ es sobreyectivo, $F|_{U}$
es una submersi\'{o}n.

Un ejemplo de esto est\'{a} dado por una funci\'{o} diferenciable
$f:\,U\rightarrow\bb{R}$ definida en un abierto de $\bb{R}^{n}$. En las
coordenadas usuales, la matriz jacobiana de $f$ est\'{a} dada por el
vector de derivadas parciales evaluadas en el punto:
\begin{align*}
	\jacobiana[x]{f} & \,=\,
		\begin{bmatrix}
			\derivada{f}{x^{1}}(x) & \cdots &
			\derivada{f}{x^{n}}(x)
		\end{bmatrix}
	\text{ .}
\end{align*}
%
La transformaci\'{o}n lineal asociada es sobreyectiva, si y s\'{o}lo si
alguna derivada es distinta de cero en el punto. Los puntos en donde el
diferencial de esta funci\'{o}n es sobreyectivo (en este caso esto quiere
decir distinto de cero) son precisamente los puntos \emph{no singulares}
de $f$.

Para ver una clase de casos en donde el diferencial es inyectivo, podemos
tomar una funci\'{o}n en la direcci\'{o}n opuesta. Si
$f:\,\bb{R}\rightarrow U\subset\bb{R}^{n}$ es diferenciable, entonces la
matriz jacobiana en este caso tambi\'{e}n est\'{a} dada por el vector de
derivadas parciales, ahora visto como una matriz de tama\~{n}o
$n\times 1$. Esta matriz ser\'{a} la matriz de una transformaci\'{o}n
lineal inyectiva, siempre y cuando, de nuevo, alguna de las derivadas
parciales no sea nula. La funci\'{o}n $f$ describe una curva --no en
tanto variedad de dimensi\'{o}n, sino en tanto parametrizaci\'{o}n--
en el espacio. Que el diferencial de $f$ sea inyectivo en un insante $t$
significa que la velocidad de la curva en dicho instante, $\dot{f}(t)$ no
es cero.

Un \emph{difeomorfismo local} es una transformaci\'{o}n suave
$F:\,M\rightarrow N$ tal que, para todo punto $p\in M$, existe un entorno
$U\subset M$ de $p$ que verifica que $F(U)\subset N$ es abierta y que
$F|_{U}:\,U\rightarrow F(U)$ es un difeomorfismo. Como los tangentes
est\'{a}n definidos localmente, si $F$ es un difeomorfismo local, entonces
$\diferencial[p]{F}:\,T_{p}M\rightarrow T_{F(p)}N$ es un isomorfismo
lineal en todo punto. En particular, todo difeomorfismo local es, a la
vez, una submersi\'{o}n y una inmersi\'{o}n. Aunque s\'{o}lo v\'{a}lido
para una transformaci\'{o}m entre variedades \emph{sin} borde, el teorema
de la funci\'{o}n inversa es la afirmaci\'{o}n rec\'{\i}proca:

\begin{teoFunInvVar}\label{thm:funinvvar}
	Sean $M$ y $N$ variedades \emph{sin} borde y sea $F:\,M\rightarrow N$
	una transformaci\'{o}n suave. Si $p\in M$ es un punto en donde
	$\diferencial[p]{F}:\,T_{p}M\rightarrow T_{F(p)}N$ es invertible,
	existen entonces entornos conexos $U_{0}\subset M$ de $p$ y
	$V_{0}\subset N$ de $F(p)$ tales que la restricci\'{o}n
	$F|_{U_{0}}:\,U_{0}\rightarrow V_{0}$ es difeomorfismo.
\end{teoFunInvVar}

\begin{proof}
	Aunque parezca trivial, el hecho de que $\diferencial[p]{F}$ es
	isomorfismo implica que $\dim\,M=\dim\,N$. Tomando coordenadas
	$(U,\varphi)$ en $p$ y $(V,\psi)$ en $F(p)$ tales que
	$F(U)\subset V$, como tanto $\varphi$ como $\psi$ son
	difeomorfismos, la matriz jacobiana
	$\jacobiana[\widehat{p}]{\widehat{F}}$ es invertible. Por el teorema
	usual de la funci\'{o}n inversa, existen entornos (conexos)
	$\widehat{U}_{0}\subset\varphi(U)$ y $\widehat{V}_{0}\subset\psi(V)$
	tales que $\widehat{F}|_{\widehat{U}_{0}}:\,%
	\widehat{U}_{0}\rightarrow\widehat{V}_{0}$ es un difeomorfismo. Si
	$U_{0}=\varphi^{-1}(\widehat{U}_{0})$ y
	$V_{0}=\psi^{-1}(\widehat{V}_{0})$, entonces $p\in U_{0}$,
	$F(p)\in V_{0}$, $U_{0}$ y $V_{0}$ son conexos y
	$F|_{U_{0}}\rightarrow V_{0}$ es difeomorfismo.
\end{proof}

A continuaci\'{o}n enunciamos algunas propiedades importantes de los
difeomorfismos locales.

\begin{propoDifeoLocal}\label{thm:propisdifeoslocales}
	\emph{(a)} La composici\'{o}n de difeomorfismos locales es un
	difeomorfismo local; \emph{(b)} el producto de dos difeomorfismos
	locales es un difeomorfismo local, su inversa est\'{a} dada por el
	producto de las respectivas transformaciones inversas;
	\emph{(c)} la restricci\'{o}n de un difeomorfismo local a un
	abierto sigue siendo un difeomorfismo local; \emph{(d)} todo
	difeomorfismo local biyectivo es un difeomorfismo.
\end{propoDifeoLocal}

Por ejemplo, para demostrar \emph{(d)}, si $F:\,M\rightarrow N$ es
biyectiva y difeomorfismo local, entonces, localmente, su inversa
$F^{-1}$ coincide con una funci\'{o}n suave: si $q\in N$ y $p\in M$
es el (\'{u}nico) punto tal que $F(p)=q$, existe un entorno $U$ de $p$
tal que $F(U)$ es abierta y $F|_{U}:\,U\rightarrow F(U)$ es difeomorfismo.
En particular, $F^{-1}|_{F(U)}=(F|_{U})^{-1}$ que es una funci\'{o}n
suave.

Para determinar si una transformaci\'{o}n (suave) es un difeomorfismo local,
alcanza con ver lo que sucede en entornos coordenados.

\begin{obsDifeoLocal}\label{obs:difeoslocaleslocal}
	Una transformaci\'{o}n suave $F:\,M\rightarrow N$ es un
	difeomorfismo local, si y s\'{o}lo si ara cada punto $p\in M$
	existe un entorno $U\subset M$ de $p$ tal que la representaci\'{o}n
	en coordenadas de $F$ en $U$ es un difeomorfismo local.
\end{obsDifeoLocal}

\begin{coroDifeoLocal}\label{thm:difeolocalsubmersioneinmersion}
	Si $F:\,M\rightarrow N$ es una transformaci\'{o}n suave, entonces
	$F$ es un difeomorismo local, si y s\'{o}lo si es una
	submersi\'{o}n y una inmersi\'{o}n.
\end{coroDifeoLocal}

\begin{proof}
	Ya demostramos que todo difeomorfismo local es submersi\'{o}n
	e inmersi\'{o}n. Rec\'{\i}procamente, si $F$ es, a la vez, una
	submersi\'{o}n y una inmersi\'{o}n, entonces
	$\diferencial[p]{F}:\,T_{p}M\rightarrow T_{F(p)}N$ es un
	isomorfismo. Por el teorema de la funci\'{o}n inversa, existen
	entornos de $p$ y de $F(p)$ tales que $F$ restringida a los mismos
	es un difeomorfismo.
\end{proof}

\begin{coroDifeoLocal}\label{thm:difeolocalsubmersionoinmersion}
	Si $F:\,M\rightarrow N$ es una transformaci\'{o}n suave,
	$\dim\,M=\dim\,N$ y $F$ es una submersi\'{o}n o una inmersi\'{o}n,
	entonces $F$ es un difeomorfismo local.
\end{coroDifeoLocal}

\begin{proof}
	Dado que las dimensiones del dominio y del codominio de $F$ son
	iguales, $F$ es una inmersi\'{o}n, si y s\'{o}lo si es una
	submersi\'{o}n.
\end{proof}
