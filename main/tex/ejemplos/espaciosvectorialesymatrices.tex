



\begin{ejemplo}\nom{Espacios vectoriales}
	Sea $E$ un espacio vectorial topol\'{o}gico real. Si $E$ es de
	dimensi\'{o}n finita $n$, equivalentemente, si $E$ es
	localmente compacto, entonces $E$ es isomorfo a $\bb{R}^{n}$ y
	todo isomorfismo \emph{lineal} de $E$ en $\bb{R}^{n}$ es un
	homeomorfismo. Es decir, $E$ tiene una \'{u}nica estructura
	de espacio topol\'{o}gico que hace que las operaciones en tanto
	espacio vectorial real sean continuas. Todo isomorfismo
	$E\rightarrow\bb{R}^{n}$ est\'{a} dado por elegir una base: dada
	una base $\{\lista{\varepsilon}{n}\}$ de $E$, sea
	$\varepsilon:\,\bb{R}^{n}\rightarrow E$ la aplicaci\'{o}n
	\begin{align*}
		\varepsilon(x) & \,=\,\sum_{i=1}^{n}\,x^{i}\varepsilon_{i}
			\,\equiv\,x^{i}\varepsilon_{i}
		\text{ .}
	\end{align*}
	%
	Seg\'{u}n lo mencionado anteriormente, $\varepsilon$ es un
	homeomorfismo y, entonces, el par $(E,\varepsilon^{-1})$ es una
	carta (continua) para $E$. Si ahora $\tilde{\varepsilon}$ denota
	el isomorfismo correspondiente a otra base
	$\{\lista{\tilde{\varepsilon}}{n}\}$, entonces existe una matriz
	invertible $\left[A_{i}^{j}\right]^{i}_{j}$ tal que
	\begin{align*}
		\varepsilon_{i} & \,=\,A_{i}^{j}\tilde{\varepsilon}_{j}
	\end{align*}
	%
	para cada $i$. El cambio de cartas (cambio de coordenadas)
	correspondiente est\'{a} dado por
	$\tilde{\varepsilon}^{-1}\circ\varepsilon(x)=\tilde{x}$,
	donde $\tilde{x}=(\lista*{\tilde{x}}{n})$ es el punto de
	$\bb{R}^{n}$ dado por
	\begin{align*}
		\tilde{x}^{j}\tilde{\varepsilon}_{j} & \,=\,
			x^{i}\varepsilon_{i}
			\,=\,x^{i}A_{i}^{j}\tilde{\varepsilon}_{j}
		\text{ .}
	\end{align*}
	%
	Es decir, para cada $j$, $\tilde{x}^{j}=A_{i}^{j}x^{i}$. En
	particular, la aplicaci\'{o}n $x\mapsto\tilde{x}$, el cambio de
	coordenadas (definido globalmente), es lineal e invertible y,
	en particular, un difeomorfismo. En definitiva, las cartas de la
	forma $(E,\varepsilon^{-1})$, donde
	$\varepsilon:\,\bb{R}^{n}\rightarrow E$ es el isomorfismo
	determinado por la base $\{\lista{\varepsilon}{n}\}$ de $E$,
	son todas (suavemente) compatibles. La estructura que estas cartas
	determinan en $E$ se denominar\'{a} la \emph{estructura usual} en $E$.
\end{ejemplo}

\begin{ejemplo}\nom{El espacio de matrices}
	El espacio de matrices $\MM{m\times n,\bb{R}}$ de tama\~{n}o
	$m\times n$ con coeficientes reales es un espacio vectorial de
	dimensi\'{o}n $mn$ y, por lo tanto, una variedad diferencial con
	su estructura usual. El espacio de matrices $\MM{m\times n,\bb{C}}$
	complejas constituye una variedad diferencial de dimensi\'{o}n $2mn$,
	pues su dimensi\'{o}n como espacio vectorial sobre $\bb{R}$ es $2mn$.
	En el caso de matrices cuadradas de tama\~{n}o $n\times n$ usaremos
	la notaci\'{o}n $\MM{n,\cdot}$.
\end{ejemplo}

\begin{ejemplo}\nom{El grupo general lineal}
	El \emph{grupo general lineal (real)}, denotado $\GL{n,\bb{R}}$
	es el conjunto de matrices invertibles con coeficientes reales.
	Dado que la funci\'{o}n $\det:\,\MM{n,\bb{R}}\rightarrow\bb{R}$
	es continua y que $\GL{n,\bb{R}}$ es el subconjunto de matrices
	con determinante no nulo, $\GL{n,\bb{R}}\subset\MM{n,\bb{R}}$ es
	un subconjunto abierto y, por lo tanto, una variedad diferencial
	de dimensi\'{o}n $\dim(\MM{n,\bb{R}})=n^{2}$.
\end{ejemplo}

\begin{ejemplo}\nom{Matrices de rango $\geq k$}
	Sea $k\geq 0$. De manera an\'{a}loga al ejemplo anterior, en
	$\MM{m\times n,\bb{R}}$, las matrices de rango mayor o igual a $k$
	forman un subconjunto abierto: una matriz tiene rango al menos $k$,
	si el determinante de alguna submatriz es distinto de cero. Por
	continuidad del determinante, existe un entorno de una tal matriz
	que verifica que, para todas las matrices de dicho abierto, el
	determinante de la misma submatriz no se anula. En definitiva, todas
	las matrices del abierto tienen rango al menos $k$. Un caso particular
	de esto es cuando $k$ es m\'{a}ximo, es decir, $k=\min\{m,n\}$.
\end{ejemplo}

\begin{ejemplo}\nom{Transformaciones lineales}
	Sean $E$ y $F$ dos espacios vectoriales de dimensi\'{o}n finita
	y sea $\lineal{E,F}$ el conjunto de transformaciones lineales
	de $E$ en $F$. Si a $E$ y $F$ se los considera espacios vectoriales
	topol\'{o}gicos reales, entonces $\lineal{E,F}$ coincide con
	el conjunto de transformaciones lineales y continuas. En general,
	como $\lineal{E,F}$ es un espacio vectorial (real) de dimensi\'{o}n
	finita, es una variedad diferencial. Eligiendo bases de $E$ y de $F$,
	se puede representar un elemento $T\in\lineal{E,F}$ como una matriz,
	lo cual determina un isomorfismo
	$\lineal{E,F}\simeq\MM{m\times n,\bb{R}}$.
\end{ejemplo}

\begin{ejemplo}\nom{El grupo lineal especial}
	El subgrupo $\SL{n,\bb{R}}$ de matrices reales de tama\~{n}o
	$n\times n$ invertibles de determinante $1$ es una variedad
	topol\'{o}gica con la topolog\'{\i}a de subespacio de
	$\GL{n,\bb{R}}$. Su dimensi\'{o}n es $n^{2}-1$. Se le puede
	dar una estructura de variedad diferencial de manera que
	la inclusi\'{o}n $\SL{n,\bb{R}}\hookrightarrow\GL{n,\bb{R}}$
	sea una inmersi\'{o}n. En consecuencia, el grupo lineal especial
	resulta una subvariedad regular de $\GL{n,\bb{R}}$.
	De manera an\'{a}loga, $\SL{n,\bb{C}}$ es una variedad topol\'{o}gica,
	si se le da la topolog\'{\i}a de subespacio de $\GL{n,\bb{C}}$ y
	se le puede dar una estructura de variedad diferencial de manera
	que sea una subvariedad regular del grupo general lineal.
\end{ejemplo}

\begin{ejemplo}\nom{El grupo de matrices ortogonales}
	El grupo $\ortogonal{n}$ de matrices \emph{ortogonales} en
	$\MM{n,\bb{R}}$ tambi\'{e}n es una variedad topol\'{o}gica,
	visto como subespacio de $\MM{n,\bb{R}}$. Se le puede dar una
	estructura de variedad diferencial de forma que resulte una
	subvariedad regular.
\end{ejemplo}
