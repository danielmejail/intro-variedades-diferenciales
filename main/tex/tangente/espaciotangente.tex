\theoremstyle{plain}
\newtheorem{propoElDiferencial}{Proposici\'{o}n}[section]
\newtheorem{lemaDerivacionesSonLocales}[propoElDiferencial]{Lema}
\newtheorem{propoDerivacionesIsomorfasI}[propoElDiferencial]{Proposici\'{o}n}
\newtheorem{propoDerivacionesIsomorfasII}[propoElDiferencial]{Proposici\'{o}n}

\theoremstyle{remark}
\newtheorem{obsDerivaciones}{Observaci\'{o}n}[section]
\newtheorem{obsSobreDerivaciones}[obsDerivaciones]{Observaci\'{o}n}
\newtheorem{obsDerivacionesLocalGlobal}[obsDerivaciones]{Observaci\'{o}n}
\newtheorem{obsDerivacionesIsomorfas}[obsDerivaciones]{Observaci\'{o}n}

%-------------

\subsection{Vectores tangentes geom\'{e}tricos}
Sea $a\in\bb{R}^{d}$ un punto en el espacio euclideo de dimensi\'{o}n $d$.
El \emph{espacio tangente geom\'{e}tricoa $\bb{R}^{d}$ en el punto $a$},
denotado $\bb{R}^{d}_{a}$ es el conjunto de pares de la forma $(a,v)$,
con $v\in\bb{R}^{d}$, es decir, $\bb{R}^{d}_{a}\simeq\{a\}\times\bb{R}^{d}$.
Los elementos de este espacio ser\'{a}n denominados \emph{vectores tangentes %
geom\'{e}tricos}. Denoteamos al punto correspondiente al par $(a,v)$
por $v_{a}$ o por $v|_{a}$. El espacio tangente geom\'{e}tico al punto $a$
es un espacio vectorial de dimensi\'{o}n $d$, si definimos las operaciones
en $\bb{R}^{d}_{a}$ trasladando los vectores al origen. Es decir, si
$v_{a},w_{a}\in\bb{R}^{d}_{a}$ y $\lambda\in\bb{R}$, entonces
\begin{align*}
	\lambda\cdot v_{a} & \,\equiv\,(\lambda\cdot v)_{a}
	\quad\text{y} \\
	v_{a} + w_{a} & \,\equiv\,(v+w)_{a}
	\text{ .}
\end{align*}
%

La base can\'{o}nica en $\bb{R}^{d}$ proporciona una base para el tangente
en $a$: $\{e_{1}|_{a},\,\dots,\,e_{d}|_{a}\}$. Notemos, adem\'{a}s, que
$\bb{R}^{d}_{a}\simeq\bb{R}^{d}$ can\'{o}nicamente v\'{\i}a $v_{a}\mapsto v$.

\subsection{Derivaciones}
Fijemos un punto $a\in\bb{R}^{d}$. Asociada a cada elemento
$v_{a}\in\bb{R}^{d}_{a}$, est\'{a} la noci\'{o}n de dervada direccional de
una funci\'{o}n: dada una funci\'{o}n $f:\,U\rightarrow\bb{R}$ definida
en un entorno $U$ de $a$, siempre que tenga sentido, definimos la
\emph{derivada direccional de $f$ en $a$ en la direcci\'{o}n del vector $v$}
como el l\'{\i}mite
\begin{align*}
	\left.\gancho{v}\right|_{a}f & \,\equiv\,
		\frac{\partial f}{\partial v}(a)\,\equiv\,
		\lim_{t\to 0}\,\frac{f(a+tv)-f(a)}{t} \\
	& \,\equiv\,\left.\gancho{t}\right|_{t=0}f(a+tv)
	\text{ .}
\end{align*}
%
Por la linealidad del l\'{\i}mite, la operaci\'{o}n $f\mapsto\gancho{v}|_{a}f$
es lineal en $f$ (en donde est\'{e} definida la aplicaci\'{o}n). Adem\'{a}s,
si las derivadas direccionales de $f$ y de $g$ existen, entonces
\begin{align*}
	\left.\gancho{v}\right|_{a}(fg) & \,=\,
		\left.\gancho{v}\right|_{a}f\cdot g(a) +
		f(a)\cdot\left.\gancho{v}\right|_{a}g
	\text{ .}
\end{align*}
%
Si $f:\,U\rightarrow\bb{R}$ es diferenciable en $a$, entonces
$\gancho{v}|_{a}f$ es lineal en $v_{a}$, es decir,
\begin{align*}
	\left(\left.\gancho{v}\right|_{a}+\lambda
		\left.\gancho{w}\right|_{a}\right)f & \,=\,
		\left.\gancho{(v+\lambda w)}\right|_{a}f
	\text{ .}
\end{align*}
%
En particular, si $v_{a}=v^{i}e_{i}|_{a}$, entonces, para toda $f$
diferenciable en $a$, definiendo $\gancho{e_{i}}|_{a}=\gancho{x^{i}}|_{a}$,
\begin{align*}
	\left.\gancho{v}\right|_{a}f & \,=\,
		v^{i}\left.\gancho{x^{i}}\right|_{a}f
	\text{ .}
\end{align*}
%

La existencia de particiones de la unidad en $\bb{R}^{d}$ permite simlificar
las definiciones, permitiendo asumir que las funciones $f$ diferenciables
cerca de un punto $a$ est\'{a}n definidas en todo el espacio y no s\'{o}lo
en un entorno del punto. De todas maneras, supongamos, para mantener
cierta generalidad, que $\cal{O}_{a}$ es un \emph{anillo de funciones %
regulares} en $a$. En general, $\cal{O}_{a}$ ser\'{a} uno de los siguientes:
\textit{(i)} $C^{\infty}(a)$, donde
\begin{align*}
	C^{\infty}(a) & \,=\,\left\lbrace (U,f)\,:\, a\in U\subset\bb{R}^{d}
		\text{ abierto},\,f\in C^{\infty}(U)\right\rbrace
	\text{ ,}
\end{align*}
%
\textit{(ii)} el espacio de pares $(U,f)$, donde $a\in U$ es un abierto de
$\bb{R}^{d}$, $f:\,U\rightarrow\bb{R}$ es diferenciable en $U$ y las derivadas
parciales $\frac{\partial f}{\partial x^{i}}:\,U\rightarrow\bb{R}$ son
diferenciables en $a$, es decir, el espacio de funciones dos veces
diferenciables en $a$, \textit{(iii)} el anillo de g\'{e}rmenes de funciones
\emph{regulares} en $a$ (para alguna interpretaci\'{o}n de ``regular''), o
bien \textit{(iv)} $C^{\infty}(U)$ o $C^{\infty}(M)$. Una
\emph{derivaci\'{o}n en $a$} en el anillo $\cal{O}_{a}$ es una
aplicaci\'{o}n $\bb{R}$-lineal $w:\,\cal{O}_{a}\rightarrow\bb{R}$ que
verifica la regla de Leibnitz: si $f,g\in\cal{O}_{a}$,
\begin{align*}
	w(fg) & \,=\,w(f)\cdot g(a)+f(a)\cdot w(g)
	\text{ .}
\end{align*}
%
El conjunto de todas las derivaciones en $a$ del anillo ($\bb{R}$-\'{a}lgebra)
$\cal{O}_{a}$ se denomina \emph{espacio de derivaciones en $a$} y lo
denotaremos $\derivaciones{\cal{O}_{a}}$, o bien
$\derivaciones[a]{\cal{O}_{a}}$, cuando sea necesario aclarar el punto,
particularmente, cuando el \'{a}lgebra $\cal{O}_{a}$ no hace referncia al
punto $a$ en donde est\'{a}n \emph{basadas} las derivaciones. Este ser\'{a}
el caso, por ejemplo, de $\cal{O}_{a}=C^{\infty}(U)$, cuando $U$
es un abierto que contiene al punto $a$.

\begin{obsDerivaciones}\label{obs:derivaciones}
	Si $w$ es una derivaci\'{o}n en $a$, entonces $w(1)=0$ y
	$w(fg)=0$, si $f(a)=g(a)=0$. Es decir, las derivaciones se anulan
	en las funciones constantes y en el cuadrado del ideal
	$\{f\in\cal{O}_{a}\,:\,f(a)=0\}$.
\end{obsDerivaciones}

Ya vimos que todo elemento del espacio tangente geom\'{e}trico
$v_{a}\in\bb{R}^{d}_{a}$ determina una derivaci\'{o}n, $\gancho{v}|_{a}f$.
Adem\'{a}s, vimos que, si nos restringimos a evaluar derivaciones en el
espacio de funciones diferenciables en $a$, la aplicaci\'{o}n
\begin{align*}
	\Phi & \,:\,v_{a}\mapsto\left(
		\left.\gancho{v}\right|_{a}:\,f\mapsto
		\left.\gancho{v}\right|_{a}f\right)
\end{align*}
%
es lineal. En particular, $\gancho{v}|_{a}=v^{i}\gancho{x^{i}}|_{a}$,
si $v_{a}=v^{i}e_{i}|_{a}$, donde $\gancho{x^{i}}|_{a}$ es la derivaci\'{o}n
correspondiente al vector can\'{o}nico $e_{i}|_{a}$, es decir, la derivada
parcial $i$-\'{e}sima.

Ahora bien, las funciones $x^{i}:\,\bb{R}^{d}\rightarrow\bb{R}$ dada por
$x^{i}(a)=a^{i}$ es diferenciable en todo punto $a$. Entonces tomando
la derivada direccional respecto de $v_{a}$, se deduce que
\begin{align*}
	\left.\gancho{v}\right|_{a}x^{i} & \,=\,
		v^{j}\left.\gancho{x^{j}}\right|_{a}x^{i} \,=\, v^{i}
	\text{ .}
\end{align*}
%
De esta igualdad podemos concluir que la aplicaci\'{o}n $\Phi$ es inyectiva.

La inyectividad de $\Phi$ se puede demostrar de otra manera, sin hacer
referencia a una base can\'{o}nica. Sea $v_{a}\in\bb{R}^{d}_{a}$. Entonces
$v_{a}$ se corresponde con el par $(a,v)$ para un cierto vector
$v\in\bb{R}^{d}$. Sea $\varphi:\,\bb{R}^{d}\rightarrow\bb{R}$ una funci\'{o}n
lineal. Como
\begin{align*}
	\varphi(a+v) -\varphi(a)-\varphi(v) & \,=\,0
\end{align*}
%
por linealidad, se deduce que $\varphi$ es diferenciable. La deriada
direccional de $\varphi$ en $a$ est\'{a} dada por
\begin{align*}
	\left.\gancho{v}\right|_{a}\varphi & \,=\,
		\lim_{t\to 0}\,\frac{\varphi(a+tv)-\varphi(a)}{t} \,=\,
		\varphi(v)
	\text{ .}
\end{align*}
%
En particular, si $\gancho{v}|_{a}\varphi=0$ para toda
$\varphi$ en el dual $\dual{\bb{R}^{d}}$, como $\dual{\bb{R}^{d}}$ separa
puntos en $\bb{R}^{d}$, debe ser $v=0$. Notemos que toda funci\'{o}n lineal
es derivable en todas direcciones, pero es derivable, si y s\'{o}lo si
es continua.

Veamos condiciones suficientes para garantizar la sobreyectividad de $\Phi$.
Sea $f:\,U\rightarrow\bb{R}$ una funci\'{o}n diferenciable en un abierto
$U$ que contiene al punto $a$. Entonces las derivadas parciales
$\frac{\partial f}{\partial x^{i}}:\,U\rightarrow\bb{R}$ est\'{a}n definidas
en $U$. Sabemos que, si $b\in U$, tomando $v=b-a$, vale que
\begin{align*}
	f(a+v) & \,=\,f(a)+\diferencial[a]{f}(v)+r(v)
\end{align*}
%
donde
\begin{align*}
	\diferencial[a]{f}(v) & \,v^{i}\left.
		\frac{\partial f}{\partial x^{i}}\right|_{a}
\end{align*}
%
y $r$ es una funci\'{o}n tal que el cociente $r(w)/|w|$ tiende a cero con
$|w|$. Equivalentemente, si $v\in\bb{R}^{d}$ y $a+v\in U$, la funci\'{o}n
\begin{align*}
	r(v) & \,=\,f(a+v)-f(a)-v^{i}
		\left.\frac{\partial f}{\partial x^{i}}\right|_{a}
\end{align*}
%
verifica que $r(v)/|v|$ tiende a cero con $v$.

Sea, para cada $i$, $g_{i}=\frac{\partial f}{\partial x^{i}}$. Por el
Teorema fundamental del c\'{a}lculo, e integrando por partes,
\begin{align*}
	f(a+v)-f(a) & \,=\,\int_{0}^{1}\,v^{i}g_{i}(a+tv)\,dt \,=\,
		v^{i}g_{i}(a+v) -\int_{0}^{1}\,tv^{i}v^{j}
		\left.\frac{\partial g_{i}}{\partial x^{j}}\right|_{a+tv}\,dt\\
	& \,=\,v^{i}g_{i}(a) +v^{i}v^{j}\int_{0}^{1}(1-t)
		\left.\frac{\partial g_{i}}{\partial x^{j}}\right|_{a+tv}\,dt
	\text{ .}
\end{align*}
%
Para que esto se v\'{a}lido, asumimos que cada $g_{i}$ es diferenciable
en ((casi todo punto de) un entorno de) $a$. Si $w$ es una derivaci\'{o}n
en $a$ en $C^{\infty}(a)$, definimos un elemento $v_{a}\in\bb{R}^{d}_{a}$
evaluando $w$ en las funciones coordenadas: para cada $i$, tomamos
$v^{i}=w(x^{i})$ y definimos $v_{a}=v^{i}e_{i}|_{a}$. En particular,
\begin{align*}
	w(x^{i}) & \,=\,v^{i} \,=\,\left.\gancho{v}\right|_{a}(x^{i})
	\text{ .}
\end{align*}
%
Sea $f\in C^{\infty}(a)$. En un entorno de $a$,
\begin{equation}
	\label{eq:taylorordendos}
	\begin{aligned}
		f(x) & \,=\, f(a) + (x^{k}-a^{k})
			\left.\frac{\partial f}{\partial x^{k}}\right|_{a} \\
		& \quad + (x^{i}-a^{i})(x^{j}-a^{j})\int_{0}^{1}\,(1-t)\left.
			\frac{\partial^{2}f}{\partial x^{j}\partial x^{i}}
			\right|_{a+t(x-a)}\,dt
	\text{ .}
	\end{aligned}
\end{equation}
%
Entonces, evaluando $w$ en $f$,
\begin{align*}
	w(f) & \,=\,w(x^{k})\left.\frac{\partial f}{\partial x^{k}}\right|_{a}
		\,=\,v^{k}\left.\frac{\partial f}{\partial x^{k}}\right|_{a}
		\,=\,\left.\gancho{v}\right|_{a}(f)
	\text{ .}
\end{align*}
%
As\'{\i}, queda demostrado que $\Phi_{a}:\,\bb{R}^{d}_{a}\rightarrow%
\derivaciones{C^{\infty}(a)}$ es un isomorfismo $\bb{R}$-lineal. Definimos
el \emph{espacio tangente a $\bb{R}^{d}$ en $a$ en sentido de derivaciones}
como el espacio $\derivaciones{C^{\infty}(a)}$. La existencia del
isomorfismo $\Phi_{a}$ justifica en cierta medida este nombre.

\begin{obsSobreDerivaciones}\label{obs:sobrederivaciones}
	El argumento anterior usando el desarrollo de Taylor con su
	expresi\'{o}n integral para el resto sigue siendo v\'{a}lido
	si asumimos que las funciones $f$ son dos veces diferenciables
	en $a$ (o $C^{2}$ en un entorno de $a$, o que las $g_{i}$ son
	diferenciables en casi todo punto de un entorno compacto de $a$\dots).
	Dicho de otra manera, si asumimos que la derivaci\'{o}n $w$ est\'{a}
	definida en un anillo m\'{a}s grande, por ejemplo, para toda $f$ dos
	veces diferenciable en (un entorno de) $a$, entonces la
	derivaci\'{o}n $\Phi_{a}(v_{a})$ coincide con $w$ al ser evaluadas
	en cualquier funci\'{o}n para la cual el desarrollo de Taylor
	\eqref{eq:taylorordendos} sea v\'{a}lido. Es decir, $\Phi_{a}$ sigue
	siendo sobreyectiva, si su codominio fuese alg\'{u}n otro espacio de
	derivaciones. Pero es casi inmediato que, si $\cal{O}_{a}\supset%
	\cal{O}'_{a}$, entonces $\derivaciones{\cal{O}_{a}}\subset%
	\derivaciones{\cal{O}'_{a}}$, es decir, condiciones menos
	restrictivas sobre el par\'{a}metro $f$ resulta en condiciones
	m\'{a}s restrictivas para $w$.
\end{obsSobreDerivaciones}

De ahora en adelante, denotaremos por $v_{a}$ tanto al elemento en el espacio
tangente geom\'{e}trico $\bb{R}^{d}_{a}$ como a la derivaci\'{o}n
correspondiente $f\mapsto\gancho{v}|_{a}f$.

\begin{obsDerivacionesLocalGlobal}\label{obs:derivacioneslocalglobal}
	El espacio de funciones suaves definidas en alg\'{u}n entorno de
	un punto $a\in\bb{R}^{d}$, $C^{\infty}(a)$ contiene al espacio de
	funciones suaves definidas en todo el espacio, es decir,
	hay una inclusi\'{o}n
	\begin{align*}
		C^{\infty}(\bb{R}^{d}) & \,\hookrightarrow C^{\infty}(a)
	\end{align*}
	%
	dada por $f\mapsto (\bb{R}^{d},f)$.

	Rec\'{\i}procamente, a los fines de estudiar derivaciones, podemos
	pensar que $C^{\infty}(a)$ est\'{a} incluido en
	$C^{\infty}(\bb{R}^{d})$. Espec\'{\i}ficamente, dada
	$(U,f)\in C^{\infty}(a)$, existe un entorno $V$ de $a$ tal que
	$\clos{V}\subset U$. Asociada al cubrimiento
	$\{U,\setcomp{\clos{V}}\}$ de $\bb{R}^{d}$, existe una partici\'{o}n
	suave de la unidad $\{\psi_{0},\psi_{1}\}$ tal que
	$\soporte{\psi_{0}}\subset U$ y $\soporte{\psi_{1}}\subset%
	\setcomp{\clos{V}}$. Definimos una funci\'{o}n $\tilde{f}$ en
	$\bb{R}^{d}$ por
	\begin{align*}
		\tilde{f}(x) & \,=\,
			\begin{cases}
				f(x)\psi(x) & \quad\text{si }x\in U\text{ ,}\\
				0 & \quad\text{si }x\not\in\clos{V}\text{ .}
			\end{cases}
	\end{align*}
	%
	Esta funci\'{o}n es suave y definida en todo $\bb{R}^{d}$. Adem\'{a}s,
	$\tilde{f}(x)=f(x)$ para todo $x\in V$. Elegir de esta manera una
	partici\'{o}n de la unidad y definir luego una extensi\'{o}n de $f$,
	\emph{define} una aplicaci\'{o}n $f\mapsto\tilde{f}$ de
	$C^{\infty}(a)$ en $C^{\infty}(\bb{R}^{d})$. Esta aplicaci\'{o}n
	est\'{a} lejos de ser inyectiva, con lo cual no da una
	inclusi\'{o}n de $C^{\infty}(a)$ en $C^{\infty}(\bb{R}^{d})$. De todas
	maneras, esta observaci\'{o}n muestra que hay, para cada elemento
	de $C^{\infty}(a)$, al menos una manera de extenderlo a un objeto en
	$C^{\infty}(\bb{R}^{d})$. Este argumento no es otra cosa que
	la proposici\'{o}n \ref{thm:extenderfuncionessuaves} en el caso en
	que el conjunto $A$ es un punto, o, m\'{a}s en general, un compacto
	en $M=\bb{R}^{d}$.
\end{obsDerivacionesLocalGlobal}

\subsection{El espacio tangente a una variedad}
El espacio tangente a $\bb{R}^{d}$ en un punto $a$, ya sea el tangente
geom\'{e}trico $\bb{R}^{d}_{a}$ o el tangente en sentido de derivaciones
$\derivaciones{C^{\infty}(a)}$, son nociones locales. Por lo tanto, haciendo
uso de las cartas compatibles de una variedad diferencial, queda m\'{a}s o
menos clara una manera de definir el espacio tangente a una variedad.
Aun as\'{\i}, la noci\'{o}n de derivaci\'{o}n es lo suficientemente abstracta
como para permitir generalizarla al contexto de variedades y dar una
definici\'{o}n \emph{intr\'{\i}nseca} del tangente, sin necesidad de un
argumento del estilo de tomar cartas.

Sea $M$ una variedad diferencial y sea $a\in M$ un punto arbitrario.
Decimos que una transformaci\'{o}n lineal $v:\,C^{\infty}(M)\rightarrow\bb{R}$
es una \emph{derivaci\'{o}n en $a$}, si satisface la regla de Leibnitz:
para todo par $f,g\in C^{\infty}(M)$, se cumple que $v(fg)=v(f)g(a)+f(a)v(g)$.
Al igual que en $\bb{R}^{d}$, toda derivaci\'{o}n en $a$ de $C^{\infty}(M)$
se anula en las constantes y, si $f(a)=g(a)=0$, entonces $v(fg)=0$ (c.~f.
la observaci\'{o}n \ref{obs:derivaciones}). Definimos el \emph{espacio %
tangente a $M$ en $a$ en sentido de derivaciones} como el espacio de
derivaciones en $a$ de $C^{\infty}(M)$ y lo denotamos
$\derivaciones[a]{C^{\infty}(M)}$, o bien $\derivaciones[a]{M}$.

Sea ahora $F:\,M\rightarrow N$ una funci\'{o}n suave. Asociada a $F$ hay una
transformaci\'{o}n lineal (notar la similitud con la \emph{definici\'{o}n}
de diferenciabilidad)
\begin{align*}
	\diferencial[a]{F} & \,:\,\derivaciones[a]{M}\,\rightarrow\,
		\derivaciones[F(a)]{N}
\end{align*}
%
dada por $\diferencial[a]{F}(v):\,f\mapsto v(f\circ F)$ para todo elemento
$f\in C^{\infty}(N)$. La aplicaci\'{o}n $\diferencial[a]{F}(v)$ es,
efectivamente una derivaci\'{o}n en $N$ en el punto $F(a)$. La
trasformaci\'{o}n lineal $\diferencial[a]{F}$ se llama el \emph{diferencial %
de $F$ en $a$}.

\begin{propoElDiferencial}\label{thm:eldiferencial}
	Si $F:\,M\rightarrow N$ es una transformaci\'{o}n suave, entonces
	$\diferencial[a]{F}:\,\derivaciones[a]{M}\rightarrow%
	\derivaciones[F(a)]{N}$ es lineal. Si $G:\,N\rightarrow P$ es
	otra funci\'{o}n suave, entonces
	\begin{align*}
		\diferencial[a]{(G\circ F)} & \,=\,
			\diferencial[F(a)]{G}\circ\diferencial[a]{F}
		\text{ .}
	\end{align*}
	%
	Para toda variedad diferencial $M$ y todo punto $a\in M$,
	\begin{align*}
		\diferencial[a]{(\id[M])} & \,=\,
			\id[{\derivaciones[a]{M}}]
		\text{ .}
	\end{align*}
	%
\end{propoElDiferencial}

El siguiente resultado ser\'{a} fundamental para poder calcular la
dimensi\'{o}n del espacio tangente a una variedad.

\begin{lemaDerivacionesSonLocales}\label{thm:derivacionessonlocales}
	Sea $M$ una variedad diferencial, sea $a\in M$ y sean
	$f,g\in C^{\infty}(M)$ funciones suaves. Si existe un
	abierto $U\subset M$, entorno de $a$, en donde $f$ y $g$ coinciden,
	entonces $vf=vg$ para toda $v\in\derivaciones[a]{C^{\infty}(M)}$.
\end{lemaDerivacionesSonLocales}

\begin{proof}
	Por linealidad de las derivaciones es suficiente ver que toda
	funci\'{o}n suave $f\in C^{\infty}(M)$ que se anula en un entorno
	$U$ de $a$ eval\'{u}a a $0$. Sea entonces $f$ una funci\'{o}n
	suave tal que $\soporte{f}\subset M\setmin\{a\}$. Sea $\psi$
	una funci\'{o}n suave tal que $0\leq\psi\leq 1$ en $M$,
	$\psi=1$ en $\soporte{f}$ y $\soporte{\psi}\subset M\setmin\{a\}$.
	Tal funci\'{o}n existe por la proposici\'{o}n \ref{thm:haychichones}.
	Notemos que $\psi(x)f(x)=f(x)$ para todo punto $x\in M$. Es decir,
	como funciones en $M$, $\psi\cdot f$ y $f$ son iguales. As\'{\i},
	dada $v:\,C^{\infty}(M)\rightarrow\bb{R}$, vale que
	$v(f)=v(\psi f)$. Si, adem\'{a}s, $v$ es una derivaci\'{o}n en $a$,
	$v(\psi f)=v(\psi)f(a)+\psi(a)v(f)=0$, ya que $f(a)=\psi(a)=0$.
	En definitiva, $v(f)=0$ para toda derivaci\'{o}n
	$v\in\derivaciones[a]{C^{\infty}(M)}$.
\end{proof}

\begin{obsDerivacionesIsomorfas}\label{obs:derivacionesisomorfas}
	Siguiendo con el comentario de la observaci\'{o}n
	\ref{obs:derivacioneslocalglobal}, demostraremos que los espacios
	$\derivaciones{C^{\infty}(a)}$ y
	$\derivaciones[a]{C^{\infty}(\bb{R}^{d})}$ son (naturalmente)
	isomorfos, es decir que la inclusi\'{o}n natural
	$C^{\infty}(\bb{R}^{d})\hookrightarrow C^{\infty}(a)$ determina un
	isomorfismo a nivel de los espacios de derivaciones en $a$, aunque
	no haya una manera clara o can\'{o}nica de incluir $C^{\infty}(a)$
	en $C^{\infty}(\bb{R}^{d})$.

	Una manera de ver que $\derivaciones[a]{C^{\infty}(\bb{R}^{d})}\simeq%
	\derivaciones{C^{\infty}(a)}$, es notar que, al igual que
	existe un isomorfismo $\Phi_{a}:\,\bb{R}^{d}_{a}\rightarrow%
	\derivaciones{C^{\infty}(a)}$, existe un isomorfismo an\'{a}logo
	$\tilde{\Phi}_{a}:\,\bb{R}^{d}_{a}\rightarrow%
	\derivaciones[a]{C^{\infty}(\bb{R}^{d})}$. Otra manera de demostrar
	que son espacios isomorfos es usar el lema
	\ref{thm:derivacionessonlocales}. Este argumento nos muestra c\'{o}mo
	en muchas ocasiones vamos a poder reemplazar el espacio de funciones
	regulares en un punto $a$, $C^{\infty}(a)$, por el espacio de
	funciones regulares en toda la variedad $C^{\infty}(\bb{R}^{d})$.
	Como este argumento est\'{a} dado exclusivamente en t\'{e}rminos de
	derivaciones, es igualmente v\'{a}lido reemplazando $\bb{R}^{d}$
	por una variedad diferencial arbitraria $M$.

	Sea $v\in\derivaciones{C^{\infty}(a)}$ y sea
	$f\in C^{\infty}(\bb{R}^{d})$ una funci\'{o}n suave. La inclusi\'{o}n
	$f\mapsto (\bb{R}^{d},f)$ nos permite definir un elemento
	$\tilde{v}\in\derivaciones[a]{C^{\infty}(\bb{R}^{d})}$ a partir de $v$:
	\begin{align*}
		\tilde{v}(f) & \,=\,v((\bb{R}^{d},f)).
	\end{align*}
	%
	Supongamos que $\tilde{v}$ es la derivaci\'{o}n cero y sea
	$(U,f)\in C^{\infty}(a)$. Sea $\tilde{f}:\,\bb{R}^{d}\rightarrow\bb{R}$
	una funci\'{o}n suave que coincide con $f$ en (la clausura de)
	cierto entorno $V$ de $a$ contenido en $U$ (c.~f. la observaci\'{o}n
	\ref{obs:derivacioneslocalglobal}). Sin importar cu\'{a}l sea
	esta extensi\'{o}n, como $\tilde{f}$ y $f$ coinciden en un entorno
	de $a$, por \ref{thm:derivacionessonlocales},
	\begin{align*}
		0 & \,=\,\tilde{v}\tilde{f} \,\equiv\,v(\bb{R}^{d},\tilde{f})
			\,=\,v(U,f)
		\text{ .}
	\end{align*}
	%
	As\'{\i}, se ve que $v$ ten\'{\i}a que ser la derivaci\'{o}n cero
	en $C^{\infty}(a)$. Es decir, $v\mapsto\tilde{v}$ es lineal e
	inyectiva.

	Sea, ahora, $w\in\derivaciones[a]{C^{\infty}(\bb{R}^{d})}$. Sea
	$v:\,C^{\infty}(a)\rightarrow\bb{R}$ la funci\'{o}n
	\begin{align*}
		v(U,f) & \,=\,w(\tilde{f})
		\text{ ,}
	\end{align*}
	%
	donde $\tilde{f}$ es una (alguna) funci\'{o}n suave, definida
	globalmente y que coincide con $f$ en un entorno de $a$ contenido en
	$U$. Nuevamente, por el lema \ref{thm:derivacionessonlocales},
	no importa cu\'{a}l sea la elecci\'{o}n $\tilde{f}$, el resultado
	es el mismo. Entonces $v$ est\'{a} bien definida y es una
	derivaci\'{o}n en $C^{\infty}(a)$. Sea
	$\tilde{v}\in\derivaciones[a]{C^{\infty}(\bb{R}^{d})}$ la
	derivaci\'{o}n determinada por $v$ y sea $g\in C^{\infty}(\bb{R}^{d})$.
	Evaluando,
	\begin{align*}
		\tilde{v}(g) & \,\equiv\,v(\bb{R}^{d},g) \,\equiv\,w(g)
		\text{ .}
	\end{align*}
	%
	Entonces $\tilde{v}$ y $w$, como funciones $C^{\infty}(\bb{R}^{d})%
	\rightarrow\bb{R}$, son iguales. En definitiva, $v\mapsto\tilde{v}$
	es un isomorfismo $\bb{R}$-lineal de $\derivaciones{C^{\infty}(a)}$
	en $\derivaciones[a]{C^{\infty}(\bb{R}^{d})}$.
\end{obsDerivacionesIsomorfas}

Sea $M$ una variedad diferencial y sea $a\in M$. Denotemos por $C^{\infty}(a)$
al espacio de funciones suaves definidas en un entorno de $a$, es decir,
$C^{infty}(a)$ es el conjunto de pares $(U,f)$ donde $U\subset M$ es
abierto y contiene a $a$ y $f\in C^{\infty}(U)$.

\begin{propoDerivacionesIsomorfasI}\label{thm:derivacionesisomorfasi}
	Los espacios de derivaciones $\derivaciones[a]{C^{\infty}(M)}$ y
	$\derivaciones{C^{\infty}(a)}$ son isomorfos.
\end{propoDerivacionesIsomorfasI}

De la misma manera en que pudimos identificar el espacio de derivaciones en
$C^{\infty}(a)$ y el espacio de derivaciones en $a$ en $C^{\infty}(M)$,
podemos identificar, dado un abierto $U\subset M$ tal que $a\in U$, los
espacios $\derivaciones[a]{M}$ y $\derivaciones[a]{U}$. La
identificaci\'{o}n, en este caso, est\'{a} dada por el isomorfismo
$\diferencial[a]{i}$, donde $i:\,U\hookrightarrow M$ es la inclusi\'{o}n.

\begin{propoDerivacionesIsomorfasII}\label{thm:derivacionesisomorfasii}
	Sea $M$ una variedad diferencial, sea $U\subset M$ un abierto y
	sea $i:\,U\hookrightarrow M$ la inclusi\'{o}n. Si $a\in U$,
	el diferencial $\diferencial[a]{i}:\,\derivaciones[a]{U}\rightarrow%
	\derivaciones[i(a)]{M}$ es un isomorfismo lineal.
\end{propoDerivacionesIsomorfasII}

La demostraci\'{o}n es an\'{a}loga al argumento dado en la obsevaci\'{o}n
\ref{obs:derivacionesisomorfas}. La identificaci\'{o}n
$v\mapsto\tilde{v}$ definida en las derivaciones $v:\,C^{\infty}(a)%
\rightarrow\bb{R}$, est\'{a} dada expl\'{\i}citamente en este caso por la
aplicaci\'{o}n diferencial $\diferencial[a]{i}$.

El isomorfismo $\bb{R}^{d}_{a}\rightarrow\derivaciones[a]{\bb{R}^{d}}$
dado por $e_{i}|_{a}\mapsto\gancho{x^{i}}|_{a}$, donde
$\{e_{1}|_{a},\,\dots,\,e_{d}|_{a}\}$ es la base can\'{o}nica de
$\bb{R}^{d}_{a}$, muestra que $\derivaciones[a]{\bb{R}^{d}}$ es una espacio
de dimensi\'{o}n finita y que su dimensi\'{o}n es $d$. Por la
proposici\'{o}n \ref{thm:derivacionesisomorfasii},
$\derivaciones[a]{U}\simeq\derivaciones[a]{\bb{R}^{d}}$ para todo abierto
$U\subset\bb{R}^{d}$ que contenga a $a$. En una variedad $M$, dada una
carta compatible $(U,\varphi)$ en un punto $a$, la funci\'{o}n
$\varphi:\,U\rightarrow\varphi(U)$ es un difeomorfismo. En particular,
por \ref{thm:eldiferencial}, el diferencial $\diferencial[a]{\varphi}:\,%
\derivaciones[a]{U}\rightarrow\derivaciones[\varphi(a)]{\varphi(U)}$ es un
isomorfismo determinado por la (elecci\'{o}n de) carta $(U,\varphi)$ en $a$.
Esto permite deducir que
\begin{align*}
	\derivaciones[a]{M} & \,\simeq\,\derivaciones[a]{U}\,\simeq\,
		\derivaciones[\varphi(a)]{\varphi(U)}
\end{align*}
%
de manera can\'{o}nica. En particular, $\dim\,\derivaciones[a]{M}=\dim\,M$.

Esta \'{u}ltima afirmaci\'{o}n no es cierta en el caso de variedades con
borde, mejor dicho, en puntos del borde de una variedad. Para determinar la
dimensi\'{o}n de los espacios tangentes a variedades con borde en puntos
del borde ser\'{a} suficiente, por el mismo argumento del p\'{a}rrafo anterior,
determinar la dimensi\'{o}n del tangente al semiespacio $\hemi[d]$ en
un punto del borde $\borde{\hemi[d]}=\{x^{d}=0\}$.
