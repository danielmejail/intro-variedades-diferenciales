



\begin{ejemplo}%{ejemGrafs}
Sea $U\subset\bb{R}^{n}$ un subconjunto abierto y sea
$f:\,U\rightarrow\bb{R}^{k}$ una funci\'{o}n \emph{continua}. El
\emph{gr\'{a}fico de $f$} es el subconjunto de $\bb{R}^{n}\times\bb{R}^{k}$
definido por
\begin{align*}
	\graf{f} & \,=\,\big\lbrace (x,y)\in\bb{R}^{n}\times\bb{R}^{k}
				\,:\,x\in U,\,y=f(x)\big\rbrace
	\text{ .}
\end{align*}
%
A este subconjunto se le da la topolog\'{\i}a de subespacio del producto.
Sea $\pi_{1}:\,\bb{R}^{n}\times\bb{R}^{k}\rightarrow\bb{R}^{n}$ la
proyecci\'{o}n en el primer factor y sea
$\varphi:\,\graf{f}\rightarrow U$ la restricci\'{o}n de $\pi_{1}$ al
gr\'{a}fico de $f$. La aplicaci\'{o}n $\varphi$ es continua, siendo
la restricci\'{o}n de una funci\'{o}n continua a un subespacio; su imagen
es el abierto $U$ de $\bb{R}^{n}$ y $\varphi^{-1}(x)=(x,f(x))$ es una inversa
para $\varphi$ definida en $U$. En definitiva, $\graf{f}$ es homeomorfo
a $U$ v\'{\i}a $\varphi$ y el par $(\graf{f},\varphi)$ es una carta global
para $\graf{f}$ que hace del mismo un espacio localmente euclideo de
dimensi\'{o}n $n$. Como $\graf{f}$ es $T_{2}$ y $N_{2}$, por ser
subespacio de $\bb{R}^{n}\times\bb{R}^{k}$, resulta ser una variedad
topol\'{o}gica de dimensi\'{o}n $n$, tambi\'{e}n.
\end{ejemplo}

\begin{ejemplo}%{ejemEsfera}
Sea $n\geq 0$ y sea $\esfera{n}$ la esfera de dimensi\'{o}n $n$ en
$\bb{R}^{n+1}$. Por ser subespacio de$\bb{R}^{n+1}$, es un espacio
$T_{2}$ y $N_{2}$. Para cada $i\in[\![1,n+1]\!]$ sea $U_{i}^{+}$ el
subconjunto de $\bb{R}^{n+1}$ definido por
\begin{align*}
	U_{i}^{+} & \,=\,\big\lbrace (\lista*{x}{n+1})\,:\,x^{i}>0
				\big\rbrace
	\text{ .}
\end{align*}
%
De manera an\'{a}loga, se define $U_{i}^{-}$ como el subconjunto en donde
la coordenada $x^{i}$ es negativa (estrictamente). Si se define una
funci\'{o}n $f:\,\bola{1}{0}\rightarrow\bb{R}$ por
\begin{align*}
	f(u) & \,=\,\sqrt{1-|u|^{2}}\text{ ,}
\end{align*}
%
entonces $f$ es continua y, para cada $i$, el subconjunto
$U_{i}^{+}\cap\esfera{n}$ de la esfera es igual al gr\'{a}fico de la
funci\'{o}n
\begin{align*}
	x^{i} & \,=\,f(x^{1},\,\dots,\,\widehat{x^{i}},\,\dots,\,x^{n+1})
\end{align*}
%
Similarmente, $U_{i}^{-}\cap\esfera{n}$ es el gr\'{a}fico de
$x^{i}=-f(x^{1},\,\dots,\,\widehat{x^{i}},\,\dots,\,x^{n+1})$. De esta
manera, se ve cada abierto $U_{i}^{\pm}\cap\esfera{n}$ de la esfera es
localmente euclideo de dimensi\'{o}n $n$, por ser gr\'{a}ficos de funciones
continuas, y que las coordenadas
$\varphi_{i}^{\pm}:\,U_{i}^{\pm}\cap\esfera\rightarrow\bola{1}{0}$ dadas por
\begin{align*}
	\varphi_{i}^{\pm}(\lista*{x}{n+1}) & \,=\,
		(x^{1},\,\dots,\,\widehat{x^{i}},\,\dots,\,x^{n+1})
\end{align*}
%
son las coordenadas correspondientes. Dado que los dominios de estas cartas
$(U_{i}^{\pm}\cap\esfera{n},\varphi_{i}^{\pm})$ cubren a $\esfera{n}$, se
deduce que $\esfera{n}$ es una variedad topol\'{o}gica de dimensi\'{o}n
$n$.

	Otro juego de cartas que tambi\'{e}n aparece con $\esfera{n}$ es
	el formado por las proyecciones estereogr\'{a}ficas. Sea $N$ el
	punto $(0,\,\dots,\,0,\,1)\in\esfera{n}\subset\bb{R}^{n+1}$ y
	sea $S=(0,\,\dots,\,0,\,-1)$ su ant\'{\i}poda. Sea
	$\sigma:\,\esfera{n}\setmin\{N\}\rightarrow\bb{R}^{n}$ la
	funci\'{o}n definida por
	\begin{align*}
		\sigma(\lista*{x}{n+1}) & \,=\,
			\frac{(\lista*{x}{n})}{1-x^{n+1}}
	\end{align*}
	%
	y sea $\tilde{\sigma}:\,\esfera{n}\setmin\{S\}\rightarrow\bb{R}^{n}$
	la funci\'{o}n
	\begin{align*}
		\tilde{\sigma}(x) & \,=\,-\sigma(-x) \,=\,
			\frac{(\lista*{x}{n})}{1+x^{n+1}}
		\text{ .}
	\end{align*}
	%
	Llamemos $H=\{x^{n+1}=0\}$. Entonces $\sigma(x)=u$, donde
	$(u,0)$ es el punto en que la recta que pasa por $N$ y por $x$
	interseca a $H$. De manera an\'{a}loga, $\tilde{\sigma}(x)=u$, donde
	$(u,0)$ es el punto en donde la recta que pasa por $S$ y por $x$
	interseca a $H$. La funci\'{o}n $\sigma$ es invertible con inversa
	\begin{align*}
		\sigma^{-1}(\lista*{u}{n}) & \,=\,
			\frac{(\lista*{2u}{n},\,|u|^{2}-1)}{|u|^{2}+1}
		\text{ .}
	\end{align*}
	%
	La composici\'{o}n $\tilde{\sigma}\circ\sigma^{-1}$ en un punto
	$u=(\lista*{u}{n})$ es igual a
	\begin{align*}
		\tilde{\sigma}(\sigma^{-1}(u)) & \,=\,
			-\sigma(-\sigma^{-1}(u)) \\
		&\,=\,\frac{(\lista*{2u}{n})}{|u|^{2}+1}\cdot
			\frac{1}{1+(\frac{|u|^{2}-1}{|u|^{2}+1})} \\
		& \,=\, \frac{1}{|u|^{2}}u
		\text{ .}
	\end{align*}
	%
\end{ejemplo}

\begin{ejemplo}%{ejemProyectivos}
El espacio proyectivo real $\proyectivo{\bb{R}}{n}$ es una variedad de
dimensi\'{o}n $n$. Una realizaci\'{o}n posible de este espacio es en tanto
el conjunto de subespacios vectoriales de dimensi\'{o}n $1$
(subvariedades lineales de dimensi\'{o}n $1$) en $\bb{R}^{n+1}$. La
topolog\'{\i}a de este espacio es la topolog\'{\i}a cociente determinada
por la aplicaci\'{o}n sobreyectiva
$\pi:\,\bb{R}^{n+1}\setmin\{0\}\rightarrow\proyectivo{\bb{R}}{n}$ que a un
punto $x\not =0$ le asigna el subespacio que genera, denotado $[x]$.

Para cada $i\in[\![1,n+1]\!]$, sea $\widetilde{U}_{i}$ el subconjunto de
$\bb{R}^{n+1}\setmin\{0\}$ donde la coordenada $x^{i}$ es no nula. Sea
$U_{i}=\pi(\widetilde{U}_{i})$ la proyecci\'{o}n correspondiente en el
espacio proyectivo. Como $\widetilde{U}_{i}$ es un abierto saturado, el
subconjunto $U_{i}$ es abierto y la restricci\'{o}n de $\pi$ a
$\widetilde{U}_{i}$ es una aplicaci\'{o}n cociente sobre $U_{i}$.
Sea $\varphi_{i}:\,U_{i}\rightarrow\bb{R}^{n}$ la aplicaci\'{o}n
dada por
\begin{align*}
	\varphi_{i}[x^{1}\,:\cdots:\,x^{n+1}] & \,=\,
		\Big(\frac{x^{1}}{x^{i}},\,\dots,\,\frac{x^{i-1}}{x^{i}},\,
			\frac{x^{i+1}}{x^{i}},\,\dots,\,\frac{x^{n+1}}{x^{i}}
			\Big)
	\text{ .}
\end{align*}
%
Dado que la composici\'{o}n $\varphi\circ\pi$ es continua (de hecho podemos
ir adelantando que es suave entre un abierto de $\bb{R}^{n+1}$ y un abierto
de $\bb{R}^{n}$), la funci\'{o}n $\varphi_{i}$ es continua, por la propiedad
caracter\'{\i}stica del cociente (de $\pi|_{\widetilde{U}_{i}}$). Esta
funci\'{o}n tiene una inversa dada por
\begin{align*}
	\varphi_{i}^{-1}(\lista*{u}{n}) & \,=\,
		[u^{1}\,:\dots:\,u^{i-1}\,:\,1\,:\,u^{i}\,:\dots:\,u^{n}]
	\text{ .}
\end{align*}
%
Esta inversa tambi\'{e}n es continua y $\varphi$ es un homeomorfismo.
La interpretaci\'{o}n geom\'{e}trica de $\varphi[x]=u$ es que el punto
$(u,1)$ es aquel por donde $[x]$ cruza al hiperplano $x^{i}=1$. Dado
que los abiertos $U_{1},\,\dots,\,U_{n+1}$ cubre al espacio proyectivo,
$\proyectivo{\bb{R}}{n}$ es localmente euclideo de dimensi\'{o}n $n$.

Para ver que los espacios proyectivos reales son espacios Hausdorff,
dados dos subespacios lineales de dimensi\'{o}n $1$ distintos, $\xi,\upsilon$,
en $\bb{R}^{n+1}$, existen abiertos \emph{c\'{o}nicos} disjuntos (el punto
$\{0\}$ se omite) cada uno de los cuales contiene a uno de ellos, es decir,
a las l\'{\i}neas. Estos abiertos son saturados respecto de la suryecci\'{o}n
natural $\pi$ y por lo tanto sus im\'{a}genes son abiertos disjuntos de
$\proyectivo{\bb{R}}{n}$ que contienen a cada uno de las clases $\xi$ y
$\upsilon$. Para que quede un poco m\'{a}s claro, podemos restringir $\pi$ a
la esferera $\esfera{n}\subset\bb{R}^{n+1}\setmin\{0\}$. La restricci\'{o}n
sigue siendo sobreyectiva y cociente. Dada una clase
$\xi\in\proyectivo{\bb{R}}{n}$, podemos tomar como representante a cualquiera
de los dos puntos de norma $1$ en $\xi\cap\esfera{n}$. Llamemos $x$ a dicho
punto. De manera similar podr\'{\i}amos tomar el punto $-x$. Si
$y\in\esfera{n}$ es un punto distinto de $x$ y de $-x$, existen abiertos
disjuntos $\widetilde{U}$ y $\widetilde{V}$ que contienen a $x$ y a $y$.
Los abiertos $-\widetilde{U}$ y $-\widetilde{V}$ contienen a los puntos
$-x$ y $-y$ y, tomando abiertos m\'{a}s peque\~{n}os de ser necesario, podemos
suponera que los cuatro abiertos son disjuntos de a pares. De esta manera,
se obtienen abiertos saturados $\widetilde{U}\cup\big(-\widetilde{U}\big)$
y $\widetilde{V}\cup\big(-\widetilde{V}\big)$ que contienen a $x$ y a $y$ y
que, adem\'{a}s, son disjuntos. Proyectando, se obtienen abiertos disjuntos
$U$ y $V$ de $\proyectivo{\bb{R}}{n}$ que contienen a $x$ y a $y$,
respectivamente.

Siendo imagen por una funci\'{o}n continua de un espacio compacto
$\pi(\esfera{n})=\proyectivo{\bb{R}}{n}$, el espacio proyectivo
es compacto para todo $n\geq 0$. Para ver que $\proyectivo{\bb{R}}{n}$ es
$N_{2}$, alcanza con notar que se puede cubrir el espacio con numerables
(finitas) cartas. En definitiva, los espacios $\proyectivo{\bb{R}}{n}$ son
variedades topol\'{o}gicas compactas.

Otra descripci\'{o}n --aunque esencialmente la misma-- del espacio
proyectivo est\'{a} dada por dejar actuar al grupo $\bb{R}^{\times}$
de reales distintos de cero sobre $\bb{R}^{n+1}\setmin\{0\}$ por
multiplicaci\'{o}n por escalares. Las \'{o}rbitas de esta acci\'{o}n son,
precisamente, los subespacios reales de dimensi\'{o}n $1$. Dado que el
cociente $\bb{R}^{\times}\backslash(\bb{R}^{n+1}\setmin\{0\})$ es homeomorfo
a $\{\pm1\}\backslash\esfera{n}$ podemos describir a $\proyectivo{\bb{R}}{n}$
como el cociente de un espacio topol\'{o}gico (de una variedad compacta) por
la acci\'{o}n de un grupo discreto (en particular actuando de manera
propiamente discontinua). Es esto lo que nos permite deducir que los
espacios proyectivos son variedades topol\'{o}gicas y, adem\'{a}s, compactas.
\end{ejemplo}

\begin{ejemplo}%{ejemEspaciosEuclideos}
Para cada entero $n\geq 0$, $\bb{R}^{n}$ es una variedad diferencial de
dimensi\'{o}n $n$. Su estructura suave est\'{a} determinada por el atlas
trivial $\{(\bb{R}^{n},\id[\bb{R}^{n}])\}$ que consta de una \'{u}nica carta.
Esta estructura diferencial en $\bb{R}^{n}$ se denominar\'{a}
\emph{estructura usual} de $\bb{R}^{n}$ o \emph{coordenadas usuales}.
Vale la pena observar que las cartas compatibles con esta estructura
son, precisamente, los pares $(U,\varphi)$ con $U$ es abierto (en la
topolog\'{\i}a usual) y $\varphi:\,U\rightarrow\bb{R}^{n}$ es difrenciable
en el sentido usual, tambi\'{e}n.
\end{ejemplo}

\begin{ejemplo}%{ejemEspaciosVectoriales}
Sea $E$ un espacio vectorial topol\'{o}gico real. Si $E$ es de dimensi\'{o}n
finita $n$, equivalentemente, si $E$ es localmente comacto, entonces $E$ es
isomorfo a $\bb{R}^{n}$ y todo isomorfismo \emph{lineal} de $E$ en
$\bb{R}^{n}$ es un homeomorfismo. Es decir, $E$ tiene una \'{u}nica estructura
de espacio topol\'{o}gico que hace que las operaciones en tanto espacio
vectorial real sean continuas. Todo ismorfismo $E\rightarrow\bb{R}^{n}$
est\'{a} dado por elegir una base: dada una base
$\{\lista{\varepsilon}{n}\}$ de $E$, sea $\varepsilon:\,\bb{R}^{n}\rightarrow E$
la aplicaci\'{o}n
\begin{align*}
	\varepsilon(x) & \,=\,\sum_{i=1}^{n}\,x^{i}\varepsilon_{i}
			\,\equiv\,x^{i}\varepsilon_{i}
	\text{ .}
\end{align*}
%
Seg\'{u}n lo mencionado anteriormente, $\varepsilon$ es un homeomorfismo
y, entonces, el par $(E,\varepsilon^{-1})$ es una carta (continua) para $E$.
Si ahora $\tilde{\varepsilon}$ denota el isomorfismo correspondiente a
otra base $\{\lista{\tilde{\varepsilon}}{n}\}$, entonces existe una matriz
invertible $\left[A_{i}^{j}\right]^{i}_{j}$ tal que
\begin{align*}
	\varepsilon_{i} & \,=\,A_{i}^{j}\tilde{\varepsilon}_{j}
\end{align*}
%
para cada $i$. El cambio de cartas (cambio de coordenadas) correspondiente
est\'{a} dado por $\tilde{\varepsilon}^{-1}\circ\varepsilon(x)=\tilde{x}$,
donde $\tilde{x}=(\lista*{\tilde{x}}{n})$ es el punto de $\bb{R}^{n}$ dado por
\begin{align*}
	\tilde{x}^{j}\tilde{\varepsilon}_{j} & \,=\,x^{i}\varepsilon_{i}
		\,=\,x^{i}A_{i}^{j}\tilde{\varepsilon}_{j}
	\text{ .}
\end{align*}
%
Es decir, para cada $j$, $\tilde{x}^{j}=A_{i}^{j}x^{i}$. En particular,
la aplicaci\'{o}n $x\mapsto\tilde{x}$, el cambio de coordenadas
(definido globalmente), es lineal e invertible y, en particular, un
difeomorfismo. En definitiva, las cartas de la forma $(E,\varepsilon^{-1})$,
donde $\varepsilon:\,\bb{R}^{n}\rightarrow E$ es el isomorfismo determinado
por la base $\{\lista{\varepsilon}{n}\}$ de $E$, son todas (suavemente)
compatibles. La estructura que estas cartas determinan en $E$ se
denominar\'{a} la \emph{estructura usual} en $E$.
\end{ejemplo}

\begin{ejemplo}
	El espacio de matrices $\MM{m\times n,\bb{R}}$ de tama\~{n}o
	$m\times n$ con coeficientes reales es un espacio vectorial de
	dimensi\'{o}n $mn$ y, por lo tanto, una variedad diferencial con
	su estructura usual. El espacio de matrices $\MM{m\times n,\bb{C}}$
	complejas constituye una variedad diferencial de dimensi\'{o}n $2mn$,
	pues su dimensi\'{o}n como espacio vectorial sobre $\bb{R}$ es $2mn$.
	En el caso de matrices cuadradas de tama\~{n}o $n\times n$ usaremos
	la notaci\'{o}n $\MM{n,\cdot}$.
\end{ejemplo}

\begin{ejemplo}
	Sea $U$ un subconjunto abierto de $\bb{R}^{n}$. Entonces $U$ es una
	variedad topol\'{o}gica de dimensi\'{o}n $n$ y la carta
	$(U,\id[U])$ define una estructura suave en $U$. En general, sea
	$M$ es una variedad diferencial y sea $U$ un abierto de $M$.
	Sea $\cal{A}$ la colecci\'{o}n de todas las cartas suaves para
	$M$, es decir, el atlas maximal que define la estructura de $M$
	como variedad diferencial. La colecci\'{o}n
	\begin{align*}
		\cal{A}_{U} & \,=\,\left\lbrace
			(V,\varphi)\in\cal{A}\,:\,V\subset U\right\rbrace
	\end{align*}
	%
	es un atlas de $U$ que es, adem\'{a}s, suavemente compatible.
	Este atlas determina naturalmente una estructura diferencial
	sobre $U$. Los subconjuntos abiertos de una variedad $M$ son
	de manera natural variedades diferenciales que denominaremos
	\emph{subvariedades abierta} de $M$.
\end{ejemplo}

\begin{ejemplo}
	El \emph{grupo general lineal (real)}, denotado $\GL{n,\bb{R}}$
	es el conjunto de matrices invertibles con coeficientes reales.
	Dado que la funci\'{o}n $\det:\,\MM{n,\bb{R}}\rightarrow\bb{R}$
	es continua y que $\GL{n,\bb{R}}$ es el subconjunto de matrices
	con determinante no nulo, $\GL{n,\bb{R}}\subset\MM{n,\bb{R}}$ es
	un subconjunto abierto y, por lo tanto, una variedad diferencial
	de dimensi\'{o}n $\dim(\MM{n,\bb{R}})=n^{2}$.
\end{ejemplo}

\begin{ejemplo}
	Sea $k\geq 0$. De manera an\'{a}loga al ejemplo anterior, en
	$\MM{m\times n,\bb{R}}$, las matrices de rango mayor o igual a $k$
	forman un subconjunto abierto: una matriz tiene rango al menos $k$,
	si el determinante de alguna submatriz es distinto de cero. Por
	continuidad del determinante, existe un entorno de una tal matriz
	que verifica que, para todas las matrices de dicho abierto, el
	determinante de la misma submatriz no se anula. En definitiva, todas
	las matrices del abierto tienen rango al menos $k$. Un caso particular
	de esto es cuando $k$ es m\'{a}ximo, es decir, $k=\min\{m,n\}$.
\end{ejemplo}

\begin{ejemplo}
	Sean $E$ y $F$ dos espacios vectoriales de dimensi\'{o}n finita
	y sea $\lineal{E,F}$ el conjunto de transformaciones lineales
	de $E$ en $F$. Si a $E$ y $F$ se los considera espacios vectoriales
	topol\'{o}gicos reales, entonces $\lineal{E,F}$ coincide con
	el conjunto de transformaciones lineales y continuas. En general,
	como $\lineal{E,F}$ es un espacio vectorial (real) de dimensi\'{o}n
	finita, es una variedad diferencial. Eligiendo bases de $E$ y de $F$,
	se puede representar un elemento $T\in\lineal{E,F}$ como una matriz,
	lo cual determina un isomorfismo
	$\lineal{E,F}\simeq\MM{m\times n,\bb{R}}$.
\end{ejemplo}

\begin{ejemplo}
	Sea $U\subset\bb{R}^{n}$ un abierto y sea $f:\,U\rightarrow\bb{R}^{k}$
	una funci\'{o}n diferenciable. El gr\'{a}fico de $f$ es una variedad
	topol\'{o}gica, dado que $f$ es continua. Dado que, adem\'{a}s, el
	gr\'{a}fico de $f$, $\graf{f}$, se puede cubrir con una \'{u}nica
	carta, $(\graf{f},\varphi)$, donde $\varphi:\,\graf{f}\rightarrow U$
	es la restricci\'{o}n de la proyecci\'{o}n en la primer coordenada,
	el gr\'{a}fico de una funci\'{o}n suave tiene una estructura suave
	de manera can\'{o}nica.
\end{ejemplo}

\begin{ejemplo}
	Las esferas son variedades diferenciales. Usando las cartas
	$(U_{i}^{\pm},\varphi_{i}^{\pm})$ se obtiene un atlas compatible.
	S\'{o}lo hay que verificar que las composiciones
	$\varphi_{i}^{\pm}\circ(\varphi_{j}^{\pm})^{-1}$ sean diferenciables.
	Esta estructura en $\esfera{n}$ se denominar\'{a} la
	\emph{estructura usual} en $\esfera{n}$.

	Las cartas correspondientes a las proyecciones esterogr\'{a}ficas
	son compatibles entre s\'{\i} y, adem\'{a}s, son compatibles
	con la estructura usual de $\esfera{n}$. Veamos primero la
	compatibilidad de $\sigma$ con $\varphi_{i}^{\epsilon}$
	para $i\not =n+1$ y $\epsilon=+\,(>)\text{ o }-\,(<)$. Por un lado,
	en este	caso,
	\begin{align*}
		\big(U_{i}^{\epsilon}\cap\esfera{n}\big)\cap
			\big(\esfera{n}\setmin\{N\}\big) & \,=\,
			U_{i}^{\epsilon}\cap\esfera{n}
		\text{ .}
	\end{align*}
	%
	Entonces, dado que
	\begin{align*}
		\varphi_{i}^{\epsilon}(U_{i}^{\epsilon}\cap\esfera{n}) &
			\,=\,\bola{1}{0}\quad\text{y} \\
		\sigma(U_{i}^{\epsilon}\cap\esfera{n}) & \,=\,
			\left\lbrace u\in\bb{R}^{n}\,:\,u^{i}\epsilon 0
				\right\rbrace
		\text{ ,}
	\end{align*}
	%
	la composici\'{o}n $\varphi_{i}^{\epsilon}\circ\sigma^{-1}:\,%
	\{u^{i}\epsilon 0\}\rightarrow\bola{1}{0}$, dada por
	\begin{align*}
		\varphi_{i}^{\epsilon}\circ\sigma^{-1}(u) & \,=\,
			\frac{(2u^{1},\,\dots,\,\widehat{2u^{i}},\,2u^{n},\,%
				|u|^{2}-1)}{|u|^{2}+1}
		\text{ ,}
	\end{align*}
	%
	es suave (, inyectiva) y su matriz de derivadas parciales es
	no singular en todo punto, como se puede verificar, derivando la
	expresi\'{o}n anterior respecto de cada una de las variables
	$u^{k}$. Esto es suficiente para concluir que la composici\'{o}n
	en el sentido inverso tambi\'{e}n es suave. En todo caso, se
	puede verificar directamente que es suave: como $i\not =n+1$,
	vale que
	\begin{align*}
		\sigma\circ(\varphi_{i}^{\epsilon})^{-1}(v) & \,=\,
			\frac{(v^{1},\,\dots,\,\sqrt{1-|v|^{2}},%
				\,\dots,\,v^{n-1})}{1-v^{n}}
		\text{ .}
	\end{align*}
	%
	Como $|v|<1$, en particular $|v^{n}|<1$ y la composici\'{o}n es
	suave de $B_{1}(0)$ en $\{u^{i}\epsilon 0\}$.

	Si $i=n+1$, $U_{n+1}^{+}\cap\esfera{n}$ contiene a $N$ pero
	$U_{n+1}^{-}\cap\esfera{n}$ no lo contiene. Veamos primero que
	$\varphi_{n+1}^{-}$ es compatible con $\sigma$: la composici\'{o}n
	\begin{align*}
		\varphi_{n+1}^{-}\circ\sigma^{-1}(u) & \,=\,
			\frac{(\lista*{2u}{n})}{|u|^{2}+1}
	\end{align*}
	%
	es suave de $\sigma(U_{n+1}^{-}\cap\esfera{n})=\bola{1}{0}$ en
	$\varphi_{n+1}^{-}(U_{n+1}^{-}\cap\esfera{n})=\bola{1}{0}$ y, en el
	sentido contrario,
	\begin{align*}
		\sigma\circ(\varphi_{n+1}^{-})^{-1}(v) & \,=\,
			\frac{(\lista*{v}{n})}{1+\sqrt{1-|v|^{2}}}
	\end{align*}
	%
	que es suave tambi\'{e}n. De manera similar, se puede comprobar que
	la \emph{otra} proyecci\'{o}n estereogr\'{a}fica $\tilde{\sigma}$
	es compatible con $\varphi_{n+1}^{+}$ y, como $\sigma$ y
	$\tilde{\sigma}$ son compatibles, por tansitividad de la relaci\'{o}n
	de compatibilidad suave, $\sigma$ es compactible con
	$\varphi_{n+1}^{+}$, tambi\'{e}n. Tambi\'{e}n se puede verificar
	directamente: si llamamos $U^{+}$ al abierto
	\begin{align*}
		U^{+} & \,=\,\big(U_{n+1}^{+}\cap\esfera{n}\big)\cap
			\big(\esfera{n}\setmin\{N\}\big)
			\text{ ,}\quad\text{entonces} \\
		\sigma(U^{+}) & \,=\,\left\lbrace u\in\bb{R}^{n}\,:\,
					|u|>1\right\rbrace
			\quad\text{y} \\
		\varphi_{n+1}^{+}(U^{+}) & \,=\,\bola{1}{0}\setmin\{0\}
		\text{ .}
	\end{align*}
	%
	Ahora bien,
	\begin{math}
		\varphi_{n+1}^{+}\circ\sigma^{-1}(u) \,=\,
			\frac{(\lista*{2u}{n})}{|u|^{2}+1}
	\end{math}
	%
	que es suave y
	\begin{math}
		\sigma\circ(\varphi_{n+1}^{+})^{-1}(v) \,=\,
			\frac{(\lista*{v}{n})}{1-\sqrt{1-|v|^{2}}}
	\end{math}
	%
	que tambi\'{e}n es suave en su dominio de definici\'{o}n.
\end{ejemplo}

\begin{ejemplo}
	El espacio proyectivo $\proyectivo{n}{\bb{R}}$ tambi\'{e}n tiene
	estructura de variedad diferencial. Los cambios de coordenadas
	son diferenciables, pues, si $i>j$,
	\begin{align*}
		\varphi_{j}\circ\varphi_{i}^{-1}(\lista*{u}{n}) & \,=\,
		\Big(\frac{u^{1}}{u^{j}},\,\dots,\,\frac{u^{j-1}}{u^{j}},\,
			\frac{u^{j+1}}{u^{j}},\,\dots,\,\frac{u^{i-1}}{u^{j}},\,
			\frac{1}{u^{j}},\,\frac{u^{i}}{u^{j}},\,\dots,\,
				\frac{u^{n}}{u^{j}}\Big) \\
		\varphi_{i}\circ\varphi_{j}^{-1}(\lista*{u}{n}) & \,=\,
		\Big(\frac{u^{1}}{u^{i-1}},\,\dots,\,\frac{u^{j-1}}{u^{i-1}},\,
			\frac{1}{u^{i-1}},\,\frac{u^{j}}{u^{i-1}},\,\dots,\,
			\frac{u^{i-2}}{u^{i-1}},\,\frac{u^{i}}{u^{i-1}},\,
			\dots,\,\frac{u^{n}}{u^{i-1}}\Big)
		\text{ .}
	\end{align*}
	%
\end{ejemplo}

\begin{ejemplo}
	De la misma manera en que se defini\'{o} el espacio proyectivo real,
	se puede definir el espacio proyectivo sobre el cuerpo de n\'{u}meros
	complejos, denotado $\proyectivo{\bb{C}}{n}$, como el conjunto de
	subespacios vectoriales \emph{complejos} de dimensi\'{o}n $1$ en
	$\bb{C}^{n+1}$. Es decir, $\proyectivo{\bb{C}}{n}$ es el
	cociente $(\bb{C}^{n+1}\setmin\{0\})/\sim$ donde dos puntos
	$z=(\lista*{z}{n+1})$ y $w=(\lista*{w}{n+1})$ est\'{a}n relacionados,
	si generan el mismo espacio \emph{sobre $\bb{C}$}, dicho de otra
	manera, si existe un n\'{u}mero complejo no nulo
	$\alpha\in\bb{C}^{\times}$ tal que $z\cdot\alpha=w$.

	En primer lugar, para ver que $\proyectivo{\bb{C}}{n}$ es una variedad
	topol\'{o}gica y para luego ver que se le puede dar una estructura
	diferencial, definimos una correspondencia. Sea
	$\Phi:\,\bb{C}^{n+1}\rightarrow\bb{R}^{2n+2}$ la funci\'{o}n dada por
	\begin{align*}
		\Phi(\lista*{z}{n+1}) & \,=\,
			(x^{1},\,y^{1},\,\dots,\,x^{n},\,y^{n})
		\text{ ,}
	\end{align*}
	%
	donde, para cada $i$, $z^{i}=x^{i}+\sqrt{-1}y^{i}$. Si $w$ es un
	punto de la forma $z\cdot\alpha$ con $\alpha=a+\sqrt{-1}b$ y
	$a,b\in\bb{R}$,
	\begin{align*}
		\Phi(w) & \,=\, \Phi(z^{1}\cdot\alpha,\,\dots,\,
					z^{n+1}\cdot\alpha) \\
		& \,=\,(x^{1}a-y^{1}b,\,x^{1}b+y^{1}a,\,\dots,\,
			x^{n+1}a-y^{n+1}b,\,x^{n+1}b+y^{n+1}a)
		\text{ .}
	\end{align*}
	%
	La acci\'{o}n de multiplicar (a derecha) por un n\'{u}mero complejo
	no nulo $\alpha$ se traduce v\'{\i}a $\Phi$ como el producto a derecha
	por una matriz:
	\begin{align*}
		\Phi(z\cdot\alpha) & \,=\,\Phi(z)\cdot m_{\alpha}
		\text{ ,}
	\end{align*}
	%
	donde $m_{\alpha}$ es la matriz \emph{real} dada por
	\begin{align*}
		m_{\alpha} & \,=\,
			\begin{bmatrix}
				\widehat{\alpha} & & \\
				& \ddots & \\
				& & \widehat{\alpha}
			\end{bmatrix}
		\quad\text{donde}\quad
		\widehat{\alpha} \,=\,
			\begin{bmatrix}
				a & b \\
				-b & a
			\end{bmatrix}
		\text{ .}
	\end{align*}
	%
	Es decir, para $\alpha\in\bb{C}^{\times}$,
	$\widehat{\alpha}\in\MM{2,\bb{R}}$ y $m_{\alpha}\in\MM{2n+2,\bb{R}}$.
	De hecho, como $\det(\widehat{\alpha})=a^{2}+b^{2}>0$,
	$\widehat{\alpha}\in\GL{2,\bb{R}}$ y $m_{\alpha}\in\GL{2n+2,\bb{R}}$.

	Dado $\alpha\in\bb{C}^{\times}$, sea
	$f_{\alpha}:\,\bb{R}^{2n+2}\setmin\{0\}\rightarrow%
		\bb{R}^{2n+2}\setmin\{0\}$ la funci\'{o}n correspondiente a
	multiplicar a derecha por la matriz $m_{\alpha}$.

	V\'{\i}a la correspondencia $\Phi$, la acci\'{o}n de
	$\bb{C}^{\times}$ en $\bb{C}^{n+1}$ est\'{a} dada por la acci\'{o}n
	diagonal del grupo de matrices
	\begin{align*}
		G & \,=\,\left\lbrace
			\begin{bmatrix}a & b\\ -b & a\end{bmatrix}\,:\,
			a^{2}+b^{2}>0\right\rbrace
		\,\simeq\,\bb{R}_{>0}\times\SO{2}
	\end{align*}
	%
	en el producto de $n+1$ copias de $\bb{R}^{2}$. Es decir,
	\begin{align*}
		\Phi((\lista*{z}{n+1})\cdot\alpha) & \,=\,
			((x^{1},y^{1})\cdot\widehat{\alpha},\,\dots,\,
			(x^{n+1},y^{n+1})\cdot\widehat{\alpha}) \\
		& \,=\,\Phi(z)\cdot m_{\alpha}
		\text{ .}
	\end{align*}
	%
	El espacio proyectivo complejo se obtiene como el espacio de
	\'{o}rbitas de $\bb{C}^{\times}$ actuando en $\bb{C}^{n+1}\setmin\{0\}$
	por multiplicaci\'{o}n por escalares, es decir, por homotecias.
	Si llamamos $M$ al conjunto de \'{o}rbitas de la acci\'{o}n
	correspondiente de $G$ en $\bb{R}^{2n+2}\setmin\{0\}$, entonces
	$\Phi$ induce una correspondencia entre $\proyectivo{\bb{C}}{n}$ y
	$M$. En definitiva, tenemos un diagrama conmutativo, en principio, de
	conjuntos:
	\begin{center}
	\begin{tikzcd}
		\bb{C}^{n+1}\setmin\{0\} \arrow[r,"\Phi"] \arrow[d,"\pi"] &
			\bb{R}^{2n+2}\setmin\{0\} \arrow[d,"q"] \\
		\proyectivo{\bb{C}}{n} \arrow[r,"F"] & M
	\end{tikzcd} .
	\end{center}

	En cuanto a la topolog\'{\i}a, $\bb{R}^{2n+2}\setmin\{0\}$
	tiene la estructura de variedad topol\'{o}gica heredada de
	$\bb{R}^{2n+2}$ en tanto subespacio abierto y a $M$ se le da la
	topolog\'{\i}a cociente. Usando las correspondencias $\Phi$ y $F$,
	damos a $\bb{C}^{n+1}$ y a $\proyectivo{\bb{C}}{n}$ las
	topolog\'{\i}as correspondientes de manera que $\Phi$ y $F$ sean
	homeomorfismos. As\'{\i}, el diagrama anterior pasa a ser un diagrama
	conmutativo de espacios topol\'{o}gicos y funciones continuas.

	El espacio topol\'{o}gico $M$ es una variedad topol\'{o}gica. Para
	mostrar esto busquemos primero cartas para $M$. Dado que las
	funciones $f_{\alpha}:\,\bb{R}^{2n+2}\setmin\{0\}%
		\rightarrow\bb{R}^{2n+2}\setmin\{0\}$ son restricciones de
	transformaciones lineales $m_{\alpha}$, son, en particular,
	continuas. Adem\'{a}s, como dichas transformaciones son invertibles,
	\begin{align*}
		f_{\alpha}\circ f_{\alpha^{-1}} & \,=\,
			f_{\alpha^{-1}}\circ f_{\alpha} \,=\,
			\id[\bb{R}^{2n+2}\setmin\{0\}]
		\text{ ,}
	\end{align*}
	%
	de lo que se deduce que el grupo $G$ act\'{u}a en
	$\bb{R}^{2n+2}\setmin\{0\}$ v\'{\i}a homeomorfismos. Una consecuencia
	de esto es que la funci\'{o}n cociente $q$ es abierta: si
	$V\subset\bb{R}^{2n+2}\setmin\{0\}$ es abierto, entonces
	\begin{align*}
		q^{-1}\big(q(V)\big) & \,=\,V\cdot G \\
		& \,=\,\bigcup_{m_{\alpha}\in G}\,V\cdot m_{\alpha}
			\,=\,\bigcup_{\alpha\in\bb{C}^{\times}}\,
				f_{\alpha}(V)
	\end{align*}
	%
	que es una uni\'{o}n de abiertos. Como $M$ tiene la topolog\'{\i}a
	cociente inducida por $q$, el conjunto $q(V)$ es abierto.
	Ahora bien, definimos, para cada $i$,
	\begin{align*}
		\widetilde{U}_{i} & \,=\,
			\left\lbrace((x^{1},y^{1}),\,\dots,\,
			(x^{n+1},y^{n+1}))\in\bb{R}^{2n+2}\,:\,
			(x^{i},y^{i})\not = 0\right\rbrace
		\text{ .}
	\end{align*}
	%
	Como las matrices $\widehat{\alpha}$ son invertibles, se ve que
	$q^{-1}(q(\widetilde{U}_{i}))=\widetilde{U}_{i}$ para todo $i$. En
	particular, $\widetilde{U}_{i}$ es un abierto saturado respecto de
	$q$ y, si llamamos $U_{i}=q(\widetilde{U}_{i})$, entonces la
	restricci\'{o}n de $q|_{i}:\,\widetilde{U}_{i}\rightarrow U_{i}$ es
	cociente.

	Para definir cartas en $M$, definimos funciones en
	$\proyectivo{\bb{C}}{n}$ an\'{a}logas a las cartas para los espacios
	proyectivos reales y las trasladamos, v\'{\i}a $F$ a $M$.
	Notemos que $\Phi^{-1}(\widetilde{U}_{i})$ es el subconjunto de
	$\bb{C}^{n+1}\setmin\{0\}$ conformado por los puntos cuya coordenada
	$i$ es distinta de cero. Si $z\in\bb{C}^{n+1}\setmin\{0\}$,
	denotaremos
	\begin{align*}
		\pi(z) & \,=\,\pi(\lista*{z}{n+1}) \,=\,
			\left[z^{1}:\,\cdots\,:z^{n+1}\right]
	\end{align*}
	%
	al punto correpondiente en el espacio proyectivo. De manera
	an\'{a}loga los puntos de $M$ los denotaremos de la siguiente manera:
	\begin{align*}
		q((x^{1},y^{1}),\,\dots,\,(x^{n+1},y^{n+1})) & \,=\,
			\left[(x^{1},y^{1}):\,\cdots\,:(x^{n+1},y^{n+1})\right]
		\text{ .}
	\end{align*}
	%
	Sea, entonces, $\varphi_{i}:\,F^{-1}(U_{i})\rightarrow\bb{C}^{n}$
	la funci\'{o}n dada por
	\begin{align*}
		\varphi_{i}\big(\left[z^{1}:\,\cdots\,:z^{n+1}\right]\big) &
			\,=\,\left(\frac{z^{1}}{z^{i}},\,\dots,\,
			\frac{z^{i-1}}{z^{i}},\,\frac{z^{i+1}}{z^{i}},
			\,\dots,\,\frac{z^{n+1}}{z^{i}}\right) \\
		& \,=\,(z^{1},\,\dots,\,z^{i-1},\,z^{i+1},\,\dots,\,z^{n+1})
			\cdot\frac{1}{z^{i}}
		\text{ .}
	\end{align*}
	%
	De la expresi\'{o}n anterior, se deduce que las funciones
	$\varphi_{i}$ son biyectivas. Veamos que son continuas. Para eso
	usamos las correspondencias $\Phi$ y $F$. En
	$\widetilde{U}_{i}\subset M$, entonces, definimos
	$\tilde{\varphi}_{i}=\Phi\circ\varphi_{i}\circ F^{-1}$. Donde
	$\Phi:\,\bb{C}^{n}\rightarrow\bb{R}^{2n}$ es an\'{a}loga a la
	funci\'{o}n $\Phi$ definida anteriormente. Cuando
	$i=n+1$, por ejemplo, si $z^{n+1}=x^{n+1}+\sqrt{-1}y^{n+1}$ y
	$\alpha=\frac{1}{z^{n+1}}$,
	\begin{align*}
		\tilde{\varphi}_{n+1}\big(\left[(x^{1},y^{1}):\,\cdots\,:
					(x^{n+1},y^{n+1})\right]\big) & \,=\,
			((x^{1},y^{1})\cdot\widehat{\alpha},\,\dots,\,
			(x^{n},y^{n})\cdot\widehat{\alpha})
		\text{ .}
	\end{align*}
	%
	Un poco m\'{a}s expl\'{\i}citamente, en el caso $i=n+1$ (para
	simplificar),
	\begin{align*}
		\tilde{\varphi}_{n+1}\big(\left[(x^{1},y^{1}):\,\cdots\,:
					(x^{n+1},y^{n+1})\right]\big) & \,=\,
		\left(ax^{1}-by^{1},\,bx^{1}+ay^{1},\right. \\
		& \left.\qquad\,\dots,\,ax^{n}-by^{n},\,bx^{n}+ay^{n}\right)
		\text{ ,}
	\end{align*}
	%
	donde $a=\frac{x^{n+1}}{(x^{n+1})^{2}+(y^{n+1})^{2}}$ y
	$b=\frac{y^{n+1}}{(x^{n+1})^{2}+(y^{n+1})^{2}}$.
	Llamamos, para simplificar, $\varphi_{i}$ a $\tilde{\varphi}_{i}$,
	tambi\'{e}n. De la expresi\'{o}n anterior, para cada $i$, la
	composici\'{o}n $\varphi_{i}\circ q$ es continua. Porque $q|_{i}$
	es cociente, $\varphi_{i}:\,U_{i}\rightarrow\bb{R}^{2n}$ es
	continua. 

	La inversa de $\varphi_{i}$ est\'{a} dada por
	$\varphi_{i}^{-1}:\,\bb{C}^{n}\rightarrow F^{-1}(U_{i})$ donde
	\begin{align*}
		\varphi_{i}^{-1}(\lista*{u}{n}) & \,=\,
			\left[u^{1}:\,\cdots\,:u^{i-1}:1:u^{i}:
				\,\cdots\,:u^{n}\right]
		\text{ .}
	\end{align*}
	%
	El elemento $1$ en la expresi\'{o}n anterior es el $1$ de $\bb{C}$.
	V\'{\i}a $\Phi$ y $F$, en t\'{e}rminos de puntos de $M$ y de
	$\bb{R}^{2n}$,
	\begin{align*}
		\varphi_{i}(\xi^{1},\upsilon^{1},\,\dots,\,
			\xi^{n},\,\upsilon^{n}) & \,=\,
			\left[(\xi^{1},\upsilon^{1}):\,\cdots\,:
			(\xi^{i-1},\upsilon^{i-1}):(1,0):
			(\xi^{i},\upsilon^{i}):\,\cdots\,
			(\xi^{n},\upsilon^{n})\right]
		\text{ .}
	\end{align*}
	%
	Notemos que $(1,0)$ es la representaci\'{o}n de $1\in\bb{C}$ en
	coordenadas reales. La funci\'{o}n $\varphi_{i}^{-1}$ es continua
	porque se puede escribir como composici\'{o}n de funciones continuas
	seg\'{u}n el siguiente diagrama:
	\begin{center}
	\begin{tikzcd}
		\bb{R}^{2n+2}\setmin\{0\}\supset\widetilde{U}_{i}
			\arrow[d,"q"'] & \\
		M\supset U_{i} & \bb{R}^{2n}\arrow[l,"\varphi_{i}^{-1}"]
					\arrow[ul,"\psi_{i}"']

	\end{tikzcd}
	\end{center}
	La funci\'{o}n $\psi_{i}$ es \'{u}nica tal que el diagrama
	conmuta y debe cumplir que
	\begin{align*}
		\psi_{i}(\xi^{1},\upsilon^{1},\,\dots,\,
			\xi^{n},\,\upsilon^{n}) & \,=\,
			(\xi^{1},\,\upsilon^{1},\,\dots,\,
			\xi^{i-1},\,\upsilon^{i-1},\,1,\,0,\,
			\xi^{i},\,\upsilon^{i},\,\dots,\,
			\xi^{n},\,\upsilon^{n})
		\text{ .}
	\end{align*}
	%
	De esta expresi\'{o}n, es inmediato que $\psi_{i}$ es continua y que
	$\varphi_{i}^{-1}=q\circ\psi_{i}$. En definitiva, las funciones
	$\varphi_{i}:\,U_{i}\rightarrow\bb{R}^{2n}$ son homeomorfismos
	y las correspondientes
	$\varphi_{i}:\,F^{-1}(U_{i})\rightarrow\bb{C}^{n}$ tambi\'{e}n, donde
	$\bb{C}^{n}$ tiene la topolog\'{\i}a inducida por la biyecci\'{o}n
	$\Phi:\,\bb{C}^{n}\rightarrow\bb{R}^{2n}$.

	% Una observaci\'{o}n importante es que las operaciones de espacio
	% vectorial complejo en $\bb{C}^{n}$ son continuas si se le da
	% la topolog\'{\i}a inducida por el homeomorfismo
	% $\Phi:\,\bb{C}^{n}\rightarrow\bb{R}^{2n}$ para todo $n\geq 0$.

	Hemos demostrado que $\proyectivo{\bb{C}}{n}$ con la topolg\'{\i}a
	cociente que se obtiene de considerar $\bb{C}^{n+1}$ como
	$\bb{R}^{2n+2}$ y a $\bb{C}^{n+1}\setmin\{0\}$ como subconjunto
	abierto es localmente euclidea de dimensi\'{o}n $2n$. De hecho,
	demostramos que $\proyectivo{\bb{C}}{n}$ es homeomorfo al
	cociente de $\bb{R}^{2n+2}\setmin\{0\}$ por la acci\'{o}n
	diagonal del grupo $\bb{R}_{>0}\times\SO{2}$.
\end{ejemplo}

