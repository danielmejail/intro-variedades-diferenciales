\theoremstyle{plain}
\newtheorem{teoParticiones}{Teorema}[section]
\newtheorem{lemaBaseDeBolasPrecompactas}[teoParticiones]{Lema}
\newtheorem{lemaConjuntoLocalmenteFinito}[teoParticiones]{Lema}
\newtheorem{teoRefinamiento}[teoParticiones]{Teorema}
\newtheorem{propoVariedadEsSigmaCompacta}[teoParticiones]{Proposici\'{o}n}
\newtheorem{lemaParticiones}[teoParticiones]{Lema}

\theoremstyle{remark}
\newtheorem{obsBaseDeBolasRegulares}[teoParticiones]{Observaci\'{o}n}

%-------------

\begin{lemaBaseDeBolasPrecompactas}\label{lema:basedebolasprecompactas}
	Toda variedad topol\'{o}gica admite una base de bolas coordenadas
	precompactas.
\end{lemaBaseDeBolasPrecompactas}

\begin{proof}
	Si $M=U$ es el dominio de una \'{u}nica carta,
	$\varphi:\,U\rightarrow\encoordenadas{U}$ es un homeomorfismo con
	alg\'{u}n abierto de $\bb R^n$. Dado un abierto arbitrario $W$, diremos
	que una bola $B=\bola{r}{x}\subset W$ es una \emph{bola premium} (o
	\emph{especial}, \emph{seleccionada}, \emph{de calidad}, etc.)
	\emph{en/para el abierto $W$}, si existe $r'>r$ tal que
	$\clos{\bola{r}{x}}\subset\bola{r'}{x}\subset W$. Consideramos el
	conjunto
	\begin{align*}
		\widetilde{\cal B} & \,:=\,
			\Big\{\bola{r}{x} \text{ premium en }
				\encoordenadas{U}\,:\,x
			\text{ tiene coordenadas racionales y } r>0
			\text{ es racional }\Big\}
		\text{ .}
	\end{align*}
	%
	Entonces $\widetilde{\cal B}$ tiene cardinal a lo sumo numerable y
	constituye una base para la topolog\'{\i}a de $\encoordenadas{U}$. Como
	$\varphi:\,U\rightarrow\encoordenadas{U}$ es un homeomorfismo, el
	conjunto
	\begin{math}
		\cal B:=\big\{ B=\varphi^{-1}(\tilde B)\,:\,
			\tilde B\in\widetilde{\cal B}\big\}
	\end{math} es una base para la topolog\'{\i}a de $U$ y es numerable.
	Por definici\'{o}n $\cal B$ est\'{a} compuesto por bolas coordenadas
	precompactas.

	En general, una variedad topol\'{o}gica arbitraria, $M$, se puede
	cubrir por una familia numerable de abiertos coordenados:
	\begin{math}
		M=\bigcup_{n\geq 1}\,U_n
	\end{math}. Si
	\begin{math}
		\cal B:=\bigcup_{n\geq 1}\,\cal B_n
	\end{math}, donde $\cal B_n$ es una base numerable de bolas
	precompactas para $U_n$, entonces $\cal B$ es numerable y constituye
	una base para la topolog\'{\i}a de $M$. Lo \'{u}nico que falta
	demostrar es que las bolas $B\in\cal B$ son precompactas \emph{en $M$}.

	Sea $B\in\cal B_n$ un elemento de la base. Por definici\'{o}n,
	$\clos B^{U_n}$ --la clausura \emph{en $U_n$} de la bola-- es compacta.
	Pero $\clos B^M$ es cerrada en $M$ y contiene a $B$, de lo que se
	deduce que $\clos B^M\cap U_n$ es cerrado en $U_n$ y contiene a $B$.
	As\'{\i},
	\begin{align*}
		\clos B^M & \,\supset\,\clos B^M\,\cap\,U_n\,\supset\,
			\clos B^{U_n}
		\text{ .}
	\end{align*}
	%
	Por otro lado, como $\clos B^{U_n}$ es compacta y ``subespacio de
	subespacio es subespacio'', $\clos B^{U_n}$ es compacta como subespacio
	de $M$. Como $M$ es $T_2$, este conjunto es cerrado en $M$. Pero
	tambi\'{e}n contiene a $B$. En consecuencia,
	$\clos B^M\subset\clos B^{U_n}$. En definitiva, ambas clausuras
	coinciden, de lo que se deduce que $B$ es precompacta.
\end{proof}

\begin{obsBaseDeBolasRegulares}\label{obs:basedebolasregulares}
	La afirmaci\'{o}n del Lema~\ref{lema:basedebolasprecompactas} sigue
	siendo v\'{a}lida si se reemplaza ``bolas coordenadas precompactas''
	por ``bolas coordenadas regulares''. Una bola coordenada regular es un
	abierto $B\subset M$ con las siguientes caracter\'{\i}sticas:
	existe una carta $(B',\varphi)$ y $r'>r>0$ tales que
	$\clos B\subset B'$ y
	\begin{align*}
		\varphi(B) \,=\,\bola{r}{x} & \quad\text{,}\quad
			\varphi(\clos B) \,=\,\clos{\bola{r}{x}}
			\quad\text{y}\quad
			\varphi(B') \,=\,\bola{r'}{x}
	\end{align*}
	%	
	(notemos que no hemos introducido la noci\'{o}n de estructura suave).
	La \'{u}nica parte del argumento que se debe modificar es
	el p\'{a}rrafo final; el resto es v\'{a}lido luego de hacer el
	reemplazo textual. Supongamos que $B\in\cal B_n$ y que
	$\varphi_n:\,U_n\rightarrow\encoordenadas U_n$ es el homeo de la carta
	compatible correspondiente. Por definici\'{o}n,
	\begin{math}
		\varphi_n(B)=\bola{r}{x}
	\end{math} para cierto $r>0$ racional y $x\in\encoordenadas U_n$ con
	coordenadas racionales y existe $r'>r$ tal que
	\begin{align*}
		\clos{\bola{r}{x}} & \,\subset\,\bola{r'}{x}\,\subset\,
			\encoordenadas U_n
		\text{ .}
	\end{align*}
	%
	Si definimos $B'=\varphi_n^{-1}(\bola{r'}{x})$, entonces,
	\emph{en $U_n$},
	\begin{align*}
		\clos B^{U_n} & \,=\,\varphi_n^{-1}\big(\clos{\bola{r}{x}}\big)
			\,\subset\,\varphi_n^{-1}\big(\bola{r'}{x}\big)
			\,=\,B'
		\text{ .}
	\end{align*}
	%
	Pero $\clos B^M=\clos B^{U_n}$, porque $B$ es precompacta en $U_n$
	(regular implica precompacta), seg\'{u}n el Lema~%
	\ref{lema:basedebolasprecompactas}. Esto quiere decir que la
	expresi\'{o}n anterior es v\'{a}lida tomando clausura en $M$, en lugar
	de clausura en $U_n$:
	\begin{align*}
		\clos B^M & \,=\,\clos B^{U_n}\,\subset\,B'
		\text{ .}
	\end{align*}
	%
\end{obsBaseDeBolasRegulares}

Si asumimos que $M$ es una variedad diferencial, podemos preguntarnos si la
estructura adicional nos permite concluir algo m\'{a}s fuerte, o si, dicho de
otra manera, este resultado tiene una versi\'{o}n ``compatible'' con la
estructura diferencial.

\begin{lemaBaseDeBolasPrecompactas}[Porisma]%
	\label{lema:basedebolasprecompactas:porisma}
	Toda variedad diferencial admite una base de bolas coordenadas (suaves)
	regulares.
\end{lemaBaseDeBolasPrecompactas}

\begin{proof}
	Reemplazar ``homeo'' por ``difeo'' en la demostraci\'{o}n anterior.
\end{proof}

\begin{lemaConjuntoLocalmenteFinito}\label{lema:conjuntolocalmentefinito}
	Sea $\cal X$ un conjunto localmente finito de subconjuntos de $M$ (un
	espacio topol\'{o}gico). Entonces
	\begin{enumerate}
		\item\label{lema:conjuntolcalmentefinito:clausuras}
			\begin{math}
				\widetilde{\cal X}:=
					\big\{\clos X\,:\,X\in\cal X\big\}
			\end{math} es localmente finito;
		\item\label{lema:conjuntolocalmentefinito:union}
			\begin{math}
				\clos{\bigcup\,\cal X}=
					\bigcup\,\widetilde{\cal X}
			\end{math}.
	\end{enumerate}
	%
\end{lemaConjuntoLocalmenteFinito}

\begin{proof}
	Sea $x\in M$. Sea $U\subset M$ un abierto al cual $x$ pertenece. Si
	$U\cap\clos X\not=\varnothing$ e $y$ es un elemento de este conjunto,
	entonces $U$ es un entorno de $y$ y, por lo tanto,
	$U\cap X\not=\varnothing$. En definitiva, para todo abierto
	$V\subset M$,
	\begin{align*}
		V\,\cap\,\clos X\,\not=\,\varnothing &
			\quad\Leftrightarrow\quad
		V\,\cap\,X\,\not=\,\varnothing
		\text{ .}
	\end{align*}
	%
	As\'{\i} $\cal X$ es localmente finito, si y s\'{o}lo si
	$\widetilde{\cal X}$ lo es. Esto demuestra
	\ref{lema:conjuntolocalmentefinito:clausuras}. En cuanto a
	\ref{lema:conjuntolocalmentefinito:union}, el conjunto
	$\clos{\bigcup\,\cal X}$ es cerrado y contiene a todo elemento de
	$\cal X$. Es decir,
	\begin{align*}
		\clos X & \,\subset\,\clos{\bigcup\,\cal X}
		\text{ ,}
	\end{align*}
	%
	si $X\in\cal X$, lo que implica
	\begin{math}
		\bigcup\,\widetilde{\cal X}\subset
			\clos{\bigcup\,\cal X}
	\end{math}. Por otro lado, si $x\in\clos{\bigcup\,\cal X}$, existe un
	entorno $U_0\subset M$ de $x$, tal que $U_0\cap X=\varnothing$ para
	casi todo $X\in\cal X$ (todos salvo finitos). Si definimos
	\begin{align*}
		\cal A_{x,0} & \,:=\,\Big\{X\in\cal X\,:\,
			U_0\cap X\not=\varnothing\Big\}
		\text{ ,}
	\end{align*}
	%
	entonces $\cal A_{x,0}$ es finito. Pero tambi\'{e}n sabemos que debe
	existir $y\in U_0\cap \big(\bigcup\,\cal X\big)$, pues $U_0$ es entorno
	de $x$. Entonces $y\in U_0$ e $y\in X$ para alg\'{u}n $X\in\cal X$.
	Este conjunto $X$ verifica $U_0\cap X\not=\varnothing$ y
	$X\in\cal A_{x,0}$.
	Llamemos $A=\bigcup\,\cal A_{x,0}$ (la uni\'{o}n de los elementos de
	esta familia (finita) de conjuntos $X\in\cal X$). Entonces
	$y\in U_0\cap A$. Eso se puede expresar de la siguiente manera:
	\begin{align*}
		U_0\,\cap\,\big(\bigcup\,\cal x\big) & \,=\,
			U_0\,\cap\,A
		\text{ .}
	\end{align*}
	%
	Si $V\subset M$ es una abierto tal que $x\in V$, $V\cap U_0$ es
	abierto, est\'{a} contenido en $U_0$ y contiene a $x$. Nuevamente,
	podemos afirmar que existe
	$y\in (V\cap U_0)\cap\big(\bigcup\,\cal X\big)$. As\'{\i},
	\begin{align*}
		(V\cap U_0)\,\cap\,\big(\bigcup\,\cal X\big) & \,\not=\,
			\varnothing \quad\text{implica} \\
		(V\cap U_0)\,\cap\,A & \,\not=\,\varnothing
			\quad\text{implica} \\
		V\,\cap\,A & \,\not=\,\varnothing
		\text{ .}
	\end{align*}
	%
	Si $x\in\clos{\bigcup\,\cal X}$, por ser $A=\bigcup\,\cal A_{x,0}$ una
	uni\'{o}n finita,
	\begin{align*}
		x & \,\in\,\clos{\bigcup\,\cal A_{x,0}} \,=\,
			\bigcup\,\big\{\clos X\,:\,X\in\cal A_{x,0}\big\}
			\,\subset\,\bigcup\,\widetilde{\cal X}
		\text{ .}
	\end{align*}
	%
\end{proof}

\begin{teoRefinamiento}\label{thm:refinamiento}
	Sea $M$ una variedad topol\'{o}gica y sea $\cal U$ un cubrimiento de
	$M$ por abiertos. Sea $\cal B$ una base para la topolg\'{\i}a de $M$.
	Existe un refinamiento numerable y localmente finito de $\cal U$
	compuesto por elementos de $\cal B$.
\end{teoRefinamiento}

\begin{propoVariedadEsSigmaCompacta}\label{propo:variedadessigmacompacta}
	Dada una variedad topol\'{o}gica $M$, existe una familia numerable de
	compactos $\{K_n\}_{n\geq 1}$ tal que
	\begin{align*}
		\interior{K_n}\,\not=\,\varnothing & \quad\text{,}\quad
			K_n\,\subset\,\interior{K_{n+1}}
			\quad\text{y}\quad
			M \,=\,\bigcup_{n\geq 1}\,K_n
		\text{ .}
	\end{align*}
	%
\end{propoVariedadEsSigmaCompacta}

\begin{proof}
	La variedad $M$ se puede cubrir por bolas precompactas. Como $M$ es
	$N_2$, existe un subcubrimiento numerable. Ordenando este
	subcubrimiento y tomando la clausura de las bolas, se obtiene la
	familia de compactos: por ejemplo, $K_1=\clos{B_1}$ se cubre por
	$B_1$ y el resto de las bolas, tomando un subcubrimiento finito, debe
	haber al menos una $B_2$ para cubrir $K_1$; numeramos $\lista[2]{B}{n}$
	y definimos $K_2$ como la uni\'{o}n de $\clos{B_i}$, $i\leq n$\dots
\end{proof}

\begin{proof}[Demostraci\'{o}n de \ref{thm:refinamiento}]
	Sea $\{K_j\}_{j\geq 1}$ una familia numerable de compactos como
	en el enunciado de la Proposici\'{o}n~%
	\ref{propo:variedadessigmacompacta}. Sean
	\begin{align*}
		F_j & \,:=\,K_{j+1}\setmin\interior{K_j} \quad\text{y} \\
		W_j & \,:=\,\interior{K_{j+2}}\setmin K_{j-1}
		\text{ .}
	\end{align*}
	%
	Si $j=1$, $K_0:=\varnothing$. Cada $F_j$ es cerrado y est\'{a}
	contenido en el compacto $K_{j+1}$ y es, por lo tanto, compacto. Para
	cada $x\in F_j$, existe $U_x\in\cal U$ tal que $x\in U_x$; elegimos,
	tambi\'{e}n, $B_x^j\in\cal B$ tal que
	\begin{align*}
		x & \,\in\,B_x^j\subset U_x\cap W_j
		\text{ .}
	\end{align*}
	%
	Como $F_j$ es compacto y $\big\{B_x^j\,:\,x\in F_j\big\}$ es un
	cubrimiento por abiertos, podemos extraer un subcubrimiento finito:
	existen $\lista{x}{n_j}\in F_j$ tales que
	\begin{align*}
		F_j & \,\subset\, B_{x_1}^j\,\cup\,\cdots\,\cup\,
			B_{x_{n_j}}^j
		\text{ .}
	\end{align*}
	%
	Definimos
	\begin{math}
		\cal V:=\big\{B_{x_i}^j\,:\,j\geq 1,\,n_j\geq i\geq 1\big\}
	\end{math}. Entonces $\cal V$ es numerable y la inclusi\'{o}n
	$B_{x_i}^j\subset U_{x_i}$ implica que es un refinamiento de $\cal U$.
	Este refinamiento est\'{a} conformado por elementos de la base
	$\cal B$, con lo que lo \'{u}nico que resta demostrar es que $\cal V$
	es localmente finito.

	Para demostrar esto, observamos que
	\begin{align*}
		|j-j'| \,\geq\,3 & \quad\Rightarrow\quad
			W_j\,\cap\,W_{j'} \,=\,\varnothing
		\text{ .}
	\end{align*}
	%
	Como $B_{x_i}^j\subset W_j$, una intersecci\'{o}n de la forma
	$B_{x_k}^j\cap B_{x_l}^{j'}$ es vac\'{\i}a salvo, posiblemente, para
	finitos valores de $l$ y $j'$. En definitiva, si $x\in M$, basta tomar
	un elemento del mismo cubrimiento $\cal V$ que lo contenga, para
	verificar que $\cal V$ es localmente finito; cualquier elemento de
	$\cal V$ que contega a $x$ es un abierto que demuestra la finitud local
	del cubrimiento en el punto.
\end{proof}

\begin{lemaParticiones}\label{lema:particiones:pendientesuave}
	La funci\'{o}n
	\begin{align*}
		f(t) & \,:=\,
			\begin{cases}
				e^{-1/t} & t>0 \\
				0 & t \leq 0
			\end{cases}
	\end{align*}
	%
	es suave.
\end{lemaParticiones}

\begin{lemaParticiones}\label{lema:particiones:escalonsuave}
	Si $r_1<r_2$ son n\'{u}meros reales, la funci\'{o}n
	\begin{align*}
		h(t) & \,:=\,\frac{f(r_2-t)}{f(r_2-t)+f(t-r_1)}
	\end{align*}
	%
	verifica: ser suave, $h=1$ en $t\leq r_1$ y $h=0$ en $r_2\leq t$.
	Adem\'{a}s, $0<h(t)<1$, si $r_1<t<r_2$.
\end{lemaParticiones}

\begin{lemaParticiones}\label{lema:particiones:chichonsuave}
	Si $0<r_1<r_2$, la funci\'{o}n
	\begin{align*}
		H(x) & \,:=\,h(|x|)
	\end{align*}
	%
	verifica: ser suave, $H=1$ en $\clos{\bola{r_1}{0}}$ y $H=0$ en
	$\bb R^n\setmin\bola{r_2}{0}$. Adem\'{a}s, $0<H(x)<1$ en
	$\bola{r_2}{0}\setmin\clos{\bola{r_1}{0}}$.
\end{lemaParticiones}

\begin{teoParticiones}[Existencia de particiones suaves de la unidad]%
	\label{thm:particiones}
	Sea $M$ una variedad (diferencial). Sea $\cal U=\{U_\alpha\}_\alpha$ un
	cubrimiento de $M$ por abiertos. Existe una partici\'{o}n de la unidad
	para $M$, subordinada a $\cal U$ y compuesta por funciones suaves.
\end{teoParticiones}

\begin{proof}
	Sean $\cal B_\alpha$ bases numerables conformadas por bolas regulares
	para cada uno de los abiertos $U_\alpha$ y sea
	$\cal B=\bigcup_\alpha\,\cal B_\alpha$. Por el Teorema del refinamiento
	(Teorema~\ref{thm:refinamiento}), existe un refinamiento de $\cal U$,
	localmente finito, numerable y compuesto por elementos de la base
	$\cal B$. Sea $\cal V=\{B_i\}_{i\geq 1}$ dicho refinamiento.

	Para cada $i$, $B_i$ es una bola coordenada regular, por lo tanto,
	existen $B_i'\supset\clos{B_i}$ bolas coordenadas y cartas
	$\varphi_i:\,B_i'\rightarrow\bola{r_i'}{0}$ tales que
	$\varphi_i(B_i)=\bola{r_i}{0}$ para cierto $r_i<r_i'$ y
	$\varphi_i(\clos{B_i})=\clos{\bola{r_i}{0}}$. Dentro de la bola
	$\bola{r_i'}{0}$, podemos definir una funci\'{o}n suave $H_i$ tal que
	$H_i=1$ en $\clos{\bola{s_i}{0}}$, $H=0$ en el complemento de
	$\bola{t_i}{0}$ y tome un valor intermedio en el anillo
	$\bola{t_i}{0}\setmin\clos{\bola{s_i}{0}}$ (eligiendo
	$0<s_i<t_i\leq r_i<r_i'$). Luego definimos $f_i:\,M\rightarrow\bb R$
	por
	\begin{align*}
		f_i & \,:=\,
			\begin{cases}
				H_i\circ\varphi_i & \text{ en } B_i' \\
				0 & \text{ en } M\setmin\clos{B_i}
			\end{cases}
		\text{ .}
	\end{align*}
	%
	Entonces $f_i$ est\'{a} bien definida y es suave. Adem\'{a}s,
	$\soporte{f_i}\subset\clos{B_i}$ (son iguales, si $t_i=r_i$). Por otro
	lado, como $\{B_i\}_i$ es localmente finita, la familia de clausuras
	$\{\clos{B_i}\}_i$ tambi\'{e}n lo es y, en particular,
	$\{\soporte{f_i}\}_i$ es localmente finita. En consecuencia, la
	expresi\'{o}n $f=\sum_i\,f_i$ est\'{a} bien definida y es suave. Esta
	funci\'{o}n verifica $f>0$: en general, $f_i\geq 0$ en su dominio de
	definici\'{o}n, todo punto $x$ pertenece a alg\'{u}n abierto $B_i$ y,
	si $t_i=r_i$, $f_i>0$ en dicho abierto.

	Finalmente, definimos $g_i=f_i/f$. Estas funciones son suaves y suman
	$1$ en $M$. Adem\'{a}s, $\soporte{g_i}=\soporte{f_i}=\clos{B_i}$ es
	compacto. Para obtener una partici\'{o}n subordinada a $\cal U$, usamos
	el Axioma de elecci\'{o}n\dots Cada conjunto $B_i$ est\'{a} incluido en
	alg\'{u}n abierto $U_\alpha$. Denotamos por $\alpha=a(i)$ alguno de
	estos \'{\i}ndices y definimos
	\begin{align*}
		\psi_\alpha & \,:=\,\sum_{i\,:\,a(i)=\alpha}\,g_i
		\text{ .}
	\end{align*}
	%
	Algunas de estas sumatorias podr\'{\i}an ser vac\'{\i}as. La finitud
	local de los soportes de las $g_i$ garantizan que estas nuevas
	funciones est\'{e}n bien definidas y sean suaves. Las funciones
	$\psi_\alpha$ tambi\'{e}n cumplen que $0\leq\psi_\alpha\leq 1$ y
	$\sum_\alpha\,\psi_\alpha=\sum_i\,g_i=1$ en $M$. Por \'{u}ltimo,
	como las $g_i$ son no negativas y estrictamente positivas en el
	correspondiente abierto $B_i$,
	\begin{align*}
		\soporte{\psi_\alpha} & \,=\,\clos{%
				\bigcup\,\{B_i\,:\,a(i)=\alpha\}} \,=\,
			\bigcup\,\{\clos{B_i}\,:\,a(i)=\alpha\} \\
		& \,\subset\,\bigcup\,\{B_i'\,:\,a(i)=\alpha\}\,\subset\,
			U_\alpha
		\text{ .}
	\end{align*}
	%
\end{proof}
