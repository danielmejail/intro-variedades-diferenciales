\theoremstyle{plain}

\theoremstyle{remark}
\newtheorem{obsSiLasCartasNoSonCompatibles}{Observaci\'{o}n}[section]

%-------------

Dados un punto $p\in M$ y su imagen $F(p)\in N$ por una transformaci\'{o}n
suave, para estudiar a la funci\'{o}n $F:\,M\rightarrow N$ cerca de $p$,
tomamos cartas $(U,\varphi)$ en $p$ y $(V,\psi)$ en $F(p)$, de manera que
$F(U)\subset V$. El hecho de que $F$ sea suave, se reflejar\'{a} en que su
expresi\'{o}n en coordenadas $\widehat{F}=\psi\circ F\circ\varphi^{-1}:\,%
\varphi(U)\rightarrow\psi(V)$ sea suave. El espacio tangente en $p$ tiene
como base a los vectores $\{\gancho{\varphi^{1}}|_{p},\,\dots,\,%
\gancho{\varphi^{n}}|_{p}\}$ que, en una funci\'{o}n suave
$f:\,U\rightarrow\bb{R}$ toman los valores
\begin{align*}
	\left.\gancho{\varphi^{i}}\right|_{p}f & \,\equiv\,
		\derivada{(f\circ\varphi^{-1})}{x^{i}}(\varphi(p))
		\,\equiv\, \diferencial[\varphi(p)]{\varphi^{-1}}
		\left(\left.\gancho{x^{i}}\right|_{\varphi(p)}\right)\,f
	\text{ .}
\end{align*}
%
Es decir, $\gancho{\varphi^{i}}|_{p}f$ es igual a la derivada parcial
$i$-\'{e}sima respecto de la base can\'{o}nica de $\bb{R}^{n}$ de la
expresi\'{o}n de $f$ en las coordenadas $\varphi=(\lista*{\varphi}{n})$.
Si queremos saber cu\'{a}l es la imagen de estos vectores v\'{\i}a el
diferencial de la transformaci\'{o}n $F$, usamos la igualdad
$\psi^{-1}\circ\widehat{F}=F\circ\varphi^{-1}$. Para simplificar la
notaci\'{o}n, llamamos $\widehat{p}=\varphi(p)$. Entonces
\begin{align*}
	\diferencial[p]{F}\Big(\left.\gancho{\varphi^{i}}\right|_{p}\Big) &
		\,=\,\diferencial[p]{F}\Big(
			\diferencial[\widehat{p}]{\varphi^{-1}}\Big(
			\left.\gancho{x^{i}}\right|_{\widehat{p}}
			\Big)
		\Big)
		\,=\,\diferencial[\widehat{p}]{(F\circ\varphi^{-1})}\Big(
			\left.\gancho{x^{i}}\right|_{\widehat{p}}
			\Big) \\
	& \,=\,\diferencial[\widehat{p}]{(\psi^{-1}\circ F)}\Big(
			\left.\gancho{x^{i}}\right|_{\widehat{p}}
			\Big)
		\,=\,\diferencial[\widehat{F}(\widehat{p})]{\psi^{-1}}\Big(
		\diferencial[\widehat{p}]{\widehat{F}}\Big(
			\left.\gancho{x^{i}}\right|_{\widehat{p}}
			\Big)
		\Big)
	\text{ .}
\end{align*}
%
Si llamamos $\lista*{y}{m}$ a las coordenadas en $\bb{R}^{m}$ y
$\gancho{y^{1}},\,\dots,\,\gancho{y^{m}}$ a las derivadas parciales en las
direcciones can\'{o}nicas y, si $f:\,\psi(V)\rightarrow\bb{R}$ es suave,
\begin{align*}
	\jacobiana[\widehat{p}]{\widehat{F}}\Big(
		\left.\gancho{x^{i}}\right|_{\widehat{p}}\Big)\,f &
		\,\equiv\,\left.\gancho{x^{i}}\right|_{\widehat{p}}
			(f\circ\widehat{F})
	\text{ ,}
\end{align*}
%
que es igual, por la regla de la cadena, a
\begin{align*}
	\derivada{f}{y^{j}}(\widehat{F}(\widehat{p}))\cdot
		\derivada{\widehat{F}^{j}}{x^{i}}(\widehat{p}) &
		\,=\,\derivada{\widehat{F}^{j}}{x^{i}}(\widehat{p})\cdot
		\left.\gancho{y^{j}}\right|_{\widehat{F}(\widehat{p})}\,f
	\text{ .}
\end{align*}
%
Notemos que $\widehat{F(p)}=\widehat{F}(\widehat{p})$. Entonces, en la base
$\{\gancho[\widehat{F(p)}]{y^{1}},\,\dots,\,\gancho[\widehat{F(p)}]{y^{n}}\}$,
\begin{align*}
	\diferencial[\widehat{p}]{\widehat{F}}\Big(
		\gancho[\widehat{p}]{x^{i}}\Big) & \,=\,
		\derivada{\widehat{F}^{j}}{x^{i}}(\widehat{p})\cdot
			\gancho[\widehat{F(p)}]{y^{j}}
	\text{ .}
\end{align*}
%
Volviendo a $F$,
\begin{align*}
	\diferencial[p]{F}\Big(\gancho[p]{\varphi^{i}}\Big) & \,=\,
		\derivada{\widehat{F}^{j}}{x^{i}}(\widehat{p})\cdot
		\diferencial[\widehat{F(p)}]{\psi^{-1}}\Big(
			\gancho[\widehat{F(p)}]{y^{j}}\Big)
		\,=\,\derivada{\widehat{F}^{j}}{x^{i}}(\widehat{p})\cdot
			\gancho[F(p)]{\psi^{j}}
	\text{ .}
\end{align*}
%
Es decir, la expresi\'{o}n del diferencial  $\diferencial[p]{F}$ en las
bases $\{\gancho[p]{\varphi^{i}}\}_{i}$ y $\{\gancho[F(p)]{\psi^{j}}\}_{j}$
es igual a la matriz jacobiana $\jacobiana[\widehat{p}]{\widehat{F}}$.

Un caso importante de todo esto es el de los cambios de carta (cambios de
coordenada, o funciones de transici\'{o}n): sea $p\in M$ un punto arbitrario
de la variedad $M$ y sean $(\widehat{U},\widehat{\varphi})$ y
$(\widetilde{U},\widetilde{\varphi})$ dos cartas en $p$. Primero,
veamos un tema de notaci\'{o}n. Si
\begin{align*}
	\widehat{\varphi} & \,=\,(\lista*{\widehat{\varphi}}{n})
	\quad\text{y}\quad
	\widetilde{\varphi}\,=\,(\lista*{\widetilde{\varphi}}{n})
	\text{ ,}
\end{align*}
%

entonces, dado un punto
$\xi=(\lista*{\xi}{n})\in\widehat{\varphi}(\widehat{U}\cap\widetilde{U})$,
la funci\'{o}n de transici\'{o}n est\'{a} dada expl\'{\i}citamente por
\begin{align*}
	\widetilde{\varphi}\circ\widehat{\varphi}^{-1}(\lista*{\xi}{n}) &
		\,=\,\big(
		\widetilde{\varphi}^{1}(\widehat{\varphi}^{-1}(\xi)),\,\dots,\,
			\widetilde{\varphi}^{n}(\widehat{\varphi}^{-1}(\xi))
			\big)
	\text{ .}
\end{align*}
%
Si pensamos en un punto $\widehat{x}\in\widehat{\varphi}(\widehat{U})$
podemos escribir $\widehat{\varphi}=(\lista*{\widehat{x}}{n})$, pensando
a las funciones coordenadas de $\widehat{\varphi}$, no como funciones en
$\bb{R}$, sino como las coordenadas de un punto en un abierto de un
espacio euclideo. An\'{a}logamente, podemos escribir
$\widetilde{\varphi}=(\lista*{\widetilde{x}}{n})$. La expresi\'{o}n
expl\'{\i}cita para las funciones de transici\'{o}n la podemos reemplazar
por una expresi\'{o}n un poco m\'{a}s clara para el cambio de coordenadas
correspondiente:
\begin{align*}
	\widetilde{\varphi}\circ\widehat{\varphi}^{-1}
		(\lista*{\widehat{x}}{n}) & \,=\,\big(
		\widetilde{x}^{1}(\widehat{x}),\,\dots,\,
		\widetilde{x}^{n}(\widehat{x})
		\big)
	\text{ .}
\end{align*}
%

Volviendo a las cartas, si las mismas son compatibles, la transformaci\'{o}n
$\widetilde{\varphi}\circ\widehat{\varphi}^{-1}:\,%
\widehat{\varphi}(\widehat{U}\cap\widetilde{U})\rightarrow%
\widetilde{\varphi}(\widehat{U}\cap\widetilde{U})$ y su inversa son
diferenciables en sentido usual. La matriz jacobiana de esta
transformaci\'{o}n, o, equivalentemente, su diferencial, est\'{a} dada por:
\begin{align*}
	\diferencial[\widehat{x}]%
		{(\widetilde{\varphi}\circ\widehat{\varphi}^{-1})}
		\Big(\gancho[\widehat{x}]{\widehat{x}^{i}}\Big) & \,=\,
		\derivada{\widetilde{x}^{j}}{\widehat{x}^{i}}(\widehat{x})
		\cdot\gancho[\widetilde{x}(\widehat{x})]{\widetilde{x}^{j}}
	\text{ ,}
\end{align*}
%
o, expl\'{\i}citamente en t\'{e}rminos de $p$,
\begin{align*}
	\diferencial[\widehat{\varphi}(p)]%
		{(\widetilde{\varphi}\circ\widehat{\varphi}^{-1})}
		\Big(\gancho[\widehat{\varphi}(p)]{\widehat{x}^{i}}\Big) &
	\,=\,\derivada%
		{(\widetilde{\varphi}\circ\widehat{\varphi}^{-1})^{j}}%
		{\widehat{x}^{i}}(\widehat{\varphi}(p))
		\cdot\gancho[\widetilde{\varphi}(p)]{\widetilde{x}^{j}}
	\text{ .}
\end{align*}
%
Por regla de la cadena y definici\'{o}n de los $\gancho{\widehat{\varphi}^{i}}$
y los $\gancho{\widetilde{\varphi}^{i}}$,
\begin{align*}
	\gancho[p]{\widehat{\varphi}^{i}} & \,=\,
	\diferencial[\widehat{\varphi}(p)]{\widehat{\varphi}^{-1}}
		\Big(
		\gancho[\widehat{\varphi}(p)]{\widehat{x}^{i}}
		\Big)
	\,=\,\diferencial[\widetilde{\varphi}(p)]{\widetilde{\varphi}^{-1}}
		\Big(
		\diferencial[\widehat{\varphi}(p)]%
			{(\widetilde{\varphi}\circ\widehat{\varphi}^{-1})}
			\Big(
			\gancho[\widehat{\varphi}(p)]{\widehat{x}^{i}}
			\Big)
		\Big) \\
	& \,=\,\derivada%
		{(\widetilde{\varphi}\circ\widehat{\varphi}^{-1})^{j}}%
		{\widehat{x}^{i}}(\widehat{\varphi}(p))
		\cdot
		\diferencial[\widetilde{\varphi}(p)]{\widetilde{\varphi}^{-1}}
			\Big(
			\gancho[\widetilde{\varphi}(p)]{\widetilde{x}^{j}}
			\Big) \\
	& \,=\,\derivada%
		{(\widetilde{\varphi}\circ\widehat{\varphi}^{-1})^{j}}%
		{\widehat{x}^{i}}(\widehat{\varphi}(p))
		\cdot
		\gancho[p]{\widetilde{\varphi}^{j}}
	\text{ ,}
\end{align*}
%
o, expresado de manera m\'{a}s concisa,
\begin{equation}
	\label{eq:cambiodecoordenadas}
	\gancho[p]{\widehat{\varphi}^{i}} \,=\,
		\derivada{\widetilde{x}^{j}}{\widehat{x}^{i}}(p)
		\cdot
		\gancho[p]{\widetilde{\varphi}^{j}}
	\text{ .}
\end{equation}
%
Notemos que estamos pensando en $\derivada{\widetilde{x}^{j}}{\widehat{x}^{i}}$
tanto como una funci\'{o}n en
$\widehat{\varphi}(\widehat{U}\cap\widetilde{U})$, como una funci\'{o}n
en $\widehat{U}\cap\widetilde{U}$.

De ahora en adelante, salvo talvez en algunos casos particulares en donde
sea necesario o conveniente hacer la distinci\'{o}n, denotaremos
$\gancho[p]{x^{i}}$ tanto a la derivada parcial $i$-\'{e}sima en un abierto
de un espacio euclideo en un punto $p$ del abierto, como a la derivaci\'{o}n
--o, mejor dicho, al \emph{campo}-- definida en el dominio de una carta
$\gancho[p]{\varphi^{i}}$ (dependiendo del punto). Es decir, con esta
notaci\'{o}n, la relaci\'{o}n entre las derivaciones provenientes de dos
cartas compatibles con intersecci\'{o}n no vac\'{\i}a, por ejemplo,
quedar\'{\i}a escrita de la siguiente manera:
\begin{align*}
	\gancho[p]{\widehat{x}^{i}} & \,=\,
		\derivada{\widetilde{x}^{j}}{\widehat{x}^{i}}(p)\cdot
		\gancho[p]{\widetilde{x}^{j}}
	\text{ .}
\end{align*}
%
Notemos, adem\'{a}s, que la expresi\'{o}n
$\derivada{\widetilde{x}^{j}}{\widehat{x}^{i}}(p)$ para cada $j$ e $i$, es,
en realidad, una funci\'{o}n de $p\in\widehat{U}\cap\widetilde{U}$,
estando definida en toda la intersecci\'{o}n. Como funci\'{o}n
$\widehat{U}\cap\widetilde{U}\rightarrow\bb{R}$, es una funci\'{o}n
continua y, m\'{a}s aun, suave.

\begin{obsSiLasCartasNoSonCompatibles}\label{obs:sinosoncompatibles}
	?`Qu\'{e} pasa si no asumimos que $\widehat{\varphi}$ y
	$\widetilde{\varphi}$ son cartas compatibles? De la misma manera
	que como se hizo antes, podemos, localmente, con cada carta,
	definir una estructura diferencial \emph{local}, \'{u}nicamente
	en $\widehat{U}$ y en $\widetilde{U}$, usando las cartas
	$(\widehat{U},\widehat{\varphi})$ y
	$(\widetilde{U},\widetilde{\varphi})$, respectivamente. En
	$\widehat{U}\cap\widetilde{U}$, hay, pues, dos estructuras posibles.
	Si $p$ pertenece a la intersecci\'{o}n, hay dos espacios
	tangentes, $T_{p}\widehat{U}$ y $T_{p}\widetilde{U}$, cada uno
	con su base can\'{o}nica
	$\{\gancho{\widehat{\varphi}^{i}}\}_{i}$ y
	$\{\gancho{\widetilde{\varphi}^{j}}\}_{j}$. Si las cartas no
	son compactibles, no hay una manera natural de relacionar estos
	espacios tangentes. Es decir, no hay, aunque no sean regulares,
	funciones $\alpha^{j}_{i}:\,%
	\widehat{U}\cap\widetilde{U}\rightarrow\bb{R}$ tales que
	\begin{align*}
		\gancho{\widehat{\varphi}^{i}} & \,=\,\alpha^{j}_{i}\cdot
			\gancho{\widetilde{\varphi}^{j}}
	\end{align*}
	%
	bajo alguna identificaci\'{o}n de los tangentes --justamente, porque
	no hay, en general, una identificaci\'{o}n.
\end{obsSiLasCartasNoSonCompatibles}
