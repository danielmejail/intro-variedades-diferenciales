\theoremstyle{plain}
\newtheorem{teoDelRango}{Teorema}[section]
\newtheorem{coroDelRangoLineal}[teoDelRango]{Corolario}
\newtheorem{coroDelRangoGlobal}[teoDelRango]{Corolario}
\newtheorem{teoInmersionConBorde}[teoDelRango]{Teorema}

\theoremstyle{remark}
\newtheorem{obsInmersionConBorde}{Observaci\'{o}n}[section]

%-------------

Empezamos enunciando y demostrando el resultado principal de esta secci\'{o}n.

\begin{teoDelRango}[del rango]\label{thm:delrango}
	Sea $F:\,M\rightarrow N$ una transformaci\'{o}n suave entre variedades
	\emph{sin} borde. Si $F$ tiene rango constante $r$, entonces, para
	cada punto $p\in M$, existen una carta $(U,\varphi)$ para $M$
	centrada en $p$ y otra carta $(V,\psi)$ para $N$ centrada en $F(p)$
	tales que $F(U)\subset V$ y $\widehat{F}=%
	\psi\circ F\circ\varphi^{-1}$ es de la forma
	\begin{align*}
		\widehat{F}(x^{1},\,\dots,\,x^{r},\,x^{r+1},\,\dots,\,x^{m}) &
			\,=\,(x^{1},\,\dots,\,x^{r},\,0,\,\dots,\,0)
		\text{ .}
	\end{align*}
	%
\end{teoDelRango}

\begin{proof}
	Sea $p\in M$ y sean $U$ y $V$ dominios de cartas en $p$ y en $F(p)$,
	respectivamente, tales que $F(U)\subset V$. Tomando coordenadas, si
	el teorema se demuestra reemplazando $M$ por $U$, $N$ por $V$ y
	$F$ por $\widehat{F}$, componiendo las cartas obtenidas con las
	anteriores, quedar\'{a} demostrado el caso general. Supongamos,
	entonces, sin p\'{e}rdida de generalidad, que $M=U\subset\bb{R}^{m}$
	y que $N=V\subset\bb{R}^{n}$ son abiertos, $p=0\in U$ y
	$F(p)=0\in V$. como $\rango{\jacobiana[p]{F}}=r$, alg\'{u}n menor
	de la matriz jacobiana de tama\~{n}o $r\times r$ es no nulo.
	Reordenando las coordenadas de $U$, podemos asumir que es el menor
	correspondiente a la submatriz
	\begin{math}
		\left[\begin{smallmatrix}
			\derivada{F^{i}}{x^{j}}
		\end{smallmatrix}\right]_{i,j\in[\![1,r]\!]}
	\end{math}.
	Sean $(\lista*{x}{r},\,\lista*{y}{m-r})$ las coordenadas en $U$ y
	sean $(\lista*{v}{r},\,\lista*{w}{n-r})$ las coordenadas en $V$. Con
	respecto a estas coordenadas,
	\begin{align*}
		F(x,y) & \,=\, (Q(x,y),R(x,y))
	\end{align*}
	%
	para ciertas funciones suaves $Q:\,U\rightarrow\bb{R}^{r}$ y
	$R:\,U\rightarrow\bb{R}^{n-r}$. Por hip\'{o}tesis, la matriz
	\begin{math}
		\left[\begin{smallmatrix}
			\derivada{Q^{i}}{x^{j}}
		\end{smallmatrix}\right]_{i,j\in[\![1,r]\!]}
	\end{math}
	es no singular. Extendemos $Q$ como en el teorema de la funci\'{o}n
	impl\'{\i}cita: sea $\Phi:\,U\rightarrow\bb{R}^{m}$ dada por
	$\Phi(x,y)=(Q(x,y),y)$. La matriz jacobiana de $\Phi$ en $(x,y)$ es
	igual a
	\begin{align*}
		\jacobiana[(x,y)]{\Phi} & \,=\,
		\sbox0{$
		\id[m-r]
		$}
		\sbox1{$
		\derivada{Q^{i}}{y^{j}}
		$}
		\left[
		\begin{array}{c|c}
			\makebox[\wd0]{$\derivada{Q^{i}}{x^{j}}$} &
				\usebox{1} \\
			\hline
			\vphantom{\usebox{1}}\makebox[\wd0]{$0$} &
				\usebox{0}
		\end{array}
		\right]
		\text{ .}
	\end{align*}
	%
	Por hip\'{o}tesis, $\left|\jacobiana[(0,0)]{\Phi}\right|\not =0$,
	con lo cual, por el teorema de la funci\'{o}n inversa, existen
	entornos (conexos) $U_{0}$ de $(0,0)$ y $\widehat{U}_{0}$ de
	$\Phi(0,0)=(0,0)$ tales que $\Phi|_{U_{0}}:\,%
	U_{0}\rightarrow\widehat{U}_{0}$ es difeomorfismo. Cambiamos
	$\widehat{U}_{0}$ por un cubo de la forma
	$\cubo{\epsilon}{0,0}\subset\widehat{U}_{0}$ y $U_{0}$ por
	$\Phi|_{U_{0}}^{-1}\big(\cubo{\epsilon}{0,0}\big)$. Sea
	$\varphi=\Phi|_{U_{0}}$.

	Ahora bien, la inversa $\varphi^{-1}:\,%
	\widehat{U}_{0}\rightarrow U_{0}$ tambi\'{e}n es de la forma
	\begin{align*}
		\varphi^{-1}(\xi,\upsilon) & \,=\,
			(A(\xi,\upsilon),B(\xi,\upsilon))
	\end{align*}
	%
	para ciertas funciones suaves $A,B$. Entonces
	\begin{align*}
		(\xi,\upsilon) & \,=\,\varphi\circ\varphi^{-1}(\xi,\upsilon)
			\,=\,(Q(A,B),B) \quad\text{y} \\
		B(\xi,\upsilon) & \,=\,\upsilon\quad\text{y} \\
		\xi & \,=\,Q(A(\xi,\upsilon),\upsilon)
		\text{ .}
	\end{align*}
	%
	Componiendo con $F$,
	\begin{align*}
		F\circ\varphi^{-1}(\xi,\upsilon) & \,=\,(Q(A,B),R(A,B))
			\,=\,(\xi,R(A(\xi,\upsilon),\upsilon))
		\text{ .}
	\end{align*}
	%
	Sea $\tilde{R}(\xi,\upsilon)=R(A(\xi,\upsilon),\upsilon)$. La matriz
	jaconiana de la composici\'{o}n en un punto $(\xi,\upsilon)$ est\'{a}
	dada por
	\begin{align*}
		\jacobiana[(0,0)]{(F\circ\varphi^{-1})} & \,=\,
		\sbox0{$\id[r]$}
		\sbox1{$\derivada{\tilde{R}^{i}}{\xi^{j}}$}
		\left[
		\begin{array}{c|c}
			\vphantom{\usebox{1}}\usebox{0} &
			\makebox[\wd0]{$0$} \\
			\hline
			\usebox{1} &
			\makebox{$\derivada{\tilde{R}^{i}}{\upsilon^{j}}$}
		\end{array}
		\right]
		\text{ .}
	\end{align*}
	%
	Como $F$ tiene rango exactamente $r$ en todo $U$ y $\varphi$ es
	difeomorfismo, $\jacobiana[(\xi,\upsilon)]{(F\circ\varphi^{-1})}$
	tiene rango $r$ en todo par $(\xi,\upsilon)$. Entonces debe valer
	que
	\begin{math}
		\left[\begin{smallmatrix}
			\derivada{\tilde{R}^{i}}{\upsilon^{j}}
		\end{smallmatrix}\right]
	\end{math}
	es la matriz nula, es decir,
	$\derivada{\tilde{R}^{i}}{\upsilon^{j}}(\xi,\upsilon)=0$ para todo
	$(\xi,\upsilon)\in\widehat{U}_{0}=\cubo{\epsilon}{0,0}$. En otras
	palabras, $\tilde{R}$ no depende de $\upsilon$. Sea
	$S(\xi)=\tilde{R}(\xi,0)$. Entonces
	\begin{align*}
		F\circ\varphi^{-1}(\xi,\upsilon) & \,=\,
			(\xi,\tilde{R}(\xi,\upsilon)) \,=\,
			(\xi,S(\xi))
		\text{ .}
	\end{align*}
	%
	Esto es casi lo que buscamos, pues $F\circ\varphi^{-1}$ es la
	identidad en las primeras $r$ coordenadas.

	Sea $V_{0}$ el subconjunto definido por
	\begin{align*}
		V_{0} & \,=\,\left\lbrace (v,w)\in V\,:\,
			(v,0)\in\widehat{U}_{0}=\cubo{\epsilon}{0,0}
			\right\rbrace
		\text{ .}
	\end{align*}
	%
	Es decir, $V_{0}$ es la preimagen por
	\begin{align*}
		\lambda v.\lambda w.(v,0) & \,:\,
			V\,\rightarrow\,\widehat{U}_{0}
		\text{ ,}
	\end{align*}
	%
	que es continua. Entonces $V_{0}$ es abierto y contiene al
	punto $(0,0)$. Ahora bien, si $(\xi,\upsilon)\in\widehat{U}_{0}$,
	entonces
	\begin{align*}
		F\circ\varphi^{-1}(\xi,\upsilon) & \,=\,
			(\xi,S(\xi))
		\text{ .}
	\end{align*}
	%
	Como $(\xi,0)\in\widehat{U}_{0}$ (porque es un cubo), vale que
	$F\circ\varphi^{-1}(\xi,\upsilon)\in V_{0}$. As\'{\i},
	$F\circ\varphi^{-1}(\widehat{U}_{0})\subset V_{0}$ y
	$F(U_{0})\subset V_{0}$. Ahora hay que definir un cambio de
	coordenadas, un difeomorfismo, en $V_{0}$ de manera que, al ser
	restringido a la imagen $F(U_{0})$, coincida con la proyecci\'{o}n
	en las primeras $r$ coordenadas. Sea
	$\psi:\,V_{0}\rightarrow\bb{R}^{n}$ dada por
	$\psi(v,w)=(v,w-S(v))$. Esta funci\'{o}n es invertible, con inversa
	dada por $\psi^{-1}(t,u)=(t,u+S(t))$. En particular,
	$\psi$ y $\psi^{-1}$ son suaves y $\psi$ es un difeomorfismo en la
	imagen. Componiendo,
	\begin{align*}
		\psi\circ F\circ\varphi^{-1}(\xi,\upsilon) & \,=\,
			\psi(\xi,S(\xi)) \,=\,(\xi,0)
		\text{ .}
	\end{align*}
	%
\end{proof}

\begin{coroDelRangoLineal}\label{thm:localmentelineal}
	Sea $F:\,M\rightarrow N$ una transformaci\'{o} suave entre variedades
	sin borde. Supongamos que $M$ es conexa. Entonces $F$ es de rango
	constante, si y s\'{o}lo si, para cada $p\in M$, existen entornos
	coordenados de $p$ y de $F(p)$ con respecto a los cuales la
	representaci\'{o}n de $F$ en coordenadas es lineal.
\end{coroDelRangoLineal}

\begin{proof}
	Si $F$ tiene rango constante, por el teorema \ref{thm:delrango},
	$F$ tiene una expresi\'{o}n lineal en coordenadas (m\'{a}s
	precisamente, es una proyecci\'{o}n). Rec\'{\i}procamente, si
	$F$ tiene una expresi\'{o}n lineal en coordenadas cerca de cada
	punto, entonces, como el rango de una transformaci\'{o}n lineal
	es constante, $\rango{F}$ es localmente constante.
	Como $M$ es conexa, $\rango{F}$ debe ser constante.
\end{proof}

\begin{coroDelRangoGlobal}\label{thm:delrangoglobal}
	Sean $M$ y $N$ variedades sin borde y sea $F:\,M\rightarrow N$ de
	rango constante $r$. si $F$ es sobreyectiva, entonces es
	submersi\'{o}n ($r=\dim\,N$); si $F$ es inyectiva, entonces
	es inmersi\'{o}n ($r=\dim\,M$); si $F$ es biyectiva, entonces es
	un difeomorfismo.
\end{coroDelRangoGlobal}

\begin{proof}
	Si $r<\dim\,N$, para cada punto $p\in M$, existen entorns coordenados
	$U_{p}$ de $p$ y $V_{p}$ de $F(p)$ tales que
	$\widehat{F}(x)=(\lista*{x}{r},\,0,\,\dots,\,0)$. Sea
	$U_{p}'\subset U_{p}$ un entorno de $p$ con clausura compacta
	contenida en $U_{p}$. Entonces $F(\clos{U_{p}'})$ es compacto y
	est\'{a} contenido en $V\cap\{y^{r+1}=\,\cdots\,=y^{n}=0\}$. Los
	conjuntos $U_{p}'$ cubren $M$. Sea $\{U_{i}\}_{i}$ un subcubrimiento
	numerable. Como los conjuntos $F(\clos{U_{i}})$ son cerrados
	nunca densos y
	\begin{align*}
		F(M) & \,=\,F\Big(\bigcup_{i\geq 1}\,\clos{U_{i}}\Big)
			\,=\,\bigcup_{i\geq 1}\,F(\clos{U_{i}})
		\text{ ,}
	\end{align*}
	%
	no puede ser que $F(M)=N$, pues $N$ es localmente compacto
	Hausdorff.

	Si $F$ no es una inmersi\'{o}n ($r<\dim\,M$), entonces, dado
	un punto $p\in M$, en un entorno suficientemente peque\~{n}o,
	\begin{align*}
		\widehat{F}(x^{1},\,\dots,\,x^{r},\,x^{r+1},\,\dots,\,x^{m})
			& \,=\,(\lista*{x}{r},\,0,\,\dots,\,0)
		\text{ .}
	\end{align*}
	%
	En particular, $\widehat{F}(0,\,dots,\,0,\,x^{r+1},\,\dots,\,x^{m})=%
	\widehat{F}(0,\,\dots,\,0,\,0,\,\dots,\,0)$ para $|x^{i}|<\epsilon$ y
	$F$ no es inyectiva.

	So $F$ es biyectiva, entonces debe ser inmersi\'{o}n y submersi\'{o}n.
	Por el teorema de la funci\'{o}n inversa, es difeomorfismo local.
	Como es biyectiva, su inversa coincide localmente con una funci\'{o}n
	suave. En definitiva, $F$ es difeomorfismo.
\end{proof}

Consideremos la inclusi\'{o}n del semiespacio $\hemi[n]$ en $\bb{R}^{n}$.
Esta funci\'{o}n es suave y su diferencial es un isomorfismo en todos los
puntos. Pero, por invarianza de dominio, no pueden ser localmente
difeomorfos. Espec\'{\i}ficamente, no hay ning\'{u}n entorno de un punto
$x\in\borde[{\hemi[n]}]$ que sea difeomorfo a un abierto de $\bb{R}^{n}$.
En otras palabras, no hay un teorema general de la funci\'{o}n inversa como
\ref{thm:funinvvar} para transformaciones $F:\,M\rightarrow N$ en donde
$\borde[M]\not =\varnothing$. Aun as\'{\i}, como vimos al definir la
estructura diferencial natural en el borde de una variedad con borde, es
posible obtener un resultado similar al teorema del rango constante en
algunos casos particulares. En el caso del borde de una variedad,
$\borde[M]\hookrightarrow M$, sabemos que, dada una carta $(U,\varphi)$
para $M$ en $p\in\borde[M]$, la composici\'{o}n
\begin{align*}
	\varphi\circ\inc[{\borde[M]}]\circ\overline{\varphi}^{-1}
		(\lista*{u}{n-1}) & \,=\,(\lista*{u}{n-1},\,0)
	\text{ ,}
\end{align*}
%
donde $\overline{\varphi}=\pi\circ\varphi\circ\inc[{\borde[M]}]:\,%
\borde[M]\cap U\rightarrow\pi(\varphi(U))$.

Sea $M$ una variedad \emph{con} borde $\borde[M]\not =\varnothing$ y sea
$F:\,M\rightarrow N$ una transformaci\'{o}n suave. Como la inclusi\'{o}n
$\inc[\interior{M}]:\,\interior{M}\hookrightarrow M$ es suave y tiene rango
m\'{a}ximo, en los puntos del interior de $M$, podemos aplicar el teorema
del rango constante, de cumplir $F$ con las condiciones del mismo, para
obtener una representaci\'{o}n de $F$ alrededor de cada punto del interior
como una proyecci\'{o}n. Pero, en un punto $p\in\borde[M]$ del borde de
$M$, esto no es cierto \emph{a priori}.

\begin{teoInmersionConBorde}[Inmersi\'{o}n de variedades con borde]%
	\label{thm:inmersionconborde}
	Sea $M$ una variedad con borde $\borde[M]\not =\varnothing$ y sea
	$m=\dim\,M$. Sea $N$ una variedad \emph{sin} borde de dimensi\'{o}n
	$n$ y sea $F:\,M\rightarrow N$ una inmersi\'{o}n suave. Si
	$p\in\borde[M]$, existe una carta $(U,\varphi)$ para $M$ en $p$
	y existe una carta $(V,\psi)$ para $N$ en $F(p)$ tales que
	$F(U)\subset V$ y
	\begin{align*}
		\widehat{F}(\lista*{x}{m}) & \,=\,
			(\lista*{x}{m},\,0,\,\dots,\,0)
		\text{ ,}
	\end{align*}
	%
	donde $\widehat{F}=\psi\circ F\circ\varphi^{-1}$.
\end{teoInmersionConBorde}

\begin{proof}
	Podemos suponer que $N=V\subset\bb{R}^{n}$ y que
	$M=U\subset\hemi[n]$ son abiertos y que $p=0$ y que $F(p)=0$.
	Por definici\'{o}n, existe $\widetilde{F}:\,W\rightarrow V$ definida
	en un abierto de $\bb{R}^{m}$ con $p\in W$ y
	$\widetilde{F}|_{W\cap U}=F$. En particular,
	\begin{align*}
		\diferencial[0]{\widetilde{F}} & \,=\,\diferencial[0]{F}
		\text{ .}
	\end{align*}
	%
	Como $F$ tiene rango m\'{a}ximo en $0$, la extensi\'{o}n
	$\widetilde{F}$ debe tener rango m\'{a}ximo en $0$. Achicando $W$,
	podemos suponer que $\widetilde{F}$ tiene rango m\'{a}ximo en
	todo el abierto $W$. Por el teorema del rango, existen cartas
	$(W_{0},\gamma_{0})$ para $\bb{R}^{m}$ centrada en $0$ y
	$(V_{0},\psi_{0})$ para $\bb{R}^{n}$ centrada en $0$ tales que
	\begin{align*}
		\psi_{0}\circ\widetilde{F}\circ\gamma_{0}^{-1}
			(\lista*{x}{m}) & \,=\,
			(\lista*{x}{m},\,0,\,\dots,\,0)
		\text{ .}
	\end{align*}
	%
	Pero, al elegir estas cartas, no hay control sobre lo que pasa con
	el borde de $M$: la imagen $\gamma_{0}(\borde[M]\cap W_{0})$
	podr\'{\i}a ser cualquier cosa --casi. Como
	$\gamma_{0}:\,W_{0}\rightarrow\widehat{W}_{0}=\gamma_{0}(W_{0})%
	\subset\bb{R}^{m}$ es un difeomorfismo, el producto
	$\gamma_{0}\times\id[\bb{R}^{n-m}]:\,W_{0}\times\bb{R}^{n-m}%
	\rightarrow\widehat{W}_{0}\times\bb{R}^{n-m}$ es un difeomorfismo,
	tambi\'{e}n. La composici\'{o}n
	$\psi=(\gamma_{0}^{-1}\times\id[\bb{R}^{n-m}])\circ\psi_{0}$ es un
	difeomorfismo definido en un entorno $V_{1}\subset V_{0}$ de $0$.
	% $V_{0}\cap(\widehat{W}_{0}\times\bb{R}^{n-m})$ de $0\in\bb{R}^{n}$.
	As\'{\i}, $(V_{1},\psi)$ es una carta para $V$ centrada en $0$ y
	\begin{align*}
		\psi\circ F (x) & \,=\,
			(\gamma_{0}^{-1}\times\id[\bb{R}^{n-m}])\circ
			(\psi_{0}\circ F\circ\gamma_{0}^{-1})\circ
			\gamma_{0}(x) \\
		& \,=\,(\gamma_{0}^{-1}\times\id[\bb{R}^{n-m}])
			(\gamma_{0}^{1}(x),\,\dots,\,\gamma_{0}^{m}(x),\,0,\,
				\dots,\,0) \,=\,(x,0)
		\text{ .}
	\end{align*}
	%
	En definitiva, usando las coordenadas originales de $M$ y $\psi$
	se obtiene una erpresentaci\'{o}n de $F$ de la forma deseada.
\end{proof}

\begin{obsInmersionConBorde}\label{obs:inmersionconborde}
	Supongamos que $F$ es submersi\'{o}n y $\borde[M]\not =\varnothing$
	como en el teorema \ref{thm:inmersionconborde}. Entonces argumentando
	de manera similar, podemos hallar cartas tales que
	\begin{align*}
		& F\circ (\gamma_{0}^{-1}\circ
			(\psi_{0}\times\id[\bb{R}^{m-n}]))(x) \\
		& \qquad\qquad
			\,=\,\psi_{0}^{-1}(\psi_{0}\circ F\circ\gamma_{0}^{-1})
				(\psi_{0}(\lista*{x}{n}),\,x^{n+1},\,\dots,\,
					x^{m}) \\
		& \qquad\qquad
			\,=\, (\lista*{x}{n})
		\text{ .}
	\end{align*}
	%
	Pero ?` es $(\psi_{0}^{-1}\times\id[\bb{R}^{m-n}])\circ\gamma_{0}$
	una carta de borde? En general, no, pues ser\'{a} de borde, si
	y s\'{o}lo si $\gamma_{0}$ lo es\dots

	?`Qu\'{e} pasa en el caso en que $\borde[N]\not=\varnothing$ y
	$\borde[M]=\varnothing$ y $F:\,M\rightarrow N$ sea submersi\'{o}n?
	En vez de considerar una extensi\'{o}n $\widetilde{F}$, habr\'{a}
	que considerar la composici\'{o}n $F_{0}=\inc[{\hemi[n]}]\circ F$.
	Como $N$ tiene borde, no est\'{a} garantizado que existan secciones
	locales suaves para $F$ y que, por lo tanto, $F$ sea abierta. Pero
	$F_{0}$ es submersi\'{o}n (una cuesti\'{o}n de n\'{u}meros y regla
	de la cadena) entre variedades \emph{sin} borde. Sea $U$ un entorno
	coordenado en $M$ centrado en $p$ y sea $V$ un entorno coordenado
	en $N$ centrado en $F(p)$. Entonces, el problema queda reducido
	al caso en que $M$ es un abierto de $\bb{R}^{m}$ y $N$ es un
	abierto de $\hemi[n]$ ($n\leq m$), $p=0$ y $F(p)=0$. Componiendo
	con la inclusi\'{o}n de $\hemi[n]$ en $\bb{R}^{n}$, la funci\'{o}n
	$F_{0}$ es submersi\'{o}n (su rango es igual al rango de $F$).
	Por el teorema del rango, existen cartas $(U_{0},\varphi_{0})$ para
	$\bb{R}^{m}$ centrada en $p=0$ y $(V_{0},\psi_{0})$ para
	$\bb{R}^{n}$ centrada en $F(p)=0$ de manera que
	\begin{align*}
		\psi_{0}\circ (\inc[{\hemi[n]}]\circ F)\circ\varphi_{0}^{-1}
			(x^{1},\,\dots,\,x^{n},\,x^{n+1},\,\dots,\,x^{m}) &
			\,=\,(\lista*{x}{n})
		\text{ .}
	\end{align*}
	%
	Completamos $\psi_{0}$ a un difeo definido en un abierto de
	$\bb{R}^{m}$: concretamente, $\psi_{0}\times\id[\bb{R}^{m-n}]$ es
	un difeomorfismo de $V_{0}\times\bb{R}^{m-n}$ en
	$\widehat{V}_{0}\times\bb{R}^{m-n}$, digamos, que es un entorno de
	$0\in\bb{R}^{m}$. Entonces la composici\'{o}n
	$(\psi_{0}^{-1}\times\id[\bb{R}^{m-n}])\circ\varphi_{0}^{-1}$ es un
	difeomorfismo definido en alg\'{u}n entorno de $0$ y vale que
	\begin{align*}
		& \inc[{\hemi[n]}]\circ F\circ
			(\varphi_{0}^{-1}\circ
				(\psi_{0}\times\id[\bb{R}^{m-n}]))
			(x^{1},\,\dots,\,x^{n},\,x^{n+1},\,\dots,\,x^{m}) \\
		&\qquad\qquad
			\,=\,\psi_{0}^{-1}\circ
				(\psi_{0}\circ F_{0}\circ\varphi_{0}^{-1})
				(\psi_{0}(x^{1},\,\dots,\,x^{n}),\,x^{n+1},\,
				\dots,\,x^{m}) \\
		& \qquad\qquad
			\,=\,\psi_{0}^{-1}(\psi_{0}^{1}(x),\,
				\dots,\,\psi_{0}^{n}(x)) \\
		& \qquad\qquad
			\,=\,(\lista*{x}{n})
		\text{ .}
	\end{align*}
	%

	Del hecho de que $F_{0}$ es submersi\'{o}n, podemos deducir que
	$F_{0}:\,U\rightarrow \widetilde{V}$ es abierta (porque existen
	secciones locales suaves), donde $\widetilde{V}\subset\bb{R}^{n}$ es
	abierto tal que $V=\hemi[n]\cap\widetilde{V}$ (podemos tomar una
	semibola y la bola correspondiente). Pero, como su imagen est\'{a}
	contenida en $\hemi[n]$, los puntos de $F_{0}(U)$ deben pertenecer al
	interior de $V$, a $\interior{\hemi[n]}\cap V$. Sabiendo que
	$F$ se correstringe a una submersi\'{o}n entre variedades sin borde,
	podemos aplicar el teorema del rango sin problema.
\end{obsInmersionConBorde}
