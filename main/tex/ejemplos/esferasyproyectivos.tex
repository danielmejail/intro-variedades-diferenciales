



\begin{ejemplo}\nom{La esfera}
	Sea $n\geq 0$ y sea $\esfera{n}$ la esfera de dimensi\'{o}n $n$ en
	$\bb{R}^{n+1}$. Por ser subespacio de$\bb{R}^{n+1}$, es un espacio
	$T_{2}$ y $N_{2}$. Para cada $i\in[\![1,n+1]\!]$ sea $U_{i}^{+}$ el
	subconjunto de $\bb{R}^{n+1}$ definido por
	\begin{align*}
		U_{i}^{+} & \,=\,\big\lbrace (\lista*{x}{n+1})\,:\,x^{i}>0
			\big\rbrace
		\text{ .}
	\end{align*}
	%
	De manera an\'{a}loga, se define $U_{i}^{-}$ como el subconjunto en
	donde la coordenada $x^{i}$ es negativa (estrictamente). Si se
	define una funci\'{o}n $f:\,\bola{1}{0}\rightarrow\bb{R}$ por
	\begin{align*}
		f(u) & \,=\,\sqrt{1-|u|^{2}}\text{ ,}
	\end{align*}
	%
	entonces $f$ es continua y, para cada $i$, el subconjunto
	$U_{i}^{+}\cap\esfera{n}$ de la esfera es igual al gr\'{a}fico de
	la funci\'{o}n
	\begin{align*}
		x^{i} & \,=\,f(x^{1},\,\dots,\,\widehat{x^{i}},\,\dots,\,
			x^{n+1})
	\end{align*}
	%
	Similarmente, $U_{i}^{-}\cap\esfera{n}$ es el gr\'{a}fico de
	$x^{i}=-f(x^{1},\,\dots,\,\widehat{x^{i}},\,\dots,\,x^{n+1})$. De
	esta manera, se ve cada abierto $U_{i}^{\pm}\cap\esfera{n}$ de la
	esfera es localmente euclideo de dimensi\'{o}n $n$, por ser
	gr\'{a}ficos de funciones continuas, y que las coordenadas
	$\varphi_{i}^{\pm}:\,U_{i}^{\pm}\cap\esfera\rightarrow\bola{1}{0}$
	dadas por
	\begin{align*}
		\varphi_{i}^{\pm}(\lista*{x}{n+1}) & \,=\,
			(x^{1},\,\dots,\,\widehat{x^{i}},\,\dots,\,x^{n+1})
	\end{align*}
	%
	son las coordenadas correspondientes. Dado que los dominios de estas
	cartas $(U_{i}^{\pm}\cap\esfera{n},\varphi_{i}^{\pm})$ cubren a
	$\esfera{n}$, se deduce que $\esfera{n}$ es una variedad
	topol\'{o}gica de dimensi\'{o}n $n$.

	Otro juego de cartas que tambi\'{e}n aparece con $\esfera{n}$ es
	el formado por las proyecciones estereogr\'{a}ficas. Sea $N$ el
	punto $(0,\,\dots,\,0,\,1)\in\esfera{n}\subset\bb{R}^{n+1}$ y
	sea $S=(0,\,\dots,\,0,\,-1)$ su ant\'{\i}poda. Sea
	$\sigma:\,\esfera{n}\setmin\{N\}\rightarrow\bb{R}^{n}$ la
	funci\'{o}n definida por
	\begin{align*}
		\sigma(\lista*{x}{n+1}) & \,=\,
			\frac{(\lista*{x}{n})}{1-x^{n+1}}
	\end{align*}
	%
	y sea $\tilde{\sigma}:\,\esfera{n}\setmin\{S\}\rightarrow\bb{R}^{n}$
	la funci\'{o}n
	\begin{align*}
		\tilde{\sigma}(x) & \,=\,-\sigma(-x) \,=\,
			\frac{(\lista*{x}{n})}{1+x^{n+1}}
		\text{ .}
	\end{align*}
	%
	Llamemos $H=\{x^{n+1}=0\}$. Entonces $\sigma(x)=u$, donde
	$(u,0)$ es el punto en que la recta que pasa por $N$ y por $x$
	interseca a $H$. De manera an\'{a}loga, $\tilde{\sigma}(x)=u$, donde
	$(u,0)$ es el punto en donde la recta que pasa por $S$ y por $x$
	interseca a $H$. La funci\'{o}n $\sigma$ es invertible con inversa
	\begin{align*}
		\sigma^{-1}(\lista*{u}{n}) & \,=\,
			\frac{(\lista*{2u}{n},\,|u|^{2}-1)}{|u|^{2}+1}
		\text{ .}
	\end{align*}
	%
	La composici\'{o}n $\tilde{\sigma}\circ\sigma^{-1}$ en un punto
	$u=(\lista*{u}{n})$ es igual a
	\begin{align*}
		\tilde{\sigma}(\sigma^{-1}(u)) & \,=\,
			-\sigma(-\sigma^{-1}(u)) \\
		&\,=\,\frac{(\lista*{2u}{n})}{|u|^{2}+1}\cdot
			\frac{1}{1+(\frac{|u|^{2}-1}{|u|^{2}+1})} \\
		& \,=\, \frac{1}{|u|^{2}}u
		\text{ .}
	\end{align*}
	%
\end{ejemplo}

\begin{ejemplo}\nom{El espacio proyectivo real}
	El espacio proyectivo real $\proyectivo{\bb{R}}{n}$ es una variedad
	de dimensi\'{o}n $n$. Una realizaci\'{o}n posible de este espacio es
	en tanto el conjunto de subespacios vectoriales de dimensi\'{o}n $1$
	(subvariedades lineales de dimensi\'{o}n $1$) en $\bb{R}^{n+1}$. La
	topolog\'{\i}a de este espacio es la topolog\'{\i}a cociente
	determinada por la aplicaci\'{o}n sobreyectiva
	$\pi:\,\bb{R}^{n+1}\setmin\{0\}\rightarrow\proyectivo{\bb{R}}{n}$
	que a un punto $x\not =0$ le asigna el subespacio que genera,
	denotado $[x]$.

	Para cada $i\in[\![1,n+1]\!]$, sea $\widetilde{U}_{i}$ el
	subconjunto de $\bb{R}^{n+1}\setmin\{0\}$ donde la coordenada
	$x^{i}$ es no nula. Sea $U_{i}=\pi(\widetilde{U}_{i})$ la
	proyecci\'{o}n correspondiente en el espacio proyectivo. Como
	$\widetilde{U}_{i}$ es un abierto saturado, el subconjunto $U_{i}$
	es abierto y la restricci\'{o}n de $\pi$ a $\widetilde{U}_{i}$ es
	una aplicaci\'{o}n cociente sobre $U_{i}$. Sea
	$\varphi_{i}:\,U_{i}\rightarrow\bb{R}^{n}$ la aplicaci\'{o}n dada por
	\begin{align*}
		\varphi_{i}[x^{1}\,:\cdots:\,x^{n+1}] & \,=\,
			\Big(\frac{x^{1}}{x^{i}},\,\dots,\,
			\frac{x^{i-1}}{x^{i}},\,\frac{x^{i+1}}{x^{i}},
			\,\dots,\,\frac{x^{n+1}}{x^{i}}
			\Big)
		\text{ .}
	\end{align*}
	%
	Dado que la composici\'{o}n $\varphi\circ\pi$ es continua (de
	hecho podemos ir adelantando que es suave entre un abierto de
	$\bb{R}^{n+1}$ y un abierto de $\bb{R}^{n}$), la funci\'{o}n
	$\varphi_{i}$ es continua, por la propiedad caracter\'{\i}stica
	del cociente (de $\pi|_{\widetilde{U}_{i}}$). Esta funci\'{o}n
	tiene una inversa dada por
	\begin{align*}
		\varphi_{i}^{-1}(\lista*{u}{n}) & \,=\,
			[u^{1}\,:\dots:\,u^{i-1}\,:\,1\,:\,
			u^{i}\,:\dots:\,u^{n}]
		\text{ .}
	\end{align*}
	%
	Esta inversa tambi\'{e}n es continua y $\varphi$ es un
	homeomorfismo. La interpretaci\'{o}n geom\'{e}trica de $\varphi[x]=u$
	es que el punto $(u,1)$ es aquel por donde $[x]$ cruza al hiperplano
	$x^{i}=1$. Dado que los abiertos $U_{1},\,\dots,\,U_{n+1}$ cubre al
	espacio proyectivo, $\proyectivo{\bb{R}}{n}$ es localmente euclideo
	de dimensi\'{o}n $n$.

	Para ver que los espacios proyectivos reales son espacios Hausdorff,
	dados dos subespacios lineales de dimensi\'{o}n $1$ distintos,
	$\xi,\upsilon$, en $\bb{R}^{n+1}$, existen abiertos
	\emph{c\'{o}nicos} disjuntos (el punto $\{0\}$ se omite) cada uno
	de los cuales contiene a uno de ellos, es decir, a las
	l\'{\i}neas. Estos abiertos son saturados respecto de la
	suryecci\'{o}n natural $\pi$ y por lo tanto sus im\'{a}genes son
	abiertos disjuntos de $\proyectivo{\bb{R}}{n}$ que contienen a
	cada uno de las clases $\xi$ y $\upsilon$. Para que quede un poco
	m\'{a}s claro, podemos restringir $\pi$ a la esferera
	$\esfera{n}\subset\bb{R}^{n+1}\setmin\{0\}$. La restricci\'{o}n
	sigue siendo sobreyectiva y cociente. Dada una clase
	$\xi\in\proyectivo{\bb{R}}{n}$, podemos tomar como representante
	a cualquiera de los dos puntos de norma $1$ en $\xi\cap\esfera{n}$.
	Llamemos $x$ a dicho punto. De manera similar podr\'{\i}amos tomar
	el punto $-x$. Si $y\in\esfera{n}$ es un punto distinto de $x$ y de
	$-x$, existen abiertos disjuntos $\widetilde{U}$ y $\widetilde{V}$
	que contienen a $x$ y a $y$. Los abiertos $-\widetilde{U}$ y
	$-\widetilde{V}$ contienen a los puntos $-x$ y $-y$ y, tomando
	abiertos m\'{a}s peque\~{n}os de ser necesario, podemos suponera
	que los cuatro abiertos son disjuntos de a pares. De esta manera,
	se obtienen abiertos saturados
	$\widetilde{U}\cup\big(-\widetilde{U}\big)$ y
	$\widetilde{V}\cup\big(-\widetilde{V}\big)$ que contienen a $x$ y a
	$y$ y que, adem\'{a}s, son disjuntos. Proyectando, se obtienen
	abiertos disjuntos $U$ y $V$ de $\proyectivo{\bb{R}}{n}$ que
	contienen a $x$ y a $y$, respectivamente.

	Siendo imagen por una funci\'{o}n continua de un espacio compacto
	$\pi(\esfera{n})=\proyectivo{\bb{R}}{n}$, el espacio proyectivo
	es compacto para todo $n\geq 0$. Para ver que
	$\proyectivo{\bb{R}}{n}$ es $N_{2}$, alcanza con notar que se
	puede cubrir el espacio con numerables (finitas) cartas. En
	definitiva, los espacios $\proyectivo{\bb{R}}{n}$ son variedades
	topol\'{o}gicas compactas.
	
	Otra descripci\'{o}n --aunque esencialmente la misma-- del espacio
	proyectivo est\'{a} dada por dejar actuar al grupo $\bb{R}^{\times}$
	de reales distintos de cero sobre $\bb{R}^{n+1}\setmin\{0\}$ por
	multiplicaci\'{o}n por escalares. Las \'{o}rbitas de esta acci\'{o}n
	son, precisamente, los subespacios reales de dimensi\'{o}n $1$. Dado
	que el cociente $\bb{R}^{\times}\backslash(\bb{R}^{n+1}\setmin\{0\})$
	es homeomorfo a $\{\pm1\}\backslash\esfera{n}$ podemos describir a
	$\proyectivo{\bb{R}}{n}$ como el cociente de un espacio
	topol\'{o}gico (de una variedad compacta) por la acci\'{o}n de un
	grupo discreto (en particular actuando de manera propiamente
	discontinua). Es esto lo que nos permite deducir que los espacios
	proyectivos son variedades topol\'{o}gicas y, adem\'{a}s, compactas.
\end{ejemplo}

\begin{ejemplo}\nom{Una estructura de variedad diferencial en la esfera}
	Las esferas son variedades diferenciales. Usando las cartas
	$(U_{i}^{\pm},\varphi_{i}^{\pm})$ se obtiene un atlas compatible.
	S\'{o}lo hay que verificar que las composiciones
	$\varphi_{i}^{\pm}\circ(\varphi_{j}^{\pm})^{-1}$ sean diferenciables.
	Esta estructura en $\esfera{n}$ se denominar\'{a} la
	\emph{estructura usual} en $\esfera{n}$.

	Las cartas correspondientes a las proyecciones esterogr\'{a}ficas
	son compatibles entre s\'{\i} y, adem\'{a}s, son compatibles
	con la estructura usual de $\esfera{n}$. Veamos primero la
	compatibilidad de $\sigma$ con $\varphi_{i}^{\epsilon}$
	para $i\not =n+1$ y $\epsilon=+\,(>)\text{ o }-\,(<)$. Por un lado,
	en este	caso,
	\begin{align*}
		\big(U_{i}^{\epsilon}\cap\esfera{n}\big)\cap
			\big(\esfera{n}\setmin\{N\}\big) & \,=\,
			U_{i}^{\epsilon}\cap\esfera{n}
		\text{ .}
	\end{align*}
	%
	Entonces, dado que
	\begin{align*}
		\varphi_{i}^{\epsilon}(U_{i}^{\epsilon}\cap\esfera{n}) &
			\,=\,\bola{1}{0}\quad\text{y} \\
		\sigma(U_{i}^{\epsilon}\cap\esfera{n}) & \,=\,
			\left\lbrace u\in\bb{R}^{n}\,:\,u^{i}\epsilon 0
				\right\rbrace
		\text{ ,}
	\end{align*}
	%
	la composici\'{o}n $\varphi_{i}^{\epsilon}\circ\sigma^{-1}:\,%
	\{u^{i}\epsilon 0\}\rightarrow\bola{1}{0}$, dada por
	\begin{align*}
		\varphi_{i}^{\epsilon}\circ\sigma^{-1}(u) & \,=\,
			\frac{(2u^{1},\,\dots,\,\widehat{2u^{i}},\,2u^{n},\,%
				|u|^{2}-1)}{|u|^{2}+1}
		\text{ ,}
	\end{align*}
	%
	es suave (, inyectiva) y su matriz de derivadas parciales es
	no singular en todo punto, como se puede verificar, derivando la
	expresi\'{o}n anterior respecto de cada una de las variables
	$u^{k}$. Esto es suficiente para concluir que la composici\'{o}n
	en el sentido inverso tambi\'{e}n es suave. En todo caso, se
	puede verificar directamente que es suave: como $i\not =n+1$,
	vale que
	\begin{align*}
		\sigma\circ(\varphi_{i}^{\epsilon})^{-1}(v) & \,=\,
			\frac{(v^{1},\,\dots,\,\sqrt{1-|v|^{2}},%
				\,\dots,\,v^{n-1})}{1-v^{n}}
		\text{ .}
	\end{align*}
	%
	Como $|v|<1$, en particular $|v^{n}|<1$ y la composici\'{o}n es
	suave de $B_{1}(0)$ en $\{u^{i}\epsilon 0\}$.

	Si $i=n+1$, $U_{n+1}^{+}\cap\esfera{n}$ contiene a $N$ pero
	$U_{n+1}^{-}\cap\esfera{n}$ no lo contiene. Veamos primero que
	$\varphi_{n+1}^{-}$ es compatible con $\sigma$: la composici\'{o}n
	\begin{align*}
		\varphi_{n+1}^{-}\circ\sigma^{-1}(u) & \,=\,
			\frac{(\lista*{2u}{n})}{|u|^{2}+1}
	\end{align*}
	%
	es suave de $\sigma(U_{n+1}^{-}\cap\esfera{n})=\bola{1}{0}$ en
	$\varphi_{n+1}^{-}(U_{n+1}^{-}\cap\esfera{n})=\bola{1}{0}$ y, en el
	sentido contrario,
	\begin{align*}
		\sigma\circ(\varphi_{n+1}^{-})^{-1}(v) & \,=\,
			\frac{(\lista*{v}{n})}{1+\sqrt{1-|v|^{2}}}
	\end{align*}
	%
	que es suave tambi\'{e}n. De manera similar, se puede comprobar que
	la \emph{otra} proyecci\'{o}n estereogr\'{a}fica $\tilde{\sigma}$
	es compatible con $\varphi_{n+1}^{+}$ y, como $\sigma$ y
	$\tilde{\sigma}$ son compatibles, por tansitividad de la relaci\'{o}n
	de compatibilidad suave, $\sigma$ es compactible con
	$\varphi_{n+1}^{+}$, tambi\'{e}n. Tambi\'{e}n se puede verificar
	directamente: si llamamos $U^{+}$ al abierto
	\begin{align*}
		U^{+} & \,=\,\big(U_{n+1}^{+}\cap\esfera{n}\big)\cap
			\big(\esfera{n}\setmin\{N\}\big)
			\text{ ,}\quad\text{entonces} \\
		\sigma(U^{+}) & \,=\,\left\lbrace u\in\bb{R}^{n}\,:\,
					|u|>1\right\rbrace
			\quad\text{y} \\
		\varphi_{n+1}^{+}(U^{+}) & \,=\,\bola{1}{0}\setmin\{0\}
		\text{ .}
	\end{align*}
	%
	Ahora bien,
	\begin{math}
		\varphi_{n+1}^{+}\circ\sigma^{-1}(u) \,=\,
			\frac{(\lista*{2u}{n})}{|u|^{2}+1}
	\end{math}
	%
	que es suave y
	\begin{math}
		\sigma\circ(\varphi_{n+1}^{+})^{-1}(v) \,=\,
			\frac{(\lista*{v}{n})}{1-\sqrt{1-|v|^{2}}}
	\end{math}
	%
	que tambi\'{e}n es suave en su dominio de definici\'{o}n.
\end{ejemplo}

\begin{ejemplo}\nom{Una estructura diferencial en el espacio proyectivo}
	El espacio proyectivo $\proyectivo{n}{\bb{R}}$ tambi\'{e}n tiene
	estructura de variedad diferencial. Los cambios de coordenadas
	son diferenciables, pues, si $i>j$,
	\begin{align*}
		\varphi_{j}\circ\varphi_{i}^{-1}(\lista*{u}{n}) & \,=\,
		\Big(\frac{u^{1}}{u^{j}},\,\dots,\,\frac{u^{j-1}}{u^{j}},\,
			\frac{u^{j+1}}{u^{j}},\,\dots,\,\frac{u^{i-1}}{u^{j}},\,
			\frac{1}{u^{j}},\,\frac{u^{i}}{u^{j}},\,\dots,\,
				\frac{u^{n}}{u^{j}}\Big) \\
		\varphi_{i}\circ\varphi_{j}^{-1}(\lista*{u}{n}) & \,=\,
		\Big(\frac{u^{1}}{u^{i-1}},\,\dots,\,\frac{u^{j-1}}{u^{i-1}},\,
			\frac{1}{u^{i-1}},\,\frac{u^{j}}{u^{i-1}},\,\dots,\,
			\frac{u^{i-2}}{u^{i-1}},\,\frac{u^{i}}{u^{i-1}},\,
			\dots,\,\frac{u^{n}}{u^{i-1}}\Big)
		\text{ .}
	\end{align*}
	%
\end{ejemplo}

\begin{ejemplo}\nom{El espacio proyectivo complejo}
	De la misma manera en que se defini\'{o} el espacio proyectivo real,
	se puede definir el espacio proyectivo sobre el cuerpo de n\'{u}meros
	complejos, denotado $\proyectivo{\bb{C}}{n}$, como el conjunto de
	subespacios vectoriales \emph{complejos} de dimensi\'{o}n $1$ en
	$\bb{C}^{n+1}$. Es decir, $\proyectivo{\bb{C}}{n}$ es el
	cociente $(\bb{C}^{n+1}\setmin\{0\})/\sim$ donde dos puntos
	$z=(\lista*{z}{n+1})$ y $w=(\lista*{w}{n+1})$ est\'{a}n relacionados,
	si generan el mismo espacio \emph{sobre $\bb{C}$}, dicho de otra
	manera, si existe un n\'{u}mero complejo no nulo
	$\alpha\in\bb{C}^{\times}$ tal que $z\cdot\alpha=w$.

	En primer lugar, para ver que $\proyectivo{\bb{C}}{n}$ es una variedad
	topol\'{o}gica y para luego ver que se le puede dar una estructura
	diferencial, definimos una correspondencia. Sea
	$\Phi:\,\bb{C}^{n+1}\rightarrow\bb{R}^{2n+2}$ la funci\'{o}n dada por
	\begin{align*}
		\Phi(\lista*{z}{n+1}) & \,=\,
			(x^{1},\,y^{1},\,\dots,\,x^{n},\,y^{n})
		\text{ ,}
	\end{align*}
	%
	donde, para cada $i$, $z^{i}=x^{i}+\sqrt{-1}y^{i}$. Si $w$ es un
	punto de la forma $z\cdot\alpha$ con $\alpha=a+\sqrt{-1}b$ y
	$a,b\in\bb{R}$,
	\begin{align*}
		\Phi(w) & \,=\, \Phi(z^{1}\cdot\alpha,\,\dots,\,
					z^{n+1}\cdot\alpha) \\
		& \,=\,(x^{1}a-y^{1}b,\,x^{1}b+y^{1}a,\,\dots,\,
			x^{n+1}a-y^{n+1}b,\,x^{n+1}b+y^{n+1}a)
		\text{ .}
	\end{align*}
	%
	La acci\'{o}n de multiplicar (a derecha) por un n\'{u}mero complejo
	no nulo $\alpha$ se traduce v\'{\i}a $\Phi$ como el producto a derecha
	por una matriz:
	\begin{align*}
		\Phi(z\cdot\alpha) & \,=\,\Phi(z)\cdot m_{\alpha}
		\text{ ,}
	\end{align*}
	%
	donde $m_{\alpha}$ es la matriz \emph{real} dada por
	\begin{align*}
		m_{\alpha} & \,=\,
			\begin{bmatrix}
				\widehat{\alpha} & & \\
				& \ddots & \\
				& & \widehat{\alpha}
			\end{bmatrix}
		\quad\text{donde}\quad
		\widehat{\alpha} \,=\,
			\begin{bmatrix}
				a & b \\
				-b & a
			\end{bmatrix}
		\text{ .}
	\end{align*}
	%
	Es decir, para $\alpha\in\bb{C}^{\times}$,
	$\widehat{\alpha}\in\MM{2,\bb{R}}$ y $m_{\alpha}\in\MM{2n+2,\bb{R}}$.
	De hecho, como $\det(\widehat{\alpha})=a^{2}+b^{2}>0$,
	$\widehat{\alpha}\in\GL{2,\bb{R}}$ y $m_{\alpha}\in\GL{2n+2,\bb{R}}$.

	Dado $\alpha\in\bb{C}^{\times}$, sea
	$f_{\alpha}:\,\bb{R}^{2n+2}\setmin\{0\}\rightarrow%
		\bb{R}^{2n+2}\setmin\{0\}$ la funci\'{o}n correspondiente a
	multiplicar a derecha por la matriz $m_{\alpha}$.

	V\'{\i}a la correspondencia $\Phi$, la acci\'{o}n de
	$\bb{C}^{\times}$ en $\bb{C}^{n+1}$ est\'{a} dada por la acci\'{o}n
	diagonal del grupo de matrices
	\begin{align*}
		G & \,=\,\left\lbrace
			\begin{bmatrix}a & b\\ -b & a\end{bmatrix}\,:\,
			a^{2}+b^{2}>0\right\rbrace
		\,\simeq\,\bb{R}_{>0}\times\SO{2}
	\end{align*}
	%
	en el producto de $n+1$ copias de $\bb{R}^{2}$. Es decir,
	\begin{align*}
		\Phi((\lista*{z}{n+1})\cdot\alpha) & \,=\,
			((x^{1},y^{1})\cdot\widehat{\alpha},\,\dots,\,
			(x^{n+1},y^{n+1})\cdot\widehat{\alpha}) \\
		& \,=\,\Phi(z)\cdot m_{\alpha}
		\text{ .}
	\end{align*}
	%
	El espacio proyectivo complejo se obtiene como el espacio de
	\'{o}rbitas de $\bb{C}^{\times}$ actuando en $\bb{C}^{n+1}\setmin\{0\}$
	por multiplicaci\'{o}n por escalares, es decir, por homotecias.
	Si llamamos $M$ al conjunto de \'{o}rbitas de la acci\'{o}n
	correspondiente de $G$ en $\bb{R}^{2n+2}\setmin\{0\}$, entonces
	$\Phi$ induce una correspondencia entre $\proyectivo{\bb{C}}{n}$ y
	$M$. En definitiva, tenemos un diagrama conmutativo, en principio, de
	conjuntos:
	\begin{center}
	\begin{tikzcd}
		\bb{C}^{n+1}\setmin\{0\} \arrow[r,"\Phi"] \arrow[d,"\pi"] &
			\bb{R}^{2n+2}\setmin\{0\} \arrow[d,"q"] \\
		\proyectivo{\bb{C}}{n} \arrow[r,"F"] & M
	\end{tikzcd} .
	\end{center}

	En cuanto a la topolog\'{\i}a, $\bb{R}^{2n+2}\setmin\{0\}$
	tiene la estructura de variedad topol\'{o}gica heredada de
	$\bb{R}^{2n+2}$ en tanto subespacio abierto y a $M$ se le da la
	topolog\'{\i}a cociente. Usando las correspondencias $\Phi$ y $F$,
	damos a $\bb{C}^{n+1}$ y a $\proyectivo{\bb{C}}{n}$ las
	topolog\'{\i}as correspondientes de manera que $\Phi$ y $F$ sean
	homeomorfismos. As\'{\i}, el diagrama anterior pasa a ser un diagrama
	conmutativo de espacios topol\'{o}gicos y funciones continuas.

	El espacio topol\'{o}gico $M$ es una variedad topol\'{o}gica. Para
	mostrar esto busquemos primero cartas para $M$. Dado que las
	funciones $f_{\alpha}:\,\bb{R}^{2n+2}\setmin\{0\}%
		\rightarrow\bb{R}^{2n+2}\setmin\{0\}$ son restricciones de
	transformaciones lineales $m_{\alpha}$, son, en particular,
	continuas. Adem\'{a}s, como dichas transformaciones son invertibles,
	\begin{align*}
		f_{\alpha}\circ f_{\alpha^{-1}} & \,=\,
			f_{\alpha^{-1}}\circ f_{\alpha} \,=\,
			\id[\bb{R}^{2n+2}\setmin\{0\}]
		\text{ ,}
	\end{align*}
	%
	de lo que se deduce que el grupo $G$ act\'{u}a en
	$\bb{R}^{2n+2}\setmin\{0\}$ v\'{\i}a homeomorfismos. Una consecuencia
	de esto es que la funci\'{o}n cociente $q$ es abierta: si
	$V\subset\bb{R}^{2n+2}\setmin\{0\}$ es abierto, entonces
	\begin{align*}
		q^{-1}\big(q(V)\big) & \,=\,V\cdot G \\
		& \,=\,\bigcup_{m_{\alpha}\in G}\,V\cdot m_{\alpha}
			\,=\,\bigcup_{\alpha\in\bb{C}^{\times}}\,
				f_{\alpha}(V)
	\end{align*}
	%
	que es una uni\'{o}n de abiertos. Como $M$ tiene la topolog\'{\i}a
	cociente inducida por $q$, el conjunto $q(V)$ es abierto.
	Ahora bien, definimos, para cada $i$,
	\begin{align*}
		\widetilde{U}_{i} & \,=\,
			\left\lbrace((x^{1},y^{1}),\,\dots,\,
			(x^{n+1},y^{n+1}))\in\bb{R}^{2n+2}\,:\,
			(x^{i},y^{i})\not = 0\right\rbrace
		\text{ .}
	\end{align*}
	%
	Como las matrices $\widehat{\alpha}$ son invertibles, se ve que
	$q^{-1}(q(\widetilde{U}_{i}))=\widetilde{U}_{i}$ para todo $i$. En
	particular, $\widetilde{U}_{i}$ es un abierto saturado respecto de
	$q$ y, si llamamos $U_{i}=q(\widetilde{U}_{i})$, entonces la
	restricci\'{o}n de $q|_{i}:\,\widetilde{U}_{i}\rightarrow U_{i}$ es
	cociente.

	Para definir cartas en $M$, definimos funciones en
	$\proyectivo{\bb{C}}{n}$ an\'{a}logas a las cartas para los espacios
	proyectivos reales y las trasladamos, v\'{\i}a $F$ a $M$.
	Notemos que $\Phi^{-1}(\widetilde{U}_{i})$ es el subconjunto de
	$\bb{C}^{n+1}\setmin\{0\}$ conformado por los puntos cuya coordenada
	$i$ es distinta de cero. Si $z\in\bb{C}^{n+1}\setmin\{0\}$,
	denotaremos
	\begin{align*}
		\pi(z) & \,=\,\pi(\lista*{z}{n+1}) \,=\,
			\left[z^{1}:\,\cdots\,:z^{n+1}\right]
	\end{align*}
	%
	al punto correpondiente en el espacio proyectivo. De manera
	an\'{a}loga los puntos de $M$ los denotaremos de la siguiente manera:
	\begin{align*}
		q((x^{1},y^{1}),\,\dots,\,(x^{n+1},y^{n+1})) & \,=\,
			\left[(x^{1},y^{1}):\,\cdots\,:(x^{n+1},y^{n+1})\right]
		\text{ .}
	\end{align*}
	%
	Sea, entonces, $\varphi_{i}:\,F^{-1}(U_{i})\rightarrow\bb{C}^{n}$
	la funci\'{o}n dada por
	\begin{align*}
		\varphi_{i}\big(\left[z^{1}:\,\cdots\,:z^{n+1}\right]\big) &
			\,=\,\left(\frac{z^{1}}{z^{i}},\,\dots,\,
			\frac{z^{i-1}}{z^{i}},\,\frac{z^{i+1}}{z^{i}},
			\,\dots,\,\frac{z^{n+1}}{z^{i}}\right) \\
		& \,=\,(z^{1},\,\dots,\,z^{i-1},\,z^{i+1},\,\dots,\,z^{n+1})
			\cdot\frac{1}{z^{i}}
		\text{ .}
	\end{align*}
	%
	De la expresi\'{o}n anterior, se deduce que las funciones
	$\varphi_{i}$ son biyectivas. Veamos que son continuas. Para eso
	usamos las correspondencias $\Phi$ y $F$. En
	$\widetilde{U}_{i}\subset M$, entonces, definimos
	$\tilde{\varphi}_{i}=\Phi\circ\varphi_{i}\circ F^{-1}$. Donde
	$\Phi:\,\bb{C}^{n}\rightarrow\bb{R}^{2n}$ es an\'{a}loga a la
	funci\'{o}n $\Phi$ definida anteriormente. Cuando
	$i=n+1$, por ejemplo, si $z^{n+1}=x^{n+1}+\sqrt{-1}y^{n+1}$ y
	$\alpha=\frac{1}{z^{n+1}}$,
	\begin{align*}
		\tilde{\varphi}_{n+1}\big(\left[(x^{1},y^{1}):\,\cdots\,:
					(x^{n+1},y^{n+1})\right]\big) & \,=\,
			((x^{1},y^{1})\cdot\widehat{\alpha},\,\dots,\,
			(x^{n},y^{n})\cdot\widehat{\alpha})
		\text{ .}
	\end{align*}
	%
	Un poco m\'{a}s expl\'{\i}citamente, en el caso $i=n+1$ (para
	simplificar),
	\begin{align*}
		\tilde{\varphi}_{n+1}\big(\left[(x^{1},y^{1}):\,\cdots\,:
					(x^{n+1},y^{n+1})\right]\big) & \,=\,
		\left(ax^{1}-by^{1},\,bx^{1}+ay^{1},\right. \\
		& \left.\qquad\,\dots,\,ax^{n}-by^{n},\,bx^{n}+ay^{n}\right)
		\text{ ,}
	\end{align*}
	%
	donde $a=\frac{x^{n+1}}{(x^{n+1})^{2}+(y^{n+1})^{2}}$ y
	$b=\frac{y^{n+1}}{(x^{n+1})^{2}+(y^{n+1})^{2}}$.
	Llamamos, para simplificar, $\varphi_{i}$ a $\tilde{\varphi}_{i}$,
	tambi\'{e}n. De la expresi\'{o}n anterior, para cada $i$, la
	composici\'{o}n $\varphi_{i}\circ q$ es continua. Porque $q|_{i}$
	es cociente, $\varphi_{i}:\,U_{i}\rightarrow\bb{R}^{2n}$ es
	continua. 

	La inversa de $\varphi_{i}$ est\'{a} dada por
	$\varphi_{i}^{-1}:\,\bb{C}^{n}\rightarrow F^{-1}(U_{i})$ donde
	\begin{align*}
		\varphi_{i}^{-1}(\lista*{u}{n}) & \,=\,
			\left[u^{1}:\,\cdots\,:u^{i-1}:1:u^{i}:
				\,\cdots\,:u^{n}\right]
		\text{ .}
	\end{align*}
	%
	El elemento $1$ en la expresi\'{o}n anterior es el $1$ de $\bb{C}$.
	V\'{\i}a $\Phi$ y $F$, en t\'{e}rminos de puntos de $M$ y de
	$\bb{R}^{2n}$,
	\begin{align*}
		\varphi_{i}(\xi^{1},\upsilon^{1},\,\dots,\,
			\xi^{n},\,\upsilon^{n}) & \,=\,
			\left[(\xi^{1},\upsilon^{1}):\,\cdots\,:
			(\xi^{i-1},\upsilon^{i-1}):(1,0):
			(\xi^{i},\upsilon^{i}):\,\cdots\,
			(\xi^{n},\upsilon^{n})\right]
		\text{ .}
	\end{align*}
	%
	Notemos que $(1,0)$ es la representaci\'{o}n de $1\in\bb{C}$ en
	coordenadas reales. La funci\'{o}n $\varphi_{i}^{-1}$ es continua
	porque se puede escribir como composici\'{o}n de funciones continuas
	seg\'{u}n el siguiente diagrama:
	\begin{center}
	\begin{tikzcd}
		\bb{R}^{2n+2}\setmin\{0\}\supset\widetilde{U}_{i}
			\arrow[d,"q"'] & \\
		M\supset U_{i} & \bb{R}^{2n}\arrow[l,"\varphi_{i}^{-1}"]
					\arrow[ul,"\psi_{i}"']

	\end{tikzcd}
	\end{center}
	La funci\'{o}n $\psi_{i}$ es \'{u}nica tal que el diagrama
	conmuta y debe cumplir que
	\begin{align*}
		\psi_{i}(\xi^{1},\upsilon^{1},\,\dots,\,
			\xi^{n},\,\upsilon^{n}) & \,=\,
			(\xi^{1},\,\upsilon^{1},\,\dots,\,
			\xi^{i-1},\,\upsilon^{i-1},\,1,\,0,\,
			\xi^{i},\,\upsilon^{i},\,\dots,\,
			\xi^{n},\,\upsilon^{n})
		\text{ .}
	\end{align*}
	%
	De esta expresi\'{o}n, es inmediato que $\psi_{i}$ es continua y que
	$\varphi_{i}^{-1}=q\circ\psi_{i}$. En definitiva, las funciones
	$\varphi_{i}:\,U_{i}\rightarrow\bb{R}^{2n}$ son homeomorfismos
	y las correspondientes
	$\varphi_{i}:\,F^{-1}(U_{i})\rightarrow\bb{C}^{n}$ tambi\'{e}n, donde
	$\bb{C}^{n}$ tiene la topolog\'{\i}a inducida por la biyecci\'{o}n
	$\Phi:\,\bb{C}^{n}\rightarrow\bb{R}^{2n}$.

	% Una observaci\'{o}n importante es que las operaciones de espacio
	% vectorial complejo en $\bb{C}^{n}$ son continuas si se le da
	% la topolog\'{\i}a inducida por el homeomorfismo
	% $\Phi:\,\bb{C}^{n}\rightarrow\bb{R}^{2n}$ para todo $n\geq 0$.

	Hemos demostrado que $\proyectivo{\bb{C}}{n}$ con la topolg\'{\i}a
	cociente que se obtiene de considerar $\bb{C}^{n+1}$ como
	$\bb{R}^{2n+2}$ y a $\bb{C}^{n+1}\setmin\{0\}$ como subconjunto
	abierto es localmente euclidea de dimensi\'{o}n $2n$. De hecho,
	demostramos que $\proyectivo{\bb{C}}{n}$ es homeomorfo al
	cociente de $\bb{R}^{2n+2}\setmin\{0\}$ por la acci\'{o}n
	diagonal del grupo $\bb{R}_{>0}\times\SO{2}$.
\end{ejemplo}

\begin{ejemplo}\nom{Una versi\'{o}n m\'{a}s resumida del ejemplo anterior}
	El \emph{espacio proyectivo complejo} de dimensi\'{o}n $n$ es el
	conjunto de $\bb{C}$-subespacios vectoriales de $\bb{C}^{n+1}$. Se
	puede realizar como el cociente de $\bb{C}^{n+1}\setmin\{0\}$
	por la relaci\'{o}n de equivalencia $z\sim w$, si generan el mismo
	subespacio, o, lo que es equivalente, si existe un n\'{u}mero
	complejo $\alpha\in\bb{C}^{\times}$ no nulo tal que $\alpha\cdot z=w$.
	Si $z\in\bb{C}^{n+1}\setmin\{0\}$, denotamos su clase por
	$z=[z^{1}:\,\cdots\,:z^{n+1}]$. Al conjunto de clases
	$\big(\bb{C}^{n+1}\setmin\{0\}\big)/\sim$ lo denotamos por
	$\proyectivo{\bb{C}}{n}$.

	Para cada $i\in[\![1,n+1]\!]$, sea $\widetilde{U}_{i}$ el conjunto
	\begin{align*}
		\widetilde{U}_{i} & \,=\,\left\lbrace
			(\lista*{z}{n+1})\in\bb{C}^{n+1}\setmin\{0\}\,:\,
			z^{i}\not =0\right\rbrace
		\text{ .}
	\end{align*}
	%
	Todo vector no nulo $z$ pertenece a alguno de estos subconjuntos.
	Sea $q:\,\bb{C}^{n+1}\setmin\{0\}\rightarrow\proyectivo{\bb{C}}{n}$
	la funci\'{o}n que a cada elemento $z$ le asigna su clase:
	\begin{align*}
		q(\lista*{z}{n+1}) & \,=\,\left[z^{1}:\,\cdots\,:z^{n+1}\right]
		\text{ .}
	\end{align*}
	%
	Sea $U_{i}=q(\widetilde{U}_{i})$. Como $q$ es sobreyectiva, los
	conjuntos $U_{i}$ cubren $\proyectivo{\bb{C}}{n}$.

	La topolog\'{\i}a que daremos al espacio proyectivo ser\'{a} la
	topolog\'{\i}a cociente determinada por $q$. Para ello, es necesario,
	en primer lugar especificar una topolog\'{\i}a en
	$\bb{C}^{n+1}\setmin\{0\}$. Cada vector $z$ complejo se puede
	representar en coordenadas reales: si $z=(\lista*{z}{n+1})$, cada
	$z^{i}$ es igual a $x^{i}+\sqrt{-1}y^{i}$ para ciertos n\'{u}meros
	reales $x^{i}$ e $y^{i}$. Sea, para $k\geq 0$, $\Phi\equiv%
	\Phi_{k}:\,\bb{C}^{k}\rightarrow\bb{R}^{2k}$ la biyecci\'{o}n
	\begin{align*}
		\Phi(\lista*{z}{k}) & \,=\,\big( (x^{1},y^{1}),\,\dots,\,
			(x^{k},y^{k})\big)
		\text{ .}
	\end{align*}
	%
	Como $\bb{R}^{2n+2}\setmin\{0\}$ es un subconjunto abierto de
	$\bb{R}^{2n+2}$ (con la topolog\'{\i}a usual), es una variedad
	topol\'{o}gica con la topolog\'{\i}a subespacio. La topolog\'{\i}a
	en $\bb{C}^{n+1}\setmin\{0\}$ es la que se obtiene de tomar
	preim\'{a}genes v\'{\i}a $\Phi$ de los abiertos del codominio.
	Notemos que esta topolog\'{\i}a coincide con la topolog\'{\i}a de
	subespacio abierto de $\bb{C}^{n+1}$, donde este \'{u}ltimo tiene
	la topolog\'{\i}a de $\bb{R}^{2n+2}$ (o la topolog\'{\i}a
	producto de $n+1$ copias de $\bb{C}$, cada una de ellas con la
	topolog\'{\i}a del plano $\bb{R}^{2}$, pues son iguales).

	Demos, entonces, a $\proyectivo{\bb{C}}{n}$ la topolog\'{\i}a
	cociente determinada por la suryecci\'{o}n $q$. Las clases de
	equivalencia en $\bb{C}^{n+1}\setmin\{0\}$ no son otra cosa
	que las \'{o}rbitas por la acci\'{o}n diagonal del grupo
	$\bb{C}^{\times}$ en los vectores complejos distintos de cero.
	En otras palabras, como las identificaciones dadas por la
	relaci\'{o}n $\sim$ y por la acci\'{o}n de $\bb{C}^{\times}$ son
	iguales en $\bb{C}^{n+1}\setmin\{0\}$, hay un homeomorfismo
	\begin{align*}
		\bb{C}^{\times}\backslash\big(\bb{C}^{n+1}\setmin\{0\}\big)
			 & \,\simeq\, \big(\bb{C}^{n+1}\setmin\{0\}\big)/\sim
		 \text{ .}
	\end{align*}
	%
	As\'{\i}, como el espacio proyectivo es un cociente por la
	acci\'{o}n de un grupo, el epimorfismo can\'{o}nico $q$ es
	una funci\'{o}n abierta. Precisamente, si $\alpha\in\bb{C}^{\times}$,
	la funci\'{o}n $z\mapsto\alpha\cdot z$ es una funci\'{o}n continua.
	Vale la pena notar que, por ahora, no estamos considerando esta
	acci\'{o}n como la acci\'{o}n de un grupo topol\'{o}gico, es decir,
	no estamos considerando la topolog\'{\i}a de $\bb{C}^{\times}$,
	s\'{o}lo estamos haciendo uso del hecho de que, para cada $\alpha$,
	la funci\'{o}n inducida es continua.
	
	De las observaciones anteriores, deducimos que los conjuntos
	$U_{i}=q(\widetilde{U}_{i})$ son abiertos en $\proyectivo{\bb{C}}{n}$.
	M\'{a}s aun, como cada $\widetilde{U}_{i}$ es saturado respecto
	de $q$, las restricciones $q_{i}=q|_{\widetilde{U}_{i}}:\,%
	\widetilde{U}_{i}\rightarrow U_{i}$ son cocientes. Estos conjuntos,
	como en el caso real, servir\'{a}n de cartas. Para demostrar esta
	afirmaci\'{o}n, definimos
	$\varphi_{i}:\,U_{i}\rightarrow\bb{C}^{n}$ por
	\begin{align*}
		\varphi_{i}\big(\left[z^{1}:\,\cdots\,:z^{n+1}\right]\big) &
			\,=\,\big(\frac{z^{1}}{z^{i}},\,\dots,\,
			\frac{z^{i-1}}{z^{i}},\,\frac{z^{i+1}}{z^{i}},
			\,\dots,\,\frac{z^{n+1}}{z^{i}}\big)
		\text{ .}
	\end{align*}
	%
	Esta aplicaci\'{o}n est\'{a} bien definida, pues la expresi\'{o}n
	de la derecha no depende del representante elegido. Adem\'{a}s,
	como la composici\'{o}n $\varphi_{i}\circ q_{i}$ es continua en
	$\widetilde{U}_{i}$, se deduce que $\varphi_{i}$ es continua,
	por la propiedad (?`universal?) del cociente. Esta aplicaci\'{o}n
	tiene una inversa, dada por la composici\'{o}n de la inclusi\'{o}n
	de una tajada en la coordenada $i$:
	\begin{align*}
		\iota_{i,1} & \,:\,
			(\lista*{z}{n})\,\mapsto\,(z^{1},\,\dots,\,z^{i-1},\,
			1,\,z^{i},\,\dots,\,z^{n})
	\end{align*}
	%
	compuesta con $q_{i}$, es decir,
	$\varphi_{i}^{-1}=q_{i}\circ\iota_{i,1}$. Como esta funci\'{o}n es una
	composici\'{o}n de funciones continuas, es continua y $\varphi_{i}$ es
	un homeomorfismo. Para asegurarnos de que $\varphi_{i}\circ q_{i}$ y
	que $\iota_{i,1}$ son funciones continuas, podemos componer con las
	``coordenadas reales'' $\Phi$: si
	$a=\frac{x^{i}}{(x^{i})^{2}+(y^{i})^{2}}$ y
	$b=\frac{y^{i}}{(x^{i})^{2}+(y^{i})^{2}}$,
	\begin{align*}
		\Phi_{n}\circ(\varphi_{i}\circ q_{i})\circ\Phi_{n+1}^{-1}
		\big((x^{1},y^{1}),\,\dots,\,(x^{n+1},y^{n+1})\big) & \,=\, \\
		\big(ax^{1}-by^{1},\,bx^{1}+ay^{1}, \,\dots,\,
		& ax^{i-1}-by^{i-1},\,bx^{i-1}+ay^{i-1},\, \\
		ax^{i+1}-by^{i+1},\,bx^{i+1}+ay^{i+1},\,\dots,\,
		& ax^{n+1}-by^{n+1},\,bx^{n+1}+ay^{n+1}\big)
		\quad\text{y} \\
		\Phi_{n+1}\circ\iota_{i,1}\circ\Phi_{n}^{-1}
		\big((x^{1},y^{1}),\,\dots,\,(x^{n},y^{n})\big) & \,=\, \\
		\big((x^{1},y^{1}),\,\dots,\,(x^{i-1},y^{i-1}), & \,(1,0),\,
		(x^{i},y^{i}),\,\dots,\,(x^{n},y^{n})\big)
		\text{ .}
	\end{align*}
	%
	Estas funciones son continuas es sus dominios de definici\'{o}n.

	Hemos demostrado que $\proyectivo{\bb{C}}{n}$ es localmente euclideo
	de dimensi\'{o}n $2n$. Tambi\'{e}n hemos hallado expl\'{\i}citamente
	un atlas \emph{finito} para el espacio. Para poder que los espacios
	proyectivos complejos son variedades topol\'{o}gicas, s\'{o}lo resta
	verificar que la topolog\'{\i}a es Hausdorff (que hay una base
	numerable para la topolog\'{\i}a, se deduce de que
	$\bb{C}^{n+1}\setmin\{0\}$ es $N_{2}$ y de que $q$ es abierta, o
	bien de que el espacio proyectivo se puede cubrir por finitas
	``cartas'' (abiertos, homeomorfos a $\bb{R}^{2n}$)). Para demostrar
	esto, usaremos el hecho \ref{thm:},
	como la funci\'{o}n cociente $q$ es abierta, ser\'{a} suficiente
	verificar que la relaci\'{o}n $\sim$ es cerrada, es decir, el
	subconjunto
	\begin{align*}
		R & \,=\,\left\lbrace (z,w)\,:\,z\sim w\right\rbrace
	\end{align*}
	%
	es cerrado en el producto de $\bb{C}^{n+1}\setmin\{0\}$ consigo
	mismo.

	Si restringimos $q$ a la esfera compleja, la aplicaci\'{o}n
	\begin{align*}
		q| & \,:\,\esfera{n}(\bb{C})\,\equiv\,
			\left\lbrace z\in\bb{C}^{n+1}\setmin\{0\}\,:\,
			|z|=1\right\rbrace\,\rightarrow\,
			\proyectivo{\bb{C}}{n}
	\end{align*}
	%
	es continua, por ser restricci\'{o}n a un subespacio de una
	funci\'{o}n continua, y es sobreyectiva. Como el dominio es
	compacto y sabemos que el codominio es Hausdorff, la restricci\'{o}n
	$q|$ es cociente. Notemos, adem\'{a}s, que $\proyectivo{\bb{C}}{n}$
	se puede realizar, tambi\'{e}n, como el cociente de
	$\esfera{n}(\bb{C})$ por la acci\'{o}n del grupo compacto
	$S^{1}=\{|\alpha|=1\}\subset\bb{C}^{\times}$. Sabemos, tambi\'{e}n,
	que la acci\'{o}n
	\begin{align*}
		S^{1}\times\esfera{n}(\bb{C}) & \,\rightarrow\,
			\esfera{n}(\bb{C})
	\end{align*}
	%
	es continua: es la restricci\'{o}n de la acci\'{o}n
	\begin{align*}
		\bb{C}^{\times}\times\big(\bb{C}^{n+1}\setmin\{0\}\big) &
		\,\rightarrow\,\bb{C}^{n+1}\setmin\{0\}
	\end{align*}
	%
	y que esta \'{u}ltima es continua (en ambos factores) se puede ver
	tomando las cartas reales $\Phi$. Finalmente, para ver que la
	relaci\'{o}n $R$ es cerrada, tomamos una sucesi\'{o}n convergente
	$\{(z_{m},w_{m})\}_{m\geq 1}$ en
	$R\cap\big(\esfera{n}(\bb{C})\times\esfera{n}(\bb{C})\big)$.
	Cada $w_{m}$ es igual a $\alpha_{m}\cdot z_{m}$ para alg\'{u}n
	$\alpha_{m}\in S^{1}$. Como $S^{1}$ es compacto, existe una
	subsucesi\'{o}n convergente. Llamemos igualmente $\alpha_{m}$ a los
	elementos de esta subsucesi\'{o}n y $z_{m}$ a los correspondientes
	t\'{e}rminos en la sucesi\'{o}n de vectores. Sean
	$\alpha=\lim_{m\to\infty},\alpha_{m}$ y $z=\lim_{m\to\infty}\,z_{m}$
	como $\esfera{n}(\bb{C})$ es compacta, es cerrada y el l\'{\i}mite
	$z$ debe pertenecer a la esfera. Pero, entonces,
	\begin{align*}
		\lim_{m\to\infty}\,(z_{m},\alpha_{m}\cdot z_{m}) & \,=\,
		(z,\alpha\cdot z)\in
		R\cap\big(\esfera{n}(\bb{C})\times\esfera{n}(\bb{C})\big)
		\text{ .}
	\end{align*}
	%
\end{ejemplo}

\begin{ejemplo}\nom{La proyecci\'{o}n en el espacio proyectivo}
	Sea $M$ una variedad diferencial y sea $\pi:\,\esfera{n}\rightarrow%
	\proyectivo{\bb{R}}{n}$ la proyecci\'{o}n can\'{o}nica.
	Dada funci\'{o}n $f:\,\proyectivo{\bb{R}}{n}\rightarrow M$,
	entonces $f$ es suave, si y s\'{o}lo si
	$f\circ\pi:\,\esfera{n}\rightarrow M$ lo es. Esto se debe a que
	hay secciones locales suaves para $\pi$. Esto es una propiedad
	caracter\'{\i}stica de las submersiones. Expl\'{\i}citamente, en
	este caso, si $U_{i}^{+}=\{x^{i}>0\}\cap\esfera{n}$ y
	$U_{i}=\{[x]\,:\,x^{i}\not =0\}\subset\proyectivo{\bb{R}}{n}$,
	entonces podemos definir, dada una clase $\xi\in U_{i}$,
	\begin{align*}
		\sigma_{i}(\xi) & \,=\,(x^{1},\,\dots,\,x^{n+1})
	\end{align*}
	%
	donde $(x^{1},\,\dots,\,x^{n+1})\in\esfera{n}$ es el representante
	de $\xi$ con $x^{i}>0$. Esta funci\'{o}n es suave. Si
	denotamos con $\varphi_{i}:\,U_{i}\rightarrow\bb{R}^{n}$ a las
	coordenadas en $U_{i}$ y con
	$\varphi_{i}^{+}:\,U_{i}^{+}\rightarrow\bola{0}{1}$ a las coordenadas
	en $U_{i}^{+}$, entonces
	\begin{align*}
		\varphi_{i}^{+}\circ\sigma_{i}\circ\varphi_{i}^{-1}
		(\lista*{u}{n}) & \,=\,
			\varphi_{i}^{+}\circ\sigma_{i}\big(
			\left[u^{1}:\,\cdots\,:u^{i-1}:1:\,\cdots\,:u^{n}
				\right]\big) \\
		& \,=\,\varphi_{i}^{+}\Big(
			\tfrac{u^{1}}{\sqrt{|u|^{2}+1}},\,\dots,\,
			\tfrac{u^{i-1}}{\sqrt{|u|^{2}+1}},\,
			\tfrac{1}{\sqrt{|u|^{2}+1}},\,\dots,\,
			\tfrac{u^{n}}{\sqrt{|u|^{2}+1}}\Big) \\
		& \,=\,\Big(\tfrac{u^{1}}{\sqrt{|u|^{2}+1}},\,\dots,\,
			\tfrac{u^{n}}{\sqrt{|u|^{2}+1}}
			\Big)
		\text{ .}
	\end{align*}
	%
	Esto demuestra que $\sigma_{i}$ es suave. Adem\'{a}s, de la
	definici\'{o}n se deduce que $\pi\circ\sigma_{i}=\id[U_{i}]$,
	Ahora bien, si $f$ es suave, $f\circ\pi$ es suave, por ser
	composici\'{o}n de funciones suaves. Rec\'{\i}procamente, si
	la composici\'{o}n $f\circ\pi$ es suave, restringiendo a cada
	$U_{i}$, vale que
	\begin{align*}
		\left.f\right|_{U_{i}} & \,=\,(f\circ\pi)\circ\sigma_{i}
		\text{ ,}
	\end{align*}
	%
	que es, nuevamente, composici\'{o}n de funciones suaves. Entonces
	$f|_{U_{i}}$ es suave para cada $i$ y $f$ es suave en
	$\proyectivo{\bb{R}}{n}$.

	En cuanto al rango de $\pi$, localmente sabemos que vale una igualdad
	de la forma $\pi\circ\sigma_{i}=\id[U_{i}]$. Tomando diferenciales,
	$\diferencial{\pi}\cdot\diferencial{\sigma_{i}} =\id[TU_{i}]$.
	Con lo cual, $\diferencial{\pi}$ es sobreyectivo y $\pi$ es una
	submersi\'{o}n. Notemos que, adem\'{a}s, $\diferencial{\sigma_{i}}$
	es inyectivo. En particular, como sabemos que las dimensiones
	de $\esfera{n}$ y de $\proyectivo{\bb{R}}{n}$ son iguales, podemos
	concluir que $\diferencial{\sigma_{i}}$ es un isomorfismo lineal
	(estamos omitiendo el punto en la notaci\'{o}n del diferencial, todo
	ocurre a nivel de los tanentes en cada punto por separado).

	Veamos cu\'{a}l es la relaci\'{o}n
	entre el rango de $f$ y el de $f\circ\pi$. Sea $\xi\in U_{i}$ un
	punto de $\proyectivo{\bb{R}}{n}$ y sea $x=\sigma_{i}(x)$. La
	ecuaci\'{o}n $f|_{U_{i}}=f\circ\pi\circ\sigma_{i}$ implica que
	\begin{align*}
		\diferencial[\xi]{f}\cdot\diferencial[\xi]{\inc[U_{i}]} &
		\,=\,
		\diferencial[x]{(f\circ\pi)}\cdot\diferencial[\xi]{\sigma_{i}}
	\end{align*}
	%
	Pero $\diferencial[\xi]{\inc[U_{i}]}$ es un isomorfismo y,
	como vimos antes, $\diferencial[\xi]{\sigma_{i}}$ tambi\'{e}n lo es.
	Entonces $\rango{f}=\rango{f\circ\pi}$ en todo punto. Rigurosamente,
	dado $x\in\esfera{n}$, el rango de $f$ en $\pi(x)$ es igual al rango
	de $f\circ\pi$ en $x$.

	Las secciones $\sigma_{i}:\,U_{i}\rightarrow U_{i}^{+}$ son,
	en realidad, difeomorfismos. Las inversas est\'{a}n dadas por las
	restricciones de $\pi$ a los abiertos $U_{i}^{+}$. En particular,
	$\diferencial[\xi]{\sigma_{i}}$ es un isomorfismo lineal, no s\'{o}lo
	una transformaci\'{o}n inyectiva que resulta isomorfismo por
	dimensi\'{o}n.
\end{ejemplo}