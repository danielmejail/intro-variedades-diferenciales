\theoremstyle{plain}
\newtheorem{propoSuavidadEsLocal}{Proposici\'{o}n}[section]
\newtheorem{propoSuaveEsConti}[propoSuavidadEsLocal]{Proposici\'{o}n}
\newtheorem{propoDelPegado}[propoSuavidadEsLocal]{Proposici\'{o}n}
\newtheorem{propoAlgunasFuncionesSuaves}[propoSuavidadEsLocal]{Proposici\'{o}n}

\theoremstyle{remark}
\newtheorem{obsComoLaUsual}{Observaci\'{o}n}[section]
\newtheorem{obsTodasSonSuaves}[obsComoLaUsual]{Observaci\'{o}n}
\newtheorem{obsOtrasCaracterizacionesDeSuavidad}[obsComoLaUsual]%
	{Observaci\'{o}n}

%-------------

Sea $M$ una variedad y sea $f:\,M\rightarrow\bb{R}$ una funci\'{o}n
arbitraria. Para describir a $f$, para poder decir algo acerca de sus
propiedades, estudiamos la funci\'{o}n en coordenadas. La
\emph{representaci\'{o}n de $f$ en coordenadas} o la funci\'{o}n $f$
\emph{en coordenadas} es cualquier composici\'{o}n de $f$ con la inversa de
una carta para $M$, es decir, algo de la forma $f\circ\varphi^{-1}$,
donde $\varphi$ es la funci\'{o}n coordenada de una carta $(U,\varphi)$
para $M$. Para hacer uso de esta idea, no es necesario que el codominio de
$f$ sea $\bb{R}$. La idea es que todo, o mucho de lo que se puede conocer de
$M$ se conoce a trav\'{e}s de las cartas.

\subsection{Funciones suaves}
Una funci\'{o}n $f:\,M\rightarrow\bb{R}$ o, m\'{a}s en general, una funci\'{o}n $f:\,M\rightarrow\bb{R}^{l}$ es una
\emph{funci\'{o}n suave}, si, para todo punto $p\in M$, existe una carta
compatible $(U,\varphi)$ para $M$ en $p$ tal que la composici\'{o}n
\begin{align*}
	f\circ\varphi^{-1} & \,:\,\varphi(U)\subset\bb{R}^{d}\,\rightarrow\,
					\bb{R}^{l}
\end{align*}
%
es suave en el sentido \emph{usual}: diremos que $f$ es diferenciable, de
clase $C^{1}$, de clase $C^{k}$, \textit{etcetera}, si todas las
composiciones $f\circ\varphi^{-1}$ tienen la propiedad correspondiente,
propiedades que dependen de la existencia de ciertos l\'{\i}mites y
de la continuidad de los mismos. En general, toda propiedad local acerca
de funciones definidas en abiertos de $\bb{R}^{d}$ se puede definir
tambi\'{e}n para funciones definidas en abiertos de variedades usando las
cartas (compatibles) para la variedad. Decimos que $f$ es \emph{regular},
si vista en coordenadas es regular.

Si $U$ es un abierto euclideo, entonces hay dos nociones de suavidad de
funciones. Si $f:\,U\rightarrow\bb{R}^{l}$ es una funci\'{o}n, podemos
decir que $f$ es suave porque es de clase $C^{k}$ en $U$ para todo $k\geq 1$,
en el sentido de que existen las derivadas parciales de orden $k$ y son
continuas para todo $k\geq 1$; o bien podemos decir que es suave porque
para todo punto $p\in U$ existe una carta $(U',\varphi)$ en $p$ tal que
$f\circ\varphi^{-1}$ es suave. Ambas nociones coinciden: si $f$ es suave en
el sentido usual, entonces, tomando la carta global $(U,\id[U])$, se ve
que $f\circ\id[U]^{-1}=f$ es suave (en el sentido usual) y que, por lo
tanto, para todo punto se puede hallar una carta tal que la funci\'{o}n
en coordenadas es suave en el sentido usual; rec\'{\i}procamente, basta
notar que, dada una carta $(V,\psi)$ para $U$, las funciones $\psi$ y
$\psi^{-1}$ son funciones suaves en sentido usual, ya que $\psi$ es la
funci\'{o}n de una carta compatible con la estructura en $U$ determinada
por el atlas $C^{\infty}$ $\{(U,\id[U])\}$.

\begin{obsComoLaUsual}\label{obs:comolausual}
Veamos esto \'{u}ltimo en detalle. La estructura diferencial usual en
$U$ es aquella determinada por el atlas que consiste en la \'{u}nica
carta $(U,\id[U])$ que se obtiene de restringir la carta $(\bb{R}^{d},\id)$
que define la estructura diferencial usual de $\bb{R}^{d}$. Sea $(V,\psi)$
una carta para $U$ compatible con esta estructura. Como $(U,\id[U])$ y
$(V,\psi)$ son cartas compatibles, las funciones
\begin{align*}
	\psi(U\cap V)\,\rightarrow\,\id[U](U\cap V) &
	\quad\text{e}\quad
	\id[U](U\cap V)\,\rightarrow\psi(U\cap V)
\end{align*}
%
son diferenciables (en el sentido usual, naturalmente). Pero estas funciones
son, precisamente, $\psi^{-1}:\,\psi(V)\rightarrow V$ y
$\psi:\,V\rightarrow\psi(V)$. Es decir, $\psi$ y $\psi^{-1}$ son suaves
en el sentido usual y $\psi$ es un difeomorfismo, en el sentido usual.
\end{obsComoLaUsual}

\begin{obsTodasSonSuaves}\label{obs:todassonsuaves}
Sea $M$ una variedad diferencial y sea $f:\,M\rightarrow\bb{R}^{l}$ una
funci\'{o}n suave. Sea $(U,\varphi)$ una carta compatible para $M$. Entonces
$f\circ\varphi^{-1}$ es suave: si $p\in U$ y $(V,\psi)$ es una carta tal
que $p\in V$ y $f\circ\psi^{-1}$ es suave,
\begin{align*}
	f\circ\varphi^{-1}|_{\varphi(U\cap V)} & \,=\,
	(f\circ\psi^{-1})|_{\psi(U\cap V)}\circ
		(\psi\circ\varphi^{-1})|_{\varphi(U\cap V)}
	\text{ .}
\end{align*}
%
Esta descomposi\'{o}n muestra que $f\circ\varphi^{-1}$ es suave ``en $p$''.
Como $p\in U$ era arbitrario, $f\circ\varphi^{-1}$ es suave.
\end{obsTodasSonSuaves}

Si $f:\,M\rightarrow\bb{R}^{l}$ es suave y $(U,\varphi)$ es una carta
(compatible) para $M$, la composici\'{o}n $\hat{f}=f\circ\varphi^{-1}$ se
denomina \emph{representaci\'{o}n de $f$ en coordenadas (respecto de la %
carta $\varphi$)}. La observaci\'{o}n \ref{obs:todassonsuaves} muestra que
la suavidad de las representaciones no depende de la carta.

\subsection{Transformaciones suaves}
Una funci\'{o}n $F:\,M\rightarrow N$ entre variedades diferenciales se dice
\emph{suave} o \emph{transformaci\'{o}n suave} (para distinguirlas de aquellas
con codominio $\bb{R}$ o $\bb{R}^{l}$), si, para todo punto $p\in M$
existen cartas $(U,\varphi)$ para $M$ en $p$ y $(V,\psi)$ para $N$ en
$F(p)$ tales que
\begin{itemize}
	\item[\i] $F(U)\subset V$ y
	\item[\i\i] $\psi\circ F\circ\varphi^{-1}:\,%
		\varphi(U)\rightarrow\psi(V)$ es suave en sentido usual entre
		abiertos de $\bb{R}^{\dim\,M}$ y de $\bb{R}^{\dim\,N}$.
\end{itemize}
%
Esta definici\'{o}n coincide con la definici\'{o}n de funci\'{o}n suave
pensando al codominio $\bb{R}^{l}$ como una variedad diferencial con su
estructura usual argumentando como en la observaci\'{o}n
\ref{obs:comolausual}. De manera similar al caso de funciones en $\bb{R}^{l}$,
llamamos \emph{representaci\'{o}n en coordenadas de $F$} a
$\hat{F}=\psi\circ F\circ\varphi^{-1}$.

\begin{obsOtrasCaracterizacionesDeSuavidad}\label{obsotrassuavidad}
	Sea $F:\,M\rightarrow N$ una funci\'{o}n. Entonces $F$ es suave, si
	y s\'{o}lo si para todo $p\in M$ existen cartas $(U,\varphi)$ en
	$p$ y $(V,\psi)$ en $F(p)$ tales que $U\cap F^{-1}(V)$ sea abierta
	en $M$ y
	\begin{align*}
		\psi\circ F\circ\varphi^{-1} & \,:\,
			\varphi(U\cap F^{-1}(V))\,\rightarrow\,\psi(V)
	\end{align*}
	%
	sea suave. Equivalentemente, $F$ es suave, si y s\'{o}lo si
	$F$ es continua y existen atlas compatibles
	$\{(U_{\alpha},\varphi_{\alpha})\}_{\alpha}$ de $M$ y
	$\{(V_{\beta},\psi_{\beta})\}_{\beta}$ de $N$ tales que las
	composiciones
	\begin{align*}
		\psi_{\beta}\circ F\circ\varphi_{\alpha}^{-1} & \,:\,
			\varphi_{\alpha}(U_{\alpha}\cap F^{-1}(V_{\beta}))
			\,\rightarrow\,\psi_{\beta}(V_{\beta})
	\end{align*}
	%
	sean suaves.
\end{obsOtrasCaracterizacionesDeSuavidad}

\subsection{Propiedades locales}
Al igual que la continuidad de funciones, suavidad es una propiedad
local.

\begin{propoSuavidadEsLocal}[Suavidad es una propiedad local]%
	\label{thm:suavidadeslocal}
	Sea $F:\,M\rightarrow N$ una funci\'{o}n entre variedades
	diferenciales. Entonces, si $F$ es suave, la restricci\'{o}n
	$F|_{U}:\,U\rightarrow N$ es suave para todo abierto $U\subset M$.
	Rec\'{\i}procamente, si todo punto admite un entorno $U$ tal que
	$F|_{U}$ sea suave, entonces $F$ es suave. Diremos que $F$ es suave
	en un punto $p\in M$, si exite un entorno $U$ de $p$ tal que la
	restricci\'{o}n $F|_{U}$ sea suave.
\end{propoSuavidadEsLocal}

Dado que las cartas son homeomorfismo podemos deducir la siguiente propiedad
\emph{deseable}.

\begin{propoSuaveEsConti}\label{thm:suaveesconti}
	Toda funci\'{o}n suave $F:\,M\rightarrow N$ es continua.
\end{propoSuaveEsConti}

\begin{proof}
	Para cada punto $p$ existen cartas $(U,\varphi)$ en $p$ y $(V,\psi)$
	en $F(p)$ tales que $F(U)\subset V$. Entonces
	\begin{align*}
		F|_{U} & \,=\,\psi^{-1}\circ (\psi\circ F\circ\varphi^{-1})
			\circ\varphi
	\end{align*}
	%
	implica que $F$ es continua restringida a $U$. Como esto es
	v\'{a}lido para todo punto $p\in M$, $F$ es continua en $M$.
\end{proof}

De manera similar, el lema del pegado para funciones continuas tambi\'{e}n
tiene su an\'{a}logo acerca de funciones suaves.

\begin{propoDelPegado}[Lema del pegado]\label{thm:delpegado}
	Sean $M$ y $N$ variedades diferenciales. Sea
	$\cal{U}=\{U_{\alpha}\}_{\alpha}$ un cubrimiento de $M$ por abiertos.
	Si, para cada $\alpha$, existe una funci\'{o}n suave
	$F_{\alpha}:\,U_{\alpha}\rightarrow N$ de manera que
	\begin{align*}
		F_{\alpha}|_{U_{\alpha}\cap U_{\beta}} & \,=\,
			F_{\beta}|_{U_{\alpha}\cap U_{\beta}}
	\end{align*}
	%
	para todo par $\alpha,\beta$, entonces existe una \'{u}nica funci\'{o}n
	suave $F:\,M\rightarrow N$ tal que $F|_{U_{\alpha}}=F_{\alpha}$
	para todo $\alpha$.
\end{propoDelPegado}

\begin{propoAlgunasFuncionesSuaves}\label{thm:algunassuaves}
	Las funciones constantes $c:\,M\rightarrow N$ son suaves. La
	identidad $\id[M]:\,M\rightarrow M$ es suave. La inclusi\'{o}n
	$U\hookrightarrow M$ de una subvariedad abierta es suave.

	Si $\lista{M}{k}$ y $N$ son variedades diferenciales (y, a lo sumo,
	una de las $M_{i}$ posee borde no vac\'{\i}o), entonces una
	funci\'{o}n $F:\,N\rightarrow M_{1}\times\,\cdots\,\times M_{k}$
	es suave, si y s\'{o}lo si las composiciones
	$\pi_{i}\circ F:\,N\rightarrow M_{i}$ son suaves.

	La composici\'{o}n de funciones suaves es suave.
\end{propoAlgunasFuncionesSuaves}

\begin{proof}
	Demostramos la \'{u}ltima afirmaci\'{o}n.
	Supongamos que $F:\,M\rightarrow N$ y que $G:\,N\rightarrow\tilde{N}$
	son funciones suaves. Sea $p\in M$. Por hip\'{o}tesis, existen cartas
	$(V,\varphi)$ en $F(p)$, $(W,\psi)$ en $G(F(p))$ tales que
	$G(V)\subset W$ y de manera que $\psi\circ G\varphi^{-1}$ es suave
	en $\varphi(V)$. Porque $F$ es continua, $F^{-1}(V)$ es abierta en
	$M$ y contiene a $p$. Existe, entonces, una carta $(U,\tilde{\varphi})$
	tal que $p\in U\subset F^{-1}(V)$. En particular,
	$\varphi\circ F\circ\tilde{\varphi}^{-1}:%
		\,\tilde{\varphi}(U)\rightarrow\varphi(V)$. Se deduce entonces
	que $G\circ F(U)\subset G(V)\subset W$ y que
	\begin{align*}
		\psi\circ(G\circ F)\circ\tilde{\varphi}^{-1} & \,=\,
			(\psi\circ G\circ\varphi^{-1})\circ
			(\varphi\circ F\tilde{\varphi}^{-1})
	\end{align*}
	%
	es suave por ser composici\'{o}n de funciones suaves entre abiertos
	de espacios euclideos.
\end{proof}
