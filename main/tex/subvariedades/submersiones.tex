\theoremstyle{plain}
\newtheorem{teoSeccionesLocales}{Teorema}[section]
\newtheorem{propoSubmersionEsAbierta}[teoSeccionesLocales]{Proposici\'{o}n}
\newtheorem{coroSubmersionSobreEsCociente}[teoSeccionesLocales]{Corolario}
\newtheorem{propoSubmersionesExtra}[teoSeccionesLocales]{Proposici\'{o}n}
\newtheorem{coroUnicidadDelCociente}[teoSeccionesLocales]{Corolario}

\theoremstyle{remark}
\newtheorem{obsCaracterizaSubmersionesSobre}{Observaci\'{o}n}[section]

%-------------

Empecemos con la siguiente definici\'{o}n: una \emph{secci\'{o}n} de un
morfismo $\pi:\,M\rightarrow N$ es un morfismo $\sigma:\,N\rightarrow M$
tal que $\pi\circ\sigma =\id[N]$, El siguiente diagrama es ilustrativo
en este sentido:
\begin{center}
\begin{tikzcd}
	M\arrow[d,"\pi"'] \\
	N\arrow[u, bend right=50, "\sigma"']
\end{tikzcd}
\end{center}

Si $\pi$ es una funci\'{o}n continua, una secci\'{o}n $\sigma$ de $\pi$ es
una funci\'{o}n \emph{continua} tal que $\pi\circ\sigma=\id[N]$; si
$\pi$ es suave, una secci\'{o}n $\sigma$ de $\pi$ es una funci\'{o}n
\emph{suave} tal que $\pi\circ\sigma=\id[N]$. En general, usaremos los
t\'{e}rminos \emph{secci\'{o}n continua} o, respectivamente,
\emph{secci\'{o}n suave} para evitar ambig\"{u}edades. una
\emph{secci\'{o}n local} de una funci\'{o}n (\emph{continua}) $\pi$ es
una funci\'{o}n (continua) $\sigma:\,U\rightarrow M$ definida en un abierto
$U\subset N$ y tal que $\pi\circ\sigma=\id[U]$.

\begin{teoSeccionesLocales}\label{thm:seccioneslocales}
	Sean $M$ y $N$ variedades diferenciales (\emph{sin} borde) y sea
	$\pi:\,M\rightarrow N$ una transformaci\'{o}n suave. Entonces
	$\pi$ es una submersi\'{o}n, si y s\'{o}lo si, para cada punto
	$p\in M$, existe una seccci\'{o}n (suave) local
	$\sigma:\,U\rightarrow M$ de $\pi$ tal que $p\in\sigma(U)$.
\end{teoSeccionesLocales}

Es decir, $\pi$ es una submersi\'{o}n, si todo punto del dominio pertenece
a la imagen de una secci\'{o}n local.

\begin{proof}
	Supongamos que $\pi:\,M\rightarrow N$  es una submersi\'{o}n y sea
	$p\in M$. Por el teorema del rango \ref{thm:delrango}, como estamos
	suponiendo que las variedades no tienen borde, existen coordenadas
	$(U,\varphi)$ en $p$ y $(V,\psi)$ en $\pi(p)=:q$ tales que
	$\pi(U)\subset V$ y
	\begin{align*}
		\widehat{\pi}(\lista*{x}{m}) & \,=\,
			\psi\circ\pi\circ\varphi^{-1}(\lista*{x}{n},\,
				x^{n+1},\,\dots,\,x^{m})
			\,=\,(\lista*{x}{n})
		\text{ .}
	\end{align*}
	%
	Achicando $U$ de ser necesario, podemos asumir que
	$\varphi(U)=\cubo[m]{\epsilon}{0}$. La imagen por $\widehat{\pi}$
	de este cubo es exactamente el cubo $\cubo[n]{\epsilon}{0}=%
	\{(\lista*{y}{n})\,:\,|y^{j}|<\epsilon\}$. Definimos entonces
	\begin{align*}
		\widehat{\sigma} & \,:\,\cubo[n]{\epsilon}{0}\,\rightarrow\,
			\cubo[m]{\epsilon}{0}\text{ ,}\quad
			\widehat{\sigma}(\lista*{y}{n})\,=\,
			(\lista*{y}{n},\,0,\,\dots,\,0)
			\quad\text{y}\\
		\sigma & \,:\,\psi^{-1}(\cubo[n]{\epsilon}{0}) \,\rightarrow\,
			\varphi^{-1}(\cubo[m]{\epsilon}{0})
				\text{ ,}\quad
			\sigma\,=\,\varphi^{-1}\circ\widehat{\sigma}\circ\psi
		\text{ .}
	\end{align*}
	%
	Entonces $\pi\circ\sigma=\id[\psi^{-1}({\cubo[n]{\epsilon}{0}})]$ y
	$\sigma$ es una secci\'{o}n suave de $\pi$ defininida en el
	abierto $\psi^{-1}(\cubo[n]{\epsilon}{0})\subset V$ y
	$\sigma(\pi(p))=p$.

	Rec\'{\i}procamente, si $p\in M$ y existe una transformaci\'{o}n
	suave $\sigma:\,U\rightarrow M$ definida en un abierto $U\subset N$
	tal que $p\in\sigma(U)$ y tal que $\pi\circ\sigma=\id[U]$, entonces,
	tomando diferencial en $q\in U$ tal que $\sigma(q)=p$ y aplicando
	la regla de la cadena,
	\begin{align*}
		\id[T_{q}N] & \,=\,\diferencial[q]{(\pi\circ\sigma)} \,=\,
			\diferencial[p]{\pi}\cdot\diferencial[q]{\sigma}
		\text{ ,}
	\end{align*}
	%
	de lo que se deduce que $\diferencial[p]{\pi}$ es sobreyectivo
	(y que $\diferencial[q]{\sigma}$ es inyectivo).
\end{proof}

Como en el caso de las inmersiones, podemos usar esta equivalencia para
definir una submersi\'{o}n topol\'{o}gica. Una funci\'{o}n continua
$\pi:\,X\rightarrow Y$ se dice \emph{submersi\'{o}n}, si admite una
cantidad suficiente de secciones locales, es decir, es tal que cada punto
$p\in X$ pertenece a la imagen de alguna secci\'{o}n local de $\pi$.

\begin{propoSubmersionEsAbierta}\label{thm:submersionesabierta}
	Una submersi\'{o}n $\pi:\,M\rightarrow N$ con $M$ y $N$ \emph{sin}
	borde, es abierta.
\end{propoSubmersionEsAbierta}

\begin{proof}
	Sea $U\subset M$ un subconjunto abierto y sea $q\in \pi(U)$. Sea
	$p\in U$ tal que $\pi(p)=q$ cualquier punto en la preimagen. Por
	el teorema \ref{thm:seccioneslocales}, existe una secci\'{o}n
	local suave $\sigma:\,W\rightarrow M$ de $\pi$ tal que $\sigma(q)=p$.
	El conjunto $\sigma^{-1}(U)$ es abierto, est\'{a} contenido en $W$
	y contiene a $q$. Pero, adem\'{a}s, si $y\in\sigma^{-1}(U)$, vale
	que $y=\pi(\sigma(y))\in\pi(U)$. Entonces $q$ pertenece a
	$\sigma^{-1}(U)$ que est\'{a} contenido en $\pi(U)$, de lo que se
	deduce que $\pi(U)$ es abierto.
\end{proof}

\begin{coroSubmersionSobreEsCociente}\label{thm:submersionsobreescociente}
	Si $\pi:\,M\rightarrow N$ es una submersi\'{o}n sobreyectiva
	($M$ y $N$ sin borde), entonces es cociente.
\end{coroSubmersionSobreEsCociente}

\begin{proof}
	Por \ref{thm:submersionesabierta}, $\pi:\,M\rightarrow N$ es
	sobreyectiva y abierta. En particular, es cociente.
\end{proof}

Las submersiones presentan algunas propiedades similares a las de los
difeomorfismos locales (c.~f. \ref{thm:propisextradifeoslocales}).

\begin{propoSubmersionesExtra}\label{thm:propisextrasubmersiones}
	Sean $M$ y $N$ variedades sin borde y sea $\pi:\,M\rightarrow N$ una
	submersi\'{o}n suave y sobreyectiva. Si $P$ es una variedad
	(con o sin borde) entonces \emph{(a)} una transformaci\'{o}n
	(no necesariamente continua) $F:\,N\rightarrow P$ es suave, si y
	s\'{o}lo si $F\circ\pi:\,M\rightarrow P$ lo es; \emph{(b)} si
	$G:\,M\rightarrow P$ es una funci\'{o}n constante en las fibras de
	$\pi$, existe una \'{u}nica funci\'{o}n $\tilde{G}:\,N\rightarrow P$
	tal que $G=\tilde{G}\circ\pi$ y $G$ es suave, si y s\'{o}lo si
	$\tilde{G}$ lo es.
\end{propoSubmersionesExtra}

Estas propiedades caracterizan a las submersiones sobreyectivas. Las
submersiones sobreyectivas son la versi\'{o}n an\'{a}loga a las
funciones cociente en espacios topol\'{o}gicos. M\'{a}s adelante veremos
qu\'{e}clase de transformaciones suaves cumplen un rol similar al de
los subespacios.

\begin{obsCaracterizaSubmersionesSobre}\label{obs:caracterizasubmersionessobre}
	Sea $\pi:\,M\rightarrow N$ como en \ref{thm:propisextrasubmersiones}.
	Si $\tilde{N}$ representa al mismo conjunto subyacente a $N$ con
	una topolog\'{\i}a posiblemente distinta (con respecto a la cual
	es una variedad topol\'{o}gica) y una estructura diferencial
	(posiblemente) distinta, pero de manera que el \'{\i}tem
	\emph{(a)} siga siendo v\'{a}lido reemplazando $N$ por $\tilde{N}$,
	entonces $\id:\,N\rightarrow\tilde{N}$ es difeomorfismo.

	Para demostrar esta afirmaci\'{o}n, consideramos los siguientes
	diagramas:
	\begin{center}
		\begin{tikzcd}
			M \arrow[r,equal] \arrow[d,"\tilde{\pi}"'] &
				M \arrow[d,"\pi"] \\
			\tilde{N} \arrow[r,"\id", shift left] &
			N \arrow[l, shift left]
		\end{tikzcd}
		%
		\begin{tikzcd}
			M \arrow[dr,dashed,"\tilde{\pi}"]
				\arrow[d,"\tilde{\pi}"'] & \\
			\tilde{N} \arrow[r,"{\id[\tilde{N}]}"'] & \tilde{N}
		\end{tikzcd}
	\end{center}

	Por hip\'{o}tesis, $\pi=\id[\rightarrow]\circ\tilde{\pi}$ es suave, si
	y s\'{o}lo si $\id[\rightarrow]$ es suave. Pero, tambi\'{e}n sabemos
	que $\pi$ es suave. Entonces $\id[\rightarrow]$ es suave. Por otro
	lado, por \emph{(a)}, $\tilde{\pi}=\id[\leftarrow]\circ\pi$ es suave,
	si y s\'{o}lo si $\id[\leftarrow]$ es suave. Pero $\tilde{\pi}$ es
	suave, pues: $\tilde{\pi}$ es suave, si y s\'{o}lo si $\id[\tilde{N}]$
	lo es, e $\id[\tilde{N}]$ siempre es suave. En definitiva,
	$\id[\leftarrow]$ tambi\'{e}n es suave, con lo que la identidad $\id$
	en el conjunto subyacente a $N$ determina un difeomorfismo entre las
	estructuras diferenciales de $N$ y de $\tilde{N}$. La conclusi\'{o}n
	es igual en el contexto de variedades topol\'{o}gicas.
\end{obsCaracterizaSubmersionesSobre}

\begin{coroUnicidadDelCociente}\label{thm:unicidaddelcociente}
	Sean $M$, $N_{1}$ y $N_{2}$ variedades \emph{sin} borde y sean
	$\pi_{1}:\,M\rightarrow N_{1}$ y $\pi_{2}:\,M\rightarrow N_{2}$
	submersiones suaves y sobreyectivas. Supongamos, adem\'{a}s, que
	$\pi_{1}(q)=\pi_{1}(q')\Leftrightarrow\pi_{2}(q)=\pi(q')$, es decir,
	cada una de las submersiones es constante en las fibras de la otra
	--dicho de otra manera, realizan las mismas identificaciones. Entonces
	existe un \'{u}nico difeomorfismo $F:\,N_{1}\rightarrow N_{2}$
	tal que $F\circ\pi_{1}=\pi_{2}$.
\end{coroUnicidadDelCociente}

