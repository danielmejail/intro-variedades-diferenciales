\theoremstyle{plain}
\newtheorem{teoDeLasFetas}{Teorema}[section]
\newtheorem{propoEmbeddingEsEmbedding}[teoDeLasFetas]{Proposici\'{o}n}
\newtheorem{propoRegularCerradaSiiPropia}[teoDeLasFetas]{Proposici\'{o}n}
\newtheorem{lemaContinuaPropia}[teoDeLasFetas]{Lema}
\newtheorem{lemaPropiaEsCerrada}[teoDeLasFetas]{Lema}
\newtheorem{teoDeLosConjuntosDeNivel}[teoDeLasFetas]{Teorema}
\newtheorem{coroDelValorRegular}[teoDeLasFetas]{Corolario}
\newtheorem{propoRegularEsLocalmenteDeNivel}[teoDeLasFetas]{Proposici\'{o}n}
\newtheorem{propoInmersaEsInmersa}[teoDeLasFetas]{Proposici\'{o}n}
\newtheorem{propoDeLasParametrizaciones}[teoDeLasFetas]{Proposici\'{o}n}

\theoremstyle{remark}
\newtheorem{obsCuandoInmersaEsRegular}{Observaci\'{o}n}[section]

%-------------

Dicho r\'{a}pidamente, una \emph{subvariedad} es un subconjunto de una
variedad que posee, a su vez, una topolog\'{\i}a y una estructura diferencial
(de manera que la inclusi\'{o}n sea suave). En esta secci\'{o}n indagamos un
poco m\'{a}s en las propiedades de los embeddings y de las inmersiones
con el objetivo de poder estudiar los distintos tipos de subvariedades.

\subsection{Subvariedades regulares}
En primer lugarm dada una variedad $M$, un subconjunto $S\subset M$ se dice
\emph{subvariedad regular} (o, a veces, \emph{subvariedad} a secas), si,
en tanto subespacio topol\'{o}gico es una variedad topol\'{o}gica
\emph{sin borde} y posee una estructura diferencial respecto de la cual
$\inc[S]:\,S\rightarrow M$ es un embedding (suave). Las subvariedades con
borde se definen de manera an\'{a}loga, pero algunos resultados v\'{a}lidos
para subvariedades sin borde no lo son para subvariedades con borde; las
definiremos m\'{a}s adelante.

La \'{u}ltima condici\'{o}n en la definici\'{o}n de subvariedad regular
se puede reemplazar por la condici\'{o}n de que $\inc[S]$ sea suave y que
el rango de $\inc[S]$ sea $\dim\,S$ (que, necesariamente, es
$\dim\,S\leq\dim\,M$), pues la condici\'{o}n de ser subespacio fue
incluida expl\'{\i}citamente antes.

Aunque parezca inmediato de las definiciones, demostramos la siguiente
proposici\'{o}n, pues, en alg\'{u}n sentido, introduce la idea de estructura
natural en una subvariedad.

\begin{propoEmbeddingEsEmbedding}\label{thm:embeddingesembedding}
	Sea $M$ una variedad diferencial y sea $N$ una variedad diferencial
	sin borde. Sea $F:\,N\rightarrow M$ un embedding suave. Sea
	$S=F(N)$ la imagen de $F$ en $M$. Entonces \emph{(a)} con la
	topolog\'{\i}a de subespacio de $M$, el subconjunto $S$ es una
	variedad topol\'{o}gica y \emph{(b)} admite una estructura
	diferencial de manera que $S$ sea una subvariedad regular de $M$ y
	$F|:\,N\rightarrow S$ sea un difeomorfismo. Esta topolog\'{\i}a
	y esta estructura diferencial en $S$ son las \'{u}nicas con
	estas propiedades.
\end{propoEmbeddingEsEmbedding}

\begin{proof}
	Le damos a $S=F(N)$ la topolog\'{\i}a de subespacio de $M$.
	De la definici\'{o}n de subvariedad regular, se deduce que esta
	es la \'{u}nica topolog\'{\i}a posible con la que $S$ puede ser
	una subvariedad regular. Como $F$ es un embedding, es, en
	particular, un subespacio y un homeomorfismo entre $N$ y $S$.
	Notemos tambi\'{e}n que, si deseamos que $F|:\,N\rightarrow S$
	sea un \emph{homeomorfismo}, como $F$ es subespacio, $S$ \emph{debe}
	tener la topolog\'{\i}a de subespacio de $M$. En particular, como
	$N$ es una variedad topol\'{o}gica, $S$ tambi\'{e}n lo es.
	
	De la misma manera, pasando a la estructura suave, si deseamos que
	$F|:\,N\rightarrow S$ sea un difeomorfismo, entonces los pares
	$(F(U),\varphi\circ F^{-1})$, donde $(U,\varphi)$ es una carta
	para $N$, deben ser cartas compatibles con la estructura de $S$.
	Notemos que la colecci\'{o}n
	\begin{align*}
		\cal{A}_{S} & \,=\,\left\lbrace (F(U),\varphi\circ F^{-1})\,:\,
			(U,\varphi)\text{ carta para } N\right\rbrace
	\end{align*}
	%
	es un atlas suavemente compatible (pues $N$ tiene una estructura
	diferencial y las cartas $(U,\varphi)$ consideradas son las
	cartas compatibles con dicha estructura). Con lo cual, la
	condici\'{o}n de que $F|:\,N\rightarrow S$ sea difeomorfismo
	impone una estructura suave sobre $S$. Finalmente, notemos que
	la inclusi\'{o}n $\inc[S]:\,S\rightarrow M$ se descompone de la
	siguiente manera:
	\begin{align*}
		\inc[S] & \,=\,F\circ (F|^{-1})
		\text{ ,}
	\end{align*}
	%
	donde $F|^{-1}$ es un difeomorfismo y $F$ es un embedding. En
	particular, $\inc[S]$ es embedding.
\end{proof}

Una subvariedad regular $S\subset M$ se dice \emph{propia}, si la
inclusi\'{o}n $\inc[S]:\,S\rightarrow M$ es una funci\'{o}n propia. Las
subvariedades propias son, exactamente, las subvariedades \emph{regulares} que
son cerradas en $M$.

\begin{propoRegularCerradaSiiPropia}\label{thm:regularcerradasiipropia}
	Sea $M$ una variedad diferencial y sea $S\subset M$ una
	subvariedad regular. Entonces $S$ es una subvariedad regular
	propia, si y s\'{o}lo si $S\subset M$ es una subconjunto cerrado.
\end{propoRegularCerradaSiiPropia}

Como toda funci\'{o}n continua y propia entre
variedades topol\'{o}gicas es cerrada, dichas funciones son, adem\'{a}s,
subespacio, con lo cual, si $\inc[S]:\,S\rightarrow M$ es continua y
propia, $\inc[S]$ es embedding (topol\'{o}gico). Es decir, no existen
subvariedades propias (tales que la inclusi\'{o}n sea propia) que no sean
subvariedades regulares (es decir, subespacios topol\'{o}gicos). Aun as\'{\i},
existen subvariedades \emph{cerradas} pero que no son regulares, es decir,
subespacios.

Antes de demostrar la proposici\'{o}n recordamos algunos lemas acerca de
funciones continuas propias.

\begin{lemaContinuaPropia}\label{thm:continuapropia}
	\emph{(a)} Sea $f:\,X\rightarrow Y$ una funci\'{o}n continua. Si
	$f$ es cerrada y las fibras $f^{-1}(y)$ con $y\in Y$ son compactas,
	entonces $f$ es propia; \emph{(b)} en particular, si $f$ es un
	embedding en un subespacio cerrado, entonces $f$ es propia;
	\emph{(c)} si $Y:T_{2}$ y $f$ admite una rettracci\'{o}n
	$g:\,Y\rightarrow X$ con $g\circ f=\id[X]$, entonces $f$ es
	propia; \emph{(d)} si $f$ es propia y $A\subset X$ es saturado
	($f^{-1}(f(A))=A$), entonces $f|_{A}:\,A\rightarrow f(A)$ es propia.
\end{lemaContinuaPropia}

\begin{proof}
	\emph{(a)}: Sea $K\subset Y$ un compacto. Sea $\{A_{i}\}_{i}$ una
	familia de subconjuntos cerrados de $f^{-1}(K)$ con la propiedad de
	la intersecci\'{o}n finita. Sea $I=\{i_{1},\,\dots,\,i_{k}\}$ un
	subconjunto finito de \'{\i}ndices. La familia
	\begin{align*}
		& \left\lbrace A_{I}:=A_{i_{1}}\cap \,\cdots\,\cap A_{i_{k}}
			\,:\, I=\{i_{1},\,\dots,\,i_{k}\}\right\rbrace
	\end{align*}
	%
	donde $I$ recorre todos los subconjuntos finitos de \'{\i}ndices
	constituye una familia de cerrados de $f^{-1}(K)$ con la propiedad
	de la intersecci\'{o}n finita, tambi\'{e}n. Como $f$ es cerrada,
	$f(A_{I})$ es cerrada en $K$ para cada $I$ (tomamos
	$\widetilde{A_{I}}$ cerrado en $X$ tal que
	$A_{I}=f^{-1}(K)\cap\widetilde{A_{I}}$, entonces $f(\widetilde{A_{I}}$
	es cerrado en $Y$ y $K\cap f(\widetilde{A_{I}})$ es igual $f(A_{I})$).
	La familia $\{f(A_{I})\}_{I}$ est\'{a} compuesta por cerrados (de $K$)
	y tales que
	\begin{align*}
		f(A_{I_{1}})\cap\,\cdots\,\cap f(A_{I_{t}}) & \,\supset\,
			f(A_{I_{1}}\cap\,\cdots\,\cap A_{I_{t}})\,\not=\,
			\varnothing
		\text{ .}
	\end{align*}
	%
	Concluimos entonces que $\{f(A_{I})\}_{I}$ tiene la propiedad de
	la intersecci\'{o}n finita, tambi\'{e}n. Como $K$ es compacto,
	existe $y\in K$ tal que $y\in\bigcap_{I}\,f(A_{I})$. Pero esto
	quiere decir que existen, para cada $I$, elementos $x_{I}\in A_{I}$
	tales que $f(x_{I})=y$, es decir, $x_{I}\in A_{I}\cap f^{-1}(y)$.
	Por hip\'{o}tesis, $f^{-1}(y)$ es compacta. Adem\'{a}s, por lo
	anterior,
	\begin{align*}
		\varnothing & \,\not=\,\big(A_{i_{1}}\cap\,\cdots\,\cap
			A_{i_{k}}\big)\cap f^{-1}(y) \,=\,
			\big(A_{i_{1}}\cap f^{-1}(y)\big) \cap\,\cdots\,\cap
			\big(A_{i_{k}}\cap f^{-1}(y)\big)
		\text{ .}
	\end{align*}
	%
	Entonces $\{A_{i}\cap f^{-1}(y)\}_{i}$ es una familia de cerrados
	en el compacto $f^{-1}(y)$ con la propiedad de intersecci\'{o}n
	finita. Por lo tanto,
	\begin{align*}
		\varnothing & \,\not=\,\bigcap_{i}\,
			\big(A_{i}\cap f^{-1}(y)\big) \,=\,
			\big(\bigcap_{i}\,A_{i}\big)\cap f^{-1}(y)
			\,\subset\, \bigcap_{i}\,A_{i}
		\text{ .}
	\end{align*}
	%

	\emph{(b)} Si asumimos que $f:\, X\rightarrow Y$ es subespacio y
	que $f(X)\subset Y$ es cerrada, entonces $f^{-1}(y)$ es vac\'{\i}a
	o consiste en un \'{u}nico punto. En cualquier caso,
	$f^{-1}(y)$ es compacta para todo $y\in Y$ y, como
	$f|:\,X\rightarrow f(X)$ es homeomorfismo, si $A\subset X$ es cerrado,
	entonces $f(A)$ es cerrada en $f(X)$ y, por lo tanto, en $Y$.

	\emph{(c)} Si $Y$ es Hausdorff y $f$ admite una retracci\'{o}n
	continua $g:\,Y\rightarrow X$, entonces, si $K\subset Y$ es compacto,
	entonces $K$ es cerrado en $Y$, $f^{-1}(K)$ es cerrado en $X$ y
	$g(K)$ es compacto (pues $g$ es continua). Pero, si $f(x)\in K$,
	entonces $x=g(f(x))$ y
	\begin{align*}
		f^{-1}(K) & \,\subset\, g(K)
		\text{ .}
	\end{align*}
	%
	Por lo tanto, $f^{-1}(K)$ es compacto.

	\emph{(d)} Si $K\subset f(A)$ es compacto, entonces $K\subset Y$
	es compacto y, por, hip\'{o}tesis, $f^{-1}(K)$ es compacto.
	Pero $f^{-1}(K)\subset f^{-1}(f(A))=A$.
\end{proof}

\begin{lemaPropiaEsCerrada}\label{thm:propiaescerrada}
	Si $f:\,x\rightarrow Y$ es continua, $Y$ es localmente compacto
	Hausdorff y, adem\'{a}s, $f$ es propia, entonces $f$ es cerrada.
\end{lemaPropiaEsCerrada}

\begin{proof}
	Sea $A\subset X$ cerrado. Sea $y\in\clos{f(A)}$ y sea $U\subset Y$
	un entorno de $y$ con clausura compacta $C$. Entonces
	$y\in\clos{f(A)\cap C}$ (si $V$ es entorno de $y$, entonces
	$V\cap U$ es abierta, $V\cap U\subset U\subset C$ y
	$(V\cap U)\cap f(A)\not=\varnothing$), con lo cual
	$(V\cap U)\cap (f(A)\cap C)\not=\varnothing$). Como $f$ es propia,
	$f^{-1}(C)$ es compacto y $A\cap f^{-1}(C)$ tambi\'{e}n, por ser
	cerrado contenido en un compacto. Como $f$ es continua,
	$f(A\cap f^{-1}(C))=f(A)\cap C$ es compacto. Como $Y$ es Hausdorff,
	$f(A)\cap C$ es cerrada e $y\in f(A)\cap C\subset f(A)$. Con lo
	cual $f(A)$ es cerrada.
\end{proof}

Ahora s\'{\i}, pasamos a la demostraci\'{o}n de la proposici\'{o}n.

\begin{proof}[Demostraci\'{o}n de \ref{thm:regularcerradasiipropia}]
	Si $S\hookrightarrow M$ es propia, $S\subset M$ es cerrado,
	por \ref{thm:propiaescerrada}. Si $S\subset M$ es cerrado,
	$\inc[S]:\,S\rightarrow M$ es subespacio cerrado y, por lo tanto,
	propia por \ref{thm:continuapropia}.
\end{proof}

\subsection{Cartas preferenciales}
Localmente, las subvariedadesregulares se pueden identificar con una
subvariedad lineal de un espacio euclideo ambiente. Para ser precisos,
llamamos, dado un abierto $U\subset\bb{R}^{n}$, \emph{feta de dimensi\'{o}n %
$k$}, o \emph{$k$-feta}, de $U$ a los subconjuntos de la forma
\begin{align*}
	S & \,=\,\left\lbrace (\lista*{x}{k},\,x^{k+1},\,\dots,\,x^{n})\in U
		\,:\, x^{k+1}=c^{k+1},\,\dots,\,x^{n}=c^{n}\right\rbrace
\end{align*}
%
para ciertas constantes reales $c^{k+1},\,\dots,\,c^{n}$. Es decir,
$S$ es igual a la intersecci\'{o}n de $U$ con un subesapcio lineal de
$\bb{R}^{n}$. En general, una $k$-feta $S$ es homeomorfa a un subconjunto
abierto de $\bb{R}^{k}$.

Dada una variedad (topol\'{o}gica o diferencial) $M$, definimos una
\emph{feta de dimensi\'{o}n $k$} de manera an\'{a}loga, tomando cartas.
Si $(U,\varphi)$ es una carta para $M$ y $S\subset U$ es un subconjunto
tal que $\varphi(S)\subset\varphi(U)\subset\bb{R}^{n}$ es una feta de
dimensi\'{o}n $k$ de $\varphi(U)$, decimos que $S$ es una feta de
dimensi\'{o}n $k$ de $U$. En ese caso, $S$ debe ser de la forma
$S=\{\varphi^{-1}(\lista*{x}{n})\,:\, x^{k+1}=c^{k+1},\,\dots,\,x^{n}=c^{n}\}$.
Como la aplicaci\'{o}n
\begin{align*}
	(\lista*{x}{n}) & \,\mapsto\, (\lista*{x}{k},\,x^{k+1}-c^{k+1},\,
		\dots,\,x^{n}-c^{n})
\end{align*}
%
es un difeomorfismo (homeomorfismo), toda $k$-feta de un abierto coordenado
$U$ es igual a la preimagen de los puntos tales que las \'{u}ltimas
coordenadas son iguales a cero.

Si, ahora, $S\subset M$ es un subconjunto arbitrario de la variedad, una
\emph{carta preferencial para $S$ en $M$} es una carta $(U,\varphi)$
tal que $S\cap U$ es, v\'{\i}a $\varphi$, una $k$-feta de $U$. Las
funciones coordenadas $(\lista*{x}{n})$ de una carta preferencial
para $S$ en $M$ se denominar\'{a}n \emph{coordenadas preferenciales}.
Si existe $k\geq 0$ tal que para todo punto del subconjunto $S$ existe un
entorno coordenado $U$ tal que $S\cap U$ es una $k$-feta de $U$, decimos que
\emph{$S$ verifica localmente (en $M$) la condici\'{o}n de ser $k$-feta} o
que \emph{es localmente una $k$-feta}. Dicho de otra manera, $S$ es localmente
una feta de dimensi\'{o}n $k$, si existe un cubrimiento de $S$ por entornos
coordenados $U$ tales que $S\cap U$ es una feta de dimensi\'{o}n $k$
de $U$. Con el nombre de carta preferencial la dimensi\'{o}n de la feta
resultante no queda especificada.

Las $k$-fetas de un abierto de $\bb{R}^{n}$ son homeomorfos a abiertos de
$\bb{R}^{k}$. Por lo tanto, es de esperar que un subconjunto de una variedad
que verifica localmente la condici\'{o}n de ser $k$-feta tenga, naturalmente,
una estructura de variedad. Por otro lado, si bien ser una subvariedad regular
es una propiedad global de un subconjunto $S\subset M$ (o, mejor dicho, de
la inclusi\'{o}n $\inc[S]$), es posible formular un criterio local para
determinar si un subconjunto admite una estructura de subvariedad regular
(es decir, determinar si $\inc[S]$ es un embedding).

\begin{teoDeLasFetas}\label{thm:delasfetas}
	Sea $M$ una variedad diferencial (sin borde) y sea $S\subset M$ una
	subvariedad regular de $M$ de dimensi\'{o}n $k$, Entonces
	para todo punto $p\in S$ existe un carta $(U,\varphi)$ para $M$
	en $p=\inc[S](p)$ tal que $S\cap U$ es una $k$-feta de $U$.
	Rec\'{\i}procamente, si $S\subset M$ es un subconjunto que
	verifica localmente la condici\'{o}n de ser una $k$-feta, entonces,
	con la topolog\'{\i}a de subespacio de $M$, $S$ es una variedad
	topol\'{o}gica de dimensi\'{o}n $k$ y, adem\'{a}s, admite una
	estructura diferencial de manera que resulta ser una subvariedad
	regular de $M$ de dimensi\'{o}n $k$.
\end{teoDeLasFetas}

\begin{proof}
	Supongamos, en primer lugar, que $S\subset M$ es una subvariedad
	regular de dimensi\'{o}n $k$. Por el teorema del rango
	\ref{thm:delrango}, como $\inc[S]:\,S\rightarrow M$ es una
	inmersi\'{o}n, si $p\in S$, existen coordenadas $(V,\psi)$ para
	$M$ en $p$ y $(U,\varphi)$ para $S$ tales que
	\begin{align*}
		\psi\circ\inc[S]\circ\varphi^{-1}(\lista*{x}{k}) &
			\,=\,(\lista*{x}{k},\,0,\,\dots,\,0)
		\text{ .}
	\end{align*}
	%
	Notemos que $\psi\circ\inc[S]\circ\varphi^{-1}=%
	\inc[U]:\,U\rightarrow V$. Es decir, $U$ se incluye en $V$ como
	una $k$-feta. El problema es que no podemos afirmar que $U$ sea
	exactamente $V\cap\{x^{k+1}=\,\cdots\,=x^{n}=0\}$.

	Tomemos $\epsilon>0$ tal que
	$U_{0}=\varphi^{-1}(\bola[k]{\epsilon}{0})\subset U$ y
	$V_{0}=\psi^{-1}(\bola[n]{\epsilon}{0})\subset V$. Como
	$U_{0}\subset S$ es abierto, existe $W\subset M$ tal que
	$U_{0}=S\cap W$. Sea $V_{1}=W\cap V_{0}$ y sea
	$\psi_{1}=\psi|_{V_{1}}$. Entonces $(V_{1},\psi_{1})$ es una carta
	para $M$ en $p$ y $S\cap V_{1}=U_{0}\cap V_{0}=U_{0}$. Entonces
	$V_{0}\cap\{x^{k+1}=\,\cdots\,=x^{n}=0\}=U_{0}$, $U_{0}$ es una
	$k$-feta de $V_{1}$ y $(V_{1},\psi_{1})$ es una carta preferencial
	para $S$ en $M$ en $p$.

	Rec\'{\i}procamente, supongamos que $S\subset M$ es un subconjunto
	que verifica localmente la condici\'{o}n de ser una feta de
	dimensi\'{o}n $k$ de $M$. En la topolog\'{\i}a de subespacio,
	$S$ es $T_{2}$ y $N_{2}$. Veamos que es localmente euclideo
	y que su dimensi\'{o}n es $k$.
	
	Sea $\pi:\,\bb{R}^{n}\rightarrow\bb{R}^{k}$ la proyecci\'{o}n en
	las primeras $k$ coordenadas. Sea $p\in S$ y sea $(U,\varphi)$ una
	carta preferencial para $S$ en $M$ en $p$. Sea $V=S\cap U$ la
	$k$-feta propio, sea $\widehat{V}=\pi\circ\varphi(V)$ lo que
	deber\'{\i}a ser la imaen de la carta para $S$ cuando \'{e}sta sea
	definida y sea $\psi=\pi\circ\varphi|_{V}:\,V\rightarrow\widehat{V}$
	la funci\'{o}n que ser\'{a} la carta. Por definici\'{o}n,
	$\varphi(V)=\varphi(U)\cap A$ para cierta subvariedad lineal
	$A\subset\bb{R}^{n}$. Supongamos que $A$ est\'{a} dada como
	el conjunto de ceros de las siguientes ecuaciones:
	\begin{align*}
		x^{k+1} \,=\,c^{k+1} \text{ ,}\dots\text{ ,}
		x^{n} \,=\,c^{n}
		\text{ .}
	\end{align*}
	%
	Como $\varphi(U)$ es abierto en $\bb{R}^{n}$, la $k$-feta
	$\varphi(V)$ es abierta en $A$. Como
	$\pi|_{A}:\,A\rightarrow\bb{R}^{k}$ es un difeomorfismo,
	$\widehat{V}=\pi(\varphi(V))$ es un subconjunto abierto de
	$\bb{R}^{k}$. Resta ver que $\psi$ es un homeomorfismo. La
	funci\'{o}n $\psi$ es continua por ser composici\'{o}n de
	funciones continuas: $\psi=\pi\circ\varphi\circ\inc[V]$; tiene
	inversa dada por
	\begin{align*}
		\varphi^{-1}\circ j\circ\inc[\widehat{V}]
			(\lista*{x}{k}) & \,=\,
		\varphi^{-1}(\lista*{x}{k},\,c^{k+1},\,\dots,\,c^{n})
		\text{ ,}
	\end{align*}
	%
	que es continua.

	En cuanto a la estructura diferencial en $S$, usamos las cartas
	reci\'{e}n definidas: dadas $(U,\varphi)$ y $(U',\varphi')$ cartas
	para $M$ tales que $V=S\cap U$ y $V'=U'\cap S$ son $k$-fetas
	de $U$ y de $U'$, respectivamente, tenemos cartas
	$(v,\psi)$ y $(V',\psi')$ en $S$ que verifican
	\begin{align*}
		\psi'\circ\psi^{-1} & \,=\, (\pi\circ\varphi'\circ\inc[V'])
			\circ (\pi\circ\varphi\circ\inc[V])^{-1} \\
		& \,=\, \pi\circ\varphi'\circ\inc[V']\circ\varphi^{-1}
			\circ j\circ\inc[\widehat{V}](\lista*{x}{k}) \\
		& \,=\,\pi\circ (\varphi'\circ\varphi^{-1})\circ
			j(\lista*{x}{k})
		\text{ .}
	\end{align*}
	%
	Pero $j$ y $\pi$ son suaves y $\varphi'$ y $\varphi$ son suavemente
	compatibles. Entonces $(V,\psi)$ y $(V',\psi')$ son compatibles.
	Con respecto a la estructura determinada por estas cartas en $S$,
	$\widehat{\inc[S]}(\lista*{x}{k})=%
	(\lista*{x}{k},\,c^{k+1},\,\dots,\,c^{n})$ localmente, que resulta
	pues una inmersi\'{o}n suave. Como $\inc[S]$ es, por definici\'{o}n
	de la topolog\'{\i}a de $S$, subespacio, la transformaci\'{o}n
	$\inc[S]$ es embedding suave.
\end{proof}

Demostraremos luego que esta estructura en $S$ es la \'{u}nica posible que
hace que $S$ sea una subvariedad de $M$. Con lo cual, si imponemos que
$S$ sea subespacio, admite una \'{u}nica estructura de subvariedad
(necesariamente regular). Pero $S$ podr\'{\i}a admitir otras topolog\'{\i}as
respecto a las cuales resulte variedad topol\'{o}gica y, por lo tanto,
podr\'{\i}a ser subvariedad (aunque no subvariedad regular). Para que
esto quede claro, habr\'{a} que introducir una noci\'{o}n m\'{a}s general
de subvariedad.

El borde de una variedad con borde es una subvariedad regular.

\subsection{Conjuntos de nivel}
Las subvariedades regulares se pueden expresar localmente como el
gr\'{a}fico de una funci\'{o}n, como tambi\'{e}n como el conjunto de ceros
de una funci\'{o}n (cartas preferenciales). Dada una transformaci\'{o}n
$\Phi:\,M\rightarrow N$ y un punto $c\in N$, decimos que
$\Phi^{-1}(c)\subset M$ es un \emph{conjunto de nivel de $\Phi$}. Todo
subconjunto cerrado de una variedad $M$ es el conjunto de ceros
($N=\bb{R}$, $c=0$) de alguna funci\'{o}n (suave). Para poder relacionar
los conceptos de conjunto de nivel y de subvariedad es necesario imponer
alguna condici\'{o}n adicional.

\begin{teoDeLosConjuntosDeNivel}
	[el conjunto de nivel de una funci\'{o}n de %
	rango constante]\label{thm:denivel}
	Sean $M$ y $N$ variedades diferenciales. Sea $\Phi:\,M\rightarrow N$
	una transformaci\'{o}n suave de rango constante $r$. Cada conjunto
	de nivel $\Phi^{-1}(c)\subset M$ es una subvariedad propia de
	codimensi\'{o}n $r$ en $M$.
\end{teoDeLosConjuntosDeNivel}

\begin{proof}
	Sea $m=\dim\,M$ y sea $n=\dim\,N$. Por el teorema del rango constante,
	para cada punto $p\in M$ existe una carta $(U,\varphi)$ para $M$
	en $p$ y existe una carta $(V,\psi)$ para $N$ en $\Phi(p)=c$ tales
	que $\Phi(U)\subset V$ y
	\begin{align*}
		\widehat{\Phi}(\lista*{x}{m}) & \,=\,
			(\lista*{x}{r},\,0,\,\dots,\,0)
		\text{ .}
	\end{align*}
	%
	En particular, el \emph{conjunto} $S=\Phi^{-1}(c)$ verifica la
	condici\'{o}n de ser localmente un $k$-feta con $k=m-r$, es
	decir, una feta de codimensi\'{o}n $r$. Por el teorema
	\ref{thm:delasfetas}, $S$ tiene estructura de subvariedad regular
	de $M$. Como $\Phi$ es continua, $S$ es cerrada en $M$ y, por lo
	tanto, subvariedad regular propia.
\end{proof}

Un caso particular de esto se da cuando $\Phi$ es una submersi\'{o}n.
En concordancia con el hecho de que las transformaciones de rango constante
son localmente como transformaciones lineales, el resultado anterior es
an\'{a}logo al resultado de \'{A}lgebra lineal que dice que, dada una
transformaci\'{o}n lineal $L:\,\bb{R}^{m}\rightarrow\bb{R}^{r}$ sobreyectiva
(o $\bb{R}^{m}\rightarrow\bb{R}^{n}$ de rango $r$), el n\'{u}cleo $\ker(L)$
es un subespacio de codimensi\'{o}n $r$, determinado por un sistema de $r$
ecuaciones lineales independientes igualadas a cero. En el caso de
variedades, si $\Phi:\,M\rightarrow N$ es una submersi\'{o}n, entonces
$\Phi^{-1}(c)$ --el an\'{a}logo del n\'{u}cleo de una t.l.-- es una
subvariedad de codimensi\'{o}n $n=\dim\,N$.

\subsection{Puntos y valores regulares}
Dada una transformaci\'{o}n suave $\Phi:\,M\rightarrow N$, decimos que
$p\in M$ es un \emph{punto regular}, si la transformaci\'{o}n lineal
$\diferencial[p]{\Phi}:\,\tangente[p]{M}\rightarrow\tangente[\Phi(p)]{N}$
es sobreyectiva. En otro caso, decimos que $p$ es un \emph{punto cr\'{\i}tico}.
El subconjunto de $M$ de puntos regulares es igual al subconjunto en donde
$\Phi$ tiene rango m\'{a}ximo. Este subconjunto es abierto en $M$ y, por lo
tanto, una subvariedad regular. Un punto $c\in N$ se dice \emph{valor %
regular} de $\Phi$, si todo punto perteneciente a $\Phi^{-1}(c)$ es un punto
regular. Si al menos un punto de la preimagen es punto cr\'{\i}tico, entonces
decimos que $c$ es un valor cr\'{\i}tico de $\Phi$. Un \emph{conjunto de %
nivel regular} de $\Phi$ es el conjunto de nivel correspondiente a un valor
regular de $\Phi$.

\begin{coroDelValorRegular}[teorema del valor regular]%
	\label{thm:delvalorregular}
	Sea $\Phi:\,M\rightarrow N$ una transformaci\'{o}n suave entre
	variedades diferenciales \emph{sin} borde. Todo conjunto de nivel
	regular de $\Phi$ es una subavariedad regular propia de $M$.
	Su codimensi\'{o}n es igual a la dimensi\'{o}n del codominio $N$.
\end{coroDelValorRegular}

\begin{proof}
	Si $c\in N$ es un valor regular y $S=\Phi^{-1}(c)$ es el conjunto de
	nivel correspondiente, entonces $S\subset M'$, donde
	$M'\subset M$ es la subvariedad regular abierta comformada por los
	puntos $p$ en los que $\diferencial[p]{\Phi}$ es sobreyectivo.
	La restricci\'{o}n $\Phi|_{M'}:\,M'\rightarrow N$ es una
	submersi\'{o}n, con lo cual, por el teorema \ref{thm:denivel},
	$\Phi^{-1}(c)\subset M'$ es una subvariedad regular. Pero
	$\inc[S]:\,\Phi^{-1}(c)\rightarrow M$ es igual a la composici\'{o}n
	$\inc[M']\circ\inc[S]':\,S\rightarrow M'\rightarrow M$. Tanto
	$\inc[S]'$ como $\inc[M']$ son embeddings. Entonces la inclusi\'{o}n
	de $S$ en $M$ lo es, tambi\'{e}n. Como $\Phi^{-1}(c)\subset M$ es
	cerrada, por continuidad, $S=\Phi^{-1}(c)$ es una subvariedad regular
	propia de $M$. Como $\codim(S,M')=\codim(S,M)$, porque
	$M'\subset M$ es abierto, se deduce que $\codim(S,M)=\dim(N)$.
\end{proof}

\begin{propoRegularEsLocalmenteDeNivel}\label{thm:regulardenivel}
	Sea $S\subset M$ un subconjunto de una variedad diferencial $M$ de
	dimensi\'{o}n $m$. Entonces $S$ es una subvariedad regular de
	dimensi\'{o}n $k$, si y s\'{o}lo si, para cada punto $p\in S$,
	es posible hallar una submersi\'{o}n $\Phi:\,U\rightarrow\bb{R}^{m-k}$
	definida en un abierto $U$ \emph{de $M$} tal que
	$\Phi^{-1}(c)=S\cap U$ para alg\'{u}n valor $c\in\bb{R}^{m-k}$.
\end{propoRegularEsLocalmenteDeNivel}

Esto es casi una reformulaci\'{o}n del criterio local de las fetas.

\begin{proof}
	Si $S\subset M$ es una subvariedad regular de dimensi\'{o}n $k$ de $M$
	y $p\in S$, podemos hallar una carta $(U,\varphi)$ para $M$ tal que
	$S\cap U$ es una feta de dimensi\'{o}n $k$ en $U$. Si
	$\varphi=(\lista*{x}{m})$, entonces $S\cap U=%
	\{x^{k+1}=\,\cdots\,=x^{n}=0\}$. Si definimos
	$\Phi:\,U\rightarrow\bb{R}^{m-k}$ por
	\begin{align*}
		\Phi(\lista*{x}{m}) & \,=\,(x^{k+1},\,\dots,\,x^{m})
		\text{ ,}
	\end{align*}
	%
	la proyecci\'{o}n en las \emph{\'{u}ltiimas} coordenadas, entonces
	$S\cap U=\Phi^{-1}(0)$. Rec\'{\i}procamente, si $S\subset M$ es un
	subconjunto de $M$ que verifica que, para cada punto, existe una
	funci\'{o}n $\Phi$ como en el enunciado y $p\in S$, entonces
	podemos elegir un abierto $U\subset M$ y una submersi\'{o}n
	$\Phi:\,U\rightarrow\bb{R}^{m-k}$, de manera que
	$S\cap U=\Phi^{-1}(c)$ para cierto valor $c\in \bb{R}^{m-k}$.
	Hay dos maneras de concluir: o bien usamos esto para definir una
	carta preferencial para $S$ en $M$ en $p$, o bien, por el teorema
	\ref{thm:delasfetas}, concluimos que $S\cap U$ es una subvariedad
	regular de $U$. En cualquier caso, $S\cap U$ es localmente en $U$
 	una feta de codimensi\'{o}n $m-k$. Como $U\subset M$ es abierto,
	concluimos que $S$ es localmente en $M$ una feta de codimensi\'{o}n
	$m-k$ (pues $\codim(U,M)=0$), o sea que $S$ es una subvariedad
	regular de dimensi\'{o}n $k$ en $M$.
\end{proof}

No sabemos, \textit{a priori}, que existan cartas preferenciales para $S$ en
$M$, pero, por el teorema del conjunto de nivel de una submersi\'{o}n, sabemos
que, para cada punto $p\in S$, existe un abierto $U$ de $M$ tal que
$p\in U$ y que $S\cap U$ es subvariedad regular de $U$. Por lo tanto,
existen cartas preferenciales para $S\cap U$ en $U$. Como $U$ es abierto
en $M$, podemos concluir que existen cartas preferenciales para $S$ en $M$
cerca de cada punto $p\in S$ y que, por lo tanto, $S$ es una subvariedad
regular de $M$.

\subsection{Subvariedades inmersas}
Una \emph{subvariedad inmersa} de una variedad (diferencial, topol\'{o}gica)
$M$ es un subconjunto $S$ con una topolog\'{\i}a con respecto a la cual
resulta una variedad topol\'{o}gica y una estructura diferencial con respecto
a la cual $\inc[S]:\,S\rightarrow M$ es una inmersi\'{o}n (suave, continua).

\begin{propoInmersaEsInmersa}\label{thm:inmersaesinmersa}
	Sea $M$ una variedad diferencial y sea $N$ una variedad diferencial
	sin borde. Sea $F:\,N\rightarrow M$ una inmersi\'{o}n suave e
	inyectiva y sea $S=F(N)\subset M$. Existen una \'{u}nica
	topolog\'{\i}a y una \'{u}nica estructura diferencial en $S$ con
	respecto a las cuales $S\subset M$ es una subvariedad inmersa y
	$F|:\,N\rightarrow S$ es un difeomorfismo.
\end{propoInmersaEsInmersa}

\begin{proof}
	Como queremos que $F$ sea un difeorfismo entre $N$ y $S$, en
	particular, la topolog\'{\i}a en $S$ deber\'{a} ser tal que
	$N$ y $S$ sean homeomorfos v\'{\i}a $F$. Definimos una topolog\'{\i}a
	en $S$ trasladando la topolog\'{\i}a de $N$: un subconjunto
	$U\subset S$ es abierto, si y s\'{o}lo si $F^{-1}(U)$ es abierto
	en $N$. Esto determina una topolog\'{\i}a en $S$ y es la \'{u}nica
	con respecto a la cual $F$ puede ser un homeomorfismo. De la misma
	manera, con respecto a la estructura diferencial, cubrimos a $S$
	con los pares $(F(U),\varphi\circ F^{-1})$, donde $(U,\varphi)$
	es una carta compatible con la estructura de $N$. Si queremos que $F$
	sea un difeomorfismo, todas estas cartas deben pertenecer a la
	estructura de $S$. Estas cartas en $N$ forman un atlas compatible y,
	por lo tanto, determinan una estructura diferencial en $S$. Finalmente,
	resta ver que, con esta topolog\'{\i}a y esta estructura suave,
	$\inc[S]:\,S\rightarrow M$ es una inmersi\'{o}n. Pero
	$\inc[S]=F\circ (F|^{-1})$, $F|$ es un difeomorfismo y $F$ es una
	inmersi\'{o}n. En particular, la inclusi\'{o}n es composici\'{o}n
	de inmersiones suaves y, en consecuencia, inmersi\'{o}n suave,
	tambi\'{e}n.
\end{proof}

\begin{obsCuandoInmersaEsRegular}\label{obs:cuandoinmersionesembedding}
	De la proposici\'{o}n \ref{thm:cuandoinmersionesembedding} podemos
	deducir los siguientes criterios para determinar si una subvariedad
	inmersa es regular: \emph{(a)} $\codim(S,M)=0$; \emph{(b)}
	$\inc[S]:\,S\hookrightarrow M$ es propia; \emph{(c)} $S$ es compacta.
	En cualquiera de estos casos, $S$ es una subvariedad regular.
	Adem\'{a}s, usando el teorema \ref{thm:embeddinglocal},
	que dice que las inmersiones son localmente embeddings, deducimos que
	toda subvariedad inmersa es localmente regular. Precisamente,
	si $M$ es una variedad diferencial y $S\subset M$ es una subvariedad
	inmersa, entonces, para cada punto $p\in S$, existe un abierto
	$V$ \emph{de $S$}, tal que $p\in V$ y $V\subset M$ es una subvariedad
	regular.
\end{obsCuandoInmersaEsRegular}

Terminamos esta secci\'{o}n definiendo el concepto de
\emph{parametrizaci\'{o}n} de una subvariedad. Sea $M$ una variedad
diferencial. Sea $S\subset M$ una subvariedad inmersa de dimensi\'{o}n $k$.
Una funci\'{o}n
\begin{align*}
	X & \,:\,U\subset\bb{R}^{k}\,\rightarrow M
\end{align*}
%
se dice \emph{parametrizaci\'{o}n (local)} de $S$, si $X:\,U\rightarrow M$
es continua en $M$, $U\subset\bb{R}^{k}$ es abierto, la imagen $X(U)\subset S$
es abierta en $S$ y $X|:\,U\rightarrow X(U)$ es un homeomorfismo
($X|:\,U\rightarrow S$ es subespacio abierta). Si $X|$ es difeomorfismo,
decimos que $X$ es \emph{parametrizaci\'{o}n local suave}.

Todo punto de una subvariedad inmersa est\'{a} en la imagen de una
parametrizaci\'{o}n local suave y toda parametrizaci\'{o}n da lugar a una
carta.

\begin{propoDeLasParametrizaciones}\label{thm:delasparametrizaciones}
	Sea $M$ una variedad diferencial y sea $S\subset M$ una
	subvariedad inmersa de dimensi\'{o}n $k$. Sea $U\subset\bb{R}^{k}$
	un subconjunto abierto. Entonces una funci\'{o}n $X:\,U\rightarrow M$
	es una parametrizaci\'{o}n local suave de $S$, si y s\'{o}lo si
	existe una carta $(V,\psi)$ para $S$ tal que $X=\inc[S]\circ\psi^{-1}$.
\end{propoDeLasParametrizaciones}

