\documentclass[11pt]{report}
\usepackage[utf8]{inputenc}

%Page dimensions
\usepackage{geometry}
\geometry{a4paper}

\usepackage{graphicx}

%More packages
\usepackage{formato_general}
\usepackage{abreviaciones}
\usepackage{nombres}
\usepackage{abreviacionesGeo}

\title{Introducci\'{o}n a las variedades diferenciales}
\author{}
\date{2020} % Activate to display a given date or no date (if empty),
	% otherwise the current date is printed


\begin{document}
\maketitle

\tableofcontents

%--------

\chapter{Variedades diferenciales}
\section{Variedades topol\'{o}gicas}
\theoremstyle{plain}
\newtheorem{teoDeLaDim}{Teorema}[section]
\newtheorem{propoAbiertoEsLocEuc}[teoDeLaDim]{Proposici\'{o}n}
\newtheorem{lemaCocienteHaus}[teoDeLaDim]{Lema}
\newtheorem{lemaCocienteLocEuc}[teoDeLaDim]{Lema}
\newtheorem{lemaBolasCoordenadas}[teoDeLaDim]{Lema}
\newtheorem{propoVariedadesSonLocArco}[teoDeLaDim]{Proposici\'{o}n}
\newtheorem{propoVarTopLocComp}[teoDeLaDim]{Proposici\'{o}n}
\newtheorem{propoVarTopParacomp}[teoDeLaDim]{Proposici\'{o}n}
\newtheorem{propoVarTopSigmacomp}[teoDeLaDim]{Proposici\'{o}n}
\newtheorem{lemaNdosImplicaSubcubNumerable}[teoDeLaDim]{Lema}

\newtheorem{teoCaractDeSubesp}[teoDeLaDim]{Teorema}
\newtheorem{teoUnicidadDeSubesp}[teoDeLaDim]{Teorema}

\newtheorem{lemaEntornoDeUnPuntoLCH}[teoDeLaDim]{Lema}
\newtheorem{lemaEntornoDeUnCompactoLCH}[teoDeLaDim]{Lema}
\newtheorem{propoUrysohnLCH}[teoDeLaDim]{Proposici\'{o}n}
\newtheorem{propoTietzeLCH}[teoDeLaDim]{Proposici\'{o}n}

\theoremstyle{remark}
\newtheorem{remarkVarTopParacomp}{Observaci\'{o}n}[section]
\newtheorem{remarkVarTopParacompI}[remarkVarTopParacomp]{Observaci\'{o}n}
\newtheorem{remarkVarTopParacompII}[remarkVarTopParacomp]{Observaci\'{o}n}
\newtheorem{remarkVarTopParacompIII}[remarkVarTopParacomp]{Observaci\'{o}n}
\newtheorem{remarkVarTopParacompIV}[remarkVarTopParacomp]{Observaci\'{o}n}

%--------------------

\subsection{Definiciones y repaso de nociones de Topolog\'{\i}a}

Sea $M$ un espacio topol\'{o}gico. Se dice que $M$ es una \emph{variedad %
topol\'{o}gica de dimensi\'{o}n $n$}, si
\begin{itemize}
	\item[(\i)] $M$ es $T_{2}$,
	\item[(\i\i)] $M$ es $N_{2}$ y
	\item[(\i\i\i)] $M$ es localmente euclideo de dimensi\'{o}n $n$,
\end{itemize}
%
es decir,
\begin{itemize}
	\item[(\i)] dados dos puntos $p,q\in M$ distintos, existen entornos
		abiertos $U$ y $V$ de $p$ y de $q$, respectivamente, tales
		que $U\cap V=\varnothing$,
	\item[(\i\i)] existe una familia numerable de abiertos de $M$,
		$\{U_{n}\}_{n\geq 1}$ que constituye una base para la
		topolog\'{\i}a de $M$ y
	\item[(\i\i\i)] dado $p\in M$, existe $U\subset M$ entorno abierto
		de $p$ homeomorfo a un abierto de $\bb{R}^{n}$.
\end{itemize}
%

El teorema de la dimensi\'{o}n implica que la noci\'{o}n de dimensi\'{o}n en
un espacio localmente euclideo est\'{a} bien definida. En particular,
la dimensi\'{o}n de una variedad topol\'{o}gica (no vac\'{\i}a\dots)
est\'{e} bien definida, tambi\'{e}n.

\begin{teoDeLaDim}[de la dimensi\'{o}n]\label{thm:deladim}
	Sean $U\subset\bb{R}^{n}$ y $V\subset\bb{R}^{m}$ abiertos no
	vac\'{\i}os y homeomorfos. Entonces $n=m$.
\end{teoDeLaDim}

Las propiedades de ser Hausdorff y de admitir una base numerable para la
topolog\'{\i}a se preservan al pasar a un subespacio. En el caso de la
propiedad de ser localmente euclideo de dimensi\'{o}n $n\geq 0$, esto
mismo es cierto para subespacios abiertos.

\begin{propoAbiertoEsLocEuc}\label{thm:abiertoesloceuc}
	Sea $X$ un espacio topol\'{o}gico y sea $Y$ un subespacio.
	\begin{itemize}
		\item[(a)] Si $X$ es $T_{2}$, $Y$ es $T_{2}$.
		\item[(b)] Si $X$ es $N_{2}$, $Y$ es $N_{2}$.
		\item[(c)] Si $X$ es localmente euclideo de dimensi\'{o}n
			$n\geq 0$ e $Y$ es abierto, entonces $Y$ es
			localmente euclideo de dimensi\'{o}n $n$.
	\end{itemize}
	%
	En particular, si $M$ es una variedad topol\'{o}gica de dimensi\'{o}n
	$n\geq 0$ y $U\subset M$ es abierto, $U$ tambi\'{e}n tiene estructura
	de variedad topol\'{o}gica de dimensi\'{o}n $n$.
\end{propoAbiertoEsLocEuc}

\begin{lemaNdosImplicaSubcubNumerable}\label{thm:subcubnum}
	Si $X$ es un espacio topol\'{o}gico $N_{2}$, entonces todo cubrimiento
	de $X$ por abiertos admite un subcubrimiento numerable.
\end{lemaNdosImplicaSubcubNumerable}

\begin{lemaCocienteHaus}\label{thm:cocientehaus}
	Sean $X$ un espacio topol\'{o}gico Hausdorff y sea
	$q:\,X\rightarrow X/R$ una funci\'{o}n cociente \emph{abierta}
	donde $R\subset X\times X$ es una relaci\'{o}n. Entonces el
	cociente $X/R$ es Hasudorff, si y s\'{o}lo si $R$ es cerrada en
	el producto.
\end{lemaCocienteHaus}

\begin{lemaCocienteLocEuc}\label{thm:cocienteloceuc}
	Sea $X$ un espacio topol\'{o}gico $N_{2}$ y sea $R$ una relaci\'{o}n
	tal que el cociente $X/R$ es localmente euclideo. Entonces $X/R$
	es, tambi\'{e}n $N_{2}$.
\end{lemaCocienteLocEuc}

\subsection{Variedades con borde y variedades con esquinas}
Una \emph{variedad con borde}, espec\'{\i}ficamente, una \emph{variedad %
topol\'{o}gica con borde} se define como un espacio topol\'{o}gico
Hausdorff y $N_{2}$ tal que todo punto del mismo tiene un entorno homeomorfo
a un abierto del semiespacio superior $\bb{H}^{d}$ para alg\'{u}n $d\geq 0$.
El valor de $d$ es la dimensi\'{o}n de la variedad (y, por un corolario del
teorema de la dimensi\'{o}n \ref{thm:deladim}, est\'{a} bien definida).
Es decir, en lugar de estar modelado localmente como $\bb{R}^{d}$, una
variedad con borde es localmente como
\begin{align*}
	\bb{H}^{d} & \,=\,\left\lbrace (\lista*{x}{d})\in\bb{R}^{d}\,:\,
				x^{d}\geq 0\right\rbrace
	\text{ .}
\end{align*}
%
Si $M$ es una variedad con borde y $p\in M$, existe un abierto $U$ de $M$
y un homeomorfismo $\varphi:\,U\rightarrow\varphi(U)$ con un abierto de
$\bb{H}^{d}$ tal que $p\in U$. Como en el caso de una variedad topol\'{o}gica,
un par $(U,\varphi)$ se denominar\'{a} \emph{carta} para $M$ en $p$.

El \emph{borde} de $\bb{H}^{d}$ en $\bb{R}^{d}$ es el conjunto de puntos
$\lista*{x}{d}$ tales que $x^{d}=0$, lo denotaremos $\partial\bb{H}^{d}$.
El \emph{interior} de $\bb{H}^{d}$ se define como el conjunto de puntos
$\lista*{x}{d}$ tales que $x^{d}>0$ y lo denotamos $\interior{\bb{H}^{d}}$.
Si $M$ es una variedad con borde, el \emph{borde} de $M$ ser\'{a} el
conjunto de puntos $p\in M$ para los cuales existe una carta $(U,\varphi)$
tal que $\varphi(p)\in\partial\bb{H}^{d}$, es decir, $\pi^{d}(\varphi(p))=0$.
El \emph{interior} de $M$ se define como el subconjunto formado por aquellos
puntos $p\in M$ para los cuales existe una carta $(U,\varphi)$ tal que
$\varphi(p)\in\interior{\bb{H}^{d}}$, es decir, $\pi^{d}(\varphi(p))>0$.
Los puntos del interior de $M$ admiten entornos homeomorfos a abiertos de
$\bb{R}^{d}$. Denotamos el borde de $M$ por $\partial M$ y su interior por
$\interior{M}$.

Si bien los conjuntos $\interior{M}$ y $\partial M$ est\'{a}n bien definidos
e, intuitivamente, deber\'{\i}an ser disjuntos, no es claro, \textit{a %
priori} que as\'{\i} lo sea.


El interior $\interior{M}$ de una variedad $M$ de dimensi\'{o}n $d$ es una
variedad de dimensi\'{o}n $d$ (sin borde), pues es un subespacio
abierto de la variedad $M$. El borde $\partial M$ tambi\'{e}n es una
variedad topol\'{o}gica (sin borde). Su dimensi\'{o}n es $d-1$: si $p$
es un punto del borde y $(U,\varphi)$ es una carta para $M$ en $p$,
entonces
\begin{align*}
	U\cap\partial M & \,=\,\left\lbrace q\in U\,:\,\pi^{d}(\varphi(q))=0
				\right\rbrace
	\text{ .}
\end{align*}
%
De esto se deduce que $(U\cap\partial M,\tilde{\varphi})$ es una carta para
$\partial M$ en $p$, donde $\tilde{\varphi}=(\lista*{\varphi}{d-1})$ --es
decir, proyectar sobre las primeras $d-1$ coordenadas la \emph{coordenada}
$\varphi$, valga la redundancia. La imagen de esta carta es un abierto de
$\bb{R}^{d-1}$ dado por intersecar el abierto $\varphi(U)$ de $\bb{R}^{d}$
con el hiperplano $\{x^{d}=0\}$ (y proyectar sobre las primeras $d-1$
coordenadas).

\begin{subsubsection}{Subespacios}
Dado un espacio topol\'{o}gico $X$, un subespacio es un subconjunto
$A\subset X$ al cual se le da la topolog\'{\i}a cuyos abiertos son,
precisamente, los subconjuntos que se obtienen de intersecar $A$ con un
abierto cualquiera de $X$. Una funci\'{o}n continua $i:\,A\rightarrow X$
entre espacios topol\'{o}gicos se dice subespacio, si es
inyectiva y determina un homeomorfismo con su imagen, es decir, $A$ e $i(A)$
son homeomorfos, donde a $i(A)$ se le da la topolog\'{\i}a subespacio de $X$.
Tambi\'{e}n se dir\'{a} que $i$ es un \emph{embedding topol\'{o}gico}.
La inclusi\'{o}n de un subespacio es una funci\'{o}n subespacio.

\begin{teoCaractDeSubesp}[Propiedad caracter\'{\i}stica de la %
	topolog\'{\i}a subespacio]\label{thm:caractdesubesp}
	Sea $X$ un espacio topol\'{o}gico y sea $\inc{A}:\,A\hookrightarrow X$
	un subespacio. Para todo espacio $Y$ y toda funci\'{o}n (conjuntista)
	$f:\,Y\rightarrow A$, la funci\'{o}n $f$ es continua, si y s\'{o}lo
	si la composici\'{o}n $\inc{A}\circ f$ es continua.
\end{teoCaractDeSubesp}

Las funciones subespacio est\'{a}n caracterizadas por la propiedad anterior.
Es decir, si $i:\,A\rightarrow X$ verifica el enunciado de la proposici\'{o}n
anterior en el lugar de $\inc{A}$, entonces $i$ es una funci\'{o}n
subespacio. M\'{a}s aun, la topolog\'{\i}a de subespacio es \'{u}nica de
manera que la propiedad se verifica.

\begin{teoUnicidadDeSubesp}[Unicidad de la topolog\'{\i}a subespacio]
	\label{thm:unicidaddesubesp}
	Sea $A$ un subconjunto de un espacio topol\'{o}gico $X$. Entonces
	la topolog\'{\i}a de subespacio en $A$ es la \'{u}nica topolog\'{\i}a
	para la cual se verifica la propiedad caracter\'{\i}stica de
	\ref{thm:caractdesubesp}.
\end{teoUnicidadDeSubesp}

\end{subsubsection}

\begin{subsubsection}{Cocientes}

\end{subsubsection}

\begin{subsubsection}{Espacios localmente compactos Hausdorff}
Sea $X$ un espacio topol\'{o}gico localmente compacto y Hausdorff
Todo punto de $X$ posee un entorno cuya clausura es compacta.

\begin{lemaEntornoDeUnPuntoLCH}\label{thm:entornodeunpuntolch}
	Sea $U\subset X$ un subconjunto abierto y sea $x\in U$. Existe
	un abierto $V$ tal que $x\in V$, $\clos{V}$ es compacta y
	$\clos{V}\subset U$.
\end{lemaEntornoDeUnPuntoLCH}

Diremos en general que un subconjunto $A\subset X$ es un entorno de un
punto $x$, si $x\in\interior{A}$. El lema anterior se puede expresar
diciendo que dado un punto $x$ y un abierto $U$ que lo contiene,
existe un entorno compacto de $x$ contenido en $U$.

\begin{proof}
	Como $X$ es localmente compacto y Hausdorff, podemos asumir que
	$\clos{U}$ es compacto, en otro caso, reemplazamos $U$ por
	$U\cap V$ donde $V$ es un entorno de $x$ con clausura copmpacta.
	Como $X$ es Hausdorff, existen abiertos en $\clos{U}$, $V$ y $W$,
	tales que $x\in V$, $\borde[U]\subset W$ y $V\cap W=\varnothing$.
	Como $U$ es abierto y $V\subset U$, $V$ es abierto en $X$. Su
	clausura $\clos{V}$ est\'{a} contenida en $U\setmin W$ y es compacta,
	por estar contenida en el compacto $\clos{U}$.
\end{proof}

\begin{lemaEntornoDeUnCompactoLCH}\label{thm:entornodeuncompactolch}
	Si $U\subset X$ es abierto y $K$ es un compacto contenido
	en $U$, entonces existe un abierto $V$ cuya clausura es
	comapcta, $K\subset V$ y $\clos{V}\subset U$.
\end{lemaEntornoDeUnCompactoLCH}

\begin{proof}
	Por \ref{thm:entornodeuncompactolch}, para cada $x\in K$ existe un
	entorno comacto $N_{x}$ de $x$ contenido en $U$. La familia
	$\{\interior{N_{x}}\}_{x\in K}$ es un cubrimiento por abiertos
	de $K$. Como $K$ es compacto, admite un subcubrimiento finito
	$\{\interior{N_{x_{1}}},\,\dots,\,\interior{N_{x_{k}}}\}$
	Si llamamos $V$ a la uni\'{o}n de los abiertos $\interior{N_{x_{i}}}$,
	entonces $K\subset V$ y $\clos{V}=\bigcup_{i=1}^{k}\,N_{x_{i}}$ es
	compacto y est\'{a} contenido en $U$.
\end{proof}

\begin{propoUrysohnLCH}\label{thm:urysohnlch}
	Sea $U\subset X$ un abierto y sea $K\subset U$ un compacto contenido
	en $U$. Existe una funci\'{o}n continua $f$ en $X$ tal que
	$0\leq f\leq 1$, $f=1$ en $K$ y $f=0$ fuera de un compacto contenido
	en $U$.
\end{propoUrysohnLCH}

\begin{proof}
	Existe un abierto $V$ con clausura compacta tal que $K\subset V$
	y $\clos{V}\subset U$, por \ref{thm:entornodeuncompactolch}.
	Como todo espacio compacto Hausdorff es $T_{4}$, existe, por
	el \emph{lema de Uryshohn}, una funci\'{o}n continua
	$f:\,\clos{V}\rightarrow[0,1]$ tal que $f=1$ en $K$ y $f=0$ en
	$\borde[V]$. Sea $\tilde{f}:\,X\rightarrow[0,1]$ la funci\'{o}n
	dada por
	\begin{align*}
		\tilde{f}(x) & \,=\,
			\begin{cases}
				f(x) & \text{ si }x\in\clos{V} \\
				0 & \text{ en otro caso.}
			\end{cases}
	\end{align*}
	%
	Esta funci\'{o}n satisface lo pedido. S\'{o}lo hay que verificar que
	sea continua. Pero si $E\subset [0,1]$ es cerrado, entonces
	\begin{align*}
		\tilde{f}^{-1}(E) & \,=\,
			\begin{cases}
				f^{-1}(E) & \text{ si}0\in E \\
				f^{-1}(E)\cup\setcomp{\clos{V}}
				\,=\,f^{-1}(E)\cup\setcomp{V} & \text{ si no.}
			\end{cases}
	\end{align*}
	%
	En cualquiera de los dos casos, $\tilde{f}^{-1}(E)$ es cerrado en
	$X$ y $\tilde{f}$ es continua.
\end{proof}

\begin{propoTietzeLCH}\label{thm:tietzelch}
	Sea $K\subset X$ compacto y sea $f$ una funci\'{o}n continua en $K$.
	Existe una extensi\'{o}n $F$ definida en $X$ que es continua y
	tiene soporte compacto.
\end{propoTietzeLCH}

\end{subsubsection}

\subsection{Propiedades b\'{a}sicas de las variedades}
Toda variedad topol\'{o}gica tiene una base numerable que consiste en bolas
coordenadas con clausura compacta (este es el \emph{lema de las bolas
coordenadas}).

\begin{lemaBolasCoordenadas}[de las bolas coordenadas]
	\label{thm:bolascoordenadas}
	Toda variedad topol\'{o}gica admite una base numerable de bolas
	precompactas.
\end{lemaBolasCoordenadas}

De este lema, se deduce la siguiente proposici\'{o}n.

\begin{propoVariedadesSonLocArco}[las variedades topol\'{o}gicas son %
	localmente arcoconexas]\label{thm:varssonlocarco}
	Sea $M$ una variedad topol\'{o}gica. Entonces
	\begin{itemize}
		\item[(a)] $M$ es localmente arcoconexa;
		\item[(b)] $M$ es conexa, si y s\'{o}lo si es arcoconexa;
		\item[(c)] las componentes conexas de $M$ coinciden con sus
			componentes arcoconexas y
		\item[(d)] $M$ tiene una cantidad numerable de componentes.
	\end{itemize}
	%
	Adem\'{a}s, cada componente es abierta y una variedad topol\'{o}gica
	conexa.
\end{propoVariedadesSonLocArco}

\begin{proof}
	Dado que las bolas en $\bb{R}^{n}$ son arcoconexas, las bolas
	coordenadas de $M$, siendo homeomorfas a bolas de $\bb{R}^{n}$,
	tambi\'{e}n lo son. Dado que, por el lema \ref{thm:bolascoordenadas},
	$M$ posee una base de bolas coordenadas, $M$ es localmente arcoconexa:
	dados un punto $p$ de $M$ y un abierto $V$ que lo contenga, existe
	un abierto de la base $B$ tal que $p\in B\subset V$ y $B$ es
	arcoconexo.

	Dado que $M$ es localmente arcoconexa, las componentes arcoconexas
	de $M$ deben ser abiertas. Como $M$ es la uni\'{o}n de dichas
	componentes, lass mismas deben ser, tambi\'{e}n, cerradas. En
	particular, $M$ es conexa, si y s\'{o}lo si es arcoconexa y,
	m\'{a}s aun, las componentes conexas coinciden con las componentes
	arcoconexas. En particular, las componentes conexas son abiertas en
	$M$ y constituyen un cubrimiento por abiertos de $M$. Como todo
	cubrimiento por abiertos admite un subcubrimiento numerable, la
	cantidad de componentes de $M$ debe ser, a lo sumo, numerable.

	Finalmente, como cada componente es abierta, tiene estructura de
	variedad topol\'{o}gica.
\end{proof}

\begin{proof}[Demostraci\'{o}n (de \ref{thm:bolascoordenadas})]
	Si $M$ es homeomorfo a un abierto $U$ de $\bb{R}^{n}$ v\'{\i}a un
	homeomorfismo $\varphi:\,M\rightarrow U$, entonces, como $U$ admite
	una base de esas caracter\'{\i}sticas, $M$ tambi\'{e}n.

	En general, $M$ admite un cubrimiento por abiertos homeomorfos a
	abiertos de $\bb{R}^{n}$. Como $M$ es $N_{2}$, admite un
	subcubrimiento numerable, existe un cubrimiento $\{U_{l}\}_{l\geq 1}$
	por abiertos de $M$ homeomorfos a abiertos de $\bb{R}^{n}$.
	Para cada $U_{l}$ existe una base numerable de bolas
	coordenadas precompactas (en $U_{l}$). La uni\'{o}n de estas bases
	constituye una base de $M$ por bolas coordenadas. Si $B$ es una de
	ellas y $B$ viene del abierto $U_{l}$ del cubrimiento, sea
	$\clos{B}$ la clausura de $B$ \emph{en $U_{l}$}. Entonces $\clos{B}$
	es compacta. Como subespacio de subespacio es subespacio, $\clos{B}$
	es un subespacio compacto de $M$. Como $M$ es $T_{2}$, esta clausura
	es cerrada en $M$. En particular, la clausura de $B$ en $U_{l}$ debe
	coincidir con la clausura en $M$ y, por lo tanto, $B\subset M$ es
	una bola coordenada precompacta.
\end{proof}

Otra consecuencia casi inmediata del lema \ref{thm:bolascoordenadas}
--aunque tambi\'{e}n es consecuencia de la propiedad de ser localmente
euclideas de las variedades-- es que las variedades topol\'{o}gicas son
localmente compactas.

\begin{propoVarTopLocComp}\label{thm:vartoploccomp}
	Sea $X$ un espacio topol\'{o}gico localmente euclideo de dimensi\'{o}n
	$n\geq 0$. Entonces $X$ es localmente compacto.
\end{propoVarTopLocComp}

Adem\'{a}s de ser localmente compactas, las variedades topol\'{o}gicas
son \emph{paracompactas}. Para dar una definici\'{o}n de esta propiedad,
es necesario definir otras nociones primero.

Sea $X$ un espacio topol\'{o}gico y sea $\cal{T}$ una colecci\'{o}n de
subconjuntos de $X$. La colecci\'{o}n $\cal{T}$ se dice \emph{localmente %
finita}, si, para todo punto $p\in X$, existe un entorno $V\subset M$ de $p$
tal que $V\cap T$ es vac\'{\i}a para todos salvo finitos elementos
$T\in\cal{T}$. Por otro lado, dado un cubrimiento $\cal{U}$ de $X$, un
\emph{refinamiento} de $\cal{U}$ es otro cubrimiento $\cal{V}$ de $X$ tal
que, para cada $V\in\cal{V}$, existe $U\in\cal{U}$ con $V\subset U$.

Ahora s\'{\i} podemos definir lo que quiere decir que un espacio
topol\'{o}gico sea paracompacto. Un espacio topol\'{o}gico $X$ se dice
paracompacto, si todo cubrimiento por abiertos de $X$ admite un refinamiento
localmente finito conformado por abiertos de $X$.

\begin{propoVarTopParacomp}\label{thm:vartopparacomp}
	Sea $M$ una variedad topol\'{o}gica. Sea $\cal{U}$ un cubrimiento
	de $M$ por abiertos y sea $\cal{B}$ una base para la topolog\'{\i}a
	de $M$. Entonces existe un refinamiento numerable y localmente
	finito de $\cal{U}$ compuesto por elementos de $\cal{B}$.
	En particular, toda variedad topol\'{o}gica es paracompacta.
\end{propoVarTopParacomp}

\begin{proof}
	En primer lugar, sea $\{K_{j}\}_{j\geq 1}$ una sucesi\'{o}n creciente
	de subconjuntos compactos de $M$ tal que
	\begin{align*}
		M & \,=\,\bigcup_{j\geq 1}\,K_{j}\quad\text{y} \\
		K_{j} & \,\subset\, K_{j+1}
	\end{align*}
	%
	para todo $j\geq 1$. Se define $K_{0}=\varnothing$ y, para $j\geq 1$,
	\begin{align*}
		F_{j} & \,=\,K_{j+1}\setmin\interior{K_{j}}\quad\text{y} \\
		W_{j} & \,=\,\interior{K_{j+2}}\setmin K_{j-1}
		\text{ .}
	\end{align*}
	%
	De esta manera, $\{W_{j}\}_{j\geq 1}$ es un cubrimiento de $M$ por
	abiertos y $W_{j}\supset F_{j}$ para todo $j$. Adem\'{a}s, cada
	$F_{j}$ es compacto y
	\begin{align*}
		W_{j}\cap W_{j'}\not =\varnothing & \,\Rightarrow\,
			|j-j'|<3\text{ .}
	\end{align*}
	%

	Sea $j\geq 1$. Para cada $x\in F_{j}$, existe un abierto
	$U_{x}\in\cal{U}$ tal que $x\in U_{x}$. Como $\cal{B}$ es una base
	para la topolog\'{\i}a de $M$, existe $B_{x}\in\cal{B}$ tal que
	$x\in B_{x}\subset U_{x}\cap W_{j}$. Entonces $F_{j}$ est\'{a}
	contenido en una uni\'{o}n de finitos abiertos b\'{a}sicos $B_{x}$.
	La colecci\'{o}n de estos abiertos, con $j$ variando en los enteros
	positivos, es una colecci\'{o}n numerable de abiertos b\'{a}sicos
	pertenecientes a $\cal{B}$. Cada elemento de esta colecci\'{o}n
	est\'{a} contenido en un elemento de $\cal{U}$ seg\'{u}n la manera
	en que fueron elegidos, por lo que constituye un refinamiento de
	$\cal{U}$. Para terminar de demostrar la proposci\'{o}n, resta
	verificar que esta colecci\'{o}n es localmente finita.

	Ahora bien, dado que cada elemento del refinamiento encontrado
	est\'{a} contenido, adem\'{a}s, en alg\'{u}n $W_{j}$ y que $W_{j}$
	s\'{o}lo interseca finitos abiertos $W_{j'}$, se deduce que cada
	$W_{j}$ contiene finitos elementos del refinamiento y que cada uno
	de estos elementos interseca a lo sumo finitos elementos distintos.
	En particular, dado un punto $p\in M$, tomando un $W_{j}$ o un
	elemento de la colecci\'{o}n que lo contenga, se deduce que existe
	un entorno de $p$ en $M$ que interseca s\'{o}lo finitos elementos
	del refinamiento, es decir, el mismo es localmente finito.
\end{proof}

\subsection{Algunas observaciones}
A continuaci\'{o}n realizamos algunas observaciones acerca de los resultados
demostrados anteriormente.

\begin{remarkVarTopParacomp}
	La proposici\'{o}n \ref{thm:vartopparacomp} dice m\'{a}s que que
	toda variedad es paracompacta. Si solamente se quiere demostrar
	la paracompacidad de una variedad topol\'{o}gica, se puede proceder
	usando algunas de las siguientes implicaciones:
	\begin{align*}
		\text{$T_{2}$ y loc. euc.} & \,\Rightarrow\,
			\text{$T_{2}$ y loc. comp.}\,\Rightarrow\,
			\text{$T_{3\frac{1}{2}}$}\\
		\text{loc. euc. y $N_{2}$} & \,\Rightarrow\,
			\text{$\sigma$-comp.} \\
		\text{loc. comp. y $N_{2}$} & \,\Rightarrow\,
			\text{$\sigma$-comp.} \\
		\text{$T_{2}$, loc. comp. y $\sigma$-comp.} & \,\Rightarrow\,
			\text{$T_{4}$} \\
		\text{$T_{2}$, loc. comp. y $\sigma$-comp.} & \,\Rightarrow\,
			\text{paracomp.} \\
		\text{$T_{2}$, loc. comp. y $N_{2}$} & \,\Rightarrow\,
			\text{metrizable} \\
		\text{$T_{4}$ y $N_{2}$} & \,\Rightarrow\,
			\text{metrizable} \\
		\text{paracomp. y loc. metrizable} & \,\Rightarrow\,
			\text{metrizable} \\
		\text{metrizable} & \,\Rightarrow\,\text{paracomp.}
	\end{align*}
	%
\end{remarkVarTopParacomp}

\begin{remarkVarTopParacompI}\label{rem:vartopparacompI}
	En la demostraci\'{o}n de la proposici\'{o}n \ref{thm:vartopparacomp},
	se usa de manera esencial que toda variedad topol\'{o}gica admite una
	sucesi\'{o}n exhaustiva de compactos $\{K_{j}\}_{j\geq 1}$, es
	decir, tales que $M=\bigcup_{j}\,K_{j}$ y que
	$K_{j}\subset\interior{K_{j+1}}$. un espacio topl\'{o}gico con
	esta propiedad se denomina $\sigma$-compacto. El hecho de que las
	variedades topol\'{o}gicas poseen esta propiedad es consecuencia de
	la siguiente proposici\'{o}n.
\end{remarkVarTopParacompI}

\begin{propoVarTopSigmacomp}\label{thm:vartopsigmacomp}
	Sea $X$ un espacio topol\'{o}gico localmente compacto Hausdorff y
	que admite una base numerable para su topolog\'{\i}a, es decir, es
	$N_{2}$. Entonces existe una sucesi\'{o}n exhaustiva de compactos
	para $X$, es decir, $X$ es $\sigma$-compacto. En particular, toda
	variedad topol\'{o}gica es $\sigma$-compacta.
\end{propoVarTopSigmacomp}

\begin{proof}
	Dado que $X$ es localmente compacto y Hausdorff, existe una base de
	abiertos con clausura compacta. Dado que, adem\'{a}s, $X$ es $N_{2}$,
	esta base, por ser un cubrimiento por abiertos de $X$, admite un
	subcubrimiento numerable. Sea $K_{1}=\clos{U_{1}}$, donde
	$\{U_{n}\}_{n}$ es un cubrimiento numerable por abiertos
	precompactos. Existe $n_{1}$ tal que
	\begin{align*}
		K_{1} & \,\subset\,U_{1}\cup\,\cdots\,\cup U_{n_{1}}
		\text{ .}
	\end{align*}
	%
	Sea $K_{2}$ la uni\'{o}n de las clausuras de los abiertos que
	aparecen en la uni\'{o}n:
	\begin{align*}
		K_{2} & \,=\,\clos{U_{1}}\cup\,\cdots\,\cup\clos{U_{n_{1}}}
		\text{ .}
	\end{align*}
	%
	Si $n_{1}<2$, se puede tomar $n_{1}=2$ y sigue siendo cierto que
	$K_{1}\subset\interior{K_{2}}$. Adem\'{a}s, eligiendo $n_{1}$ de
	esta manera, $K_{2}\supset U_{2}$.

	Habiendo definido $K_{1},\,\dots,\,K_{j}$ compactos tales que
	$K_{i}\subset\interior{K_{i+1}}$ y tales que $U_{i}\subset K_{i}$, sea
	$K_{j+1}=\clos{U_{1}}\cup\,\cdots\,\cup\clos{U_{n_{j}}}$, donde
	$n_{j}$ es tal que $K_{j}\subset U_{1}\cup\,\cdots\,\cup U_{n_{j}}$.
	Se puede suponer que $n_{j}\geq j+1$, de manera que
	$K_{j}\subset\interior{K_{j+1}}$ y que $K_{j+1}\supset U_{j+1}$,
	tambi\'{e}n. La sucesi\'{o}n que se obtiene as\'{\i} es una
	sucesi\'{o}n exhaustiva de $X$ por compactos.
\end{proof}

\begin{remarkVarTopParacompII}\label{rem:vartopparacompII}
	Si $X$ es un espacio topol\'{o}gico localmente compacto y tal que
	todo cubrimiento por abiertos admite un subcubrimiento numerable,
	para cada punto $p\in X$ existen un abierto $U_{p}$ y un compacto
	$C_{p}$ tales que $p\in U_{p}\subset C_{p}$. La colecci\'{o}n
	$\{U_{p}\,:\,p\in\ X\}$ es un cubrimiento de $X$ y admite, pues, un
	subcubrimiento numerable, $\{U_{n}\}_{n\geq 1}$. Sea
	$\{C_{n}\}_{n\geq 1}$ la familia de compactos correspondiente.
	Sea $K_{1}=C_{1}$. Como $K_{1}$ es compacto, existe $n_{1}$
	(que se puede suponer mayor o igual a $2$) tal que
	\begin{align*}
		K_{1} & \,\subset\, U_{1}\cup\,\cdots\,\cup U_{n_{1}}
		\text{ .}
	\end{align*}
	%
	Sea $K_{2}=C_{1}\cup\,\cdots\,\cup C_{n_{1}}$. Entonces
	\begin{align*}
		K_{1} & \,\subset\,U_{1}\cup\,\cdots\,\cup U_{n_{1}}
			\,\subset\, C_{1}\cup\,\cdots\,\cup C_{n_{1}}
		\text{ .}
	\end{align*}
	%
	As\'{\i}, $K_{1}\subset\interior{K_{2}}$ y $K_{2}\supset U_{2}$.
	Inductivamente, queda definida una sucesi\'{o}n $\{K_{j}\}_{j\geq 1}$
	de compactos tales que $K_{j}\subset\interior{K_{j+1}}$ y
	$K_{j}\supset U_{j}$. Como $\{U_{j}\}_{j\geq 1}$ es un cubrimiento
	de $X$, se deduce que
	\begin{align*}
		X & \,=\,\bigcup_{j\geq 1}\,K_{j}
		\text{ .}
	\end{align*}
	%
	El espacio topol\'{o}gico $X$ admite una sucesi\'{o}n exhaustiva de
	compactos. En definitiva, hemos demostrado una versi\'{o}n un poco
	m\'{a}s general de \ref{thm:vartopsigmacomp}.
\end{remarkVarTopParacompII}

\begin{propoVarTopSigmacomp}\label{thm:vartopsigmacompbis}
 Si $X$ es un espacio topol\'{o}gico localmente compacto y es tal
 que todo cubrimiento por abiertos admite un subcubriento numerable,
 entonces $X$ es \emph{$\sigma$-compacto}. En este contexto,
 esto quiere decir que $X$ admite una sucesi\'{o}n exhaustiva de
 compactos.
\end{propoVarTopSigmacomp}

\begin{remarkVarTopParacompIII}\label{rem:vartopparacompIII}
	La propiedad de ser localmente compacta de una variedad topol\'{o}gica
	se puede demostrar sin el lema de las bolas coordenadas. Mejor dicho,
	las conclusiones del lema se pueden deducir asumiendo \'{u}nicamente
	que $M$ es un espacio topol\'{o}gico localmente euclideo. La
	cardinalidad de la base es consecuencia de que $M$ admite un
	subcubrimiento numerable porque es, adem\'{a}s, $N_{2}$.

	Si decimos que $X$ es un espacio topol\'{o}gico localmente euclideo
	de dimensi\'{o}n $n\geq 0$, estamos diciendo que $X$ admite un
	cubrimiento por abiertos homeomorfos a abiertos de $\bb{R}^{n}$.
	Es decir, existe una familia de pares $(U,\varphi)$ donde
	$U$ es abierto de $X$ y $\varphi:\,U\rightarrow\bb{R}^{n}$ determina
	un homeomorfismo entre $U$ y un abierto de $\bb{R}^{n}$ y los
	abiertos $U$ cubren a $X$. Por cada uno de estos pares,
	$\varphi(U)$ es un abierto de $\bb{R}^{n}$ homeomorfo a $U$. Dado
	que $\varphi(U)$ admite una base numerable de bolas precompactas
	$\{B_{k}\}_{k\geq 1}$, la familia $\{\varphi^{-1}(B_{k})\}_{k\geq 1}$
	es una base numerable para la topolog\'{\i}a de $U$, porque $\varphi$
	es un homeomorismo. Pero $U$ tiene la topolog\'{\i}a subespacio
	de $X$ y cada \emph{bola} $\varphi^{-1}(B_{k})$ tiene clausura
	compacta \emph{en $U$}. Es decir, $\clos{\varphi^{-1}(B_{k})}^{U}$
	es compacta como subespacio de $U$. Pero subespcio de subespacio
	es subespacio, por lo tanto, $\clos{\varphi^{-1}(B_{k})}^{U}$ debe ser
	compacto como subespacio de $X$, aunque talvez no coincida con la
	clausura de $\varphi^{-1}(B_{k})$ en $X$.

	En definitiva, por cada par $(U,\varphi)$ con $U\subset X$ abierto
	y $\varphi:\,U\rightarrow\bb{R}^{n}$ un homeomorfismo con un abierto
	de $\bb{R}^{n}$, existe una base numerable de bolas de $X$ y por
	cada una de ellas un compacto de $U$ (y, por lo tanto, de $X$) que
	la contiene. Es decir, cada bola de la base $U$ es precompacta en
	el sentido de que est\'{a} contenida en un subespacio compacto.
	Agrupando las bases asociadas a cada par $(U,\varphi)$ se obtiene
	una base para $X$.

	Si ahora asumimos que todo cubrimiento de $X$ admite un subcubrimiento
	numerable, entonces $X$ admite un cubrimiento numerable por bolas
	tales que cada una de ellas est\'{e} contenida en un subespacio
	compacto. Si $X$ es, adem\'{a}s, Hausdorff, se puede asumir que
	dichos compactos son las clausuras (en $X$ o en el abierto $U$
	correspondiente, pues son iguales) de las respectivas bolas
	$\varphi^{-1}(B)$ de $X$. Obtenemos as\'{\i} una demostraci\'{o}n
	de la proposici\'{o}n \ref{thm:vartoploccomp} y una generalizaci\'{o}n
	del lema \ref{thm:bolascoordenadas} de las bolas coordenadas.
\end{remarkVarTopParacompIII}

\begin{remarkVarTopParacompIV}\label{rem:vartopparacompIV}
	Si $X$ es un espacio topol\'{o}gico $\sigma$-compacto y que,
	adem\'{a}s es Hausdorff, entonces, siguiendo el argumento en
	la demostraci\'{o}n de la proposici\'{o}n \ref{thm:vartopparacomp},
	se deduce que, dado un cubrimiento por abiertos y una base,
	existe un refinamiento numerable localmente finito del cubrimiento
	por elementos de la base. Es decir, todo espacio $\sigma$-compacto
	y Hausdorff es paracompacto.
\end{remarkVarTopParacompIV}

\subsection{Cartas coordenadas}
Sea $M$ una variedad topol\'{o}gica de dimensi\'{o}n $n$. Una
\emph{carta coordenada} (mapa coordenado, mapa, coordenada, carta, sistema
de coordenadas, etc.) para/en/de $M$ es un par $(U,\varphi)$ donde
$U\subset M$ es abierto y $\varphi:\,U\rightarrow\bb{R}^{n}$ es un
homemorfismo sobre su imagen. Tambi\'{e}n se puede definir como una terna
$(U,\tilde{U},\varphi)$ donde $U$ es abierto en $M$, $\tilde{U}$ es abierto
en $\bb{R}^{n}$ y $\varphi:\,U\rightarrow\tilde{U}$ es un homeomorfismo.
Dado un punto $p\in M$, se dice que una carta $(U,\varphi)$ \emph{est\'{a} %
centrada} en $p$, si $p\in U$ y $\varphi(p)=0$. Dada una carta $(U,\varphi)$,
$U$ se denomina el \emph{dominio coordenado} de la carta y $\varphi$ el
\emph{mapa coordenado}. Las funciones
\begin{align*}
	x^{k} & \,=\,\pi^{k}\circ\varphi\,:\,U\,\rightarrow\,\bb{R}
	\text{ ,}
\end{align*}
%
donde $\pi^{k}:\,\bb{R}^{n}\rightarrow\bb{R}$ denota la proyecci\'{o}n en la
coordenada $k$, son las \emph{funciones coordenadas} o \emph{coordenadas %
locales en $U$}.

Con esta noci\'{o}n, podemos reformular la definici\'{o}n d variedad
topol\'{o}gica: una variedad topol\'{o}gica es un espacio topol\'{o}gico
Hausdorff $M$ que admite una base numerable para su topolog\'{\i}a y
que posee, adem\'{a}s, un cubrimiento por abiertos $\{U_{\alpha}\}_{\alpha}$,
donde los conjuntos $U_{\alpha}$ son domnios de una carta coordenada
$(U_{\alpha},\varphi_{\alpha})$ para $M$.

\subsection{Par\'{e}ntesis: el grupo fundamental de una variedad}
Sea $M$ una variedad topol\'{o}gica y sea $\cal{B}$ una base numerable de
bolas coordenadas precompactas. Sea $p\in M$ un punto arbitrario y sea
$f:\,[0,1]\rightarrow M$ un camino cerrado basado en $p$, es decir, una
funci\'{o}n continua tal que $f(0)=f(1)=p$. Dado que $\cal{B}$ constituye
un cubrimiento por abiertos de $M$ y, en particular, de $f([0,1])$,
existen finitos puntos $\{a_{0},\,\dots,\,a_{k}\}$
tales que $a_{0}=0<a_{1}<\cdots<a_{k-1}<a_{k}=1$ y existen tambi\'{e}n bolas
$B_{1},\,\dots,\,B_{k}$ pertenecientes a $\cal{B}$ tales que
\begin{align*}
	f([a_{t-1},a_{t}]) & \,\subset\,B_{t}
	\text{ .}
\end{align*}
%
El camino $f$ se factoriza como un producto de caminos
\begin{align*}
	f & \,\sim\,f_{1}\sqcdot\,\cdots\,\sqcdot f_{k}
	\text{ ,}
\end{align*}
%
donde $f_{t}=f|_{[a_{t-1},a_{t}]}$ (reparametriado adecuadamente para que su
dominio sea $[0,1]$). Sea $x\in B_{t-1}\cap B_{t}$ un punto de la misma
componente conexa de $B_{t-1}\cap B_{t}$ que $f(a_{t-1})=:x_{t-1}$.
Existe un camino contenido en $B_{t}$ de $x_{t-1}$ a $x$. Sea $g_{t-1}$ tal
camino. As\'{\i},
\begin{align*}
	f & \,\sim\, f_{1}\sqcdot\,\cdots\,\sqcdot f_{k} \\
	& \,\sim\,(f_{1}\sqcdot g_{1})\sqcdot
		(\reverse{g_{1}}\sqcdot f_{2}\sqcdot g_{2})
		\sqcdot\,\cdots\,\sqcdot (\reverse{g_{k-1}}\sqdot f_{k})
	\text{ ,}
\end{align*}
%
donde $\reverse{g}$ es el camino inverso de $g$.

Sea $B\in\cal{B}$ un elemento arbitrario de la base. Para cada $B'\in\cal{B}$,
posiblemente igual a $B$, se elige un punto en cada componente conexa
de $B\cap B'$. Como las componentes conecas de dicha intersecci\'{o}n son
numerables en cantidad y $\cal{B}$ contiene numerables bolas, son numerables
los puntos elegidos. Llamemos a estos puntos \emph{puntos especiales}. Si
$B\in\cal{B}$ y si $x',x''\in B$ son dos puntos especiales ($x'\in B'\cap B$
y $x''\in B''\cap B$, por ponerles un nombre), sea $h_{x',x''}^{B}$ un camino
contenido en $B$ con $h_{x',x''}^{B}(0)=x'$ y $h_{x',x''}^{B}(1)=x''$.
Por cada terna $(B,x',x'')$ se elige un camino, que se denominar\'{a}
\emph{camino especial}, contenido en $B$ de $x'$ a $x''$. La cantidad de
caminos as\'{\i} elegidos es, pues, numerable.

Por otro lado, se puede asumir que el punto $p$ es uno de los puntos
especiales: en primer lugar, la elecci\'{o}n de los puntos especiales
fue realizada sin menci\'{o}n de $p$; en segundo lugar, dado un punto
$p\in M$ arbitrario, existe un punto especial $x$ en la misma componente
conexa de $M$ que $p$, entonces $\pi(M,p)\simeq\pi(M,x)$. Por el
argumento del p\'{a}rrafo anterior, para cada $t\in[\![1,k-1]\!]$,
existe un camino $g_{t}$ contenido en $B_{t}\cap B_{t+1}$ con origen en
$x_{t}=f(a_{t})$ que termina en un punto $x\in B_{t}\cap B_{t+1}$ en la
misma componente conexa de $B_{t}\cap B_{t+1}$ que $x_{t}$. Ahora bien,
si $g_{0}$ es el camino constante fijo en $x_{0}=f(a_{0})=f(0)=p$ y $g_{k}$
es el camino constante fijo en $x_{k}=f(a_{k})=f(1)=p$, vale que
\begin{align*}
	f & \,\sim\,f_{1}\sqcdot\,\cdots\,\sqcdot f_{k} \\
	& \,\sim\, (\reverse{g_{0}}\sqcdot f_{1}\sqcdot g_{1})\sqcdot
		(\reverse{g_{1}}\sqcdot f_{2}\sqcdot g_{2})\sqcdot
		\,\cdots\,\sqcdot (\reverse{g_{k-1}}\sqcdot f_{k}\sqcdot g_{k})
	\text{ .}
\end{align*}
%
Pero $\reverse{g_{t-1}}\sqcdot f_{t}\sqcdot g_{t}$ es un camino contenido en
$B_{t}$, que es simplemente conexa, que comienza en un punto especial
$x_{t}'$ y termina en otro punto especial $x_{t}''$. En particular,
\begin{align*}
	\reverse{g_{t-1}}\sqcdot f_{t}\sqcdot g_{t} &
		\,\sim\, h_{x_{t}',x_{t}''}^{B_{t}}\quad\text{y} \\
	f & \,\sim\, h_{x_{1}',x_{1}''}^{B_{1}}\sqcdot\,\cdots\,\sqcdot
		h_{x_{k}',x_{k}''}^{B_{k}}
	\text{ .}
\end{align*}
%
En definitiva, todo loop basado en $p$ es homot\'{o}pico a un producto
finito de caminos especiales, lo que implica que $\pi(M,p)$ es, a lo sumo,
numerable.


%
\section{Estructuras diferenciales}
\theoremstyle{plain}
\newtheorem{propoAtlasMax}{Proposici\'{o}n}[section]
\newtheorem{lemaDeLasBolasRegulares}[propoAtlasMax]{Lema}
\newtheorem{lemaDeLasCartas}[propoAtlasMax]{Lema}

\theoremstyle{remark}
\newtheorem{remarkAtlasMax}{Observaci\'{o}n}[section]

%--------------------

\subsection{Atlas suaves y estructuras diferenciales}
Sea $M$ una variedad topol\'{o}gica y sean $(U,\varphi)$ y $(V,\psi)$ dos
cartas coordenadas. Si $U\cap V$ es no vac\'{\i}a, las funciones
$\varphi$ y $\psi$ restringidas a la intersecci\'{o}n $U\cap V$ son
homeomorfismos
\begin{align*}
	\varphi| & \,:\,U\cap V\,\rightarrow\,\varphi(U\cap V)\quad\text{y} \\
	\psi| & \,:\, U\cap V\,\rightarrow\,\psi(U\cap V)
\end{align*}
%
entre $U\cap V$ y $\varphi(U\cap V)$ y $\psi(U\cap V)$, respectivamente. En
particular, la composici\'{o}n
\begin{align*}
	\varphi\circ\psi^{-1} & \,:\,\psi(U\cap V)\,\rightarrow\,
		\varphi(U\cap V)
\end{align*}
%
es un homeomorfismo denominado \emph{mapa de transici\'{o}n} o \emph{cambio %
de coordenadas}. Las cartas $(U,\varphi)$ y $(V,\psi)$ se dicen
\emph{suavemente compatibles} (o, simplemente, compatibles), si
$\varphi\circ\psi^{-1}$ es un difeomorfismo, es decir, si
$\varphi\circ\psi^{-1}$ y $\psi\circ\varphi^{-1}$ son funciones suaves
entre los correspondientes abiertos de $\bb{R}^{n}$. Un \emph{atlas para $M$}
es una colecci\'{o}n de cartas que cubren $M$. Un atlas se dice
\emph{atlas suave} o \emph{atlas (suavemente) compatible}, si todo par
de cartas del atlas es un par compatible. Un atlas (suave) se dice
\emph{maximal} o \emph{completo} (respecto de la propiedad de ser suave),
si no est\'{a} propiamente contenido en otro atlas (suave); equivalentemente,
si cualquier carta compatible con las cartas del atlas ya formaba parte del
atlas. Una \emph{estructura suave} en $M$ es un atlas suave maximal. Las
variedad topol\'{o}gica $M$, junto con una estructura suave se denomina
\emph{variedad suave} o \emph{variedad diferencial}.

\begin{propoAtlasMax}\label{thm:atlasmax}
	Todo atlas suave est\'{a} contenido en un \'{u}nico atlas suave
	maximal. Dos atlas suaves est\'{a}n contenidos en el mismo atlas
	maximal, si y s\'{o}lo si su uni\'{o}n es un atlas suave.

	Equivalentemente, en t\'{e}rminos de estructura suave, todo
	atlassuave en $M$ determina una \'{u}nica estructura suave en $M$.
	Dos atlas suaves determinan la misma estructura suave, si y s\'{o}lo
	si su uni\'{o}n es un atlas suave.
\end{propoAtlasMax}

\begin{remarkAtlasMax}\label{rem:atlasmax}
	Equivalentemente, se define una relaci\'{o}n de equivalencia entre
	atlas suaves de la siguiente manera: dos atlas suaves se dicen
	equivalentes, si se uni\'{o}nes un atlas suave. Esto, efectivamente,
	determina una relaci\'{o}n de equivalencia entre atlas suaves y
	las clases de equivalencia se corresponden, exactamente, con lo atlas
	suavves maximales. Una estructura suave en $M$ se puede definir,
	tambi\'{e}n, como una clase de equivalencia de atlas suaves.
\end{remarkAtlasMax}

Para determinar si un atlas es suave, hay que verifica que todo cambio de
coordenadas $\varphi\circ\psi^{-1}$ sea un difeomorfismo. Pero alcanza
con verificar que cada uno de ellos es una transformaci\'{o}n suave entre
abiertos euclideos, ya que la inversa de un cambio $\varphi\circ\psi^{-1}$
es el cambio de coordenadas $\psi\circ\varphi^{-1}$. Por otra parte, para
determinar si dos cartas $(U,\varphi)$ y $(V,\psi)$ son compatibles, hay
que verificar que ambos mapas de transici\'{o}n sean suaves. Pero es
suficiente verificar que uno de ellos, digamos, sea suave, inyectivo y que
su jacobiano sea no nulo en todo punto de su dominio $\psi(U\cap V)$.

\subsection{Representaciones locales en coordenadas}
Sea $M$ es una variedad diferencial. Una carta contenida en el atlas maximal,
es decir, una carta compatible con la estructura diferencial se denomina
\emph{carta suave} para la variedad $M$. El dominio se denomina \emph{entorno %
coordenado}. Si $\varphi(U)$ es una bola o un cubo de $\bb{R}^{n}$, se
dice que $U$ es una \emph{bola, o un cubo coordenado}. Decimos que una bola
coordenada, o un cubo coordenado, est\'{a} centrada en un punto $p\in M$,
si $\varphi(p)=0$, donde $\varphi$ es la funci\'{o}n correspondiente de
la carta.

Sea $B\subset M$ una bola coordenada. Se dice que $B$ es una bola coordenada
\emph{regular}, si existe otra bola coordenada $B'$ y coordenadas suaves
$\varphi$ tales que $B'\supset\clos{B}$ y
\begin{align*}
	\varphi(B) & \,=\,\bola{r}{0}\text{ ,} \\
	\varphi(\clos{B}) & \,=\,\clos{\bola{r}{0}}\quad\text{y} \\
	\varphi(B') & \,=\,\bola{r'}{0}
	\text{ .}
\end{align*}
%
Esta noci\'{o}n tiene sentido, incluso si $\varphi$ es meramente un
homeomorfismo, sin tener en cuenta la estructura diferencial en $M$,
pero nos concentraremos en coordenadas compatibles con dicha estructra.

\begin{lemaDeLasBolasRegulares}\label{thm:delasbolasregulares}
	Toda variedad diferencial tiene una base numerable de bolas
	coordenadas regulares.
\end{lemaDeLasBolasRegulares}

Todo lo anterior sigue cierto reemplazando bolas por cubos.

\subsection{El lema de las cartas}
El siguiente lema es \'{u}til en la construcci\'{o}n de nuevas variedades,
al definir nuevos objetos y determinar si son, o no, variedades diferenciales.

\begin{lemaDeLasCartas}[de las cartas]\label{thm:delascartas}
	Sea $M$ un conjunto y supongamos dados \textit{(a)} una colecci\'{o}n
	$\{U_{\alpha}\}_{\alpha}$ de subconjuntos de $M$ y \textit{(b)},
	para cada \'{\i}ndice $\alpha$, una funci\'{o}n
	$\varphi_{\alpha}:\,U_{\alpha}\rightarrow\bb{R}^{n}$ que cumplen
	con las siguientes condiciones:
	\begin{itemize}
		\item[(i)] por cada $\alpha$, $\varphi_{\alpha}$ determina
			una biyecci\'{o}n entre $U_{\alpha}$ y un abierto
			$\varphi_{\alpha}(U_{\alpha})$ de $\bb{R}^{n}$;
		\item[(ii)] dados $\alpha,\beta$, tanto
			$\varphi_{\alpha}(U_{\alpha}\cap U_{\beta})$, como
			$\varphi_{\beta}(U_{\alpha}\cap U_{\beta})$ son
			abiertos de $\bb{R}^{n}$;
		\item[(iii)] si, para $\alpha,\beta$,
			$U_{\alpha}\cap U_{\beta}$ es no vac\'{\i}a, entonces
			\begin{align*}
				\varphi_{\alpha}\circ\varphi_{\beta}^{-1} &
				\,:\,
				\varphi_{\beta}(U_{\beta}\cap U_{\alpha})
				\,\rightarrow\,
				\varphi_{\alpha}(U_{\beta}\cap U_{\alpha})
			\end{align*}
			%
			es suave;
		\item[(iv)] existe una subcolecci\'{o}n numerable
			$\{U_{n}\}_{n\geq 1}$ tal que
			\begin{align*}
				M & \,=\,\bigcup_{n\geq 1}\,U_{n}
				\quad\text{y}
			\end{align*}
			%
		\item[(v)] si $p$ y $q$ son puntos distintos de $M$, o bien
			existe $\alpha$ tal que $p,q\in U_{\alpha}$, o bien
			existen $\alpha,\beta$ tales que $p\in U_{\alpha}$,
			$q\in U_{\beta}$ y
			$U_{\alpha}\cap U_{\beta}=\varnothing$.
	\end{itemize}
	%
	Entonces $M$ admite una \'{u}nica estructura suave tal que
	$(U_{\alpha},\varphi_{\alpha})$ sea una carta compatible para todo
	$\alpha$.
\end{lemaDeLasCartas}

\begin{proof}
	Si se quiere que cada par $(U_{\alpha},\varphi_{\alpha})$ sea una
	compatible de $M$ (suponiendo que $M$ es una variedad diferencial),
	debe ser, en particular, una carta coordenada para la estructura de
	variedad topol\'{o}gica subyacente. Esto implica que cada
	$U_{\alpha}$ debe ser abierto y que cada aplicaci\'{o}n
	$\varphi_{\alpha}$ debe ser un homeomorfismo entre $U_{\alpha}$ y un
	abierto de $\bb{R}^{n}$. Como $M=\bigcup_{\alpha}\,U_{\alpha}$,
	dado $p\in M$, existe $\alpha$ tal que $p\in U_{\alpha}$. As\'{\i},
	tomando una base de entornos para $\varphi_{\alpha}(p)$ en
	$\bb{R}^{n}$ y tomando preimagen por $\varphi_{\alpha}$, se
	deber\'{\i}a obtener una base de entornos para $p$ en $U_{\alpha}$
	y, porque $U_{\alpha}\subset M$ deber\'{\i}a ser abierto, estas bases
	deber\'{\i}an dar una base para la topolog\'{\i}a de $M$
	(asumiendo que $M$ es una variedad topol\'{o}gica).

	Se define la siguiente topolog\'{\i}a en $M$. Sea
	\begin{align*}
		\cal{B} & \,=\,\bigcup_{\alpha}\,
			\left\lbrace\varphi_{\alpha}^{-1}(V)\,:\,
				V\subset\bb{R}^{n}\text{ abierto}\right\rbrace
		\text{ .}
	\end{align*}
	%
	Esta colecci\'{o}n tiene las propiedades de base para una
	topolog\'{\i}a en $M$: en primer lugar,
	\begin{align*}
		M & \,=\,\bigcup_{\alpha}\,U_{\alpha} \,=\,
			\bigcup_{\alpha}\,
			\varphi^{-1}\big(\varphi_{\alpha}(U_{\alpha})\big)
	\end{align*}
	%
	y $\varphi_{\alpha}(U_{\alpha})\in\cal{B}$ para todo $\alpha$; en
	segundo lugar, dado $p\in M$ y dados $V,W\subset\bb{R}^{n}$
	abiertos tales que
	$p\in\varphi_{\alpha}^{-1}(V)\cap\varphi_{\beta}^{-1}(W)$, vale
	la igualdad
	\begin{align*}
		\varphi_{\alpha}^{-1}\big(
			V\cap\varphi_{\alpha}\circ\varphi_{\beta}^{-1}(W)
			\big) & \,=\,
			\varphi_{\alpha}^{-1}(V)\cap\varphi_{\beta}^{-1}(W)
			\,\ni\,p
		\text{ .}
	\end{align*}
	%
	Pero tambi\'{e}n
	\begin{align*}
		\varphi_{\alpha}\circ\varphi_{\beta}^{-1}(W) & \,=\,
		(\varphi_{\beta}\circ\varphi_{\alpha})^{-1}(W)
	\end{align*}
	%
	y como $\varphi_{\beta}\circ\varphi_{\alpha}^{-1}:\,%
		\varphi_{\alpha}(U_{\alpha}\cap U_{\beta})\rightarrow%
		\varphi_{\beta}(U_{\alpha}\cap U_{\beta})$ es suave, es,
	en particular, continua y
	$\varphi_{\alpha}\circ\varphi_{\beta}^{-1}(W)$ es abierto en
	$\varphi_{\alpha}(U_{\alpha}\cap U_{\beta})$. Como este \'{u}ltimo
	conjunto es abierto en $\bb{R}^{n}$, el subconjunto
	$\varphi_{\alpha}\circ\varphi_{\beta}^{-1}(W)$ es abierto en
	$\bb{R}^{n}$, tambi\'{e}n. En definitiva, la intersecci\'{o}n
	$\varphi_{\alpha}^{-1}(V)\cap\varphi_{\beta}^{-1}(W)\in\cal{B}$
	y $\cal{B}$ es base para una topolog\'{\i}a, la topolog\'{\i}a m\'{a}s
	peque\~{n}a que la contiene. Resta ver que, con esta topolog\'{\i}a
	$M$ es efectivamente una variedad topol\'{o}gica.
	
	Antes de demostrarlo, notemos que esta topolog\'{\i}a en $M$ es la
	m\'{a}s peque\~{n}a que hace de $M$ una variedad topol\'{o}gica y
	tal que los pares $(U_{\alpha},\varphi_{\alpha})$ sean cartas
	coordenadas. Si, por otro lado, $M$ tiene una estructura de
	variedad topol\'{o}gica tal que estos pares sean cartas coordenadas,
	entonces, dado un abierto $U\subset M$, podemos descomponerlo
	intersecando con los dominios coordenados $U_{\alpha}$:
	$U=\bigcup_{\alpha}\,U_{\alpha}\cap U$. Como las funciones
	coordenadas $\varphi_{\alpha}$ son homeomorfismos, cada t\'{e}rmino
	$U_{\alpha}\cap U$ pertenece a la colecci\'{o}n $\cal{B}$. En
	definitiva, $\cal{B}$ es base para la topolog\'{\i}a de $M$.
	Por lo tanto, la topolog\'{\i}a determinada por $\cal{B}$ es la
	\'{u}nica topolog\'{\i}a que hace que $M$ tenga estructura de
	variedad topol\'{o}gica y que los pares
	$(U_{\alpha},\varphi_{\alpha})$ sean cartas para $M$. Dado que la
	colecci\'{o}n de cartas $\{(U_{\alpha},\varphi_{\alpha})\}_{\alpha}$
	constituye, por hip\'{o}tesis, un atlas compatible para $M$,
	la estructura diferenciable tambi\'{e}n es \'{u}nica: precisamente,
	es la (\'{u}nica) estructura detereminada por este atlas.

	Por \textit{(i)}, $M$ es localmente euclidea de dmensi\'{o}n $n$;
	por \textit{(v)} es $T_{2}$ y por \textit{(iv)} es $N_{2}$.
	Entonces $M$ tiene estructura de variedad topol\'{o}gica. Finalmente,
	por \textit{(ii)} y \textit{(iii)}, $\cal{A}=%
	\{(U_{\alpha},\varphi_{\alpha})\}_{\alpha}$ es un atlas $C^{\infty}$
	para $M$.
\end{proof}



%
\section{Variedades con borde y variedades con esquinas}
\theoremstyle{plain}
\newtheorem{teoInvarianzaDeLasEsquinas}{Teorema}[section]

\theoremstyle{remark}
\newtheorem{obsElBordeConEsquinasNoEsVariedad}{Observaci\'{o}n}[section]

%--------------------

\subsection{Variedades con bordes}
Una \emph{variedad con borde}, espec\'{\i}ficamente, una \emph{variedad %
topol\'{o}gica con borde} se define como un espacio topol\'{o}gico
Hausdorff y $N_{2}$ tal que todo punto del mismo tiene un entorno homeomorfo
a un abierto del semiespacio superior $\hemi[d]$ para alg\'{u}n $d\geq 0$.
El valor de $d$ es la dimensi\'{o}n de la variedad (y, por un corolario del
teorema de la dimensi\'{o}n \ref{thm:deladim}, est\'{a} bien definida).
Es decir, en lugar de estar modelado localmente como $\bb{R}^{d}$, una
variedad con borde es localmente como
\begin{align*}
	\hemi[d] & \,=\,\left\lbrace (\lista*{x}{d})\in\bb{R}^{d}\,:\,
				x^{d}\geq 0\right\rbrace
	\text{ .}
\end{align*}
%
Si $M$ es una variedad con borde y $p\in M$, existe un abierto $U$ de $M$
y un homeomorfismo $\varphi:\,U\rightarrow\varphi(U)$ con un abierto de
$\hemi[d]$ tal que $p\in U$. Como en el caso de una variedad topol\'{o}gica,
un par $(U,\varphi)$ se denominar\'{a} \emph{carta} para $M$ en $p$.

El \emph{borde} de $\hemi[d]$ en $\bb{R}^{d}$ es el conjunto de puntos
$(\lista*{x}{d})$ tales que $x^{d}=0$, lo denotaremos $\borde[{\hemi[d]}]$.
El \emph{interior} de $\hemi[d]$ se define como el conjunto de puntos
$(\lista*{x}{d})$ tales que $x^{d}>0$ y lo denotamos $\interior{\hemi[d]}$.
Si $M$ es una variedad con borde, el \emph{borde} de $M$ ser\'{a} el
conjunto de puntos $p\in M$ para los cuales existe una carta $(U,\varphi)$
tal que $\varphi(p)\in\borde[{\hemi[d]}]$, es decir, $\pi^{d}(\varphi(p))=0$.
El \emph{interior} de $M$ se define como el subconjunto formado por aquellos
puntos $p\in M$ para los cuales existe una carta $(U,\varphi)$ tal que
$\varphi(p)\in\interior{\hemi[d]}$, es decir, $\pi^{d}(\varphi(p))>0$.
Los puntos del interior de $M$ admiten entornos homeomorfos a abiertos de
$\bb{R}^{d}$. Denotamos el borde de $M$ por $\borde[M]$ y su interior por
$\interior{M}$.

Si bien los conjuntos $\interior{M}$ y $\borde[M]$ est\'{a}n bien definidos
e, intuitivamente, deber\'{\i}an ser disjuntos, no es claro, \textit{a %
priori} que as\'{\i} lo sea.

El interior $\interior{M}$ de una variedad $M$ de dimensi\'{o}n $d$ es una
variedad de dimensi\'{o}n $d$ (sin borde), pues es un subespacio
abierto de la variedad $M$. El borde $\borde[M]$ tambi\'{e}n es una
variedad topol\'{o}gica (sin borde). Su dimensi\'{o}n es $d-1$: si $p$
es un punto del borde y $(U,\varphi)$ es una carta para $M$ en $p$,
entonces
\begin{align*}
	U\cap\borde[M] & \,=\,\left\lbrace q\in U\,:\,\pi^{d}(\varphi(q))=0
				\right\rbrace
	\text{ .}
\end{align*}
%
De esto se deduce que $(U\cap\borde[M],\tilde{\varphi})$ es una carta para
$\borde[M]$ en $p$, donde $\tilde{\varphi}=(\lista*{\varphi}{d-1})$ --es
decir, proyectar sobre las primeras $d-1$ coordenadas la \emph{coordenada}
$\varphi$, valga la redundancia. La imagen de esta carta es un abierto de
$\bb{R}^{d-1}$ dado por intersecar el abierto $\varphi(U)$ de $\bb{R}^{d}$
con el hiperplano $\{x^{d}=0\}$ (y proyectar sobre las primeras $d-1$
coordenadas).

Dada una variedad topol\'{o}gica con borde $M$ y  una carta $(U,\varphi)$,
decimos que esta carta es una \emph{carta del interior}, si
$\varphi(U)\subset\hemi[d]$ es un abierto contenido en el interior del
semiespacio, es decir, $\varphi(U)\cap\borde[{\hemi[d]}]=\varnothing$. Si,
en cambio, $\varphi(U)\cap\borde[{\hemi[d]}]\not=\varnothing$, decimos que
$(U,\varphi)$ es una \emph{carta de borde}. Dado que los abiertos del interior
$\interior{\hemi[d]}$ del semiespacio son homeomorfos a abiertos de
$\bb{R}^{d}$ y \textit{vice versa}, tambi\'{e}n se denominar\'{a}n
\emph{cartas} para $M$ a los pares $(U,\varphi)$, donde $U\subset M$ es un
abierto y $\varphi:\,U\rightarrow\bb{R}^{d}$ es un homeomorfismo con un
abierto euclideo. Espec\'{\i}ficamente, estas cartas ser\'{a}n cartas
de interior, tambi\'{e}n. Finalmente, decimos que un abierto $U\subset M$
es una \emph{semibola coordenada}, si es el dominio de una carta
$(U,\varphi)$ para $M$ tal que $\varphi(U)\cap\borde[{\hemi[d]}]\not=%
\varnothing$ (es decir, una carta de borde) y $\varphi(U)=%
\bola{r}{x}\cap\hemi[d]$ para alg\'{u}n n\'{u}mero $r>0$ y alg\'{u}n punto
$x\in\borde[{\hemi[d]}]$. Un abierto $B\subset M$ se dice
\emph{semibola regular}, si existe una semibola coordenada $(B',\varphi)$
tal que $\clos{B}\subset B'$ y
\begin{align*}
	\varphi(B) & \,=\,\bola{r}{0}\cap\hemi[d]\text{ ,}\\
	\varphi(\clos{B}) & \,=\,\clos{\bola{r}{0}}\cap\hemi[d]\quad\text{y} \\
	\varphi(B') & \,=\,\bola{r'}{0}\cap\hemi[d]
\end{align*}
%
para ciertos n\'{u}meros $r'>r>0$.

\subsection{Estructuras diferenciales en variedades con borde}
Una \emph{estructura diferencial (o suave) en/de/para $M$} se define como un
atlas suavemente compatible maximal. Un \emph{atlas} para $M$ es un conjunto
de cartas que cubre a $M$. Dos cartas se dicen \emph{suavemente compatibles},
si los cambios de coordenadas en ambas direcciones son suaves. Un
atlas en $M$ se dice \emph{suavemente compatible}, si todo par de cartas
del atlas es un par compatible. Resta definir la noci\'{o}n de suavidad o
regularidad para una funci\'{o}n definida en un abierto de $\hemi[d]$.

Sea $A\subset\bb{R}^{n}$ un subconjunto arbitrario y sea
$F:\,A\rightarrow\bb{R}^{k}$ una funci\'{o}n. Se dice que $F$ es \emph{suave}
o \emph{diferenciable} o \emph{regular}, si, dado $x\in A$, existe una
funci\'{o}n $\widetilde{F}:\,B\rightarrow\bb{R}^{k}$ suave, diferenciable,
regular, definida en un entorno $B$ de $x$, tal que
$\widetilde{F}|_{B\cap A}=F|_{B\cap A}$. En particular, si $U\subset\hemi[d]$
es un subconjunto abierto, una funci\'{o}n $F:\,U\rightarrow\bb{R}^{k}$ es
suave, si, para cada punto $x\in U$, existe un abierto $\widetilde{U}$ de
$\bb{R}^{d}$ tal que $x\in\widetilde{U}$ y una funci\'{o}n suave
$\widetilde{F}:\,\widetilde{U}\rightarrow\bb{R}^{k}$ que coincide con
$F$ en $\widetilde{U}\cap U$. Notemos que la noci\'{o}n de diferenciabilidad
depende del dominio de definici\'{o}n de la funci\'{o}n; precisamente,
depende de cu\'{a}l es el espacio euclideo ambiente del cual $A$ es
subespacio. Si $(U,\varphi)$ y $(V,\psi)$ son cartas con borde para
una variedad con borde $M$, entonces las mismas son compatibles, si,
o bien $U\cap V=\varnothing$, o bien $U\cap V\not=\varnothing$ y
los cambios de coordenadas
\begin{align*}
	\varphi\circ\psi^{-1} & \,:\,\psi(U\cap V)\,\rightarrow\,
		\varphi(U\cap V)\quad\text{y} \\
	\psi\circ\varphi^{-1} & \,:\,\varphi(U\cap V)\,\rightarrow\,
		\psi(U\cap V)
\end{align*}
%
son suaves. Seg\'{u}n la definici\'{o}n anterior, esto quiere decir,
definiendo $f=\psi\circ\varphi^{-1}$, que, dado $x\in \varphi(U\cap V)$,
existe un abierto $B\subset\bb{R}^{d}$ tal que $x\in B$ y una
funci\'{o}n suave $\tilde{f}:\,B\rightarrow\bb{R}^{d}$ tal que
$\tilde{f}|_{B\cap\varphi(U\cap V)}=f|_{B\cap\varphi(U\cap V)}$ y,
lo mismo para la inversa $f^{-1}$. El entorno $B$ debe ser un abierto de
$\bb{R}^{d}$, porque $\varphi(U\cap V)$ es un subespacio de $\bb{R}^{d}$, y
$\tilde{f}$ tiene que tomar valores en $\bb{R}^{d}$, porque,
de la misma manera, $\psi(U\cap V)$ es un subespacio de $\bb{R}^{d}$.

Si $F:\,U\subset\hemi[d]\rightarrow\bb{R}^{k}$ es una funci\'{o}n suave,
entonces $F$ restringida a $U\cap\interior{\hemi[d]}$ es suave en el
sentido usual y las derivadas parciales de $F$ en puntos del borde quedan
determinadas por los valores en $\interior{\hemi[d]}$, independientemente
de la extensi\'{o}n, por la continuidad de las derivadas (de $F$ en el
interior y de la extensi\'{o}n en el entorno del punto).

Una variedad topol\'{o}gica con borde $M$ junto con una estructura
diferencial en $M$ se denomina \emph{variedad diferencial con borde}.
Una carta en $M$ se dice \emph{compatible}, si pertenece al atlas
maximal correspondiente a la estructura en $M$.
El lema \ref{thm:delascartas} sigue siendo v\'{a}lido, si se reemplaza
el espacio euclideo que modela localmente a la variedad por
un semiespacio. El resultado es que queda determinada una estructura de
variedad diferencial \emph{con borde}.

\subsection{Variedades con borde}
Sea $\esquina{d}$ el subconjunto de $\bb{R}^{d}$ de puntos cuyas coordenadas
son no negativas:
\begin{align*}
	\esquina{d} & \,=\,\left\lbrace(\lista*{x}{d})\in\bb{R}^{d}\,:\,
		x^{1}\geq 0,\,\dots,\,x^{d}\geq 0\right\rbrace
	\text{ .}
\end{align*}
%
Topol\'{o}gicamente, $\esquina{d}$ y $\hemi[d]$ son homeomorfos, como lo
son, por ejemplo, un cuadrado y un c\'{\i}rculo. La diferencia desde el
punto de vista geom\'{e}trico est\'{a} en la estructura diferencial.
El homeomorfismo entre la esquina y el semiespacio nos permitir\'{\i}a
trasladar al estructura dferencial de $\hemi[d]$ a $\esquina{d}$ ya
que $\esquina{d}$ es una variedad topol\'{o}gica con borde. Pero esta
estrucutura no ser\'{\i}a compatible con la topolog\'{\i}a de
$\esquina{d}$ como subespacio de $\bb{R}^{d}$.

Sea $M$ una variedad topol\'{o}gica (con borde) de dimensi\'{o}n $d$.
Una \emph{carta de esquina (o con esquinas)} para $M$ es un par
$(U,\varphi)$ tal que $U\subset M$ es abierto y
$\varphi:\,U\rightarrow\widehat{U}\subset\esquina{d}$ es un homeomorfismo
entre $U$ y un abierto $\widehat{U}$ de $\esquina{d}$. Notemos que,
componiendo una carta de borde de $M$ con un homeomorfismo, se obtiene
una carta de esquina de $M$ (esto no quiere decir que estos homeomorfismos
terminen siendo suaves). Un \emph{atlas (con esquinas)} en $M$ es un
conjunto de cartas (cartas de interior, cartas de borde \emph{y} cartas
con esquinas) que cubren a $M$. Dos cartas (posiblemente con esquinas)
$(U,\varphi)$ y $(V,\psi)$ para $M$ se dicen \emph{suavemente compatibles},
si $V\cap U=\varnothing$, o $V\cap U\not=\varnothing$ y los cambios de
coordenadas
\begin{align*}
	\varphi\circ\psi^{-1} & \,:\,\psi(V\cap U)\,\rightarrow\,
		\varphi(V\cap U) \quad\text{y} \\
	\psi\circ\varphi^{-1} & \,:\,\varphi(V\cap U)\,\rightarrow\,
		\psi(V\cap U)
	\text{ ,}
\end{align*}
%
que son homeomorfismos, son suaves. Los dominios y codominios de estas
composiciones son abiertos de $\esquina{d}$ en este caso. Como en el caso
de variedades con borde, decimos que una funci\'{o}n
$A\subset\bb{R}^{n}\rightarrow\bb{R}^{k}$ es suave si se puede extender
en un entorno de cada punto de su dominio de definici\'{o}n a una funci\'{o}n
suave. Recordemos que la extensi\'{o}n a considerar depende del espacio
euclideo ambiente del que $A$ sea subespacio. En el caso de los cambios
de coordenadas, si $f=\psi\circ\varphi^{-1}$, por ejemplo,
que $f:\,\varphi(U\cap V)\rightarrow\psi(U\cap V)$ sea suave quiere decir
que, dado $x\in\varphi(U\cap V)$, existe un abierto $B\subset\bb{R}^{d}$
tal que $x\in B$ y una extensi\'{o}n $\tilde{f}:\,B\rightarrow\bb{R}^{d}$
que coincide con $f$ en $\varphi(U\cap V)\cap B$ y que es suave.

Una \emph{estructura diferenial (o suave) con esquinas} en una variedad
topol\'{o}gica (con borde) $M$ es un atlas (con esquinas) suavemente
compatible maximal. Una variedad topol\'{o}gica (con borde), junto con una
estructura diferencial con esquinas, se denomina \emph{variedad diferencial %
con esquinas}. Dada una variedad diferencial con esquinas $M$, una carta
\emph{compatible} es una carta perteneciente a la estructura de $M$,
ya sea de interior, de borde o de esquina.

Vale la pena notar que el borde (de variedad) de una variedad est\'{a}
definido en t\'{e}rminos de la topolog\'{\i}a de la misma. En particular,
el borde de una variedad con esquinas es el conjunto de puntos $p$ para los
cuales existe un homeomorfismo $\varphi$ entre un entorno del punto en la
variedad y un abierto de $\hemi[d]$, de forma tal que
$\varphi(p)\in\borde[{\hemi[d]}]$. Por ejemplo,
\begin{align*}
	\borde[\esquina{d}] & \,=\,\left\lbrace (\lista*{x}{d})\in
		\esquina{d} \,:\,x^{1}=0\text{ o}\dots\text{ o }x^{d}=0
		\right\rbrace
	\text{ .}
\end{align*}
%
Las \emph{esquinas} de $\esquina{d}$ son los puntos $(\lista*{x}{d})$ tales
que al menos dos coordenadas se anulan.

\begin{teoInvarianzaDeLasEsquinas}\label{thm:invarianzadelasesquinas}
	Sea $M$ una variedad diferencial con esquinas de dimensi\'{o}n
	$d\geq 2$. Sea $p\in M$ y sea $(U,\varphi)$ una carta (compatible)
	en $p$. Si $\varphi(p)\in\esquina{d}$ pertenece a las esquinas de
	$\esquina{d}$, entonces, dada cualquier otra carta compatible
	$(V,\psi)$ en $p$, $\psi(p)$ tambi\'{e}n pertenece a las esquinas.
\end{teoInvarianzaDeLasEsquinas}

\begin{proof}
	Supongamos que $\psi(p)$ no pertenece a las esquinas de $\esquina{d}$.
	Como $\varphi(p)$ si es un punto de las esquinas, podemos suponer,
	reordenando las coordenadas, que $\varphi(p)=%
	(\lista*{x}{k},\,0,\,\dots,\,0)$ (en particular, $k\leq d-2$).
	Como $\psi(V)\subset\esquina{d}$ es abierto y $\psi(p)$ tiene, al
	menos, $d-1$ coordenadas no nulas, existe un subespacio lineal
	$S\subset\bb{R}^{d}$ de dimensi\'{o}n $d-1$ tal que
	$\psi(p)\in\psi(V)\cap A$ y que $S=\psi(V)\cap A$ es abierto en $A$.
	Esto es cierto, aun si $\psi(p)$ tiene a lo sumo $2$ coodenadas nulas,
	pero, como estamos suponiendo que $\psi(p)$ tiene a lo sumo una
	coordenada nula, es decir, que $p$ es un punto del borde pero no de la
	esquina o un punto del interior, podemos elegir $A$ de la forma
	$A=\{x^{i}=0\}$ para alg\'{u}n (\'{u}nico) $i$, si
	$\varphi(p)\in\borde[\esquina{d}]$ o de manera arbitraria,
	si $\psi(p)\in\interior{\esquina{d}}$.
	
	Sea $S'=S\cap\psi(U\cap V)=A\cap\psi(U\cap V)$ y sea
	$\alpha:\,S'\rightarrow\bb{R}^{d}$ la restricci\'{o}n de
	$\varphi\circ\psi^{-1}$ a $S'$. Dado que $\varphi\circ\psi^{-1}$
	es un difeomorfismo (es suave con inversa suave), por la regla de
	la cadena,
	\begin{align*}
		(\psi\circ\varphi^{-1})\circ\alpha & \,=\,\id[S']
			\quad\text{y} \\
		\jacobiana[\alpha(x)]{(\psi\circ\varphi^{-1})}\cdot
			\jacobiana[x]{\alpha} & \,=\,I
		\text{ .}
	\end{align*}
	%
	donde $I$ es una matriz que es la identidad en los vectores de $A$.
	Definiendo adecuadamente los tangentes en las esquinas, podr\'{\i}amos
	hablar del diferencial, en lugar de usar la matriz jacobiana, pero
	es esencialmente lo mismo: las derivadas parciales en
	$\borde[\esquina{d}]$ est\'{a}n determinadas por su valor en el
	interior. En particular, se deduce que $\jacobiana[x]{\alpha}$ es
	una matriz de rango m\'{a}ximo (es decir, la transformaci\'{o}n
	lineal asociada es inyectiva). Con tales definiciones, deber\'{\i}amos
	tener $\id[T_{x}S']$ en lugar de la matriz $I$. Como $\dim\,S=d-1$,
	vale que $\rango{\jacobiana[x]{\alpha}}=d-1$. Entonces existe un
	vector $v=(\lista*{v}{d})\in\bb{R}^{d}$ que pertenece al espacio
	columna de la matriz $\jacobiana[x]{\alpha}$ y tal que
	$v^{d-1}\not =0$ o $v^{d}\not =0$. Reordenando o multiplicando, de
	ser necesario, por $-1$, podemos asumir que $v^{d}<0$.

	Sea $\gamma:\,(-\epsilon,\epsilon)\rightarrow S$ una curva suave
	con origen en $\psi(p)$ y velocidad $\dot{\gamma}(0)$ tal que
	$\jacobiana[\psi(p)]{\alpha}(\dot{\gamma}(0))=v$. En particular,
	la \'{u}ltima coordenada de la composici\'{o}n $\alpha\circ\gamma(t)$
	verifica
	\begin{align*}
		(\alpha\circ\gamma(t))^{d} & \,<\,0
	\end{align*}
	%
	para $t\in (-\epsilon,\epsilon)$ suficientemente chico. Esto
	contradice el hecho de que $\alpha$ tiene imagen en $\esquina{d}$
	en donde todas las coordenadas son no negativas.
\end{proof}

Con este teorema, podemos definir sin ambig\"{u}edad la noci\'{o}n
de \emph{puntos de borde}, es decir puntos tales que, respecto de alguna
(y por lo tanto toda) carta compatible $(U,\varphi)$, en coordenadas,
$\varphi(p)$ pertenece a las esquinas de $\esquina{d}$. Un punto del borde,
como antes, es un punto para el cual debe valer que
$\varphi(p)\in\borde[\esquina{d}]$, bajo cualquier carta $\varphi$ (con
codominio un abierto de $\esquina{d}$).

\begin{obsElBordeConEsquinasNoEsVariedad}\label{thm:bordenoesvariedad}
	A diferencia de las variedades (diferenciales) con borde, el borde
	de una variedad con esquinas no es una variedad (con ni sin esquinas).
	Basta considerar $\esquina{d}$ (para $d\geq 2$). En este caso,
	\begin{align*}
		\borde[\esquina{d}] & \,=\,H_{1}\cup\,\cdots\,\cup H_{d}
		\text{ ,}
	\end{align*}
	%
	donde $H_{i}=\{(\lista*{x}{d})\in\esquina{d}\,:\,x^{i}=0\}$. Notemos
	que los subconjuntos $H_{i}$ s\'{\i} son variedades. Precisamente,
	$H_{i}$ es una variedad con esquinas de dimensi\'{o}n $d-1$.
\end{obsElBordeConEsquinasNoEsVariedad}

%

%--------

\chapter{Transformaciones suaves}
\section{Funciones y transformaciones suaves}
\theoremstyle{plain}
\newtheorem{propoSuavidadEsLocal}{Proposici\'{o}n}[section]
\newtheorem{propoSuaveEsConti}[propoSuavidadEsLocal]{Proposici\'{o}n}
\newtheorem{propoDelPegado}[propoSuavidadEsLocal]{Proposici\'{o}n}
\newtheorem{propoAlgunasFuncionesSuaves}[propoSuavidadEsLocal]{Proposici\'{o}n}

\theoremstyle{remark}
\newtheorem{obsComoLaUsual}{Observaci\'{o}n}[section]
\newtheorem{obsTodasSonSuaves}[obsComoLaUsual]{Observaci\'{o}n}
\newtheorem{obsOtrasCaracterizacionesDeSuavidad}[obsComoLaUsual]%
	{Observaci\'{o}n}

%-------------

Sea $M$ una variedad y sea $f:\,M\rightarrow\bb{R}$ una funci\'{o}n
arbitraria. Para describir a $f$, para poder decir algo acerca de sus
propiedades, estudiamos la funci\'{o}n en coordenadas. La
\emph{representaci\'{o}n de $f$ en coordenadas} o la funci\'{o}n $f$
\emph{en coordenadas} es cualquier composici\'{o}n de $f$ con la inversa de
una carta para $M$, es decir, algo de la forma $f\circ\varphi^{-1}$,
donde $\varphi$ es la funci\'{o}n coordenada de una carta $(U,\varphi)$
para $M$. Para hacer uso de esta idea, no es necesario que el codominio de
$f$ sea $\bb{R}$. La idea es que todo, o mucho de lo que se puede conocer de
$M$ se conoce a trav\'{e}s de las cartas.

\subsection{Funciones suaves}
Una funci\'{o}n $f:\,M\rightarrow\bb{R}$ o, m\'{a}s en general, una funci\'{o}n $f:\,M\rightarrow\bb{R}^{l}$ es una
\emph{funci\'{o}n suave}, si, para todo punto $p\in M$, existe una carta
compatible $(U,\varphi)$ para $M$ en $p$ tal que la composici\'{o}n
\begin{align*}
	f\circ\varphi^{-1} & \,:\,\varphi(U)\subset\bb{R}^{d}\,\rightarrow\,
					\bb{R}^{l}
\end{align*}
%
es suave en el sentido \emph{usual}: diremos que $f$ es diferenciable, de
clase $C^{1}$, de clase $C^{k}$, \textit{etcetera}, si todas las
composiciones $f\circ\varphi^{-1}$ tienen la propiedad correspondiente,
propiedades que dependen de la existencia de ciertos l\'{\i}mites y
de la continuidad de los mismos. En general, toda propiedad local acerca
de funciones definidas en abiertos de $\bb{R}^{d}$ se puede definir
tambi\'{e}n para funciones definidas en abiertos de variedades usando las
cartas (compatibles) para la variedad. Decimos que $f$ es \emph{regular},
si vista en coordenadas es regular.

Si $U$ es un abierto euclideo, entonces hay dos nociones de suavidad de
funciones. Si $f:\,U\rightarrow\bb{R}^{l}$ es una funci\'{o}n, podemos
decir que $f$ es suave porque es de clase $C^{k}$ en $U$ para todo $k\geq 1$,
en el sentido de que existen las derivadas parciales de orden $k$ y son
continuas para todo $k\geq 1$; o bien podemos decir que es suave porque
para todo punto $p\in U$ existe una carta $(U',\varphi)$ en $p$ tal que
$f\circ\varphi^{-1}$ es suave. Ambas nociones coinciden: si $f$ es suave en
el sentido usual, entonces, tomando la carta global $(U,\id[U])$, se ve
que $f\circ\id[U]^{-1}=f$ es suave (en el sentido usual) y que, por lo
tanto, para todo punto se puede hallar una carta tal que la funci\'{o}n
en coordenadas es suave en el sentido usual; rec\'{\i}procamente, basta
notar que, dada una carta $(V,\psi)$ para $U$, las funciones $\psi$ y
$\psi^{-1}$ son funciones suaves en sentido usual, ya que $\psi$ es la
funci\'{o}n de una carta compatible con la estructura en $U$ determinada
por el atlas $C^{\infty}$ $\{(U,\id[U])\}$.

\begin{obsComoLaUsual}\label{obs:comolausual}
Veamos esto \'{u}ltimo en detalle. La estructura diferencial usual en
$U$ es aquella determinada por el atlas que consiste en la \'{u}nica
carta $(U,\id[U])$ que se obtiene de restringir la carta $(\bb{R}^{d},\id)$
que define la estructura diferencial usual de $\bb{R}^{d}$. Sea $(V,\psi)$
una carta para $U$ compatible con esta estructura. Como $(U,\id[U])$ y
$(V,\psi)$ son cartas compatibles, las funciones
\begin{align*}
	\psi(U\cap V)\,\rightarrow\,\id[U](U\cap V) &
	\quad\text{e}\quad
	\id[U](U\cap V)\,\rightarrow\psi(U\cap V)
\end{align*}
%
son diferenciables (en el sentido usual, naturalmente). Pero estas funciones
son, precisamente, $\psi^{-1}:\,\psi(V)\rightarrow V$ y
$\psi:\,V\rightarrow\psi(V)$. Es decir, $\psi$ y $\psi^{-1}$ son suaves
en el sentido usual y $\psi$ es un difeomorfismo, en el sentido usual.
\end{obsComoLaUsual}

\begin{obsTodasSonSuaves}\label{obs:todassonsuaves}
Sea $M$ una variedad diferencial y sea $f:\,M\rightarrow\bb{R}^{l}$ una
funci\'{o}n suave. Sea $(U,\varphi)$ una carta compatible para $M$. Entonces
$f\circ\varphi^{-1}$ es suave: si $p\in U$ y $(V,\psi)$ es una carta tal
que $p\in V$ y $f\circ\psi^{-1}$ es suave,
\begin{align*}
	f\circ\varphi^{-1}|_{\varphi(U\cap V)} & \,=\,
	(f\circ\psi^{-1})|_{\psi(U\cap V)}\circ
		(\psi\circ\varphi^{-1})|_{\varphi(U\cap V)}
	\text{ .}
\end{align*}
%
Esta descomposi\'{o}n muestra que $f\circ\varphi^{-1}$ es suave ``en $p$''.
Como $p\in U$ era arbitrario, $f\circ\varphi^{-1}$ es suave.
\end{obsTodasSonSuaves}

Si $f:\,M\rightarrow\bb{R}^{l}$ es suave y $(U,\varphi)$ es una carta
(compatible) para $M$, la composici\'{o}n $\hat{f}=f\circ\varphi^{-1}$ se
denomina \emph{representaci\'{o}n de $f$ en coordenadas (respecto de la %
carta $\varphi$)}. La observaci\'{o}n \ref{obs:todassonsuaves} muestra que
la suavidad de las representaciones no depende de la carta.

\subsection{Transformaciones suaves}
Una funci\'{o}n $F:\,M\rightarrow N$ entre variedades diferenciales se dice
\emph{suave} o \emph{transformaci\'{o}n suave} (para distinguirlas de aquellas
con codominio $\bb{R}$ o $\bb{R}^{l}$), si, para todo punto $p\in M$
existen cartas $(U,\varphi)$ para $M$ en $p$ y $(V,\psi)$ para $N$ en
$F(p)$ tales que
\begin{itemize}
	\item[\i] $F(U)\subset V$ y
	\item[\i\i] $\psi\circ F\circ\varphi^{-1}:\,%
		\varphi(U)\rightarrow\psi(V)$ es suave en sentido usual entre
		abiertos de $\bb{R}^{\dim\,M}$ y de $\bb{R}^{\dim\,N}$.
\end{itemize}
%
Esta definici\'{o}n coincide con la definici\'{o}n de funci\'{o}n suave
pensando al codominio $\bb{R}^{l}$ como una variedad diferencial con su
estructura usual argumentando como en la observaci\'{o}n
\ref{obs:comolausual}. De manera similar al caso de funciones en $\bb{R}^{l}$,
llamamos \emph{representaci\'{o}n en coordenadas de $F$} a
$\hat{F}=\psi\circ F\circ\varphi^{-1}$.

\begin{obsOtrasCaracterizacionesDeSuavidad}\label{obsotrassuavidad}
	Sea $F:\,M\rightarrow N$ una funci\'{o}n. Entonces $F$ es suave, si
	y s\'{o}lo si para todo $p\in M$ existen cartas $(U,\varphi)$ en
	$p$ y $(V,\psi)$ en $F(p)$ tales que $U\cap F^{-1}(V)$ sea abierta
	en $M$ y
	\begin{align*}
		\psi\circ F\circ\varphi^{-1} & \,:\,
			\varphi(U\cap F^{-1}(V))\,\rightarrow\,\psi(V)
	\end{align*}
	%
	sea suave. Equivalentemente, $F$ es suave, si y s\'{o}lo si
	$F$ es continua y existen atlas compatibles
	$\{(U_{\alpha},\varphi_{\alpha})\}_{\alpha}$ de $M$ y
	$\{(V_{\beta},\psi_{\beta})\}_{\beta}$ de $N$ tales que las
	composiciones
	\begin{align*}
		\psi_{\beta}\circ F\circ\varphi_{\alpha}^{-1} & \,:\,
			\varphi_{\alpha}(U_{\alpha}\cap F^{-1}(V_{\beta}))
			\,\rightarrow\,\psi_{\beta}(V_{\beta})
	\end{align*}
	%
	sean suaves.
\end{obsOtrasCaracterizacionesDeSuavidad}

\subsection{Propiedades locales}
Al igual que la continuidad de funciones, suavidad es una propiedad
local.

\begin{propoSuavidadEsLocal}[Suavidad es una propiedad local]%
	\label{thm:suavidadeslocal}
	Sea $F:\,M\rightarrow N$ una funci\'{o}n entre variedades
	diferenciales. Entonces, si $F$ es suave, la restricci\'{o}n
	$F|_{U}:\,U\rightarrow N$ es suave para todo abierto $U\subset M$.
	Rec\'{\i}procamente, si todo punto admite un entorno $U$ tal que
	$F|_{U}$ sea suave, entonces $F$ es suave. Diremos que $F$ es suave
	en un punto $p\in M$, si exite un entorno $U$ de $p$ tal que la
	restricci\'{o}n $F|_{U}$ sea suave.
\end{propoSuavidadEsLocal}

Dado que las cartas son homeomorfismo podemos deducir la siguiente propiedad
\emph{deseable}.

\begin{propoSuaveEsConti}\label{thm:suaveesconti}
	Toda funci\'{o}n suave $F:\,M\rightarrow N$ es continua.
\end{propoSuaveEsConti}

\begin{proof}
	Para cada punto $p$ existen cartas $(U,\varphi)$ en $p$ y $(V,\psi)$
	en $F(p)$ tales que $F(U)\subset V$. Entonces
	\begin{align*}
		F|_{U} & \,=\,\psi^{-1}\circ (\psi\circ F\circ\varphi^{-1})
			\circ\varphi
	\end{align*}
	%
	implica que $F$ es continua restringida a $U$. Como esto es
	v\'{a}lido para todo punto $p\in M$, $F$ es continua en $M$.
\end{proof}

De manera similar, el lema del pegado para funciones continuas tambi\'{e}n
tiene su an\'{a}logo acerca de funciones suaves.

\begin{propoDelPegado}[Lema del pegado]\label{thm:delpegado}
	Sean $M$ y $N$ variedades diferenciales. Sea
	$\cal{U}=\{U_{\alpha}\}_{\alpha}$ un cubrimiento de $M$ por abiertos.
	Si, para cada $\alpha$, existe una funci\'{o}n suave
	$F_{\alpha}:\,U_{\alpha}\rightarrow N$ de manera que
	\begin{align*}
		F_{\alpha}|_{U_{\alpha}\cap U_{\beta}} & \,=\,
			F_{\beta}|_{U_{\alpha}\cap U_{\beta}}
	\end{align*}
	%
	para todo par $\alpha,\beta$, entonces existe una \'{u}nica funci\'{o}n
	suave $F:\,M\rightarrow N$ tal que $F|_{U_{\alpha}}=F_{\alpha}$
	para todo $\alpha$.
\end{propoDelPegado}

\begin{propoAlgunasFuncionesSuaves}\label{thm:algunassuaves}
	Las funciones constantes $c:\,M\rightarrow N$ son suaves. La
	identidad $\id[M]:\,M\rightarrow M$ es suave. La inclusi\'{o}n
	$U\hookrightarrow M$ de una subvariedad abierta es suave.

	Si $\lista{M}{k}$ y $N$ son variedades diferenciales (y, a lo sumo,
	una de las $M_{i}$ posee borde no vac\'{\i}o), entonces una
	funci\'{o}n $F:\,N\rightarrow M_{1}\times\,\cdots\,\times M_{k}$
	es suave, si y s\'{o}lo si las composiciones
	$\pi_{i}\circ F:\,N\rightarrow M_{i}$ son suaves.

	La composici\'{o}n de funciones suaves es suave.
\end{propoAlgunasFuncionesSuaves}

\begin{proof}
	Demostramos la \'{u}ltima afirmaci\'{o}n.
	Supongamos que $F:\,M\rightarrow N$ y que $G:\,N\rightarrow\tilde{N}$
	son funciones suaves. Sea $p\in M$. Por hip\'{o}tesis, existen cartas
	$(V,\varphi)$ en $F(p)$, $(W,\psi)$ en $G(F(p))$ tales que
	$G(V)\subset W$ y de manera que $\psi\circ G\varphi^{-1}$ es suave
	en $\varphi(V)$. Porque $F$ es continua, $F^{-1}(V)$ es abierta en
	$M$ y contiene a $p$. Existe, entonces, una carta $(U,\tilde{\varphi})$
	tal que $p\in U\subset F^{-1}(V)$. En particular,
	$\varphi\circ F\circ\tilde{\varphi}^{-1}:%
		\,\tilde{\varphi}(U)\rightarrow\varphi(V)$. Se deduce entonces
	que $G\circ F(U)\subset G(V)\subset W$ y que
	\begin{align*}
		\psi\circ(G\circ F)\circ\tilde{\varphi}^{-1} & \,=\,
			(\psi\circ G\circ\varphi^{-1})\circ
			(\varphi\circ F\tilde{\varphi}^{-1})
	\end{align*}
	%
	es suave por ser composici\'{o}n de funciones suaves entre abiertos
	de espacios euclideos.
\end{proof}

%
\section{Particiones de la unidad}
\theoremstyle{plain}
\newtheorem{lemaLocFin}{Lema}[section]
\newtheorem{propoParticionesLCH}[lemaLocFin]{Proposici\'{o}n}
\newtheorem{propoParticionesSigmaCompLCH}[lemaLocFin]{Proposici\'{o}n}

\newtheorem{lemaParticionesEnRI}[lemaLocFin]{Lema}
\newtheorem{lemaParticionesEnRII}[lemaLocFin]{Lema}
\newtheorem{lemaParticionesEnRIII}[lemaLocFin]{Lema}

\newtheorem{coroParticionesVarTop}[lemaLocFin]{Corolario}
\newtheorem{propoParticionesVarDif}[lemaLocFin]{Proposici\'{o}n}

\newtheorem{propoHayChichones}[lemaLocFin]{Proposici\'{o}n}
\newtheorem{propoExtenderFuncionesSuaves}[lemaLocFin]{Proposici\'{o}n}
\newtheorem{coroExhaustiva}[lemaLocFin]{Corolario}

\theoremstyle{remark}

%-------------

El objetivo de esta secci\'{o}n es demostrar que las variedades diferenciales
adminten particiones de la unidad subordinadas a cualquier cubrimiento por
abiertos. Comenzamos por el siguiente lema.

\begin{lemaLocFin}\label{thm:locfin}
	Sea $X$ un espacio topol\'{o}gico y sea $\{X_{\alpha}\}_{\alpha}$
	una colecci\'{o}n de subconjuntos. Entonces, si
	$\{X_{\alpha}\}_{\alpha}$ es localmente finito en $X$, la colecci\'{o}n
	$\{\clos{X_{\alpha}}\}_{\alpha}$ tambi\'{e}n lo es y, adem\'{a}s,
	\begin{align*}
		\clos{\bigcup_{\alpha}\,X_{\alpha}} & \,=\,
			\bigcup_{\alpha}\,\clos{X_{\alpha}}
		\text{ .}
	\end{align*}
	%
\end{lemaLocFin}

\begin{proof}
	Sea $p\in X$. Si $\{X_{\alpha}\}_{\alpha}$ es localmente finito,
	existe un abierto $U\subset X$ tal que $p\in U$ y
	$U\cap X_{\alpha}=\varnothing$ para todo $\alpha$ salvo finitos.
	Si $x\in U\cap\clos{X_{\alpha}}$, entonces existe un abierto
	$V$ tal que $x\in V$ y $V\subset U$. Como $x\in\clos{X_{\alpha}}$,
	la intersecci\'{o}n $V\cap X_{\alpha}$ es no vac\'{\i}a. Pero entonces
	$U\cap X_{\alpha}\not=\varnothing$. En definitiva, el abierto $U$
	interseca a lo sumo finitos subconjuntos $\clos{X_{\alpha}}$.

	En cuanto a la \'{u}ltima afirmaci\'{o}n, como
	$X_{\beta}\subset\bigcup_{\alpha}\,X_{\alpha}$ para todo $\beta$,
	$\clos{X_{\beta}}\subset\clos{\bigcup_{\alpha}\,X_{\alpha}}$ para
	todo $\beta$. Rec\'{\i}procamente, si $\{X_{\alpha}\}_{\alpha}$
	es localmente finita y $x\in\clos{\bigcup_{\alpha}\,X_{\alpha}}$,
	entonces existe un abierto $U$ tal que $x\in U$ y
	$U\cap\clos{X_{\alpha}}=\varnothing$ para todos salvo finitos
	$\alpha$. Por otro lado, como $x$ pertenece a la clausura de
	$\bigcup_{\alpha}\,X_{\alpha}$,
	\begin{align*}
		U\cap\bigcup_{\alpha}\,X_{\alpha} & \,=\,
		\bigcup_{\alpha}\,(U\cap X_{\alpha})
		\,=\, \big(U\cap X_{\alpha_{1}}\big)\cup\,\cdots\,\cup
			\big(U\cap X_{\alpha_{k}}\big) \\
		& \,=\, U\cap\big(X_{\alpha_{1}}\cup\,\cdots\,
			\cup X_{\alpha_{k}}\big)
	\end{align*}
	%
	es no vac\'{\i}a. Dicho de otra manera, si $\alpha\not =\alpha_{i}$
	para alg\'{u}n $i$, entonces $U\cap X_{\alpha}=\varnothing$ y, al
	menos para un valor de $i$, $U\cap X_{\alpha_{i}}\not=\varnothing$.
	Adem\'{a}s, lo mismo es cierto si se reemplaza $U$ por alg\'{u}n otro
	abierto $U'\subset U$ tal que $p\in U'$. En definitiva,
	\begin{align*}
		x & \,\in\,\clos{X_{\alpha_{1}}\cup\,\cdots\,
			\cup X_{\alpha_{k}}} \,=\,
			\clos{X_{\alpha_{1}}}\cup\,\cdots\,
			\cup\clos{X_{\alpha_{k}}}
		\text{ .}
	\end{align*}
	%
\end{proof}

Sea $X$ un espacio topol\'{o}gico y sea $\cal{U}=\{U_{\alpha}\}_{\alpha}$
un cubrimiento por abiertos de $X$. Una \emph{partici\'{o}n de la unidad %
para $X$ subordinada al cubrimiento $\cal{U}$} es una familia
$\{\psi_{\alpha}\}_{\alpha}$ de funciones $\psi_{\alpha}:\, X\rightarrow\bb{R}$
que cumplen:
\begin{itemize}
	\item[\i] $0\leq \psi_{\alpha}\leq 1$ en $X$ para todo $\alpha$;
	\item[\i\i] $\soporte{\psi_{\alpha}}\subset U_{\alpha}$;
	\item[\i\i\i] la colecci\'{o}n $\{\soporte{\psi_{\alpha}}\}_{\alpha}$
		es localmente finita; y
	\item[\i v] $\sum_{\alpha}\,\psi_{\alpha} =1$ en $X$.
\end{itemize}
%
La suma en \textit{(iv)} est\'{a} bien definida pues, siendo los soportes
localente finitos, para todo $x\in X$, $\psi_{\alpha}(x)=0$ para todo
$\alpha$ salvo finitos. Una partici\'{o}n de la unidad se dir\'{a}
\emph{suave}, si las funciones $\psi_{\alpha}$ son todas suaves.

\subsection{Particiones en variedades topol\'{o}gicas}
Empecemos recordando algunos resultados para espacios localmente compactos.

\begin{propoParticionesLCH}\label{thm:particioneslch}
	Sea $X$ un espacio topol\'{o}gico localmente compacto Hausdorff y sea
	$K\subset X$ un compacto. Sea $\cal{U}=\{\lista{U}{k}\}$ un
	cubrimiento de $K$ por abiertos de $X$. Entonces existe una
	partici\'{o}n de la unidad para $K$ subordinada a $\cal{U}$.
\end{propoParticionesLCH}

\begin{proof}
	Por el lema \ref{thm:entornodeunpuntolch}, existen, para cada
	$x\in K$, abiertos $V_{x}$ tales que $x\in V_{x}$,
	$\clos{V_{x}}\subset U_{j}$ para alg\'{u}n $j$ y $\clos{V_{x}}$ sea
	compacto. Como $K$ es compacto, existen $\lista{x}{m}$ tales que
	$K\subset\bigcup_{i=1}^{m}\,\interior{N_{x_{i}}}$. Para cada
	$j\in[\![1,k]\!]$, definimos
	\begin{align*}
		F_{j} & \,=\,
			\bigcup\,\left\lbrace N_{x_{i}}\,:\,
			i\in[\![1,m]\!],\,N_{x_{i}}\subset U_{j}\right\rbrace
		\text{ .}
	\end{align*}
	%
	Entonces cada $F_{j}$ est\'{a} contenido en $U_{j}$ y es compacto.
	Por \ref{thm:urysohnlch}, existen funciones continuas
	$\lista{g}{k}$ de soporte compacto en $X$ tales que
	$0\leq g_{j}\leq 1$, $g_{j}=1$ en $F_{j}$ y
	$\soporte{g_{j}}\subset U_{j}$. En particular, en $K$,
	$\sum_{j}\,g_{j}\geq 1$ y $K\subset\{\sum_{j}\,g_{j} >0\}=:U$.
	Aplicando \ref{thm:urysohnlch} nuevamente, se obtiene una funci\'{o}n
	$f$ de soporte compacto contenido en el abierto $U$, que toma valores
	entre $0$ y $1$, que restringida a $K$ es constantemente $1$.

	Sea $g_{k+1}:=1-f$. Entonces $\sum_{j=1}^{k+1}\,g_{j}>0$ en todo el
	espacio $X$. Para $j\leq k$, sea $h_{j}$ la funci\'{o}n
	\begin{align*}
		h_{j} & \,=\,\frac{g_{j}}{\sum_{t=1}^{k+1}\,g_{t}}
		\text{ .}
	\end{align*}
	%
	Entonces $\soporte{h_{j}}=\soporte{g_{j}}\subset U_{j}$ y
	$\sum_{j=1}^{k}\,h_{j}=1$ en $K$.
\end{proof}

\begin{propoParticionesSigmaCompLCH}\label{thm:particionessigmacomplch}
	Si $X$ es un espacio localmente compacto Hausdorff y $\sigma$-compacto,
	entonces, dado un cubrimiento por abiertos $\cal{U}$ de $X$,
	existe una partici\'{o}n de la unidad para $X$ subordinada a
	$\cal{U}$ que consiste en funciones de soporte comapcto.
\end{propoParticionesSigmaCompLCH}

Como corolario del resultado anterior, deducimos la existencia de particiones
de la unidad para variedades topol\'{o}gicas.

\begin{coroParticionesVarTop}\label{thm:particionesvartop}
	Sea $M$ una variedad topol\'{o}gica y sea $\cal{U}$ un cubrimiento
	por abiertos de $M$. Existe una partici\'{o}n de la unidad
	para $M$ subordinada a $\cal{U}$.
\end{coroParticionesVarTop}

\begin{proof}
	Las variedades topol\'{o}gicas son espacios localmente compactos
	Hausdorff y $\sigma$-compactos. Una dmostraci'{o}n un poco
	m\'{a}s \emph{expl\'{\i}cita} podr\'{\i}a ir por el lado de
	la exitencia de bolas coordenadas regulares y de poder refinar
	cualquier cubrimiento por uncubrimiento que est\'{e} conformado
	por tales abiertos.
\end{proof}

\subsection{Particiones en variedades diferenciales}
Para establecer la existencia de particiones suaves de la unidad, primero
es necesario demostrar los resultados espec\'{\i}ficos para los espacios
euclideos.

\begin{lemaParticionesEnRI}\label{thm:particionesenri}
	La funci\'{o}n $f:\,\bb{R}\rightarrow\bb{R}$ dada por
	$f(t)=\indica{\bb{R}_{>0}}(t)e^{-\frac{1}{t}}$ es suave.
\end{lemaParticionesEnRI}

\begin{lemaParticionesEnRII}\label{thm:particionesenrii}
	Dados $r_{1}<r_{2}$ n\'{u}meros reales, existe una funci\'{o}n
	suave $h:\,\bb{R}\rightarrow\bb{R}$ tal que $h(t)=0$ si
	$t\geq r_{2}$, $h(t)=1$ si $t\leq r_{1}$ y $0 <h< 1$ en
	otro caso.
\end{lemaParticionesEnRII}

\begin{proof}
	Una funci\'{o}n con estas propiedades es aquella dada por
	\begin{align*}
		h(t) & \,=\,\frac{f(r_{2}-t)}{f(t-r_{1})+f(r_{2}-t)}
		\text{ ,}
	\end{align*}
	%
	donde $f$ es la funci\'{o}n del lema \ref{thm:particionesenri}.
\end{proof}

\begin{lemaParticionesEnRIII}\label{thm:particionesenriii}
	Dados n\'{u}meros reales $0<r_{1}< r_{2}$, existe una funci\'{o}n
	suave $H:\,\bb{R}^{d}\rightarrow\bb{R}$ tal que $H(x)=0$ si
	$x\not\in\bola{r_{2}}{0}$, $H(x)=1$ si $x\in\clos{\bola{r_{1}}{0}}$
	y $0<H<1$ en otro caso.
\end{lemaParticionesEnRIII}

\begin{proof}
	Una funci\'{o}n con estas propiedades es $H(x)=h(|x|)$.
\end{proof}

\begin{propoParticionesVarDif}\label{thm:particionesvardif}
	Sea $M$ una variedad diferencial y sea
	$\cal{U}=\{U_{\alpha}\}_{\alpha}$ un cubrimiento por abiertos. Existe
	una partici\'{o}n suave de la unidad para $M$ subordinada a $\cal{U}$.
\end{propoParticionesVarDif}

\begin{proof}
	Los abiertos $U_{\alpha}$ son subvariedades abiertas de $M$.
	En particular, cada uno de ellos admite una base $\cal{B}_{\alpha}$
	de bolas coordenadas regulares. La uni\'{o}n de dichas bases,
	$\cal{B}=\bigcup_{\alpha}\,\cal{B}_{\alpha}$ es una base para $M$.
	Por lo tanto, existe un refinamiento $\cal{V}$ numerable y
	localmente finito de $\cal{U}$ compuesto por bolas coordenadas
	regulares de $\cal{B}$. Sea 
	\begin{align*}
		\cal{V}_{\alpha} & \,=\,\left\lbrace
			B\in\cal{B}\,:\,B\subset \cal{B}_{\alpha}
			\right\rbrace
		\text{ .}
	\end{align*}
	%
	Cada bola de $\cal{V}_{\alpha}$ es una bola coordenada regular en
	$U_{\alpha}$ y, por lo tanto, existen una bola coordenada
	$(B',\varphi)$ (compatible con la estructura diferncial de
	$U_{\alpha}$) tal que $\clos{B}\subset B'\subset U_{\alpha}$,
	$\varphi(B)=\bola{r_{1}}{0}$ y $\varphi(B')=\bola{r_{2}}{0}$
	para ciertos n\'{u}meros reales $r_{2}>r_{1}>0$. Definimos
	entonces una funci\'{o}n $f_{B}:\,M\rightarrow\bb{R}$ por
	\begin{align*}
		f_{B} & \,=\,
			\begin{cases}
				H\circ\varphi & \text{ en }B' \\
				0 & \text{ en }M\setmin\clos{B}\text{ .}
			\end{cases}
	\end{align*}
	%
	La funci\'{o}n $H$ que aparece en la definici\'{o}n de $f_{B}$ es
	la funci\'{o}n suave dada por el lema \ref{thm:particionesenriii}
	para los valores de $r_{2}$ y $r_{1}$ correspondientes (es decir, la
	definici\'{o}n de $H$ depende de $B$). Cada $f_{B}$ es suave en $M$ y
	$\soporte{f_{B}}=\clos{B}$.

	Como $\cal{V}$ es localmente finito, $\{\clos{B}\,:\,B\in\cal{V}\}$
	es localmente finito por \ref{thm:locfin}. Sea
	$f=\sum_{B\in\cal{V}}\,f_{B}$. Esta funci\'{o}n est\'{a} bien
	definida y es suave en $M$. Adem\'{a}s, como $f_{B}\geq 0$ para todo
	en $M$ y es estrictamente positiva en $B$ para todo $B\in\cal{V}$,
	la suma $f$ es estrictamente positiva en $M$. EN particular,
	si definimos $g_{B}=f_{B}/f$, esta funci\'{o}n es suave en $M$,
	$0\leq g_{B}\leq 1$ y $\sum_{B\in\cal{V}}\,g_{B}=1$.

	Para obtener una partici\'{o}n de la unidad subordinada al cubrimiento
	$\cal{U}$ es necesario reinexar y agrupar la funciones $g_{B}$.
	Para cada $B\in\cal{V}$, sea $a(B)$ alg\'{u}n \'{\i}ndice tal que
	$B'\subset U_{a(B)}$ (por ejemplo, si $B\in\cal{V}_{\alpha}$, podemos
	tomar $a(B)=\alpha$ ya que $B$ es una bola coordenada regular de
	$U_{\alpha}$ y $B'\subset U_{\alpha}$). Para cada $\alpha$ se define
	\begin{align*}
		\psi_{\alpha} & \,=\,\sum_{B\in\cal{V}\,:\,a(B)=\alpha}\,g_{B}
		\text{ .}
	\end{align*}
	%
	Si la suma es vac\'{\i}a la funci\'{o}n se define como la funci\'{o}n
	$0$ en $M$. En particular, las funciones $\psi_{\alpha}$ son
	suaves en $M$, $0\leq\psi_{\alpha}\leq 1$,
	\begin{align*}
		\soporte{\psi_{\alpha}} & \,=\,
			\clos{\bigcup_{B\,:\,a(B)=\alpha}\,B}
			\,=\,\bigcup_{B\,:\,a(B)=\alpha}\,\clos{B}
			\,\subset\, U_{\alpha}
		\text{ .}
	\end{align*}
	%
	Adem\'{a}s, $\{\soporte{\psi_{\alpha}}\}$ es una familia localmente
	finita y $\sum_{\alpha}\,\psi_{\alpha}=\sum_{B}\,g_{B}=1$ en $M$.

	Si $M$ es una variedad con borde, las bolas regulares $B$ pueden
	ser, en realidad, semibolas regulares. Aun as\'{\i}, esto quiere
	decir que para cada $B$ semibola regular de $U_{\alpha}$,
	existen $B'\subset U_{\alpha}$, $\varphi:\,B'\rightarrow\bb{R}^{d}$
	de manera que $(B',\varphi)$ es una carta comatible para $U_{\alpha}$,
	$\clos{B}\subset B'$ y
	\begin{align*}
		\varphi(B) & \,=\,\bola{r_{1}}{0}\cap\{x^{d}\geq 0\}
		\quad\text{y} \\
		\varphi(B') & \,=\,\bola{r_{2}}{0}\cap\{x^{d}\geq 0\}
	\end{align*}
	%
	con $r_{2}>r_{1}>0$. En particular, podemos tomar la funci\'{o}n
	suave $H$ correspondiente a las bolas de radios $r_{1}$ y $r_{2}$
	y definir $f_{B}$ como antes. Estas funciones $f_{B}$ siguen siendo
	suaves porque, donde no es cero,
	$f_{B}\circ\varphi^{-1}=H|_{\varphi(B')}$, que es suave en la semibola
	$\varphi(B')$, porque $H$ es una extensi\'{o}n suave a $\bb{R}^{d}$.
	El resto de la demostraci\'{o}n contin\'{u}a de la misma manera.
\end{proof}

\subsection{Algunos corolarios}
Sea $M$ una variedad diferencial. Como en el lema de Tietze para funciones
continuas, queremos ver si es posible extender funciones definidas en un
subconjunto de $M$ de manera suave.

\begin{propoHayChichones}\label{thm:haychichones}
	Sea $M$ una variedad diferencial, sea $A\subset M$ un subconjunto
	cerrado y sea $U\supset A$ un abierto que lo contiene. Entonces
	existe una funci\'{o}n suave $\psi:\,M\rightarrow\bb{R}$
	tal que $\soporte{\psi}\subset U$, $\psi=1$ en $A$ y $0\leq\psi\leq 1$
	en $M$.
\end{propoHayChichones}

\begin{proof}
	Sea $U_{0}=U$ y sea $U_{1}=M\setmin A$. Sea $\{\psi_{0},\psi_{1}\}$
	una partici\'{o}n suave de la unidad subordinada al cubrimiento
	$\{U_{0},U_{1}\}$. Como $\psi_{1}=0$ en $A$ y $\psi_{0}+\psi_{1}=1$
	en $M$, debe ser $\psi_{0}=1$ en $A$.
\end{proof}

Sea $M$ una variedad y sea $f:\,A\rightarrow N$ una funci\'{o}n definida
en un subconjunto arbitrario $A\subset M$. La noci\'{o}n de suavidad de
la funci\'{o}n $f$ requiere que el dominio de la misma sea, o bien toda
la variedad $M$, o bien un abierto $U\subset M$. No hemos definido aun
lo que significa que $f$ sea suave si su dominio de definici\'{o}n es
un subconjunto arbitrario de $M$. Decimos que $f:\,A\rightarrow N$ es
suave, si, dado $p\in A$, existe un abierto $W\subset M$ tal que $p\in W$
y una extensi\'{o}n suave $\tilde{f}:\,W\rightarrow N$, es decir,
$\tilde{f}$ es suave de $W$ en $N$ y $\tilde{f}=f$ en $W\cap A$.

\begin{propoExtenderFuncionesSuaves}\label{thm:extenderfuncionessuaves}
	Sea $M$ una variedad diferencial y sea $A\subset M$ un subconjunto
	cerrado. Si $f:\,A\rightarrow\bb{R}^{l}$ es suave, para todo abierto
	$U\supset A$ existe una funci\'{o}n suave
	$\tilde{f}:\,M\rightarrow\bb{R}^{l}$ tal que
	$\tilde{f}|_{A}=f$ y $\soporte{\tilde{f}}\subset U$.
\end{propoExtenderFuncionesSuaves}

\begin{proof}
	Para cada punto $p\in A$, existe un entorno $W_{p}$ de $p$
	y una funci\'{o}n suave $\tilde{f}_{p}:\,W_{p}\rightarrow\bb{R}^{l}$
	tales que $\tilde{f}_{p}$ coincide con $f$ en $W_{p}\cap A$. Podemos
	asumir que $W_{p}\subset U$, reemplazando $W_{p}$ por $W_{p}\cap U$.
	La familia de abiertos $\{W_{p}\,:\,p\in A\}$, junto con $M\setmin A$,
	es un cubrimiento por abiertos de $M$. Sea
	$\{\psi_{p}\,:\,p\in A\}\cup\{\psi_{0}\}$ una partici\'{o}n de a
	unidad subordinada a la cubrimiento, con $\soporte{\psi_{p}}%
	\subset W_{p}$ y $\soporte{\psi_{0}}\subset M\setmin A$.

	Para cada $p\in A$, el producto $\psi_{p}\tilde{f}_{p}$ coincide
	con $\tilde{f}_{p}$ en (la clausura de) alg\'{u}n entorno de $p$
	dentro de $W_{p}$. Adem\'{a}s, como $\soporte{\psi_{p}}\subset W_{p}$,
	podemos extender $\psi_{p}\tilde{f}_{p}$ a una funci\'{o}n
	definida en $M$, por cero fuera de $W_{p}$ (se define por cero
	fuera del soporte de $\psi_{p}$ y, donde los abiertos $W_{p}$ y
	$M\setmin\soporte{\psi_{p}}$ se intersecan, las definiciones
	coinciden). Esta extensi\'{o}n es suave por \ref{thm:delpegado}.
	Porque los soportes de las funciones $\psi_{p}$ forman una
	colecci\'{o}n localmente finita, la funci\'{o}n definida por la
	expresi\'{o}n
	\begin{align*}
		\tilde{f}(x) & \,=\,\sum_{p\in A}\,\psi_{p}(x)\tilde{f}_{p}(x)
	\end{align*}
	%
	est\'{a} bien definida y es suave. Si tomamos $x\in A$, entonces
	$\psi_{0}(x)=0$ y $\tilde{f}_{p}(x)=f(x)$ para todo $p\in A$.
	Entonces $\tilde{f}(x)=f(x)$ porque $\sum_{p\in A}\,\psi_{p}(x)=1$.

	En cuanto al soporte de $\tilde{f}$,
	\begin{align*}
		\soporte{\tilde{f}} & \,\subset\,
			\clos{\bigcup_{p\in A}\,\soporte{\psi_{p}}}
			\,=\,\bigcup_{p\in A}\,\soporte{\psi_{p}}
			\,\subset\, U
		\text{ .}
	\end{align*}
	%
\end{proof}

\begin{coroExhaustiva}\label{thm:exhaustiva}
	Toda variedad (diferencial) admite una funci\'{o}n exhaustiva (suave).
\end{coroExhaustiva}

Una \emph{funci\'{o}n exhaustiva} para una variedad $M$ es una funci\'{o}n
(continua) $f:\,M\rightarrow\bb{R}$ tal que los conjuntos
$\{f\leq c\}=f^{-1}(\left(-\infty,c\right])$ sean compactos para todo
$c\in\bb{R}$.

\begin{proof}
	Sea $M$ una variedad diferencial. Sabemos que existe una familia
	numerable de subconjuntos abiertos con clausura compacta
	$\{V_{j}\}_{j\geq 1}$ que cubren a $M$. Tambi\'{e}n sabemos que,
	por ser cubrimiento de $M$, admite una partici\'{o}n (suave) de la
	unidad $\{\psi_{j}\}_{j\geq 1}$ subordinada al mismo.

	Definimos $f:\,M\rightarrow\bb{R}$ por
	\begin{align*}
		f(x) & \,=\,\sum_{j\geq 1}\,j\psi_{j}(x)
		\text{ .}
	\end{align*}
	%
	Esta funci\'{o}n est\'{a} bien definida porque los soportes de
	las funciones $\psi_{j}$ son una colecci\'{o}n localmente finita
	(y es suave). Adem\'{a}s, $f\geq\sum_{j}\,\psi_{j}=1$ en $M$.

	Para ver que $f$ es una funci\'{o}n exhaustiva, sea $c\in\bb{R}$.
	Veamos que $\{f\leq c\}$ est\'{a} contenida en una uni\'{o}n finita
	de compactos $\clos{V_{j}}$. Si $N>c$ es un entero cualquiera
	mayor que $c$ y si $x\not\in\bigcup_{j=1}^{N}\,\clos{V_{j}}$, entonces
	\begin{align*}
		f(x) & \,=\,\sum_{j\geq N+1}\,j\psi_{j}(x) \,\geq\,
			N\sum_{j\geq N+1}\,\psi_{j}(x) \,\geq\,N>c
		\text{ .}
	\end{align*}
	%
	Dicho de otra manera, si $f(x)\leq c$, entonces
	$x\in\bigcup_{j=1}^{N}\,\clos{V_{j}}$. Como $\{f\leq c\}$ es cerrado
	y est\'{a} contenido en un compacto, resulta ser compacto.
\end{proof}

%

%--------

\chapter{Espacio tangente y fibrado tangente}
\section{El espacio tangente}
\theoremstyle{plain}
\newtheorem{propoElDiferencial}{Proposici\'{o}n}[section]
\newtheorem{lemaDerivacionesSonLocales}[propoElDiferencial]{Lema}
\newtheorem{propoDerivacionesIsomorfasI}[propoElDiferencial]{Proposici\'{o}n}
\newtheorem{propoDerivacionesIsomorfasII}[propoElDiferencial]{Proposici\'{o}n}

\theoremstyle{remark}
\newtheorem{obsDerivaciones}{Observaci\'{o}n}[section]
\newtheorem{obsSobreDerivaciones}[obsDerivaciones]{Observaci\'{o}n}
\newtheorem{obsDerivacionesLocalGlobal}[obsDerivaciones]{Observaci\'{o}n}
\newtheorem{obsDerivacionesIsomorfas}[obsDerivaciones]{Observaci\'{o}n}

%-------------

\subsection{Vectores tangentes geom\'{e}tricos}
Sea $a\in\bb{R}^{d}$ un punto en el espacio euclideo de dimensi\'{o}n $d$.
El \emph{espacio tangente geom\'{e}tricoa $\bb{R}^{d}$ en el punto $a$},
denotado $\bb{R}^{d}_{a}$ es el conjunto de pares de la forma $(a,v)$,
con $v\in\bb{R}^{d}$, es decir, $\bb{R}^{d}_{a}\simeq\{a\}\times\bb{R}^{d}$.
Los elementos de este espacio ser\'{a}n denominados \emph{vectores tangentes %
geom\'{e}tricos}. Denoteamos al punto correspondiente al par $(a,v)$
por $v_{a}$ o por $v|_{a}$. El espacio tangente geom\'{e}tico al punto $a$
es un espacio vectorial de dimensi\'{o}n $d$, si definimos las operaciones
en $\bb{R}^{d}_{a}$ trasladando los vectores al origen. Es decir, si
$v_{a},w_{a}\in\bb{R}^{d}_{a}$ y $\lambda\in\bb{R}$, entonces
\begin{align*}
	\lambda\cdot v_{a} & \,\equiv\,(\lambda\cdot v)_{a}
	\quad\text{y} \\
	v_{a} + w_{a} & \,\equiv\,(v+w)_{a}
	\text{ .}
\end{align*}
%

La base can\'{o}nica en $\bb{R}^{d}$ proporciona una base para el tangente
en $a$: $\{e_{1}|_{a},\,\dots,\,e_{d}|_{a}\}$. Notemos, adem\'{a}s, que
$\bb{R}^{d}_{a}\simeq\bb{R}^{d}$ can\'{o}nicamente v\'{\i}a $v_{a}\mapsto v$.

\subsection{Derivaciones}
Fijemos un punto $a\in\bb{R}^{d}$. Asociada a cada elemento
$v_{a}\in\bb{R}^{d}_{a}$, est\'{a} la noci\'{o}n de dervada direccional de
una funci\'{o}n: dada una funci\'{o}n $f:\,U\rightarrow\bb{R}$ definida
en un entorno $U$ de $a$, siempre que tenga sentido, definimos la
\emph{derivada direccional de $f$ en $a$ en la direcci\'{o}n del vector $v$}
como el l\'{\i}mite
\begin{align*}
	\left.\gancho{v}\right|_{a}f & \,\equiv\,
		\frac{\partial f}{\partial v}(a)\,\equiv\,
		\lim_{t\to 0}\,\frac{f(a+tv)-f(a)}{t} \\
	& \,\equiv\,\left.\gancho{t}\right|_{t=0}f(a+tv)
	\text{ .}
\end{align*}
%
Por la linealidad del l\'{\i}mite, la operaci\'{o}n $f\mapsto\gancho{v}|_{a}f$
es lineal en $f$ (en donde est\'{e} definida la aplicaci\'{o}n). Adem\'{a}s,
si las derivadas direccionales de $f$ y de $g$ existen, entonces
\begin{align*}
	\left.\gancho{v}\right|_{a}(fg) & \,=\,
		\left.\gancho{v}\right|_{a}f\cdot g(a) +
		f(a)\cdot\left.\gancho{v}\right|_{a}g
	\text{ .}
\end{align*}
%
Si $f:\,U\rightarrow\bb{R}$ es diferenciable en $a$, entonces
$\gancho{v}|_{a}f$ es lineal en $v_{a}$, es decir,
\begin{align*}
	\left(\left.\gancho{v}\right|_{a}+\lambda
		\left.\gancho{w}\right|_{a}\right)f & \,=\,
		\left.\gancho{(v+\lambda w)}\right|_{a}f
	\text{ .}
\end{align*}
%
En particular, si $v_{a}=v^{i}e_{i}|_{a}$, entonces, para toda $f$
diferenciable en $a$, definiendo $\gancho{e_{i}}|_{a}=\gancho{x^{i}}|_{a}$,
\begin{align*}
	\left.\gancho{v}\right|_{a}f & \,=\,
		v^{i}\left.\gancho{x^{i}}\right|_{a}f
	\text{ .}
\end{align*}
%

La existencia de particiones de la unidad en $\bb{R}^{d}$ permite simlificar
las definiciones, permitiendo asumir que las funciones $f$ diferenciables
cerca de un punto $a$ est\'{a}n definidas en todo el espacio y no s\'{o}lo
en un entorno del punto. De todas maneras, supongamos, para mantener
cierta generalidad, que $\cal{O}_{a}$ es un \emph{anillo de funciones %
regulares} en $a$. En general, $\cal{O}_{a}$ ser\'{a} uno de los siguientes:
\textit{(i)} $C^{\infty}(a)$, donde
\begin{align*}
	C^{\infty}(a) & \,=\,\left\lbrace (U,f)\,:\, a\in U\subset\bb{R}^{d}
		\text{ abierto},\,f\in C^{\infty}(U)\right\rbrace
	\text{ ,}
\end{align*}
%
\textit{(ii)} el espacio de pares $(U,f)$, donde $a\in U$ es un abierto de
$\bb{R}^{d}$, $f:\,U\rightarrow\bb{R}$ es diferenciable en $U$ y las derivadas
parciales $\frac{\partial f}{\partial x^{i}}:\,U\rightarrow\bb{R}$ son
diferenciables en $a$, es decir, el espacio de funciones dos veces
diferenciables en $a$, \textit{(iii)} el anillo de g\'{e}rmenes de funciones
\emph{regulares} en $a$ (para alguna interpretaci\'{o}n de ``regular''), o
bien \textit{(iv)} $C^{\infty}(U)$ o $C^{\infty}(M)$. Una
\emph{derivaci\'{o}n en $a$} en el anillo $\cal{O}_{a}$ es una
aplicaci\'{o}n $\bb{R}$-lineal $w:\,\cal{O}_{a}\rightarrow\bb{R}$ que
verifica la regla de Leibnitz: si $f,g\in\cal{O}_{a}$,
\begin{align*}
	w(fg) & \,=\,w(f)\cdot g(a)+f(a)\cdot w(g)
	\text{ .}
\end{align*}
%
El conjunto de todas las derivaciones en $a$ del anillo ($\bb{R}$-\'{a}lgebra)
$\cal{O}_{a}$ se denomina \emph{espacio de derivaciones en $a$} y lo
denotaremos $\derivaciones{\cal{O}_{a}}$, o bien
$\derivaciones[a]{\cal{O}_{a}}$, cuando sea necesario aclarar el punto,
particularmente, cuando el \'{a}lgebra $\cal{O}_{a}$ no hace referncia al
punto $a$ en donde est\'{a}n \emph{basadas} las derivaciones. Este ser\'{a}
el caso, por ejemplo, de $\cal{O}_{a}=C^{\infty}(U)$, cuando $U$
es un abierto que contiene al punto $a$.

\begin{obsDerivaciones}\label{obs:derivaciones}
	Si $w$ es una derivaci\'{o}n en $a$, entonces $w(1)=0$ y
	$w(fg)=0$, si $f(a)=g(a)=0$. Es decir, las derivaciones se anulan
	en las funciones constantes y en el cuadrado del ideal
	$\{f\in\cal{O}_{a}\,:\,f(a)=0\}$.
\end{obsDerivaciones}

Ya vimos que todo elemento del espacio tangente geom\'{e}trico
$v_{a}\in\bb{R}^{d}_{a}$ determina una derivaci\'{o}n, $\gancho{v}|_{a}f$.
Adem\'{a}s, vimos que, si nos restringimos a evaluar derivaciones en el
espacio de funciones diferenciables en $a$, la aplicaci\'{o}n
\begin{align*}
	\Phi & \,:\,v_{a}\mapsto\left(
		\left.\gancho{v}\right|_{a}:\,f\mapsto
		\left.\gancho{v}\right|_{a}f\right)
\end{align*}
%
es lineal. En particular, $\gancho{v}|_{a}=v^{i}\gancho{x^{i}}|_{a}$,
si $v_{a}=v^{i}e_{i}|_{a}$, donde $\gancho{x^{i}}|_{a}$ es la derivaci\'{o}n
correspondiente al vector can\'{o}nico $e_{i}|_{a}$, es decir, la derivada
parcial $i$-\'{e}sima.

Ahora bien, las funciones $x^{i}:\,\bb{R}^{d}\rightarrow\bb{R}$ dada por
$x^{i}(a)=a^{i}$ es diferenciable en todo punto $a$. Entonces tomando
la derivada direccional respecto de $v_{a}$, se deduce que
\begin{align*}
	\left.\gancho{v}\right|_{a}x^{i} & \,=\,
		v^{j}\left.\gancho{x^{j}}\right|_{a}x^{i} \,=\, v^{i}
	\text{ .}
\end{align*}
%
De esta igualdad podemos concluir que la aplicaci\'{o}n $\Phi$ es inyectiva.

La inyectividad de $\Phi$ se puede demostrar de otra manera, sin hacer
referencia a una base can\'{o}nica. Sea $v_{a}\in\bb{R}^{d}_{a}$. Entonces
$v_{a}$ se corresponde con el par $(a,v)$ para un cierto vector
$v\in\bb{R}^{d}$. Sea $\varphi:\,\bb{R}^{d}\rightarrow\bb{R}$ una funci\'{o}n
lineal. Como
\begin{align*}
	\varphi(a+v) -\varphi(a)-\varphi(v) & \,=\,0
\end{align*}
%
por linealidad, se deduce que $\varphi$ es diferenciable. La deriada
direccional de $\varphi$ en $a$ est\'{a} dada por
\begin{align*}
	\left.\gancho{v}\right|_{a}\varphi & \,=\,
		\lim_{t\to 0}\,\frac{\varphi(a+tv)-\varphi(a)}{t} \,=\,
		\varphi(v)
	\text{ .}
\end{align*}
%
En particular, si $\gancho{v}|_{a}\varphi=0$ para toda
$\varphi$ en el dual $\dual{\bb{R}^{d}}$, como $\dual{\bb{R}^{d}}$ separa
puntos en $\bb{R}^{d}$, debe ser $v=0$. Notemos que toda funci\'{o}n lineal
es derivable en todas direcciones, pero es derivable, si y s\'{o}lo si
es continua.

Veamos condiciones suficientes para garantizar la sobreyectividad de $\Phi$.
Sea $f:\,U\rightarrow\bb{R}$ una funci\'{o}n diferenciable en un abierto
$U$ que contiene al punto $a$. Entonces las derivadas parciales
$\frac{\partial f}{\partial x^{i}}:\,U\rightarrow\bb{R}$ est\'{a}n definidas
en $U$. Sabemos que, si $b\in U$, tomando $v=b-a$, vale que
\begin{align*}
	f(a+v) & \,=\,f(a)+\diferencial[a]{f}(v)+r(v)
\end{align*}
%
donde
\begin{align*}
	\diferencial[a]{f}(v) & \,v^{i}\left.
		\frac{\partial f}{\partial x^{i}}\right|_{a}
\end{align*}
%
y $r$ es una funci\'{o}n tal que el cociente $r(w)/|w|$ tiende a cero con
$|w|$. Equivalentemente, si $v\in\bb{R}^{d}$ y $a+v\in U$, la funci\'{o}n
\begin{align*}
	r(v) & \,=\,f(a+v)-f(a)-v^{i}
		\left.\frac{\partial f}{\partial x^{i}}\right|_{a}
\end{align*}
%
verifica que $r(v)/|v|$ tiende a cero con $v$.

Sea, para cada $i$, $g_{i}=\frac{\partial f}{\partial x^{i}}$. Por el
Teorema fundamental del c\'{a}lculo, e integrando por partes,
\begin{align*}
	f(a+v)-f(a) & \,=\,\int_{0}^{1}\,v^{i}g_{i}(a+tv)\,dt \,=\,
		v^{i}g_{i}(a+v) -\int_{0}^{1}\,tv^{i}v^{j}
		\left.\frac{\partial g_{i}}{\partial x^{j}}\right|_{a+tv}\,dt\\
	& \,=\,v^{i}g_{i}(a) +v^{i}v^{j}\int_{0}^{1}(1-t)
		\left.\frac{\partial g_{i}}{\partial x^{j}}\right|_{a+tv}\,dt
	\text{ .}
\end{align*}
%
Para que esto se v\'{a}lido, asumimos que cada $g_{i}$ es diferenciable
en ((casi todo punto de) un entorno de) $a$. Si $w$ es una derivaci\'{o}n
en $a$ en $C^{\infty}(a)$, definimos un elemento $v_{a}\in\bb{R}^{d}_{a}$
evaluando $w$ en las funciones coordenadas: para cada $i$, tomamos
$v^{i}=w(x^{i})$ y definimos $v_{a}=v^{i}e_{i}|_{a}$. En particular,
\begin{align*}
	w(x^{i}) & \,=\,v^{i} \,=\,\left.\gancho{v}\right|_{a}(x^{i})
	\text{ .}
\end{align*}
%
Sea $f\in C^{\infty}(a)$. En un entorno de $a$,
\begin{equation}
	\label{eq:taylorordendos}
	\begin{aligned}
		f(x) & \,=\, f(a) + (x^{k}-a^{k})
			\left.\frac{\partial f}{\partial x^{k}}\right|_{a} \\
		& \quad + (x^{i}-a^{i})(x^{j}-a^{j})\int_{0}^{1}\,(1-t)\left.
			\frac{\partial^{2}f}{\partial x^{j}\partial x^{i}}
			\right|_{a+t(x-a)}\,dt
	\text{ .}
	\end{aligned}
\end{equation}
%
Entonces, evaluando $w$ en $f$,
\begin{align*}
	w(f) & \,=\,w(x^{k})\left.\frac{\partial f}{\partial x^{k}}\right|_{a}
		\,=\,v^{k}\left.\frac{\partial f}{\partial x^{k}}\right|_{a}
		\,=\,\left.\gancho{v}\right|_{a}(f)
	\text{ .}
\end{align*}
%
As\'{\i}, queda demostrado que $\Phi_{a}:\,\bb{R}^{d}_{a}\rightarrow%
\derivaciones{C^{\infty}(a)}$ es un isomorfismo $\bb{R}$-lineal. Definimos
el \emph{espacio tangente a $\bb{R}^{d}$ en $a$ en sentido de derivaciones}
como el espacio $\derivaciones{C^{\infty}(a)}$. La existencia del
isomorfismo $\Phi_{a}$ justifica en cierta medida este nombre.

\begin{obsSobreDerivaciones}\label{obs:sobrederivaciones}
	El argumento anterior usando el desarrollo de Taylor con su
	expresi\'{o}n integral para el resto sigue siendo v\'{a}lido
	si asumimos que las funciones $f$ son dos veces diferenciables
	en $a$ (o $C^{2}$ en un entorno de $a$, o que las $g_{i}$ son
	diferenciables en casi todo punto de un entorno compacto de $a$\dots).
	Dicho de otra manera, si asumimos que la derivaci\'{o}n $w$ est\'{a}
	definida en un anillo m\'{a}s grande, por ejemplo, para toda $f$ dos
	veces diferenciable en (un entorno de) $a$, entonces la
	derivaci\'{o}n $\Phi_{a}(v_{a})$ coincide con $w$ al ser evaluadas
	en cualquier funci\'{o}n para la cual el desarrollo de Taylor
	\eqref{eq:taylorordendos} sea v\'{a}lido. Es decir, $\Phi_{a}$ sigue
	siendo sobreyectiva, si su codominio fuese alg\'{u}n otro espacio de
	derivaciones. Pero es casi inmediato que, si $\cal{O}_{a}\supset%
	\cal{O}'_{a}$, entonces $\derivaciones{\cal{O}_{a}}\subset%
	\derivaciones{\cal{O}'_{a}}$, es decir, condiciones menos
	restrictivas sobre el par\'{a}metro $f$ resulta en condiciones
	m\'{a}s restrictivas para $w$.
\end{obsSobreDerivaciones}

De ahora en adelante, denotaremos por $v_{a}$ tanto al elemento en el espacio
tangente geom\'{e}trico $\bb{R}^{d}_{a}$ como a la derivaci\'{o}n
correspondiente $f\mapsto\gancho{v}|_{a}f$.

\begin{obsDerivacionesLocalGlobal}\label{obs:derivacioneslocalglobal}
	El espacio de funciones suaves definidas en alg\'{u}n entorno de
	un punto $a\in\bb{R}^{d}$, $C^{\infty}(a)$ contiene al espacio de
	funciones suaves definidas en todo el espacio, es decir,
	hay una inclusi\'{o}n
	\begin{align*}
		C^{\infty}(\bb{R}^{d}) & \,\hookrightarrow C^{\infty}(a)
	\end{align*}
	%
	dada por $f\mapsto (\bb{R}^{d},f)$.

	Rec\'{\i}procamente, a los fines de estudiar derivaciones, podemos
	pensar que $C^{\infty}(a)$ est\'{a} incluido en
	$C^{\infty}(\bb{R}^{d})$. Espec\'{\i}ficamente, dada
	$(U,f)\in C^{\infty}(a)$, existe un entorno $V$ de $a$ tal que
	$\clos{V}\subset U$. Asociada al cubrimiento
	$\{U,\setcomp{\clos{V}}\}$ de $\bb{R}^{d}$, existe una partici\'{o}n
	suave de la unidad $\{\psi_{0},\psi_{1}\}$ tal que
	$\soporte{\psi_{0}}\subset U$ y $\soporte{\psi_{1}}\subset%
	\setcomp{\clos{V}}$. Definimos una funci\'{o}n $\tilde{f}$ en
	$\bb{R}^{d}$ por
	\begin{align*}
		\tilde{f}(x) & \,=\,
			\begin{cases}
				f(x)\psi(x) & \quad\text{si }x\in U\text{ ,}\\
				0 & \quad\text{si }x\not\in\clos{V}\text{ .}
			\end{cases}
	\end{align*}
	%
	Esta funci\'{o}n es suave y definida en todo $\bb{R}^{d}$. Adem\'{a}s,
	$\tilde{f}(x)=f(x)$ para todo $x\in V$. Elegir de esta manera una
	partici\'{o}n de la unidad y definir luego una extensi\'{o}n de $f$,
	\emph{define} una aplicaci\'{o}n $f\mapsto\tilde{f}$ de
	$C^{\infty}(a)$ en $C^{\infty}(\bb{R}^{d})$. Esta aplicaci\'{o}n
	est\'{a} lejos de ser inyectiva, con lo cual no da una
	inclusi\'{o}n de $C^{\infty}(a)$ en $C^{\infty}(\bb{R}^{d})$. De todas
	maneras, esta observaci\'{o}n muestra que hay, para cada elemento
	de $C^{\infty}(a)$, al menos una manera de extenderlo a un objeto en
	$C^{\infty}(\bb{R}^{d})$. Este argumento no es otra cosa que
	la proposici\'{o}n \ref{thm:extenderfuncionessuaves} en el caso en
	que el conjunto $A$ es un punto, o, m\'{a}s en general, un compacto
	en $M=\bb{R}^{d}$.
\end{obsDerivacionesLocalGlobal}

\subsection{El espacio tangente a una variedad}
El espacio tangente a $\bb{R}^{d}$ en un punto $a$, ya sea el tangente
geom\'{e}trico $\bb{R}^{d}_{a}$ o el tangente en sentido de derivaciones
$\derivaciones{C^{\infty}(a)}$, son nociones locales. Por lo tanto, haciendo
uso de las cartas compatibles de una variedad diferencial, queda m\'{a}s o
menos clara una manera de definir el espacio tangente a una variedad.
Aun as\'{\i}, la noci\'{o}n de derivaci\'{o}n es lo suficientemente abstracta
como para permitir generalizarla al contexto de variedades y dar una
definici\'{o}n \emph{intr\'{\i}nseca} del tangente, sin necesidad de un
argumento del estilo de tomar cartas.

Sea $M$ una variedad diferencial y sea $a\in M$ un punto arbitrario.
Decimos que una transformaci\'{o}n lineal $v:\,C^{\infty}(M)\rightarrow\bb{R}$
es una \emph{derivaci\'{o}n en $a$}, si satisface la regla de Leibnitz:
para todo par $f,g\in C^{\infty}(M)$, se cumple que $v(fg)=v(f)g(a)+f(a)v(g)$.
Al igual que en $\bb{R}^{d}$, toda derivaci\'{o}n en $a$ de $C^{\infty}(M)$
se anula en las constantes y, si $f(a)=g(a)=0$, entonces $v(fg)=0$ (c.~f.
la observaci\'{o}n \ref{obs:derivaciones}). Definimos el \emph{espacio %
tangente a $M$ en $a$ en sentido de derivaciones} como el espacio de
derivaciones en $a$ de $C^{\infty}(M)$ y lo denotamos
$\derivaciones[a]{C^{\infty}(M)}$, o bien $\derivaciones[a]{M}$.

Sea ahora $F:\,M\rightarrow N$ una funci\'{o}n suave. Asociada a $F$ hay una
transformaci\'{o}n lineal (notar la similitud con la \emph{definici\'{o}n}
de diferenciabilidad)
\begin{align*}
	\diferencial[a]{F} & \,:\,\derivaciones[a]{M}\,\rightarrow\,
		\derivaciones[F(a)]{N}
\end{align*}
%
dada por $\diferencial[a]{F}(v):\,f\mapsto v(f\circ F)$ para todo elemento
$f\in C^{\infty}(N)$. La aplicaci\'{o}n $\diferencial[a]{F}(v)$ es,
efectivamente una derivaci\'{o}n en $N$ en el punto $F(a)$. La
trasformaci\'{o}n lineal $\diferencial[a]{F}$ se llama el \emph{diferencial %
de $F$ en $a$}.

\begin{propoElDiferencial}\label{thm:eldiferencial}
	Si $F:\,M\rightarrow N$ es una transformaci\'{o}n suave, entonces
	$\diferencial[a]{F}:\,\derivaciones[a]{M}\rightarrow%
	\derivaciones[F(a)]{N}$ es lineal. Si $G:\,N\rightarrow P$ es
	otra funci\'{o}n suave, entonces
	\begin{align*}
		\diferencial[a]{(G\circ F)} & \,=\,
			\diferencial[F(a)]{G}\circ\diferencial[a]{F}
		\text{ .}
	\end{align*}
	%
	Para toda variedad diferencial $M$ y todo punto $a\in M$,
	\begin{align*}
		\diferencial[a]{(\id[M])} & \,=\,
			\id[{\derivaciones[a]{M}}]
		\text{ .}
	\end{align*}
	%
\end{propoElDiferencial}

El siguiente resultado ser\'{a} fundamental para poder calcular la
dimensi\'{o}n del espacio tangente a una variedad.

\begin{lemaDerivacionesSonLocales}\label{thm:derivacionessonlocales}
	Sea $M$ una variedad diferencial, sea $a\in M$ y sean
	$f,g\in C^{\infty}(M)$ funciones suaves. Si existe un
	abierto $U\subset M$, entorno de $a$, en donde $f$ y $g$ coinciden,
	entonces $vf=vg$ para toda $v\in\derivaciones[a]{C^{\infty}(M)}$.
\end{lemaDerivacionesSonLocales}

\begin{proof}
	Por linealidad de las derivaciones es suficiente ver que toda
	funci\'{o}n suave $f\in C^{\infty}(M)$ que se anula en un entorno
	$U$ de $a$ eval\'{u}a a $0$. Sea entonces $f$ una funci\'{o}n
	suave tal que $\soporte{f}\subset M\setmin\{a\}$. Sea $\psi$
	una funci\'{o}n suave tal que $0\leq\psi\leq 1$ en $M$,
	$\psi=1$ en $\soporte{f}$ y $\soporte{\psi}\subset M\setmin\{a\}$.
	Tal funci\'{o}n existe por la proposici\'{o}n \ref{thm:haychichones}.
	Notemos que $\psi(x)f(x)=f(x)$ para todo punto $x\in M$. Es decir,
	como funciones en $M$, $\psi\cdot f$ y $f$ son iguales. As\'{\i},
	dada $v:\,C^{\infty}(M)\rightarrow\bb{R}$, vale que
	$v(f)=v(\psi f)$. Si, adem\'{a}s, $v$ es una derivaci\'{o}n en $a$,
	$v(\psi f)=v(\psi)f(a)+\psi(a)v(f)=0$, ya que $f(a)=\psi(a)=0$.
	En definitiva, $v(f)=0$ para toda derivaci\'{o}n
	$v\in\derivaciones[a]{C^{\infty}(M)}$.
\end{proof}

\begin{obsDerivacionesIsomorfas}\label{obs:derivacionesisomorfas}
	Siguiendo con el comentario de la observaci\'{o}n
	\ref{obs:derivacioneslocalglobal}, demostraremos que los espacios
	$\derivaciones{C^{\infty}(a)}$ y
	$\derivaciones[a]{C^{\infty}(\bb{R}^{d})}$ son (naturalmente)
	isomorfos, es decir que la inclusi\'{o}n natural
	$C^{\infty}(\bb{R}^{d})\hookrightarrow C^{\infty}(a)$ determina un
	isomorfismo a nivel de los espacios de derivaciones en $a$, aunque
	no haya una manera clara o can\'{o}nica de incluir $C^{\infty}(a)$
	en $C^{\infty}(\bb{R}^{d})$.

	Una manera de ver que $\derivaciones[a]{C^{\infty}(\bb{R}^{d})}\simeq%
	\derivaciones{C^{\infty}(a)}$, es notar que, al igual que
	existe un isomorfismo $\Phi_{a}:\,\bb{R}^{d}_{a}\rightarrow%
	\derivaciones{C^{\infty}(a)}$, existe un isomorfismo an\'{a}logo
	$\tilde{\Phi}_{a}:\,\bb{R}^{d}_{a}\rightarrow%
	\derivaciones[a]{C^{\infty}(\bb{R}^{d})}$. Otra manera de demostrar
	que son espacios isomorfos es usar el lema
	\ref{thm:derivacionessonlocales}. Este argumento nos muestra c\'{o}mo
	en muchas ocasiones vamos a poder reemplazar el espacio de funciones
	regulares en un punto $a$, $C^{\infty}(a)$, por el espacio de
	funciones regulares en toda la variedad $C^{\infty}(\bb{R}^{d})$.
	Como este argumento est\'{a} dado exclusivamente en t\'{e}rminos de
	derivaciones, es igualmente v\'{a}lido reemplazando $\bb{R}^{d}$
	por una variedad diferencial arbitraria $M$.

	Sea $v\in\derivaciones{C^{\infty}(a)}$ y sea
	$f\in C^{\infty}(\bb{R}^{d})$ una funci\'{o}n suave. La inclusi\'{o}n
	$f\mapsto (\bb{R}^{d},f)$ nos permite definir un elemento
	$\tilde{v}\in\derivaciones[a]{C^{\infty}(\bb{R}^{d})}$ a partir de $v$:
	\begin{align*}
		\tilde{v}(f) & \,=\,v((\bb{R}^{d},f)).
	\end{align*}
	%
	Supongamos que $\tilde{v}$ es la derivaci\'{o}n cero y sea
	$(U,f)\in C^{\infty}(a)$. Sea $\tilde{f}:\,\bb{R}^{d}\rightarrow\bb{R}$
	una funci\'{o}n suave que coincide con $f$ en (la clausura de)
	cierto entorno $V$ de $a$ contenido en $U$ (c.~f. la observaci\'{o}n
	\ref{obs:derivacioneslocalglobal}). Sin importar cu\'{a}l sea
	esta extensi\'{o}n, como $\tilde{f}$ y $f$ coinciden en un entorno
	de $a$, por \ref{thm:derivacionessonlocales},
	\begin{align*}
		0 & \,=\,\tilde{v}\tilde{f} \,\equiv\,v(\bb{R}^{d},\tilde{f})
			\,=\,v(U,f)
		\text{ .}
	\end{align*}
	%
	As\'{\i}, se ve que $v$ ten\'{\i}a que ser la derivaci\'{o}n cero
	en $C^{\infty}(a)$. Es decir, $v\mapsto\tilde{v}$ es lineal e
	inyectiva.

	Sea, ahora, $w\in\derivaciones[a]{C^{\infty}(\bb{R}^{d})}$. Sea
	$v:\,C^{\infty}(a)\rightarrow\bb{R}$ la funci\'{o}n
	\begin{align*}
		v(U,f) & \,=\,w(\tilde{f})
		\text{ ,}
	\end{align*}
	%
	donde $\tilde{f}$ es una (alguna) funci\'{o}n suave, definida
	globalmente y que coincide con $f$ en un entorno de $a$ contenido en
	$U$. Nuevamente, por el lema \ref{thm:derivacionessonlocales},
	no importa cu\'{a}l sea la elecci\'{o}n $\tilde{f}$, el resultado
	es el mismo. Entonces $v$ est\'{a} bien definida y es una
	derivaci\'{o}n en $C^{\infty}(a)$. Sea
	$\tilde{v}\in\derivaciones[a]{C^{\infty}(\bb{R}^{d})}$ la
	derivaci\'{o}n determinada por $v$ y sea $g\in C^{\infty}(\bb{R}^{d})$.
	Evaluando,
	\begin{align*}
		\tilde{v}(g) & \,\equiv\,v(\bb{R}^{d},g) \,\equiv\,w(g)
		\text{ .}
	\end{align*}
	%
	Entonces $\tilde{v}$ y $w$, como funciones $C^{\infty}(\bb{R}^{d})%
	\rightarrow\bb{R}$, son iguales. En definitiva, $v\mapsto\tilde{v}$
	es un isomorfismo $\bb{R}$-lineal de $\derivaciones{C^{\infty}(a)}$
	en $\derivaciones[a]{C^{\infty}(\bb{R}^{d})}$.
\end{obsDerivacionesIsomorfas}

Sea $M$ una variedad diferencial y sea $a\in M$. Denotemos por $C^{\infty}(a)$
al espacio de funciones suaves definidas en un entorno de $a$, es decir,
$C^{infty}(a)$ es el conjunto de pares $(U,f)$ donde $U\subset M$ es
abierto y contiene a $a$ y $f\in C^{\infty}(U)$.

\begin{propoDerivacionesIsomorfasI}\label{thm:derivacionesisomorfasi}
	Los espacios de derivaciones $\derivaciones[a]{C^{\infty}(M)}$ y
	$\derivaciones{C^{\infty}(a)}$ son isomorfos.
\end{propoDerivacionesIsomorfasI}

De la misma manera en que pudimos identificar el espacio de derivaciones en
$C^{\infty}(a)$ y el espacio de derivaciones en $a$ en $C^{\infty}(M)$,
podemos identificar, dado un abierto $U\subset M$ tal que $a\in U$, los
espacios $\derivaciones[a]{M}$ y $\derivaciones[a]{U}$. La
identificaci\'{o}n, en este caso, est\'{a} dada por el isomorfismo
$\diferencial[a]{i}$, donde $i:\,U\hookrightarrow M$ es la inclusi\'{o}n.

\begin{propoDerivacionesIsomorfasII}\label{thm:derivacionesisomorfasii}
	Sea $M$ una variedad diferencial, sea $U\subset M$ un abierto y
	sea $i:\,U\hookrightarrow M$ la inclusi\'{o}n. Si $a\in U$,
	el diferencial $\diferencial[a]{i}:\,\derivaciones[a]{U}\rightarrow%
	\derivaciones[i(a)]{M}$ es un isomorfismo lineal.
\end{propoDerivacionesIsomorfasII}

La demostraci\'{o}n es an\'{a}loga al argumento dado en la obsevaci\'{o}n
\ref{obs:derivacionesisomorfas}. La identificaci\'{o}n
$v\mapsto\tilde{v}$ definida en las derivaciones $v:\,C^{\infty}(a)%
\rightarrow\bb{R}$, est\'{a} dada expl\'{\i}citamente en este caso por la
aplicaci\'{o}n diferencial $\diferencial[a]{i}$.

El isomorfismo $\bb{R}^{d}_{a}\rightarrow\derivaciones[a]{\bb{R}^{d}}$
dado por $e_{i}|_{a}\mapsto\gancho{x^{i}}|_{a}$, donde
$\{e_{1}|_{a},\,\dots,\,e_{d}|_{a}\}$ es la base can\'{o}nica de
$\bb{R}^{d}_{a}$, muestra que $\derivaciones[a]{\bb{R}^{d}}$ es una espacio
de dimensi\'{o}n finita y que su dimensi\'{o}n es $d$. Por la
proposici\'{o}n \ref{thm:derivacionesisomorfasii},
$\derivaciones[a]{U}\simeq\derivaciones[a]{\bb{R}^{d}}$ para todo abierto
$U\subset\bb{R}^{d}$ que contenga a $a$. En una variedad $M$, dada una
carta compatible $(U,\varphi)$ en un punto $a$, la funci\'{o}n
$\varphi:\,U\rightarrow\varphi(U)$ es un difeomorfismo. En particular,
por \ref{thm:eldiferencial}, el diferencial $\diferencial[a]{\varphi}:\,%
\derivaciones[a]{U}\rightarrow\derivaciones[\varphi(a)]{\varphi(U)}$ es un
isomorfismo determinado por la (elecci\'{o}n de) carta $(U,\varphi)$ en $a$.
Esto permite deducir que
\begin{align*}
	\derivaciones[a]{M} & \,\simeq\,\derivaciones[a]{U}\,\simeq\,
		\derivaciones[\varphi(a)]{\varphi(U)}
\end{align*}
%
de manera can\'{o}nica. En particular, $\dim\,\derivaciones[a]{M}=\dim\,M$.

Esta \'{u}ltima afirmaci\'{o}n no es cierta en el caso de variedades con
borde, mejor dicho, en puntos del borde de una variedad. Para determinar la
dimensi\'{o}n de los espacios tangentes a variedades con borde en puntos
del borde ser\'{a} suficiente, por el mismo argumento del p\'{a}rrafo anterior,
determinar la dimensi\'{o}n del tangente al semiespacio $\hemi[d]$ en
un punto del borde $\borde{\hemi[d]}=\{x^{d}=0\}$.

%
\section{Algunas cuentas en coordenadas}
\theoremstyle{plain}

\theoremstyle{remark}
\newtheorem{obsSiLasCartasNoSonCompatibles}{Observaci\'{o}n}[section]

%-------------

Dados un punto $p\in M$ y su imagen $F(p)\in N$ por una transformaci\'{o}n
suave, para estudiar a la funci\'{o}n $F:\,M\rightarrow N$ cerca de $p$,
tomamos cartas $(U,\varphi)$ en $p$ y $(V,\psi)$ en $F(p)$, de manera que
$F(U)\subset V$. El hecho de que $F$ sea suave, se reflejar\'{a} en que su
expresi\'{o}n en coordenadas $\widehat{F}=\psi\circ F\circ\varphi^{-1}:\,%
\varphi(U)\rightarrow\psi(V)$ sea suave. El espacio tangente en $p$ tiene
como base a los vectores $\{\gancho{\varphi^{1}}|_{p},\,\dots,\,%
\gancho{\varphi^{n}}|_{p}\}$ que, en una funci\'{o}n suave
$f:\,U\rightarrow\bb{R}$ toman los valores
\begin{align*}
	\left.\gancho{\varphi^{i}}\right|_{p}f & \,\equiv\,
		\derivada{(f\circ\varphi^{-1})}{x^{i}}(\varphi(p))
		\,\equiv\, \diferencial[\varphi(p)]{\varphi^{-1}}
		\left(\left.\gancho{x^{i}}\right|_{\varphi(p)}\right)\,f
	\text{ .}
\end{align*}
%
Es decir, $\gancho{\varphi^{i}}|_{p}f$ es igual a la derivada parcial
$i$-\'{e}sima respecto de la base can\'{o}nica de $\bb{R}^{n}$ de la
expresi\'{o}n de $f$ en las coordenadas $\varphi=(\lista*{\varphi}{n})$.
Si queremos saber cu\'{a}l es la imagen de estos vectores v\'{\i}a el
diferencial de la transformaci\'{o}n $F$, usamos la igualdad
$\psi^{-1}\circ\widehat{F}=F\circ\varphi^{-1}$. Para simplificar la
notaci\'{o}n, llamamos $\widehat{p}=\varphi(p)$. Entonces
\begin{align*}
	\diferencial[p]{F}\Big(\left.\gancho{\varphi^{i}}\right|_{p}\Big) &
		\,=\,\diferencial[p]{F}\Big(
			\diferencial[\widehat{p}]{\varphi^{-1}}\Big(
			\left.\gancho{x^{i}}\right|_{\widehat{p}}
			\Big)
		\Big)
		\,=\,\diferencial[\widehat{p}]{(F\circ\varphi^{-1})}\Big(
			\left.\gancho{x^{i}}\right|_{\widehat{p}}
			\Big) \\
	& \,=\,\diferencial[\widehat{p}]{(\psi^{-1}\circ F)}\Big(
			\left.\gancho{x^{i}}\right|_{\widehat{p}}
			\Big)
		\,=\,\diferencial[\widehat{F}(\widehat{p})]{\psi^{-1}}\Big(
		\diferencial[\widehat{p}]{\widehat{F}}\Big(
			\left.\gancho{x^{i}}\right|_{\widehat{p}}
			\Big)
		\Big)
	\text{ .}
\end{align*}
%
Si llamamos $\lista*{y}{m}$ a las coordenadas en $\bb{R}^{m}$ y
$\gancho{y^{1}},\,\dots,\,\gancho{y^{m}}$ a las derivadas parciales en las
direcciones can\'{o}nicas y, si $f:\,\psi(V)\rightarrow\bb{R}$ es suave,
\begin{align*}
	\jacobiana[\widehat{p}]{\widehat{F}}\Big(
		\left.\gancho{x^{i}}\right|_{\widehat{p}}\Big)\,f &
		\,\equiv\,\left.\gancho{x^{i}}\right|_{\widehat{p}}
			(f\circ\widehat{F})
	\text{ ,}
\end{align*}
%
que es igual, por la regla de la cadena, a
\begin{align*}
	\derivada{f}{y^{j}}(\widehat{F}(\widehat{p}))\cdot
		\derivada{\widehat{F}^{j}}{x^{i}}(\widehat{p}) &
		\,=\,\derivada{\widehat{F}^{j}}{x^{i}}(\widehat{p})\cdot
		\left.\gancho{y^{j}}\right|_{\widehat{F}(\widehat{p})}\,f
	\text{ .}
\end{align*}
%
Notemos que $\widehat{F(p)}=\widehat{F}(\widehat{p})$. Entonces, en la base
$\{\gancho[\widehat{F(p)}]{y^{1}},\,\dots,\,\gancho[\widehat{F(p)}]{y^{n}}\}$,
\begin{align*}
	\diferencial[\widehat{p}]{\widehat{F}}\Big(
		\gancho[\widehat{p}]{x^{i}}\Big) & \,=\,
		\derivada{\widehat{F}^{j}}{x^{i}}(\widehat{p})\cdot
			\gancho[\widehat{F(p)}]{y^{j}}
	\text{ .}
\end{align*}
%
Volviendo a $F$,
\begin{align*}
	\diferencial[p]{F}\Big(\gancho[p]{\varphi^{i}}\Big) & \,=\,
		\derivada{\widehat{F}^{j}}{x^{i}}(\widehat{p})\cdot
		\diferencial[\widehat{F(p)}]{\psi^{-1}}\Big(
			\gancho[\widehat{F(p)}]{y^{j}}\Big)
		\,=\,\derivada{\widehat{F}^{j}}{x^{i}}(\widehat{p})\cdot
			\gancho[F(p)]{\psi^{j}}
	\text{ .}
\end{align*}
%
Es decir, la expresi\'{o}n del diferencial  $\diferencial[p]{F}$ en las
bases $\{\gancho[p]{\varphi^{i}}\}_{i}$ y $\{\gancho[F(p)]{\psi^{j}}\}_{j}$
es igual a la matriz jacobiana $\jacobiana[\widehat{p}]{\widehat{F}}$.

Un caso importante de todo esto es el de los cambios de carta (cambios de
coordenada, o funciones de transici\'{o}n): sea $p\in M$ un punto arbitrario
de la variedad $M$ y sean $(\widehat{U},\widehat{\varphi})$ y
$(\widetilde{U},\widetilde{\varphi})$ dos cartas en $p$. Primero,
veamos un tema de notaci\'{o}n. Si
\begin{align*}
	\widehat{\varphi} & \,=\,(\lista*{\widehat{\varphi}}{n})
	\quad\text{y}\quad
	\widetilde{\varphi}\,=\,(\lista*{\widetilde{\varphi}}{n})
	\text{ ,}
\end{align*}
%

entonces, dado un punto
$\xi=(\lista*{\xi}{n})\in\widehat{\varphi}(\widehat{U}\cap\widetilde{U})$,
la funci\'{o}n de transici\'{o}n est\'{a} dada expl\'{\i}citamente por
\begin{align*}
	\widetilde{\varphi}\circ\widehat{\varphi}^{-1}(\lista*{\xi}{n}) &
		\,=\,\big(
		\widetilde{\varphi}^{1}(\widehat{\varphi}^{-1}(\xi)),\,\dots,\,
			\widetilde{\varphi}^{n}(\widehat{\varphi}^{-1}(\xi))
			\big)
	\text{ .}
\end{align*}
%
Si pensamos en un punto $\widehat{x}\in\widehat{\varphi}(\widehat{U})$
podemos escribir $\widehat{\varphi}=(\lista*{\widehat{x}}{n})$, pensando
a las funciones coordenadas de $\widehat{\varphi}$, no como funciones en
$\bb{R}$, sino como las coordenadas de un punto en un abierto de un
espacio euclideo. An\'{a}logamente, podemos escribir
$\widetilde{\varphi}=(\lista*{\widetilde{x}}{n})$. La expresi\'{o}n
expl\'{\i}cita para las funciones de transici\'{o}n la podemos reemplazar
por una expresi\'{o}n un poco m\'{a}s clara para el cambio de coordenadas
correspondiente:
\begin{align*}
	\widetilde{\varphi}\circ\widehat{\varphi}^{-1}
		(\lista*{\widehat{x}}{n}) & \,=\,\big(
		\widetilde{x}^{1}(\widehat{x}),\,\dots,\,
		\widetilde{x}^{n}(\widehat{x})
		\big)
	\text{ .}
\end{align*}
%

Volviendo a las cartas, si las mismas son compatibles, la transformaci\'{o}n
$\widetilde{\varphi}\circ\widehat{\varphi}^{-1}:\,%
\widehat{\varphi}(\widehat{U}\cap\widetilde{U})\rightarrow%
\widetilde{\varphi}(\widehat{U}\cap\widetilde{U})$ y su inversa son
diferenciables en sentido usual. La matriz jacobiana de esta
transformaci\'{o}n, o, equivalentemente, su diferencial, est\'{a} dada por:
\begin{align*}
	\diferencial[\widehat{x}]%
		{(\widetilde{\varphi}\circ\widehat{\varphi}^{-1})}
		\Big(\gancho[\widehat{x}]{\widehat{x}^{i}}\Big) & \,=\,
		\derivada{\widetilde{x}^{j}}{\widehat{x}^{i}}(\widehat{x})
		\cdot\gancho[\widetilde{x}(\widehat{x})]{\widetilde{x}^{j}}
	\text{ ,}
\end{align*}
%
o, expl\'{\i}citamente en t\'{e}rminos de $p$,
\begin{align*}
	\diferencial[\widehat{\varphi}(p)]%
		{(\widetilde{\varphi}\circ\widehat{\varphi}^{-1})}
		\Big(\gancho[\widehat{\varphi}(p)]{\widehat{x}^{i}}\Big) &
	\,=\,\derivada%
		{(\widetilde{\varphi}\circ\widehat{\varphi}^{-1})^{j}}%
		{\widehat{x}^{i}}(\widehat{\varphi}(p))
		\cdot\gancho[\widetilde{\varphi}(p)]{\widetilde{x}^{j}}
	\text{ .}
\end{align*}
%
Por regla de la cadena y definici\'{o}n de los $\gancho{\widehat{\varphi}^{i}}$
y los $\gancho{\widetilde{\varphi}^{i}}$,
\begin{align*}
	\gancho[p]{\widehat{\varphi}^{i}} & \,=\,
	\diferencial[\widehat{\varphi}(p)]{\widehat{\varphi}^{-1}}
		\Big(
		\gancho[\widehat{\varphi}(p)]{\widehat{x}^{i}}
		\Big)
	\,=\,\diferencial[\widetilde{\varphi}(p)]{\widetilde{\varphi}^{-1}}
		\Big(
		\diferencial[\widehat{\varphi}(p)]%
			{(\widetilde{\varphi}\circ\widehat{\varphi}^{-1})}
			\Big(
			\gancho[\widehat{\varphi}(p)]{\widehat{x}^{i}}
			\Big)
		\Big) \\
	& \,=\,\derivada%
		{(\widetilde{\varphi}\circ\widehat{\varphi}^{-1})^{j}}%
		{\widehat{x}^{i}}(\widehat{\varphi}(p))
		\cdot
		\diferencial[\widetilde{\varphi}(p)]{\widetilde{\varphi}^{-1}}
			\Big(
			\gancho[\widetilde{\varphi}(p)]{\widetilde{x}^{j}}
			\Big) \\
	& \,=\,\derivada%
		{(\widetilde{\varphi}\circ\widehat{\varphi}^{-1})^{j}}%
		{\widehat{x}^{i}}(\widehat{\varphi}(p))
		\cdot
		\gancho[p]{\widetilde{\varphi}^{j}}
	\text{ ,}
\end{align*}
%
o, expresado de manera m\'{a}s concisa,
\begin{equation}
	\label{eq:cambiodecoordenadas}
	\gancho[p]{\widehat{\varphi}^{i}} \,=\,
		\derivada{\widetilde{x}^{j}}{\widehat{x}^{i}}(p)
		\cdot
		\gancho[p]{\widetilde{\varphi}^{j}}
	\text{ .}
\end{equation}
%
Notemos que estamos pensando en $\derivada{\widetilde{x}^{j}}{\widehat{x}^{i}}$
tanto como una funci\'{o}n en
$\widehat{\varphi}(\widehat{U}\cap\widetilde{U})$, como una funci\'{o}n
en $\widehat{U}\cap\widetilde{U}$.

De ahora en adelante, salvo talvez en algunos casos particulares en donde
sea necesario o conveniente hacer la distinci\'{o}n, denotaremos
$\gancho[p]{x^{i}}$ tanto a la derivada parcial $i$-\'{e}sima en un abierto
de un espacio euclideo en un punto $p$ del abierto, como a la derivaci\'{o}n
--o, mejor dicho, al \emph{campo}-- definida en el dominio de una carta
$\gancho[p]{\varphi^{i}}$ (dependiendo del punto). Es decir, con esta
notaci\'{o}n, la relaci\'{o}n entre las derivaciones provenientes de dos
cartas compatibles con intersecci\'{o}n no vac\'{\i}a, por ejemplo,
quedar\'{\i}a escrita de la siguiente manera:
\begin{align*}
	\gancho[p]{\widehat{x}^{i}} & \,=\,
		\derivada{\widetilde{x}^{j}}{\widehat{x}^{i}}(p)\cdot
		\gancho[p]{\widetilde{x}^{j}}
	\text{ .}
\end{align*}
%
Notemos, adem\'{a}s, que la expresi\'{o}n
$\derivada{\widetilde{x}^{j}}{\widehat{x}^{i}}(p)$ para cada $j$ e $i$, es,
en realidad, una funci\'{o}n de $p\in\widehat{U}\cap\widetilde{U}$,
estando definida en toda la intersecci\'{o}n. Como funci\'{o}n
$\widehat{U}\cap\widetilde{U}\rightarrow\bb{R}$, es una funci\'{o}n
continua y, m\'{a}s aun, suave.

\begin{obsSiLasCartasNoSonCompatibles}\label{obs:sinosoncompatibles}
	?`Qu\'{e} pasa si no asumimos que $\widehat{\varphi}$ y
	$\widetilde{\varphi}$ son cartas compatibles? De la misma manera
	que como se hizo antes, podemos, localmente, con cada carta,
	definir una estructura diferencial \emph{local}, \'{u}nicamente
	en $\widehat{U}$ y en $\widetilde{U}$, usando las cartas
	$(\widehat{U},\widehat{\varphi})$ y
	$(\widetilde{U},\widetilde{\varphi})$, respectivamente. En
	$\widehat{U}\cap\widetilde{U}$, hay, pues, dos estructuras posibles.
	Si $p$ pertenece a la intersecci\'{o}n, hay dos espacios
	tangentes, $T_{p}\widehat{U}$ y $T_{p}\widetilde{U}$, cada uno
	con su base can\'{o}nica
	$\{\gancho{\widehat{\varphi}^{i}}\}_{i}$ y
	$\{\gancho{\widetilde{\varphi}^{j}}\}_{j}$. Si las cartas no
	son compactibles, no hay una manera natural de relacionar estos
	espacios tangentes. Es decir, no hay, aunque no sean regulares,
	funciones $\alpha^{j}_{i}:\,%
	\widehat{U}\cap\widetilde{U}\rightarrow\bb{R}$ tales que
	\begin{align*}
		\gancho{\widehat{\varphi}^{i}} & \,=\,\alpha^{j}_{i}\cdot
			\gancho{\widetilde{\varphi}^{j}}
	\end{align*}
	%
	bajo alguna identificaci\'{o}n de los tangentes --justamente, porque
	no hay, en general, una identificaci\'{o}n.
\end{obsSiLasCartasNoSonCompatibles}

%
\section{Curvas en variedades y relaci\'{o}n con el tangente}
\theoremstyle{plain}
\newtheorem{propoTodoTangenteEsVelocidad}{Proposici\'{o}n}[section]

\theoremstyle{remark}

%-------------

Dada una variedad $M$, una \emph{curva en $M$} (o \emph{curva parametrizada})
es una funci\'{o}n $\gamma:\,J\rightarrow M$ definida en un intervalo
$J\subset\bb{R}$. El intervalo $J$ tiene una estructura de variedad
diferencial (posiblemente con borde) y podemos preguntarnos si $\gamma$ es
regular. Si $\gamma$ es de clase $C^{k}$ ($k\geq 1$) o suave, dado
$t_{0}\in J$, llamamos \emph{velocidad de la curva $\gamma$} en $t=t_{0}$ a
su derivada en $t_{0}$, es decir, al elemento de $T_{\gamma(t_{0})}M$ dado por
\begin{align*}
	\gamma'(t_{0}) & \,\equiv\,\dot{\gamma}(t_{0}) \,\equiv\,
		\diferencial[t_{0}]{\gamma}\Big(\gancho[t_{0}]{t}\Big)
	\text{ ,}
\end{align*}
%
donde $\diferencial[t_{0}]{\gamma}$ es el diferencial de $\gamma$, vista
como funci\'{o}n entre las variedades $J$ y $M$, en el punto $t_{0}\in J$
y $\gancho[t_{0}]{t}$ es el elemento de la base de $T_{t_{0}}J$.

Dado $p\in M$, decimos que $\gamma$ es una curva con origen en $p$, si
$0\in J$ y $\gamma(0)=p$, o si, m\'{a}s en general, fijado $t_{0}\in J$,
$\gamma(t_{0})=p$. Dada una carta $(U,\varphi)$ para $M$ en $p$, sabemos que,
al ser $\gamma$ suave, su expresi\'{o}n en coordenadas, $\varphi\circ\gamma=%
(\lista*{\gamma}{n})$ es suave. Si $f:\,U\rightarrow\bb{R}$ es una
funci\'{o}n suave,
\begin{align*}
	\dot{\gamma}(t_{0})\,f & \,\equiv\,\diferencial[t_{0}]{\gamma}
		\Big(\gancho[t_{0}]{t}\Big)\,f \,=\,
		(f\circ\gamma)'(t_{0})
\end{align*}
%
y, en coordenadas,
\begin{align*}
	\dot{\gamma}(t_{0})\,f & \,=\,
		\diferencial[\varphi(\gamma(t_{0}))]{\varphi^{-1}}\Big(
		\diferencial[t_{0}]{(\varphi\circ\gamma)}\Big(
		\gancho[t_{0}]{t}\Big)\Big)\,f \\
	& \,=\,\diferencial[\varphi(\gamma(t_{0}))]{\varphi^{-1}}\Big(
		\derivada{(\varphi\circ\gamma)^{k}}{t}(t_{0})\cdot
		\gancho[\varphi(\gamma(t_{0}))]{x^{k}}\Big)\,f \\
	& \,=\,\derivada{\gamma^{k}}{t}(t_{0})\cdot
		\diferencial[\varphi(\gamma(t_{0}))]{\varphi^{-1}}
		\Big(\gancho[\varphi(\gamma(t_{0}))]{x^{k}}\Big)\,f \\
	& \,=\,\derivada{\gamma^{k}}{t}(t_{0})\cdot
		\gancho[\varphi(\gamma(t_{0}))]{\varphi^{k}}\,f
	\text{ .}
\end{align*}
%
En definitiva,
\begin{equation}
	\label{eq:velocidaddeunacurva}
	\dot{\gamma}(t_{0}) \,=\,\derivada{\gamma^{k}}{t}(t_{0})\cdot
		\gancho[\varphi(\gamma(t_{0}))]{\varphi^{k}}
\end{equation}
%
donde $\gamma^{k}=(\varphi\circ\gamma)^{k}=\varphi^{k}\circ\gamma$.

\begin{propoTodoTangenteEsVelocidad}\label{thm:todotangenteesvelocidad}
	Sea $M$ una variedad diferencial y sea $p\in M$. Si $v\in T_{p}M$
	es un vector tangente en $p$, existe una curva
	$\gamma:\,J\rightarrow M$ con origen en $p$ tal que
	$\dot{\gamma}(t_{0})=v$.
\end{propoTodoTangenteEsVelocidad}

\begin{proof}
	Sea $(U,\varphi)$ una carta en $p$. Sea $\widehat{v}\in%
	T_{\widehat{p}}\varphi(U)\simeq\bb{R}^{n}$ el vector dado por
	$\widehat{v}=(\lista*{v}{n})=v^{i}\gancho[\widehat{p}]{x^{i}}$,
	donde $\lista*{v}{n}$ son las coordenadas de $v$ en la base
	$\gancho[p]{\varphi^{i}}\equiv\gancho[p]{x^{i}}$. Es decir, si
	$v=v^{i}\gancho[p]{x^{i}}$, definimos
	$\widehat{v}=v^{i}\gancho[\widehat{p}]{x^{i}}$.

	Supongamos que $p\in\interior{M}$. Sea $\widehat{\gamma}:\,%
	(-\epsilon,\epsilon)\rightarrow\varphi(U)$ la curva dada por
	$\widehat{\gamma}(t)=(v^{1}t,\,\dots,\,v^{n}t)+\widehat{p}$ y
	sea $\gamma=\varphi^{-1}\circ\widehat{\gamma}$. En $t=0$,
	$\gamma(0)=p$ y
	\begin{align*}
		\dot{\gamma}(0) & \,=\,v^{i}\gancho[p]{x^{i}} \,=\,v
		\text{ .}
	\end{align*}
	%
	Si $U$ es una carta del borde de $M$ y $p\in\borde[M]$, no es cierto
	que $\widehat{\gamma}(t)\in\hemi[n]$ para todo
	$t\in(-\epsilon,\epsilon)$, excepto en el caso $v^{n}=0$. En otro
	caso, si $v^{n}>0$, entonces, la restricci\'{o}n de
	$\widehat{\gamma}$ a $[0,\epsilon)$ s\'{\i} es una curva
	contenida en $\varphi(U)$ y si $v^{n}<0$, entonces su
	restricci\'{o}n a $(-\epsilon,0]$ lo es. Definiendo $\gamma$
	como la composici\'{o}n de $\varphi^{-1}$ con la restricci\'{o}n
	correspondiente, se obtiene una curva en $M$ con origen en $p$
	y velocidad $\dot{\gamma}(0)=v$.
\end{proof}

Las curvas suaves en una variedad dan una idea local de la estructura de la
misma. En este sentido, las curvas suaves tienen dos aplicaciones
principales: por un lado, son \'{u}tiles para el c\'{a}lculo de diferenciales
de transformaciones suaves, como veremos a continuaci\'{o}n; por otro
lado, aunque todav\'{\i}a no haya sido definida esta noci\'{o}n, dada
una subvariedad $S$ de una variedad ambiente $M$, la interpretaci\'{o}n
de los vectores tangentes como velocidades o clases de equivalencia de
curvas suaves nos permitir\'{a} identificar el espacio tangente a $S$ en
un punto $p\in S$ con un subespacio vectorial del espacio tangente a $M$
en $p$ (ver el cap\'{\i}tulo relacionado con subvariedades).

Si queremos saber c\'{o}mo afecta una transformaci\'{o}n suave
$F:\,M\rightarrow N$ a una curva $\gamma:\,J\rightarrow M$, s\'{o}lo
debemos componer: $F\circ\gamma:\,J\rightarrow N$ es una curva suave y
\begin{equation}
	\label{eq:diferencialporcurvas}
	\begin{aligned}
		(F\circ\gamma)'(t_{0}) & \,\equiv\,
			\diferencial[t_{0}]{(F\circ\gamma)}
				\Big(\gancho[t_{0}]{t}\Big) \,=\,
			\diferencial[\gamma(t_{0})]{F}\cdot
			\diferencial[t_{0}]{\gamma}
				\Big(\gancho[t_{0}]{t}\Big) \\
		& \,=\,\diferencial[\gamma(t_{0})]{F}
			\big(\dot{\gamma}(t_{0})\big)
		\text{ .}
	\end{aligned}
\end{equation}
%
Este resultado se puede usar de dos maneras, o con dos prop\'{o}sitos
distintos: o bien para determinar la velocidad de la curva
$F\circ\gamma$, conociendo $\dot{\gamma}(t_{0})$ y
$\diferencial[\gamma(t_{0})]{F}$, o bien para calcular el diferencial
$\diferencial[p]{F}$ estudiando el efecto que tiene sobre las curvas
$\gamma$ con origen en $p$. En esta segunda situaci\'{o}n, dado un
vector tangente $v_{p}\in T_{p}M$, si tomamos una curva suave
$\gamma:\,J\rightarrow M$ con origen en $p$ y velocidad $v_{p}$, entonces
$\diferencial[p]{F}(v_{p})=(F\circ\gamma)'(t_{0})$.

%
\section{El fibrado tangente}
\theoremstyle{plain}
\newtheorem{propoFibradoTangente}{Proposici\'{o}n}[section]
\newtheorem{propoElDiferencialFuntorial}[propoFibradoTangente]{Proposici\'{o}n}

\theoremstyle{remark}

%-------------


El \emph{fibrado tangente} es la construcci\'{o}n que permite estudiar de
manera coherente los espacios tangentes a una variedad. Si $M$ es una
variedad diferencial, como conjunto, el fibrado tangente a/de $M$ es la
uni\'{o}n disjunta de todos los espacios tangentes:
\begin{align*}
	TM & \,=\,\sqcup_{p\in M}\,T_{p}M
	\text{ .}
\end{align*}
%
A los elementos de $TM$ los denotamos $(p,v)$, donde $p\in M$ y $v\in T_{p}M$.
El fibrado tangente tiene asociada una proyecci\'{o}n
$\pi:\,TM\rightarrow M$ que a un par $(p,v)$ le asigna $\pi(p,v)=p$. La
coherencia a la que se hizo alusi\'{o}n se refiere a la posibilidad de dar a
$TM$ una estructura natural de variedad diferencial.

\begin{propoFibradoTangente}\label{thm:fibradotangente}
	Si $M$ es una variedad diferencial de dimensi\'{o}n $n$, el fibrado
	tangente $TM$ tiene una topolog\'{\i}a y una estructura diferencial
	naturales de manera que, con ellas, $TM$ sea una variedad diferencial
	de dimensi\'{o}n $2\cdot n$ y respecto a las cuales
	$\pi:\,TM\rightarrow M$ sea diferenciable.
\end{propoFibradoTangente}

\begin{proof}
	Para demostrar esta proposici\'{o}n, haremos uso del lema de las
	cartas (lema \ref{thm:delascartas}). Sea $(U,\varphi)$ una carta
	para $M$. Su preimagen v\'{\i}a $\pi$ es el conjunto
	$\pi^{-1}(U)$ de pares $(p,v_{p})$ con $p\in U$ y $v_{p}\in T_{p}M$.
	Es decir, $\pi^{-1}(U)$ consiste en los vectores tangentes a $M$ en
	puntos de $U$. En $\pi^{-1}(U)$, definimos una aplicaci\'{o}n
	$\widetilde{\varphi}:\,\pi^{-1}(U)\rightarrow\bb{R}^{2n}$ por
	\begin{align*}
		\widetilde{\varphi}(p,v_{p}) & \,=\,
			(\lista*{x}{n},\,\lista*{v}{n})
		\text{ ,}
	\end{align*}
	%
	donde $\varphi(p)=(\lista*{x}{n})\in\varphi(U)\subset\bb{R}^{n}$ y
	$v_{p}=v^{i}\gancho[p]{x^{i}}\in T_{p}M$ es la escritura de $v_{p}$
	en la base can\'{o}nica $\gancho[p]{x^{1}},\,\dots,\,\gancho[p]{x^{n}}$
	(o, expl\'{\i}citamente, $\gancho[p]{\varphi^{i}}$) del espacio
	tangente a $M$ en $p$. La imagen $\widetilde{\varphi}(\pi^{-1}(U))=%
	\varphi(U)\times\bb{R}^{n}$ es un abierto de
	$\bb{R}^{n}\times\bb{R}^{n}$ en biyecci\'{o}n con $\pi^{-1}(U)$
	v\'{\i}a $\widetilde{\varphi}$: su inversa est\'{a} dada por
	\begin{align*}
		\widetilde{\varphi}^{-1}(\lista*{x}{n},\,\lista*{v}{n}) &
		\,=\,\Big(
		\varphi^{-1}(x),\,v^{i}\gancho[\varphi^{-1}(x)]{x^{i}}\Big)
		\text{ .}
	\end{align*}
	%
	Dadas dos cartas $(U,\varphi)$ y $(V,\psi)$ con intersecci\'{o}n
	no vac\'{\i}a, se cumple que $\pi^{-1}(U)\cap\pi^{-1}(V)\not =%
	\varnothing$ y
	\begin{align*}
		\widetilde{\varphi}(\pi^{-1}(U)\cap\pi^{-1}(V)) & \,=\,
			\varphi(U\cap V)\times\bb{R}^{n}\quad\text{y} \\
		\widetilde{\psi}(\pi^{-1}(U)\cap\pi^{-1}(V)) & \,=\,
			\psi(U\cap V)\times\bb{R}^{n}
		\text{ .}
	\end{align*}
	%
	Estos conjuntos son abiertos en $\bb{R}^{2n}$ y las funciones de
	transici\'{o}n $\widetilde{\varphi}\circ\widetilde{\psi}^{-1}$ y
	$\widetilde{\psi}\circ\widetilde{\varphi}^{-1}$ son suaves, pues:
	\begin{align*}
		\widetilde{\psi}\circ\widetilde{\varphi}^{-1}
			(\lista*{x}{n},\,\lista*{v}{n}) & \,=\,
			\Big(
			\widetilde{x}^{1}(x),\,\dots,\,\widetilde{x}^{n}(x),\,
			\derivada{\widetilde{x}^{1}}{x^{i}}(x)v^{i},\,\dots,\,
			\derivada{\widetilde{x}^{n}}{x^{i}}(x)v^{i}
			\Big)
		\text{ .}
	\end{align*}
	%
	(Notemos que, si $M$ es $C^{1}$, entonces $TM$ no llegar\'{a} a ser
	una variedad $C^{1}$, pues las funciones
	$\derivada{\widetilde{x}^{j}}{x^{i}}$ son meramente continuas).
	Por otro lado, si $\{U_{i}\}_{i}$ es un cubrimiento (numerable) de $M$
	por cartas compatibles, entonces $\{\pi^{-1}(U_{i})\}_{i}$ ser\'{a}
	un cubrimiento (numerable) de $TM$. Finalmente, para verificar la
	hip\'{o}tesis de separabilidad ($T_{2}$) del lema, si $(p,v_{p})$ y
	$(q,w_{q})$ son dos puntos distintos en $TM$ con $p\not =q$,
	entonces, eligiendo cartas con dominios disjuntos que separen a
	$p$ y a $q$ obtenemos, tomando preimagen por $\pi$, conjuntos
	disjuntos en $TM$ de la forma $\pi^{-1}(U)$, donde $U$ es el dominio
	de una carta, que separan a $(p,v_{p})$ y a $(q,w_{q})$. Si, por
	otro lado, $p=q$, entonces, tomando cualquier carta $U$ que
	contenga a $p$, el conjunto $pi^{-1}(U)$ contiene a ambos puntos
	del fibrado. Con esto se terminan de verificar las hip\'{o}tesis
	del lema \ref{thm:delascartas}.

	Para ver que $\pi:\,TM\rightarrow M$ es suave, basta tomar una
	carta $(U,\varphi)$ para $M$ y la carta correspondiente
	$(\pi^{-1}(U),\widetilde{\varphi})$ para $TM$. Con respecto a estas
	coordenadas,
	\begin{align*}
		\varphi\circ\pi\widetilde{\varphi}^{-1}
			(\lista*{x}{n},\,\lista*{v}{n}) & \,=\,(\lista*{x}{n})
		\text{ ,}
	\end{align*}
	%
	es decir, $\widehat{\pi}$ es, tambi\'{e}n, la proyecci\'{o}n en las
	primeras coordenadas. En el caso de variedades con borde, podemos
	tomar la expresi\'{o}n $(v,p)$ para que, al tomar coordenadas,
	$\widetilde{\varphi}(v_{p},p)=(\lista*{v}{n},\,\lista*{x}{n})\in %
	\bb{R}^{n}\times\hemi[n]=\hemi[2n]$.
\end{proof}

Una funci\'{o}n suave $F:\,M\rightarrow N$ determina una funci\'{o}n suave a
nivel de los fibrados tangentes. Esta funci\'{o}n es el \emph{diferencial}
(o \emph{diferencial global} o \emph{diferencial total}) de $F$,
$\diferencial{F}:\,TM\rightarrow TN$, dado por
\begin{align*}
	\diferencial{F}(v_{p},p) & \,=\,\diferencial[p]{F}(v_{p})\,\in\,
		T_{F(p)}N\quad\text{, o bien} \\
	\diferencial{F}(v_{p},p) & \,=\,(\diferencial[p]{F}(v_{p}),F(p))
\end{align*}
%
En coordenadas,
\begin{align*}
	\widehat{\diferencial{F}}(\lista*{v}{n},\,\lista*{x}{n}) & \,=\,
		\Big(
		\derivada{\widehat{F}^{1}}{x^{i}}(x)v^{i},\,\dots,\,
		\derivada{\widehat{F}^{m}}{x^{i}}(x)v^{i},\,
		\widehat{F}^{1}(x),\,\dots,\,\widehat{F}^{m}(x)
		\Big)
		\text{ .}
\end{align*}
%
Al igual que en el caso de la diferencial en un punto, $\diferencial{F}$
posee las siguientes propiedades funtoriales:

\begin{propoElDiferencialFuntorial}\label{thm:eldiferencialfuntorial}
	\emph{(a)} Si $F:\,M\rightarrow N$ y $G:\,N\rightarrow\tilde{N}$ son
	transformaciones suaves, la composici\'{o}n
	$G\circ F:\,M\rightarrow\tilde{N}$ es suave y
	$\diferencial{(G\circ F)}=\diferencial{G}\cdots\diferencial{F}$;
	\emph{(b)} la identidad $\id[M]:\,M\rightarrow M$ es suave y su
	diferencial est\'{a} dado por $\diferencial{\id[M]}=\id[TM]$;
	\emph{(c)} si $F:\,M\rightarrow N$ es un difeomorfismo, entonces
	$\diferencial{F}:\,TM\rightarrow TN$ es un difeomorfismo y
	$(\diferencial{F})^{-1}=\diferencial{(F^{-1})}$.
\end{propoElDiferencialFuntorial}

%

%--------

\chapter{El teorema del rango constante}
\section{Los teoremas de la funci\'{o}n inversa y de la funci\'{o}n impl\'{\i}cita}
\theoremstyle{plain}
\newtheorem{teoFunInversa}{Teorema}[section]
\newtheorem{coroFunInversa}[teoFunInversa]{Corolario}
\newtheorem{teoFunImplicita}[teoFunInversa]{Teorema}

\theoremstyle{remark}

%-------------

En esta secci\'{o}n recordamos los teoremas de la funci\'{o}n inversa y
de la funci\'{o}n impl{\i}cita para funciones definidas en abiertos
de un espacio euclideo.

\begin{teoFunInversa}[de la funci\'{o}n inversa]\label{thm:funinversa}
	Sean $U,V\subset\bb{R}^{d}$ abiertos. Sea $F:\,U\rightarrow V$ una
	funci\'{o}n de clase $C^{k}(U,V)$ ($k\geq 1$). Si la matriz
	jacobiana de $F$ en un punto $p\in U$,
	\begin{align*}
		\jacobiana[p]{F} & \,\equiv\,\jacobiana{F}(p)\,\equiv\,
			\left.\jacobiana{F}\right|_{p}
		\,=\,	\begin{bmatrix}
				\derivada{F^{1}}{x^{1}}(p) & \cdots &
					\derivada{F^{1}}{x^{d}}(p) \\
				\vdots & \ddots & \vdots \\
				\derivada{F^{d}}{x^{1}}(p) & \cdots &
					\derivada{F^{d}}{x^{d}}(p)
			\end{bmatrix}
		\,=\,	\left[
			\begin{array}{ccc}
				& \jacobiana[p]{F^{1}} & \\
				\hline
				& \vdots & \\
				\hline
				& \jacobiana[p]{F^{d}} &
			\end{array}
			\right]
		\text{ ,}
	\end{align*}
	%
	es invertible, existen entornos conexos $U_{0}$ y $V_{0}$ de
	$p$ y de $F(p)$, respectivamente, tales que
	$F_{0}\equiv F|_{U_{0}}^{V_{0}}:\,U_{0}\rightarrow V_{0}$
	es un difeomorfismo.
\end{teoFunInversa}

\begin{proof}
	Sin p\'{e}rdida de generalidad, $p=0$ y $F(0)=0$, componiendo con
	las traslaciones $x\mapsto x+p$ e $y\mapsto y-F(p)$. Tambi\'{e}n
	podemos asumir que $\jacobiana{F}(0)=\id[\bb{R}^{d}]$, componiendo
	con el difeomorfismo $y\mapsto \jacobiana{F}(0)^{-1}\cdot y$
	(esto s\'{o}lo es posible porque el tangente de un abierto de
	$\bb{R}^{d}$ se identifica con $\bb{R}^{d}$, el espacio ambiente,
	de manera natural). M\'{a}s aun,
	$\det\circ\jacobiana{F}:\,U\rightarrow\bb{R}$ es continua y no
	nula en $p=0$. Se puede suponer que $\jacobiana{F}$ es invertible
	en $U$, achicando $U$, de ser necesario.

	Sea $H(x)=x-F(x)$. Esta funci\'{o}n es diferenciable y
	$\jacobiana{H}(0)=0$. Por continuidad de $\jacobiana{H}$, para cierto
	$\delta>0$, vale que $\|\jacobiana{H}(x)\|\leq 1/2$ en la bola
	cerrada $\clos{\bola{\delta}{0}}$. Entonces
	\begin{equation}
		\label{eq:cotafuninversa}
		\begin{aligned}
		|H(x')-H(x)| & \,\leq\,\frac{1}{2}|x'-x|\quad\text{y} \\
		|x'-x| & \,\leq\,2|F(x')-F(x)|\text{ .}
		\end{aligned}
	\end{equation}
	%
	Esto muestra que $F$ es inyectiva en $\clos{\bola{\delta}{0}}$.

	Sea ahora $y$ un elemento fijo perteneciente a
	$\bola{\delta/2}{0}\subset V$. Sea $G(x)=y+H(x)$ definida en $U$.
	Si $|x|\leq\delta$, entonces, como $H(0)=0$,
	\begin{align*}
		|G(x)| & \,\leq\,|y|+|H(x)|\,<\,\frac{\delta}{2}+\frac{1}{2}|x|
			\,\leq\,\delta
		\text{ .}
	\end{align*}
	%
	En particular,
	$G(\clos{\bola{\delta}{0}})\subset\clos{\bola{\delta}{0}}$ y,
	como $|G(x')-G(x)|\leq 1/2|x'-x|$, se deduce que $G$ es una
	contracci\'{o}n en un espacio m\'{e}trico completo. Por el teorema
	del punto fijo, existe un (\'{u}inico) punto
	$x\in\clos{\bola{\delta}{0}}$ tal que $G(x)=x$. Pero esto significa
	que $F(x)=y$. Adem\'{a}s, para este punto, $|x|=|G(x)|<\delta$.
	Como la desigualdad es estricta, se deduce que todo punto
	$y\in\bola{\delta/2}{0}$ es imagen de un \'{u}nico punto
	$x\in\bola{\delta}{0}$ v\'{\i}a $F$.

	Sea $V_{0}=\bola{\delta/2}{0}\subset V$ y sea
	$U_{0}=\bola{\delta}\cap F^{-1}(V_{0})$. Entonces
	$F_{0}=F|:\,U_{0}\rightarrow V_{0}$ es biyectiva. La inversa
	$F_{0}^{-1}$ existe y es continua por \eqref{eq:cotafuninversa}.
	Resta ver que $F_{0}^{-1}$ es suave.

	Sean $y_{0}\in V_{0}$, $x_{0}\in U_{0}$ tales que $F_{0}(x_{0})=y_{0}$
	y sea $L$ la transformaci\'{o}n lineal dada por
	$\jacobiana{F_{0}}(x_{0})$. Dados $y\in V_{0}\setmin\{y_{0}\}$ y
	el punto correspondiente $x\in U_{0}$ tal que $F(x)=y$, el
	cociente incremental de $F_{0}^{-1}$ en $y_{0}$ verifica:
	\begin{align*}
		\frac{F_{0}^{-1}(y)-F_{0}^{-1}(y_{0}) -L^{-1}(y-y_{0})}%
			{|y-y_{0}|} & \,=\, \\
		\frac{|x-x_{0}|}{|y-y_{0}|} & \cdot
			L^{-1}\left(
			-\frac{F_{0}(x)-F_{0}(x_{0})-L(x-x_{0})}{|x-x_{0}|}
			\right)
		\text{ ,}
	\end{align*}
	%
	por la linealidad de $L^{-1}$. Como $L^{-1}$ es lineal entre espacios
	de dimensi\'{o}n finita y, de nuevo, por \eqref{eq:cotafuninversa},
	$\|L^{-1}\|<\infty$ y
	\begin{align*}
		\frac{F_{0}^{-1}(y)-F_{0}^{-1}(y_{0})-L^{-1}(y-y_{0})}%
			{|y-y_{0}|} & \,\leq\, \\
		\frac{1}{2}\|L^{-1}\| & \cdot
			\left|
			-\frac{F_{0}(x)-F_{0}(x_{0})-L(x-x_{0})}{|x-x_{0}|}
			\right|
		\text{ ,}
	\end{align*}
	%
	que tiende a cero, si $y\to y_{0}$, pues, en ese caso, $x\to x_{0}$.
	As\'{\i}, $F_{0}^{-1}$ es diferenciable y su diferenciale es igual a
	\begin{align*}
		\jacobiana{(F_{0}^{-1})}(y_{0}) & \,=\,L^{-1} \,=\,
			\big[\jacobiana{F_{0}}(x_{0})\big]^{-1} \,=\,
			\big[\jacobiana{F_{0}}(F_{0}^{-1}(y_{0}))\big]^{-1}
		\text{ .}
	\end{align*}
	%
	Esto es cierto para todo punto $y_{0}\in V_{0}$. Por otro lado, la
	funci\'{o}n $y\mapsto\jacobiana{(F_{0}^{-1})}(y)$ se descompone de
	la siguiente manera:
	\begin{align*}
		y & \,\mapsto\,F_{0}^{-1}(y)\,\mapsto\,
			(\jacobiana{F_{0}})(F_{0}^{-1}(y)) \,\mapsto\,
			\big[(\jacobiana{F_{0}})(F_{0}^{-1}(y))\big]^{-1}
		\text{ ,}
	\end{align*}
	%
	como composici\'{o}n de funciones continuas (porque $F_{0}$ es
	$C^{1}$, $F_{0}^{-1}$ es continua y a inversi\'{o}n de matrices es
	continua (suave) en los coeficientes (por Cramer)). Entonces las
	derivadas parciales de $F_{0}^{-1}$, las componenetes de
	$\jacobiana{F_{0}^{-1}}$, son continuas y $F_{0}^{-1}$ es de clase
	$C^{1}$. En general, si $F_{0}^{-1}$ es de clase $C^{t}(V_{0},U_{0})$
	y $F_{0}$ es de clase $C^{t+1}(U_{0},V_{0})$, el argumento anterior
	muestra que $F_{0}^{-1}$ es $C^{t+1}(V_{0},U_{0})$, que sus
	derivadas parciales de orden $t+1$ existen y que son continuas.
	Inductivamente, $F_{0}^{-1}$ es tan regular como $F_{0}$. En
	particular, si $F_{0}$ es $C^{\infty}$, $F_{0}^{-1}$ tambi\'{e}n
	lo es.
\end{proof}

\begin{coroFunInversa}\label{thm:coroinversa}
	Sea $U\subset\bb{R}^{d}$ un abierto y sea $F:\,U\rightarrow\bb{R}^{d}$
	una funci\'{o}n de clase $C^{k}$ ($k\geq 1$) o suave. Si
	$\det(\jacobiana{F})\not =0$ en $U$, entonces \emph{(a)} $F$ es
	abierta y \emph{(b)} si $F$ es inyectiva, entonces
	$F:\,U\rightarrow F(U)$ es invertible con inversa $C^{k}$ (o suave).
\end{coroFunInversa}

\begin{proof}
	Sea $p\in U$. Por hip\'{o}tesis, $\jacobiana{F}(p)$ es invertible.
	Por el Teorema de la funci\'{o}n inversa \ref{thm:funinversa},
	existen abiertos $U_{p}\subset U$ y $V_{p}\subset\bb{R}^{d}$ tales
	que $p\in U_{p}$, $F(p)\in V_{p}$ y la restricci\'{o}n
	$F|:\,U_{p}\rightarrow V_{p}$ es un difeomorfismo. El subconjunto
	$V_{p}$ es abierto y est\'{a} contenido en $F(U)$. Por lo tanto,
	si ahora tomamos un punto arbitrario $q\in F(U)$ y un punto $p\in U$
	talque $F(p)=q$, el abierto correspondiente $V_{p}$ es un entorno
	de $q$ contenido en $F(U)$. En definitiva, $F(U)$ es un subespacio
	abierto de la imagen $\bb{R}^{d}$. Si $U_{0}\subset U$ es un
	subconjunto abierto, reemplazando $U$ por $U_{0}$ en el argumento
	anterior, se ve que $F(U_{0})$ es abierto en la imagen. Entonces
	$F$ es una funci\'{o}n abierta.

	En cuanto al \'{\i}tem \emph{(b)}, si $F$ es inyectiva, la
	correstricci\'{o}n de $F$ a $F(U)$ es invertible. En un punto
	$p\in U$, si $q=F(p)$ y $U_{p}$ y $V_{p}$ son los abiertos
	difeomorfos dados por el teorema \ref{thm:funinversa}, la
	inversa de $F$ restringida a $V_{p}$, $F^{-1}|_{V_{p}}$, conincide
	con la inversa de la restricci\'{o}n $F|:\,U_{p}\rightarrow V_{p}$.
	Pero esta funci\'{o}n es $C^{k}$, invertible y con inversa $C^{k}$.
	As\'{\i}, como $F$ es globalmente invertible y esta inversa coincide
	localmente con funciones $C^{k}$, debe ser $C^{k}$ tambi\'{e}n.
\end{proof}

Pasamos ahora al Teorema de la funci\'{o}n impl\'{\i}cita.

\begin{teoFunImplicita}[de la funci\'{o}n impl\'{\i}cita]%
	\label{thm:funimplicita}
	Sea $U\subset\bb{R}^{d}\times\bb{R}^{l}$ un abierto. Sea
	$\Phi:\,U\rightarrow\bb{R}^{l}$ una funci\'{o}n suave (o de clase
	$C^{k}$) y sean $c\in\bb{R}^{l}$ y $(a,b)\in U$ un punto en la
	preimagen $\Phi(a,b)=c$. Si la transformaci\'{o}n determinada por
	la matriz
	\begin{align*}
		L & \,=\,
			\begin{bmatrix}
				\derivada{\Phi^{1}}{y^{1}} & \cdots &
					\derivada{\Phi^{1}}{y^{l}} \\
				& \vdots & \\
				\derivada{\Phi^{l}}{y^{1}} & \cdots &
					\derivada{\Phi^{l}}{y^{l}}
			\end{bmatrix}
		\,=\,\begin{bmatrix}\derivada{\Phi^{i}}{y^{j}}\end{bmatrix}
	\end{align*}
	%
	de derivadas parciales respecto de las variables $\lista*{y}{l}$ es
	no singular en $(a,b)$, entonces existen entornos $V_{0}$ de $a$
	y $W_{0}$ de $b$ y una funci\'{o}n suave $F:\,V_{0}\rightarrow W_{0}$
	tales que 
	\begin{align*}
		\Phi^{-1}(c)\cap\big(V_{0}\times W_{0}\big) & \,=\,\Graf{F}
			\,=\,\left\lbrace (x,F(x))\,:\,x\in V_{0}\right\rbrace
		\text{ .}
	\end{align*}
	%
	Dicho de otra manera, en $V_{0}\times W_{0}$, un punto $(x,y)$
	verifica $\Phi(x,y)=c$, si y s\'{o}lo si $y=F(x)$.
\end{teoFunImplicita}

\begin{proof}
	Sea define $\Psi:\,U\rightarrow\bb{R}^{d}\times\bb{R}^{l}$ por
	$\Psi(x,y)=(x,\Phi(x,y))$. La matriz jacobiana de $\Psi$ es igual a
	\begin{align*}
			%
		\jacobiana{\Psi} & \,=\,
		\sbox0{$
		\begin{matrix}
			\derivada{\Psi^{1}}{x^{1}} & \cdots &
				\derivada{\Psi^{1}}{x^{d}} \\
			& \vdots & \\
			\derivada{\Psi^{d}}{x^{1}} & \cdots &
				\derivada{\Psi^{d}}{x^{d}}
		\end{matrix}
		$}
		\sbox1{$
		\begin{matrix}
			\derivada{\Psi^{1}}{y^{1}} & \cdots &
				\derivada{\Psi^{1}}{y^{l}} \\
			& \vdots & \\
			\derivada{\Psi^{d}}{y^{1}} & \cdots &
				\derivada{\Psi^{d}}{y^{l}}
		\end{matrix}
		$}
		\sbox2{$
		\begin{matrix}
			\derivada{\Psi^{d+1}}{x^{1}} & \cdots &
				\derivada{\Psi^{d+1}}{x^{d}} \\
			& \vdots & \\
			\derivada{\Psi^{d+l}}{x^{1}} & \cdots &
				\derivada{\Psi^{d+l}}{x^{d}}
		\end{matrix}
		$}
		\sbox3{$
		\begin{matrix}
			\derivada{\Psi^{d+1}}{y^{1}} & \cdots &
				\derivada{\Psi^{d+1}}{y^{l}} \\
			& \vdots & \\
			\derivada{\Psi^{d+l}}{y^{1}} & \cdots &
				\derivada{\Psi^{d+l}}{y^{l}}
		\end{matrix}
		$}
			\left[
			\begin{array}{c|c}
				\usebox{0} & \usebox{1} \\
			\hline
				\usebox{2} & \usebox{3}
			\end{array}
			\right]
		\,=\,
		\sbox4{$\derivada{\Phi^{i}}{x^{j}}$}
		\sbox5{$\derivada{\Phi^{i}}{y^{j}}$}
		\left[
		\begin{array}{c|c}
			\vphantom{\usebox{4}}%
				\makebox[\wd4]{$\id[\bb{R}^{d}]$} & \\
			\hline
			\usebox{4} & \usebox{5}
		\end{array}
		\right]
		% \begin{bmatrix}
			% \id[\bb{R}^{d}] & \\
			% \derivada{\Phi^{i}}{x^{j}} &
				% \derivada{\Phi^{i}}{y^{j}}
		% \end{bmatrix}
		\text{ .}
	\end{align*}
	%
	Como la submatriz inferior derecha es invertible, por el teorema
	\ref{thm:funinversa}, existen entornos $U_{0}\subset U$ de $(a,b)$
	y $\widehat{U}_{0}\subset\bb{R}^{d}\times\bb{R}^{l}$ de $\Psi(a,b)$
	ambos conexos y tales que $\Psi|:\,U_{0}\rightarrow\widehat{U}_{0}$
	es invertible y con inversa $C^{k}$. Podemos tomar $U_{0}$
	de la forma $V\times W$ achicando, de ser necesario, y
	$\widehat{U}_{0}=\Psi(U_{0})=\Psi(V\times W)$.

	La inversa de $\Psi$ restringida e $U_{0}$ es de la forma
	\begin{align*}
		\Psi^{-1}(\xi,\upsilon) & \,=\,
			(A(\xi,\upsilon),B(\xi,\upsilon))
	\end{align*}
	%
	para ciertas funciones de clase $C^{k}$ definidas en $\widehat{U}_{0}$.
	En particular, estas funciones deben verificar
	\begin{align*}
		(\xi,\upsilon) & \,=\, \Psi\circ\Psi^{-1}(\xi,\upsilon) \,=\,
			(A,\Phi(A,B))
		\text{ .}
	\end{align*}
	%
	As\'{\i}, se ve que $A(\xi,\upsilon)=\xi$ y que
	$\Psi^{-1}(\xi,\upsilon)=(\xi,B(\xi,\upsilon))$.

	Sea, ahora, $\inc[c]:\,V\rightarrow\bb{R}^{d}\times\bb{R}^{l}$ la
	funci\'{o}n $\inc[c](x)=(x,c)$. Sea $V_{0}$ el subconjunto
	\begin{align*}
		V_{0} & \,=\,\left\lbrace x\in V\,:\,(x,c)\in\widehat{U}_{0}
			\right\rbrace
		\,=\,\inc[c]^{-1}(\widehat{U}_{0})
		\text{ .}
	\end{align*}
	%
	Como $\inc[c]$ es continua (m\'{a}s aun, es un embedding), $V_{0}$
	es abierto en $V$. Tomamos $W_{0}=W$ y definimos
	$F:\,V_{0}\rightarrow W_{0}$ por $x\mapsto B(x,c)$. Es decir,
	$F=\pi_{2}\circ\Psi\circ\inc[c]$, donde $\pi_{2}$ es la
	proyecci\'{o}n $\bb{R}^{d}\times\bb{R}^{l}\rightarrow\bb{R}^{l}$
	en el segundo factor. En particular, $F$ es de clase $C^{k}$ (tan
	regular como $\Psi$). Adem\'{a}s, si $x\in V_{0}$,
	\begin{align*}
		(x,c) & \,=\,\Psi\circ\Psi^{-1}(x,c) \,=\,\Psi(x,B(x,c)) \\
		& \,=\,(x,\Phi(x,F(x)))\quad\text{y} \\
		c & \,=\,\Phi(x,F(x))
		\text{ .}
	\end{align*}
	%
	Finalmente, si $(x,y)\in V_{0}\times W_{0}$, es tal que $\Phi(x,y)=c$,
	\begin{align*}
		\Psi(x,y) & \,=\,(x,\Phi(x,y)) \,=\, (x,c)\quad\text{y} \\
		(x,y) & \,=\,\Psi^{-1}(x,c) \,=\, (x,B(x,c)) \\
		& \,=\,(x,F(x))
		\text{ .}
	\end{align*}
	%
\end{proof}

%
\section{El rango de una transformaci\'{o}n}
\theoremstyle{plain}
\newtheorem{teoFunInvVar}{Teorema}[section]
\newtheorem{propoDifeoLocal}[teoFunInvVar]{Proposici\'{o}n}
\newtheorem{coroDifeoLocal}[teoFunInvVar]{Corolario}

\theoremstyle{remark}
\newtheorem{obsDifeoLocal}{Observaci\'{o}n}[section]

%-------------

Sea $F:\,M\rightarrow N$ una transformaci\'{o}n suave. Dado un punto $p\in M$,
el \emph{rango de $F$ en $p$} se define como el rango de la transformaci\'{o}n
lineal asociada $\diferencial[p]{F}:\,T_{p}M\rightarrow T_{F(p)}N$. Este
n\'{u}mero es igual al rango de la matriz jacobiana de $F$ en cualquiera de
sus representaciones en coordenadas $\jacobiana[\widehat{p}]{\widehat{F}}$,
siendo \'{e}sta la matriz de la transformaci\'{o}n lineal $\diferencial[p]{F}$
con respecto a las bases determinadas por tomar cartas compatibles en $p$ y en
$F(p)$, por lo que, equivalentemente, podr\'{\i}amos definir el rango de
esta manera, sin hacer referencia directamente a los tangentes y al
diferencial. Denotamos el rango de $F$ en $p$ por $\rango[p]{F}$ y,
por definici\'{o}n $\rango[p]{F}=\rango{\diferencial[p]{F}}$.

Si existe $r\geq 0$ tal que $\rango{\diferencial[p]{F}}=r$, para todo punto
$p$, se dice que $F$ \emph{tiene rango constante $r$ en $M$} o que
\emph{es de rango constante}. En todo caso, el rango de $F$ est\'{a}
acotado:
\begin{align*}
	0 & \,\leq\,\rango{F}\,\leq\,\min\{\dim\,M,\dim\,N\}
	\text{ .}
\end{align*}
%
Si la cota superior es alcanzada en un punto $p\in M$, se dice que $F$
\emph{tiene rango m\'{a}ximo en $p$}. Esto se puede deber a cualquiera de
dos cosas: o bien $\diferencial[p]{F}:\,T_{p}M\rightarrow T_{F(p)}N$ es
inyectivo ($\rango{\diferencial[p]{F}}=\dim\,M$), o bien es sobreyectivo
($\rango{\diferencial[p]{F}}=\dim\,N$).

El rango de una transformaci\'{o}n suave $F$ verifica que
\begin{align*}
	\{\rango{F}>k\} & \,=\,\{\rango{F}\geq k+1\}
\end{align*}
%
es abierto para todo $k\geq 0$. En particular, si $F$ tiene rango m\'{a}ximo
en $p$, entonces tendr\'{a} rango m\'{a}ximo en todo un entorno del punto,
en particular, el rango de $F$ ser\'{a} constante en el entorno. Si
$\diferencial[p]{F}$ es sobreyectivo para todo punto $p$ en el dominio de
$F$, decimos que $F$ es una \emph{submersi\'{o}n}; si es inyectivo, decimos
que es una \emph{inmersi\'{o}n}. Con estas definiciones, podemos afirmar
que, si $\diferencial[p]{F}$ es inyectivo, entonces la restricci\'{o}n
$F|_{U}:\,U\rightarrow N$ es una inmersi\'{o}n en alg\'{u}n entorno $U$ de
$p$. An\'{a}logamente, si $\diferencial[p]{F}$ es sobreyectivo, $F|_{U}$
es una submersi\'{o}n.

Un ejemplo de esto est\'{a} dado por una funci\'{o} diferenciable
$f:\,U\rightarrow\bb{R}$ definida en un abierto de $\bb{R}^{n}$. En las
coordenadas usuales, la matriz jacobiana de $f$ est\'{a} dada por el
vector de derivadas parciales evaluadas en el punto:
\begin{align*}
	\jacobiana[x]{f} & \,=\,
		\begin{bmatrix}
			\derivada{f}{x^{1}}(x) & \cdots &
			\derivada{f}{x^{n}}(x)
		\end{bmatrix}
	\text{ .}
\end{align*}
%
La transformaci\'{o}n lineal asociada es sobreyectiva, si y s\'{o}lo si
alguna derivada es distinta de cero en el punto. Los puntos en donde el
diferencial de esta funci\'{o}n es sobreyectivo (en este caso esto quiere
decir distinto de cero) son precisamente los puntos \emph{no singulares}
de $f$.

Para ver una clase de casos en donde el diferencial es inyectivo, podemos
tomar una funci\'{o}n en la direcci\'{o}n opuesta. Si
$f:\,\bb{R}\rightarrow U\subset\bb{R}^{n}$ es diferenciable, entonces la
matriz jacobiana en este caso tambi\'{e}n est\'{a} dada por el vector de
derivadas parciales, ahora visto como una matriz de tama\~{n}o
$n\times 1$. Esta matriz ser\'{a} la matriz de una transformaci\'{o}n
lineal inyectiva, siempre y cuando, de nuevo, alguna de las derivadas
parciales no sea nula. La funci\'{o}n $f$ describe una curva --no en
tanto variedad de dimensi\'{o}n, sino en tanto parametrizaci\'{o}n--
en el espacio. Que el diferencial de $f$ sea inyectivo en un insante $t$
significa que la velocidad de la curva en dicho instante, $\dot{f}(t)$ no
es cero.

Un \emph{difeomorfismo local} es una transformaci\'{o}n suave
$F:\,M\rightarrow N$ tal que, para todo punto $p\in M$, existe un entorno
$U\subset M$ de $p$ que verifica que $F(U)\subset N$ es abierta y que
$F|_{U}:\,U\rightarrow F(U)$ es un difeomorfismo. Como los tangentes
est\'{a}n definidos localmente, si $F$ es un difeomorfismo local, entonces
$\diferencial[p]{F}:\,T_{p}M\rightarrow T_{F(p)}N$ es un isomorfismo
lineal en todo punto. En particular, todo difeomorfismo local es, a la
vez, una submersi\'{o}n y una inmersi\'{o}n. Aunque s\'{o}lo v\'{a}lido
para una transformaci\'{o}m entre variedades \emph{sin} borde, el teorema
de la funci\'{o}n inversa es la afirmaci\'{o}n rec\'{\i}proca:

\begin{teoFunInvVar}\label{thm:funinvvar}
	Sean $M$ y $N$ variedades \emph{sin} borde y sea $F:\,M\rightarrow N$
	una transformaci\'{o}n suave. Si $p\in M$ es un punto en donde
	$\diferencial[p]{F}:\,T_{p}M\rightarrow T_{F(p)}N$ es invertible,
	existen entonces entornos conexos $U_{0}\subset M$ de $p$ y
	$V_{0}\subset N$ de $F(p)$ tales que la restricci\'{o}n
	$F|_{U_{0}}:\,U_{0}\rightarrow V_{0}$ es difeomorfismo.
\end{teoFunInvVar}

\begin{proof}
	Aunque parezca trivial, el hecho de que $\diferencial[p]{F}$ es
	isomorfismo implica que $\dim\,M=\dim\,N$. Tomando coordenadas
	$(U,\varphi)$ en $p$ y $(V,\psi)$ en $F(p)$ tales que
	$F(U)\subset V$, como tanto $\varphi$ como $\psi$ son
	difeomorfismos, la matriz jacobiana
	$\jacobiana[\widehat{p}]{\widehat{F}}$ es invertible. Por el teorema
	usual de la funci\'{o}n inversa, existen entornos (conexos)
	$\widehat{U}_{0}\subset\varphi(U)$ y $\widehat{V}_{0}\subset\psi(V)$
	tales que $\widehat{F}|_{\widehat{U}_{0}}:\,%
	\widehat{U}_{0}\rightarrow\widehat{V}_{0}$ es un difeomorfismo. Si
	$U_{0}=\varphi^{-1}(\widehat{U}_{0})$ y
	$V_{0}=\psi^{-1}(\widehat{V}_{0})$, entonces $p\in U_{0}$,
	$F(p)\in V_{0}$, $U_{0}$ y $V_{0}$ son conexos y
	$F|_{U_{0}}\rightarrow V_{0}$ es difeomorfismo.
\end{proof}

A continuaci\'{o}n enunciamos algunas propiedades importantes de los
difeomorfismos locales.

\begin{propoDifeoLocal}\label{thm:propisdifeoslocales}
	\emph{(a)} La composici\'{o}n de difeomorfismos locales es un
	difeomorfismo local; \emph{(b)} el producto de dos difeomorfismos
	locales es un difeomorfismo local, su inversa est\'{a} dada por el
	producto de las respectivas transformaciones inversas;
	\emph{(c)} la restricci\'{o}n de un difeomorfismo local a un
	abierto sigue siendo un difeomorfismo local; \emph{(d)} todo
	difeomorfismo local biyectivo es un difeomorfismo.
\end{propoDifeoLocal}

Por ejemplo, para demostrar \emph{(d)}, si $F:\,M\rightarrow N$ es
biyectiva y difeomorfismo local, entonces, localmente, su inversa
$F^{-1}$ coincide con una funci\'{o}n suave: si $q\in N$ y $p\in M$
es el (\'{u}nico) punto tal que $F(p)=q$, existe un entorno $U$ de $p$
tal que $F(U)$ es abierta y $F|_{U}:\,U\rightarrow F(U)$ es difeomorfismo.
En particular, $F^{-1}|_{F(U)}=(F|_{U})^{-1}$ que es una funci\'{o}n
suave.

Para determinar si una transformaci\'{o}n (suave) es un difeomorfismo local,
alcanza con ver lo que sucede en entornos coordenados.

\begin{obsDifeoLocal}\label{obs:difeoslocaleslocal}
	Una transformaci\'{o}n suave $F:\,M\rightarrow N$ es un
	difeomorfismo local, si y s\'{o}lo si ara cada punto $p\in M$
	existe un entorno $U\subset M$ de $p$ tal que la representaci\'{o}n
	en coordenadas de $F$ en $U$ es un difeomorfismo local.
\end{obsDifeoLocal}

\begin{coroDifeoLocal}\label{thm:difeolocalsubmersioneinmersion}
	Si $F:\,M\rightarrow N$ es una transformaci\'{o}n suave, entonces
	$F$ es un difeomorismo local, si y s\'{o}lo si es una
	submersi\'{o}n y una inmersi\'{o}n.
\end{coroDifeoLocal}

\begin{proof}
	Ya demostramos que todo difeomorfismo local es submersi\'{o}n
	e inmersi\'{o}n. Rec\'{\i}procamente, si $F$ es, a la vez, una
	submersi\'{o}n y una inmersi\'{o}n, entonces
	$\diferencial[p]{F}:\,T_{p}M\rightarrow T_{F(p)}N$ es un
	isomorfismo. Por el teorema de la funci\'{o}n inversa, existen
	entornos de $p$ y de $F(p)$ tales que $F$ restringida a los mismos
	es un difeomorfismo.
\end{proof}

\begin{coroDifeoLocal}\label{thm:difeolocalsubmersionoinmersion}
	Si $F:\,M\rightarrow N$ es una transformaci\'{o}n suave,
	$\dim\,M=\dim\,N$ y $F$ es una submersi\'{o}n o una inmersi\'{o}n,
	entonces $F$ es un difeomorfismo local.
\end{coroDifeoLocal}

\begin{proof}
	Dado que las dimensiones del dominio y del codominio de $F$ son
	iguales, $F$ es una inmersi\'{o}n, si y s\'{o}lo si es una
	submersi\'{o}n.
\end{proof}

%
\section{M\'{a}s propiedades de los difeomorfismos locales}
\theoremstyle{plain}
\newtheorem{coroFunInvVar}{Corolario}[section]
\newtheorem{propoAbiertoEsSubvariedadRegular}[coroFunInvVar]{Proposici\'{o}n}
\newtheorem{propoBordeEsSubvariedadRegular}[coroFunInvVar]{Proposici\'{o}n}
\newtheorem{propoDiferencialInvertibleCaeEnElInterior}[coroFunInvVar]%
	{Proposici\'{o}n}
\newtheorem{coroCaeEnElInterior}[coroFunInvVar]{Corolario}
\newtheorem{propoDifeoLocalExtra}[coroFunInvVar]{Proposici\'{o}n}

\theoremstyle{remark}

%-------------

Bajo ciertas condiciones sobre $F:\,M\rightarrow N$ podemos garantizar
la validez de las conclusiones del teorema de la funci\'{o}n inversa en
contextos un poco m\'{a}s generales.

\begin{coroFunInvVar}\label{thm:funinvvarconborde}
	Sea $F:\,M\rightarrow N$ una transformaci\'{o}n suave, donde
	$M$ es una variedad \emph{sin} borde y $N$ es arbitraria. Si
	$F(M)\subset\interior{N}$ y $p\in M$ es tal que $\diferencial[p]{F}$
	es invertible, entonces existen entornos conexos $p\in U_{0}$
	y $F(p)\in V_{0}$ tales que $F|_{U_{0}}:\,U_{0}\rightarrow V_{0}$
	es un difeomorfismo.
\end{coroFunInvVar}

El contenido de este corolario es un caso particular del caso en que una
transformaci\'{o}n suave tiene imagen en una \emph{subvariedad regular}.
Sabemos que, en una variedad $M$, el interior $\interior{M}$ es un
subconjunto abierto y que --con la topolog\'{\i}a de subespacio y la
estructura heredada de $M$-- es una variedad diferencial de la misma
dimensi\'{o}n que $M$. El borde $\borde[M]$ de $M$ es un subconjunto cerrado
al cual se le puede dar, naturalmente tambi\'{e}n, una estructura de
variedad diferencial de dimensi\'{o}n $\dim(M)-1$. Como se ver\'{a} luego,
con dicha estructura, $\borde[M]$ resulta una subvariedad regular de $M$,
debido a la presencia de \emph{cartas preferenciales} (de hecho, \'{e}stas
son las cartas utilizadas para definir la estructura a la que se hizo
alusi\'{o}n).

Repasemos la demostraci\'{o}n de estas afirmaciones. Sea $n=\dim\,M$.
Por definici\'{o}n, un punto $p\in M$ pertenece al interior $\interior{M}$,
si existe una \emph{carta de interior} en $p$, es decir, una carta
$(U,\varphi)$ \emph{para $M$} tal que $p\in U$ y $\varphi(U)$ sea un
abierto de $\bb{R}^{n}$. En particular, de esto se deduce que $\interior{M}$
es un subconjunto abierto de $M$, pues, por ejemplo, dados $p$ y $(U,\varphi)$
como antes, vale que $U\subset\interior{M}$. La colecci\'{o}n de cartas
\begin{align*}
	\cal{A}_{0} & \,=\,\{(U,\varphi)\text{ carta para }M\,:\,
		\varphi(U)\subset\bb{R}^{n}\text{ es abierto}\}
	\text{ ,}
\end{align*}
%
es un atlas compatible para un subconjunto de $M$, precisamente para
el interior de $M$. El subconjunto $\interior{M}$ con la topolog\'{\i}a
de subespacio y la estructura determinada por dicho atlas resulta
una variedad diferencial. La inclusi\'{o}n
$\inc[\interior{M}]:\,\interior{M}\hookrightarrow M$ es suave y tiene
la propiedad de que una funci\'{o}n $F:\,N\rightarrow\interior{M}$ es suave,
si y s\'{o}lo si $\inc[\interior{M}]\circ F:\,N\rightarrow M$ lo es.
Esto es cierto, m\'{a}s en general, para cualquier abierto de $M$.

\begin{propoAbiertoEsSubvariedadRegular}\label{thm:abiertosubvarreg}
	Sea $M$ una variedad diferencial y sea $U\subset M$ un abierto al
	que se le da la estructura diferencial heredada de $M$, es decir,
	$U$ tiene la estructura dada por el atlas compatible
	\begin{align*}
		\cal{A}_{U} & \,=\,
		\left\lbrace (V,\psi)\text{ carta para } M\,:\,
			V\subset U\right\rbrace
		\text{ .}
	\end{align*}
	%
	Entonces la inclusi\'{o}n $\inc[U]:\,U\hookrightarrow M$ es suave,
	$\diferencial[p]{\inc[U]}$ es isomorfismo y, adem\'{a}s, una
	funci\'{o}n $F:\,N\rightarrow U$ es una transformaci\'{o}n suave,
	si y s\'{o}lo si $\inc[U]\circ F$ lo es.
\end{propoAbiertoEsSubvariedadRegular}

Aunque demostraremos luego un resultado m\'{a}s general, damos ahora una
demostraci\'{o}n de esta proposici\'{o}n.

\begin{proof}
	Sea $p\in U$ y sea $(V,\psi)\in\cal{A}_{U}$ con $p\in V$.
	Entonces $(V,\psi)$ tambi\'{e}n es una carta en $p$ en tanto
	punto de $M$ y $\psi\circ\inc[U]\circ\psi^{-1}:\,%
	\psi(V)\rightarrow\psi(V)$ es igual, como funci\'{o}n, a
	$\id[\psi(V)]$, que es suave. Que el diferencial
	$\diferencial[p]{\inc[U]}:\,T_{p}U\rightarrow T_{p}M$ es un
	isomorfismo, se vio en \ref{thm:derivacionesisomorfasii}.

	Sea ahora $F:\,N\rightarrow U$ una transformaci\'{o}n arbitraria.
	En primer lugar, como $U$ es un subespacio de $M$,
	$F:\,N\rightarrow U$ es continua, si y s\'{o}lo si
	$\inc[U]\circ F:\,N\rightarrow M$ lo es. Si asumimos que $F$ es
	suave, entonces $\inc[U]\circ F$ es suave por ser composici\'{o}n
	de funciones suaves. Si, rec\'{\i}procamente, $\inc[U]\circ F$ es
	suave y $(\widetilde{V},\widetilde{\psi})$ es una carata en
	$p=F(q)=\inc\circ F(q)$ para $M$ y
	$(\widetilde{U},\widetilde{\varphi})$ es una carta en $q$ para $N$,
	sabemos que la composici\'{o}n
	\begin{align*}
		\widetilde{\psi}\circ (\inc[U]\circ F)\circ
			\widetilde{\varphi}^{-1} & \,:\,
			\widetilde{\varphi}\big(
			(\inc[U]\circ F)^{-1}(\widetilde{V})\cap\widetilde{U}
			\big)\,\rightarrow\,\widetilde{\psi}(\widetilde{V})
	\end{align*}
	%
	es suave. Ahora bien, $\widetilde{V}\cap U$ es abierto en $U$ y
	$(\inc\circ F)^{-1}(\widetilde{V})=F^{-1}(\widetilde{V}\cap U)$ es
	abierto porque $F$ es continua. La carta $(V,\psi)$, donde
	$V=\widetilde{V}\cap U$ y $\psi=\widetilde{\psi}|_{V}$, pertenece
	a $\cal{A}_{U}$ y $\psi\circ F\circ\widetilde{\varphi}^{-1}:\,%
	\widetilde{\varphi}(F^{-1}(V)\cap\widetilde{U})\rightarrow\psi(V)$
	es igual a
	$\widetilde{\psi}\circ\inc\circ F\circ\widetilde{\varphi}^{-1}$
	que es suave.
\end{proof}

Pasemos ahora a $\borde[M]$. Por definici\'{o}n, los puntos del borde son
aquellos puntos $p\in M$ que verifican que existe un abierto $U\subset M$
tal que $p\in U$ y una funci\'{o}n continua $\varphi$ de $U$ en un
abierto de $\hemi[n]$ tal que $\varphi(p)\in\borde[{\hemi[n]}]$. Por el teorema
de invarianza del dominio, sabemos que $\borde[M]=M\setmin\interior{M}$ y
que, por lo tanto, $\borde[M]$ es cerrado. Pero, adem\'{a}s, si $(U,\varphi)$
es una carta para $M$,
\begin{align*}
	\varphi(U\cap\borde[M]) & \,=\,\borde[{\hemi[n]}]\cap\varphi(U)
\end{align*}
%
exactamente, es decir, si $p'\in U$ y $\varphi(p')\not\in\borde[{\hemi[n]}]$,
entonces $p'$ debe ser un punto del interior de $M$. Dicho de otra manera,
a nivel de cartas para $M$, $\interior{M}$ se ve como $\interior{\hemi[n]}$
y $\borde[M]$ como $\borde[{\hemi[n]}]$. Para cada punto $p\in\borde[M]$
del borde, existe una carta del borde $(U_{p},\varphi_{p})$ para $M$ en
$p$. Si $\varphi_{p}=(\lista*{x}{n})$, entonces
\begin{align*}
	\borde[M]\cap U_{p} & \,=\,\{x^{n}=0\}\quad\text{e} \\
	\interior{M}\cap U_{p} & \,=\,\{x^{n}>0\}
	\text{ .}
\end{align*}
%
Para cada una de estas cartas consideramos la proyecci\'{o}n en las primeras
$n-1$ coordenadas de la restricci\'{o}n a $\borde[M]$. Es decir, para cada
$p$ definimos una funci\'{o}n $\overline{\varphi_{p}}:\,%
\borde[M]\cap U_{p}\rightarrow\bb{R}^{n-1}$ por
\begin{align*}
	\overline{\varphi_{p}} & \,=\,\pi\circ\varphi_{p}\circ
		\inc[{\borde[M]}\cap U_{p}]
	\text{ .}
\end{align*}
%
La imagen de esta funci\'{o}n es igual a
\begin{align*}
	\overline{\varphi_{p}}(\borde[M]\cap U_{p}) & \,=\,\pi(x^{n}=0)]
	\text{ ,}
\end{align*}
%
que es un abierto de $\bb{R}^{n-1}$. Sea
$\overline{U_{p}}=\borde[M]\cap U_{p}$ y sea
\begin{align*}
	\cal{A}_{\borde[M]} & \,=\,\left\lbrace
		(\overline{U_{p}},\overline{\varphi_{p}})\,:\,
		p\in\borde[M]\right\rbrace
	\text{ .}
\end{align*}
%
La colecci\'{o}n $\cal{A}_{\borde[M]}$ cubre a $\borde[M]$ y las funciones
$\overline{\varphi_{p}}$ son homeomorfismos con abiertos de $\bb{R}^{n-1}$.
Dados $p,q\in\borde[M]$ con
$\overline{U_{p}}\cap\overline{U_{q}}\not=\varnothing$, la composici\'{o}n
$\overline{\varphi_{p}}\circ\overline{\varphi_{q}}^{-1}:\,%
\overline{\varphi_{p}}(\overline{U_{p}}\cap\overline{U_{q}})\rightarrow%
\overline{\varphi_{q}}(\overline{U_{p}}\cap\overline{U_{q}})$ es igual a
\begin{align*}
	\overline{\varphi_{p}}\circ\overline{\varphi_{q}}^{-1}
		(\lista*{u}{n-1}) & \,=\,
		\pi\circ\varphi_{p}\inc[\overline{U_{p}}](\varphi_{q}^{-1}(
			\lista*{u}{n-1},\,0)) \\
	& \,=\,\pi\varphi_{p}\varphi_{q}^{-1}(\lista*{u}{n-1},\,0) \\
	& \,=\,\pi\circ(\varphi_{p}\circ\varphi_{q}^{-1})\circ j_{0}
		(\lista*{u}{n-1})
	\text{ ,}
\end{align*}
%
donde $j_{0}(\lista*{u}{n-1})=(\lista*{u}{n-1},\,0)\in%
\overline{\varphi_{q}}(\overline{U_{q}})$. Esta composici\'{o}n es suave
en las coordenadas $\lista*{u}{n-1}$. Vemos, entonces, que $\borde[M]$ tiene
una estructura de variedad diferencial determinada por el \emph{atlas}
$\cal{A}_{\borde[M]}$. Este atlas consiste, esencialmente, en las
restricciones de las cartas compatibles con la estructura de $M$ al borde,
de maner an\'{a}loga a lo que se hizo con $\interior{M}$ o, en general,
con un abierto $U$ de $M$.

\begin{propoBordeEsSubvariedadRegular}\label{thm:bordesubvarreg}
	Sea $M$ una variedad dierencial. La inclusi\'{o}n
	$\inc[{\borde[M]}]:\,\borde[M]\hookrightarrow M$ es suave. Dada
	una funci\'{o}n $F:\,N\rightarrow\borde[M]$ arbitraria,
	$F$ es suave, si y s\'{o}lo si $\inc[{\borde[M]}]\circ F$ lo es.
\end{propoBordeEsSubvariedadRegular}

\begin{proof}
	Notemos que, como $\borde[M]$ es un subespacio topol\'{o}gico,
	$F:\,N\rightarrow\borde[M]$ es continua, si y s\'{o}lo si
	$\inc\circ F$ es continua. Sea $p\in\borde[M]$ y sea $(U,\varphi)$
	una carta para $M$ en $p$. Como $p$ es un punto del borde, cualquiera
	sea la carta $(U,\varphi)$ en $p$, debe valer que
	$\varphi(p)\in\varphi(U)\cap\borde[{\hemi[n]}]$. Notemos, adem\'{a}s,
	que, si $\overline{U}=\borde[M]\cap U$ y
	$\overline{\varphi}=\pi\circ\varphi\circ\inc$, el par
	$(\overline{U},\overline{\varphi})$ es una carta compatible para
	$\borde[M]$. Con respecto a las cartas $(U,\varphi)$ en $M$ y
	$(\overline{U},\overline{\varphi})$ en $\borde[M]$,
	\begin{align*}
		\varphi\circ\inc[{\borde[M]}]\circ\overline{\varphi}^{-1}
			(\lista*{u}{n-1}) & \,=\,(\lista*{u}{n-1},\,0)
			\,=\,j(\lista*{u}{n-1})
		\text{ .}
	\end{align*}
	%
	Con lo cual, $\inc[{\borde[M]}]:\,\borde[M]\hookrightarrow M$ es
	suave. Localmente, $\inc[{\borde[M]}]$ es la inclusi\'{o}n de un
	abierto de $\bb{R}^{n-1}$ como la tajada con $\{x^{n}=0\}$ en
	un abierto de $\hemi[n]$ (o de $\bb{R}^{n}$).

	Si $F:\,N\rightarrow\borde[M]$ es suave, entonces
	$\inc\circ F:\,N\rightarrow M$ es suave por ser composici\'{o}n
	de funciones suaves. Si, rec\'{\i}procamente, $\inc\circ F$ es suave,
	entonces debe ser suave en sentido usual con respecto a cualquier
	par de cartas para $N$ y para $M$. Sea $F(q)=p\in\borde[M]$. Sean
	$(V,\psi)$ una carta para $M$ en $p$ y $(U,\varphi)$ una carta
	para $N$ en $q$. Definimos $\overline{V}=V\cap\borde[M]$ y
	$\overline{\psi}=\pi\circ\psi\circ\inc[{\borde[M]}]$, como antes.
	En el abierto $\varphi(F^{-1}(V)\cap U)$,
	\begin{align*}
		\overline{\psi}\circ F\circ\varphi^{-1} & \,=\,
			\pi\circ\psi\circ(\inc[{\borde[M]}]\circ F)\circ
			\varphi^{-1}
		\text{ .}
	\end{align*}
	%
	Como $\psi\circ\inc\circ F\circ\varphi^{-1}$ es suave y
	$\pi:\,\hemi[n]\rightarrow\bb{R}^{n-1}$ tambi\'{e}n lo es, la
	representaci\'{o}n $\overline{\psi}\circ F\circ\varphi^{-1}$ es
	suave. Notemos que $\varphi(F^{-1}(V)\cap U)$ es abierto
	porque $F$ es continua como funci\'{o}n en $\borde[M]$.
\end{proof}

Volvamos, ahora, a la demostraci\'{o}n corolario.

\begin{proof}[Demostraci\'{o}n de \ref{thm:funinvvarconborde}]
	Si $F:\,M\rightarrow N$ es suave y $F(M)\subset\interior{N}$,
	entonces $F|:\,M\rightarrow\interior{N}$ es suave e $\interior{N}$
	es abierto en $N$. Adem\'{a}s, $\borde[\interior{N}]=\varnothing$,
	con lo cual podemos intentar aplocar el teorema \ref{thm:funinvvar}.
	Notemos que $F=\inc[\interior{N}]\circ F|$. Por la
	funtorialidad del diferencial,
	\begin{align*}
		\diferencial[p]{F} & \,=\,
			\diferencial[F(p)]{\inc[\interior{N}]}\circ
			\diferencial[p]{(F|)}
		\text{ .}
	\end{align*}
	%
	Como $\diferencial[F(p)]{\inc[\interior{N}]}$ es un isomorfismo,
	en particular, es inyectivo y
	\begin{align*}
		\rango{\diferencial[p]{F}} & \,=\,
			\rango{\diferencial[p]{(F|)}}
		\text{ .}
	\end{align*}
	%
	Por hip\'{o}tesis, $\diferencial[p]{F}$ es invertible. Entonces
	$\diferencial[p]{(F|)}$ lo es, tambi\'{e}n. Por el teorema
	\ref{thm:funinvvar}, existen abiertos $V_{0}\subset\interior{N}$
	y $U_{0}\subset M$ tales que $p\in U_{0}$, $F(p)\in V_{0}$ y
	$(F|)|_{U_{0}}:\,U_{0}\rightarrow V_{0}$ es difeomorfismo. Pero,
	como $\interior{N}$ es abierto en $N$, el conjunto $V_{0}$ es abieto
	en $N$. En definitiva, existen entornos conexos $U_{0}$ de $p$ en
	$M$ y $V_{0}$ de $F(p)$ en $N$ (conexos) tales que
	$F|_{U_{0}}:\,U_{0}\rightarrow V_{0}$ es difeomorfismo.
\end{proof}

\begin{propoDiferencialInvertibleCaeEnElInterior}\label{thm:caeenelinterior}
	Sea $M$ una variedad \emph{sin} borde y sea $N$ una variedad
	(arbitraria). Sea $F:\,M\rightarrow N$ una transformaci\'{o}n
	suave. Si $p\in M$ es tal que $\diferencial[p]{F}$ es no
	singular, entonces $F(p)\in\interior{N}$.
\end{propoDiferencialInvertibleCaeEnElInterior}

Esta proposici\'{o}n nos permite omitir la hip\'{o}tesis
$F(M)\subset\interior{N}$ en el enunciado del corolario
\ref{thm:funinvvarconborde}.

\begin{proof}
	Sean $m=\dim\,M$ y $n=\dim\,N$.
	Supongamos que $F(p)\in\borde[N]$. Sea $(U,\varphi)$ una carta
	para $M$ en $p$, con $\varphi(p)=0$ y $\varphi(U)=\bola[m]{1}{0}$
	y sea $(V,\psi)$ carta para $N$ en $F(p)$ con $\psi(F(p))=0$
	y $\psi(V)=\bola[n]{1}{0}\cap\hemi[n]$. Expresado de manera
	m\'{a}s concisa, $U$ es una bola coordenada centrada en $p$
	y $V$ es una semibola coordenada centrada en $F(p)$. Supongamos que
	elegimos los entornos de las cartas de manera que se cumpla
	que $F(U)\subset V$, para simplificar. Sea
	$\widehat{F}=\psi\circ F\circ\varphi^{-1}:\,%
	\varphi(U)\rightarrow\psi(V)$ la representaci\'{o}n
	de $F$ en estas coordenadas. Por hip\'{o}tesis,
	$\diferencial[p]{F}:\,T_{p}M\rightarrow T_{F(p)}N$ es un isomorfismo
	(en $p$), con lo que la matriz jacobiana
	$\jacobiana{\widehat{F}}:\,\bb{R}^{m}\rightarrow\bb{R}^{n}$
	tambi\'{e}n lo es. En particular, $m=n$. Pero, adem\'{a}s,
	como $\inc=\inc[{\hemi[n]}]:\,\hemi[n]\rightarrow\bb{R}^{n}$ es suave,
	$\inc\circ \widehat{F}:\bola[n]{1}{0}\rightarrow\bola[n]{1}{0}$ es
	suave y
	\begin{align*}
		\jacobiana[\widehat{F}(\widehat{p})]{\inc}\cdot
			\jacobiana[\widehat{p}]{\widehat{F}} & \,=\,
			\jacobiana[\widehat{p}]{(\inc\circ \widehat{F})}
		\text{ .}
	\end{align*}
	%
	Como $\rango{\jacobiana[\widehat{F(p)}]{\inc}}=n$, es decir,
	$\jacobiana[\widehat{F(p)}]{\inc}$ es invertible,
	la matriz $\jacobiana[\widehat{p}]{(\inc\circ\widehat{F})}$
	tambi\'{e}n debe serlo. Por el teorema de la funci\'{o}n inversa,
	existen abiertos $\widehat{U}_{0}\subset\varphi(U)=\bola{1}{0}$ y
	$W_{0}\subset\bola{1}{0}$ tales que
	$\inc\circ\widehat{F}|_{\widehat{U}_{0}}:\,%
	\widehat{U}_{0}\rightarrow W_{0}$ es difeomorfismo. Pero
	$W_{0}=\inc\circ\widehat{F}(\widehat{U}_{0})\subset\inc(\hemi[n])$
	y el punto $0$ pertenece a $\inc\circ\widehat{F}(\widehat{U}_{0})$ y,
	entonces, $W_{0}$ no puede ser abierto pues todo entorno de $0$ en
	$\bb{R}^{n}$ contiene puntos que no pertenecen a $\hemi[n]$.
\end{proof}

\begin{coroCaeEnElInterior}\label{thm:corocaeenelinterior}
	Si $M$ es una variedad \emph{sin} borde y $N$ es una variedad
	diferencial posiblemente con borde, entonces
	\emph{(a)} una transformaci\'{o}n suave $F:\,M\rightarrow N$ es
	difeomorfismo local, si y s\'{o}lo si es submersi\'{o}n e
	inmersi\'{o}n; \emph{(b)} si $\dim\,M=\dim\,N$ y $F$ es
	submersi\'{o}n o inmersi\'{o}n, entonces $F$ es difeomorfismo local.
\end{coroCaeEnElInterior}

\begin{proof}
	Si $F$ es difeomorfismo local, entonces $\diferencial[p]{F}$ es
	isomorfismo lineal. En particular, $\dim\,M=\dim\,N$ y $F$
	tiene rango m\'{a}ximo en todo punto $p$. Rec\'{\i}procamente, si $F$
	submersi\'{o}n e inmersi\'{o}n, $\diferencial[p]{F}$ es isomorfismo
	en todo punto $p$. Por la proposici\'{o}n \ref{thm:caeenelinterior},
	$F(M)\subset\interior{N}$ y, por el corolario
	\ref{thm:funinvvarconborde} $F$ es difeomorfismo local. Esto
	demuestra \emph{(a)}. En cuanto a \emph{(b)}, si $\dim\,M=\dim\,N$,
	entonces las condiciones para ser submersi\'{o}n y para ser
	inmersi\'{o}n coinciden. Con lo cual, si se sabe, por ejemplo, que
	$F$ es submersi\'{o}n, entonces debe ser inmersi\'{o}n y, por
	\emph{(a)}, debe ser difeomorfismo local.
\end{proof}

\begin{propoDifeoLocalExtra}\label{thm:propisextradifeoslocales}
	Sean $M,N,P,P'$ variedades diferenciales con o sin borde. Sea
	$F:\,M\rightarrow N$ un difeomorfismo local. Entonces
	\emph{(a)} si $G:\,P\rightarrow M$ es continua, entonces
	$G$ es suave, si y s\'{o}lo si $F\circ G$ lo es; \emph{(b)} si
	$F$ es sobreyectiva y $G':\,N\rightarrow P'$ e s una funci\'{o}n
	arbitraria, entonces $G'$ es (continua y) suave, si y s\'{o}lo si
	$G'\circ F$ lo es.
\end{propoDifeoLocalExtra}

\begin{proof}
	\emph{(a)} Supongamos que $F\circ G$ es suave y que $G$ es continua.
	Si $p\in P$, como $F$ es un difeomorfismo local, existe un entorno
	$U_{0}\subset M$ de $G(p)$ y existe un entorno $V_{0}\subset N$ de
	$F(G(p))$ tales que $F|_{U_{0}}:\,U_{0}\rightarrow V_{0}$ es
	difeomorfismo. Sea $V\subset V_{0}$ el dominio de una carta $(V,\psi)$
	para $N$ en $F(G(p))$. Sea $U=F^{-1}(V)$ y sea
	$\varphi:\,U\rightarrow\bb{R}^{n}$ dada por $\varphi=\psi\circ F|_{U}$.
	Entonces $(U,\varphi)$ es una carta para $M$ en $G(p)$ contenida en
	$U_{0}$. Como $G$ es continua, $G^{-1}(U)\subset P$ es abierto y
	contiene a $p$. Sea $(W,\gamma)$ una carta en $p$ con
	$W\subset G^{-1}(U)$. Como $F\circ G$ es suave,
	\begin{align*}
		\psi\circ(F\circ G)\circ\gamma^{-1} & \,:\,\gamma(W)\,
			\rightarrow\,\psi(V)
	\end{align*}
	%
	es suave. Pero $\psi\circ(F\circ G)\circ\gamma^{-1}=%
	(\psi\circ F\circ\varphi^{-1})\circ (\varphi\circ G\circ\gamma^{-1})$
	y $\psi\circ F\circ\varphi^{-1}$ es difeomorfismo (de hecho,
	$\psi\circ F\circ\varphi^{-1}=\psi$). Entonces
	$\varphi\circ G\circ\gamma^{-1}$ es suave, de lo que se deduce que
	$G$ es suave.

	\emph{(b)}sea $F:\,M\rightarrow N$ un difeomorfismo local
	sobreyectivo y sea $G':\,N\rightarrow P'$ una funci\'{o}n tal que
	$G'\circ F:\,M\rightarrow P'$ es suave. Sea $p\in N$. Por
	sobreyectividad, existe $q\in M$ tal que $F(q)=p$. Como
	$G'\circ F$ es suave, existen cartas $(U,\varphi)$ para $M$ en
	$q$ y $(V,\psi)$ para $P'$ en $G'(p)$ tales que
	$G'\circ F(U)\subset V$ y $\psi\circ (G'\circ F)\circ\varphi^{-1}:\,%
	\varphi(U)\rightarrow\psi(V)$ es suave. Como $F$ es un difeomorfismo
	local, $F$ es abierta y $F(U)\subset N$ es un subconjunto abierto
	que contiene a $p$. Podemos tomar, entonces, una carta $(W,\gamma)$
	para $N$ en $p$, con dominio $W\subset F(U)$. En particular,
	\begin{align*}
		G'(W) & \,\subset\,G'(F(U))\,\subset\, V\quad\text{y} \\
		\psi\circ(G'\circ F)\circ\varphi^{-1} & \,=\,
			(\psi\circ G'\circ\gamma^{-1})\circ
			(\gamma\circ F\circ\varphi^{-1})
		\text{ .}
	\end{align*}
	%
	Tomando la carta $(U,\varphi)$ de manera que
	$F|_{U}:\,U\rightarrow F(U)$ sea difeomorfismo --esto se puede hacer
	si primero fijamos entornos $U_{0}$ de $q$ y $V_{0}$ de $p$ de manera
	que $F|_{U_{0}}:\,U_{0}\rightarrow V_{0}$ sea difeomorfismo y
	eligiendo $(U,\varphi)$ con $U\subset U_{0}$--, la composici\'{o}n
	$\gamma\circ F\circ\varphi^{-1}$ resulta ser un difeomorfismo y,
	entonces, $\psi\circ G'\circ\gamma^{-1}$ debe ser suave, por ser
	composici\'{o}n de dos funciones suaves:
	\begin{align*}
		\psi\circ G'\circ\gamma^{-1} & \,=\,
			(\psi\circ (G'\circ F)\circ\varphi^{-1})\circ
			(\gamma\circ F\circ\varphi^{-1})^{-1}
		\text{ .}
	\end{align*}
	%
\end{proof}

%
\section{El teorema del rango constante}
\theoremstyle{plain}
\newtheorem{teoDelRango}{Teorema}[section]
\newtheorem{coroDelRangoLineal}[teoDelRango]{Corolario}
\newtheorem{coroDelRangoGlobal}[teoDelRango]{Corolario}
\newtheorem{teoInmersionConBorde}[teoDelRango]{Teorema}

\theoremstyle{remark}
\newtheorem{obsInmersionConBorde}{Observaci\'{o}n}[section]

%-------------

Empezamos enunciando y demostrando el resultado principal de esta secci\'{o}n.

\begin{teoDelRango}[del rango]\label{thm:delrango}
	Sea $F:\,M\rightarrow N$ una transformaci\'{o}n suave entre variedades
	\emph{sin} borde. Si $F$ tiene rango constante $r$, entonces, para
	cada punto $p\in M$, existen una carta $(U,\varphi)$ para $M$
	centrada en $p$ y otra carta $(V,\psi)$ para $N$ centrada en $F(p)$
	tales que $F(U)\subset V$ y $\widehat{F}=%
	\psi\circ F\circ\varphi^{-1}$ es de la forma
	\begin{align*}
		\widehat{F}(x^{1},\,\dots,\,x^{r},\,x^{r+1},\,\dots,\,x^{m}) &
			\,=\,(x^{1},\,\dots,\,x^{r},\,0,\,\dots,\,0)
		\text{ .}
	\end{align*}
	%
\end{teoDelRango}

\begin{proof}
	Sea $p\in M$ y sean $U$ y $V$ dominios de cartas en $p$ y en $F(p)$,
	respectivamente, tales que $F(U)\subset V$. Tomando coordenadas, si
	el teorema se demuestra reemplazando $M$ por $U$, $N$ por $V$ y
	$F$ por $\widehat{F}$, componiendo las cartas obtenidas con las
	anteriores, quedar\'{a} demostrado el caso general. Supongamos,
	entonces, sin p\'{e}rdida de generalidad, que $M=U\subset\bb{R}^{m}$
	y que $N=V\subset\bb{R}^{n}$ son abiertos, $p=0\in U$ y
	$F(p)=0\in V$. como $\rango{\jacobiana[p]{F}}=r$, alg\'{u}n menor
	de la matriz jacobiana de tama\~{n}o $r\times r$ es no nulo.
	Reordenando las coordenadas de $U$, podemos asumir que es el menor
	correspondiente a la submatriz
	\begin{math}
		\left[\begin{smallmatrix}
			\derivada{F^{i}}{x^{j}}
		\end{smallmatrix}\right]_{i,j\in[\![1,r]\!]}
	\end{math}.
	Sean $(\lista*{x}{r},\,\lista*{y}{m-r})$ las coordenadas en $U$ y
	sean $(\lista*{v}{r},\,\lista*{w}{n-r})$ las coordenadas en $V$. Con
	respecto a estas coordenadas,
	\begin{align*}
		F(x,y) & \,=\, (Q(x,y),R(x,y))
	\end{align*}
	%
	para ciertas funciones suaves $Q:\,U\rightarrow\bb{R}^{r}$ y
	$R:\,U\rightarrow\bb{R}^{n-r}$. Por hip\'{o}tesis, la matriz
	\begin{math}
		\left[\begin{smallmatrix}
			\derivada{Q^{i}}{x^{j}}
		\end{smallmatrix}\right]_{i,j\in[\![1,r]\!]}
	\end{math}
	es no singular. Extendemos $Q$ como en el teorema de la funci\'{o}n
	impl\'{\i}cita: sea $\Phi:\,U\rightarrow\bb{R}^{m}$ dada por
	$\Phi(x,y)=(Q(x,y),y)$. La matriz jacobiana de $\Phi$ en $(x,y)$ es
	igual a
	\begin{align*}
		\jacobiana[(x,y)]{\Phi} & \,=\,
		\sbox0{$
		\id[m-r]
		$}
		\sbox1{$
		\derivada{Q^{i}}{y^{j}}
		$}
		\left[
		\begin{array}{c|c}
			\makebox[\wd0]{$\derivada{Q^{i}}{x^{j}}$} &
				\usebox{1} \\
			\hline
			\vphantom{\usebox{1}}\makebox[\wd0]{$0$} &
				\usebox{0}
		\end{array}
		\right]
		\text{ .}
	\end{align*}
	%
	Por hip\'{o}tesis, $\left|\jacobiana[(0,0)]{\Phi}\right|\not =0$,
	con lo cual, por el teorema de la funci\'{o}n inversa, existen
	entornos (conexos) $U_{0}$ de $(0,0)$ y $\widehat{U}_{0}$ de
	$\Phi(0,0)=(0,0)$ tales que $\Phi|_{U_{0}}:\,%
	U_{0}\rightarrow\widehat{U}_{0}$ es difeomorfismo. Cambiamos
	$\widehat{U}_{0}$ por un cubo de la forma
	$\cubo{\epsilon}{0,0}\subset\widehat{U}_{0}$ y $U_{0}$ por
	$\Phi|_{U_{0}}^{-1}\big(\cubo{\epsilon}{0,0}\big)$. Sea
	$\varphi=\Phi|_{U_{0}}$.

	Ahora bien, la inversa $\varphi^{-1}:\,%
	\widehat{U}_{0}\rightarrow U_{0}$ tambi\'{e}n es de la forma
	\begin{align*}
		\varphi^{-1}(\xi,\upsilon) & \,=\,
			(A(\xi,\upsilon),B(\xi,\upsilon))
	\end{align*}
	%
	para ciertas funciones suaves $A,B$. Entonces
	\begin{align*}
		(\xi,\upsilon) & \,=\,\varphi\circ\varphi^{-1}(\xi,\upsilon)
			\,=\,(Q(A,B),B) \quad\text{y} \\
		B(\xi,\upsilon) & \,=\,\upsilon\quad\text{y} \\
		\xi & \,=\,Q(A(\xi,\upsilon),\upsilon)
		\text{ .}
	\end{align*}
	%
	Componiendo con $F$,
	\begin{align*}
		F\circ\varphi^{-1}(\xi,\upsilon) & \,=\,(Q(A,B),R(A,B))
			\,=\,(\xi,R(A(\xi,\upsilon),\upsilon))
		\text{ .}
	\end{align*}
	%
	Sea $\tilde{R}(\xi,\upsilon)=R(A(\xi,\upsilon),\upsilon)$. La matriz
	jaconiana de la composici\'{o}n en un punto $(\xi,\upsilon)$ est\'{a}
	dada por
	\begin{align*}
		\jacobiana[(0,0)]{(F\circ\varphi^{-1})} & \,=\,
		\sbox0{$\id[r]$}
		\sbox1{$\derivada{\tilde{R}^{i}}{\xi^{j}}$}
		\left[
		\begin{array}{c|c}
			\vphantom{\usebox{1}}\usebox{0} &
			\makebox[\wd0]{$0$} \\
			\hline
			\usebox{1} &
			\makebox{$\derivada{\tilde{R}^{i}}{\upsilon^{j}}$}
		\end{array}
		\right]
		\text{ .}
	\end{align*}
	%
	Como $F$ tiene rango exactamente $r$ en todo $U$ y $\varphi$ es
	difeomorfismo, $\jacobiana[(\xi,\upsilon)]{(F\circ\varphi^{-1})}$
	tiene rango $r$ en todo par $(\xi,\upsilon)$. Entonces debe valer
	que
	\begin{math}
		\left[\begin{smallmatrix}
			\derivada{\tilde{R}^{i}}{\upsilon^{j}}
		\end{smallmatrix}\right]
	\end{math}
	es la matriz nula, es decir,
	$\derivada{\tilde{R}^{i}}{\upsilon^{j}}(\xi,\upsilon)=0$ para todo
	$(\xi,\upsilon)\in\widehat{U}_{0}=\cubo{\epsilon}{0,0}$. En otras
	palabras, $\tilde{R}$ no depende de $\upsilon$. Sea
	$S(\xi)=\tilde{R}(\xi,0)$. Entonces
	\begin{align*}
		F\circ\varphi^{-1}(\xi,\upsilon) & \,=\,
			(\xi,\tilde{R}(\xi,\upsilon)) \,=\,
			(\xi,S(\xi))
		\text{ .}
	\end{align*}
	%
	Esto es casi lo que buscamos, pues $F\circ\varphi^{-1}$ es la
	identidad en las primeras $r$ coordenadas.

	Sea $V_{0}$ el subconjunto definido por
	\begin{align*}
		V_{0} & \,=\,\left\lbrace (v,w)\in V\,:\,
			(v,0)\in\widehat{U}_{0}=\cubo{\epsilon}{0,0}
			\right\rbrace
		\text{ .}
	\end{align*}
	%
	Es decir, $V_{0}$ es la preimagen por
	\begin{align*}
		\lambda v.\lambda w.(v,0) & \,:\,
			V\,\rightarrow\,\widehat{U}_{0}
		\text{ ,}
	\end{align*}
	%
	que es continua. Entonces $V_{0}$ es abierto y contiene al
	punto $(0,0)$. Ahora bien, si $(\xi,\upsilon)\in\widehat{U}_{0}$,
	entonces
	\begin{align*}
		F\circ\varphi^{-1}(\xi,\upsilon) & \,=\,
			(\xi,S(\xi))
		\text{ .}
	\end{align*}
	%
	Como $(\xi,0)\in\widehat{U}_{0}$ (porque es un cubo), vale que
	$F\circ\varphi^{-1}(\xi,\upsilon)\in V_{0}$. As\'{\i},
	$F\circ\varphi^{-1}(\widehat{U}_{0})\subset V_{0}$ y
	$F(U_{0})\subset V_{0}$. Ahora hay que definir un cambio de
	coordenadas, un difeomorfismo, en $V_{0}$ de manera que, al ser
	restringido a la imagen $F(U_{0})$, coincida con la proyecci\'{o}n
	en las primeras $r$ coordenadas. Sea
	$\psi:\,V_{0}\rightarrow\bb{R}^{n}$ dada por
	$\psi(v,w)=(v,w-S(v))$. Esta funci\'{o}n es invertible, con inversa
	dada por $\psi^{-1}(t,u)=(t,u+S(t))$. En particular,
	$\psi$ y $\psi^{-1}$ son suaves y $\psi$ es un difeomorfismo en la
	imagen. Componiendo,
	\begin{align*}
		\psi\circ F\circ\varphi^{-1}(\xi,\upsilon) & \,=\,
			\psi(\xi,S(\xi)) \,=\,(\xi,0)
		\text{ .}
	\end{align*}
	%
\end{proof}

\begin{coroDelRangoLineal}\label{thm:localmentelineal}
	Sea $F:\,M\rightarrow N$ una transformaci\'{o} suave entre variedades
	sin borde. Supongamos que $M$ es conexa. Entonces $F$ es de rango
	constante, si y s\'{o}lo si, para cada $p\in M$, existen entornos
	coordenados de $p$ y de $F(p)$ con respecto a los cuales la
	representaci\'{o}n de $F$ en coordenadas es lineal.
\end{coroDelRangoLineal}

\begin{proof}
	Si $F$ tiene rango constante, por el teorema \ref{thm:delrango},
	$F$ tiene una expresi\'{o}n lineal en coordenadas (m\'{a}s
	precisamente, es una proyecci\'{o}n). Rec\'{\i}procamente, si
	$F$ tiene una expresi\'{o}n lineal en coordenadas cerca de cada
	punto, entonces, como el rango de una transformaci\'{o}n lineal
	es constante, $\rango{F}$ es localmente constante.
	Como $M$ es conexa, $\rango{F}$ debe ser constante.
\end{proof}

\begin{coroDelRangoGlobal}\label{thm:delrangoglobal}
	Sean $M$ y $N$ variedades sin borde y sea $F:\,M\rightarrow N$ de
	rango constante $r$. si $F$ es sobreyectiva, entonces es
	submersi\'{o}n ($r=\dim\,N$); si $F$ es inyectiva, entonces
	es inmersi\'{o}n ($r=\dim\,M$); si $F$ es biyectiva, entonces es
	un difeomorfismo.
\end{coroDelRangoGlobal}

\begin{proof}
	Si $r<\dim\,N$, para cada punto $p\in M$, existen entorns coordenados
	$U_{p}$ de $p$ y $V_{p}$ de $F(p)$ tales que
	$\widehat{F}(x)=(\lista*{x}{r},\,0,\,\dots,\,0)$. Sea
	$U_{p}'\subset U_{p}$ un entorno de $p$ con clausura compacta
	contenida en $U_{p}$. Entonces $F(\clos{U_{p}'})$ es compacto y
	est\'{a} contenido en $V\cap\{y^{r+1}=\,\cdots\,=y^{n}=0\}$. Los
	conjuntos $U_{p}'$ cubren $M$. Sea $\{U_{i}\}_{i}$ un subcubrimiento
	numerable. Como los conjuntos $F(\clos{U_{i}})$ son cerrados
	nunca densos y
	\begin{align*}
		F(M) & \,=\,F\Big(\bigcup_{i\geq 1}\,\clos{U_{i}}\Big)
			\,=\,\bigcup_{i\geq 1}\,F(\clos{U_{i}})
		\text{ ,}
	\end{align*}
	%
	no puede ser que $F(M)=N$, pues $N$ es localmente compacto
	Hausdorff.

	Si $F$ no es una inmersi\'{o}n ($r<\dim\,M$), entonces, dado
	un punto $p\in M$, en un entorno suficientemente peque\~{n}o,
	\begin{align*}
		\widehat{F}(x^{1},\,\dots,\,x^{r},\,x^{r+1},\,\dots,\,x^{m})
			& \,=\,(\lista*{x}{r},\,0,\,\dots,\,0)
		\text{ .}
	\end{align*}
	%
	En particular, $\widehat{F}(0,\,dots,\,0,\,x^{r+1},\,\dots,\,x^{m})=%
	\widehat{F}(0,\,\dots,\,0,\,0,\,\dots,\,0)$ para $|x^{i}|<\epsilon$ y
	$F$ no es inyectiva.

	So $F$ es biyectiva, entonces debe ser inmersi\'{o}n y submersi\'{o}n.
	Por el teorema de la funci\'{o}n inversa, es difeomorfismo local.
	Como es biyectiva, su inversa coincide localmente con una funci\'{o}n
	suave. En definitiva, $F$ es difeomorfismo.
\end{proof}

Consideremos la inclusi\'{o}n del semiespacio $\hemi[n]$ en $\bb{R}^{n}$.
Esta funci\'{o}n es suave y su diferencial es un isomorfismo en todos los
puntos. Pero, por invarianza de dominio, no pueden ser localmente
difeomorfos. Espec\'{\i}ficamente, no hay ning\'{u}n entorno de un punto
$x\in\borde[{\hemi[n]}]$ que sea difeomorfo a un abierto de $\bb{R}^{n}$.
En otras palabras, no hay un teorema general de la funci\'{o}n inversa como
\ref{thm:funinvvar} para transformaciones $F:\,M\rightarrow N$ en donde
$\borde[M]\not =\varnothing$. Aun as\'{\i}, como vimos al definir la
estructura diferencial natural en el borde de una variedad con borde, es
posible obtener un resultado similar al teorema del rango constante en
algunos casos particulares. En el caso del borde de una variedad,
$\borde[M]\hookrightarrow M$, sabemos que, dada una carta $(U,\varphi)$
para $M$ en $p\in\borde[M]$, la composici\'{o}n
\begin{align*}
	\varphi\circ\inc[{\borde[M]}]\circ\overline{\varphi}^{-1}
		(\lista*{u}{n-1}) & \,=\,(\lista*{u}{n-1},\,0)
	\text{ ,}
\end{align*}
%
donde $\overline{\varphi}=\pi\circ\varphi\circ\inc[{\borde[M]}]:\,%
\borde[M]\cap U\rightarrow\pi(\varphi(U))$.

Sea $M$ una variedad \emph{con} borde $\borde[M]\not =\varnothing$ y sea
$F:\,M\rightarrow N$ una transformaci\'{o}n suave. Como la inclusi\'{o}n
$\inc[\interior{M}]:\,\interior{M}\hookrightarrow M$ es suave y tiene rango
m\'{a}ximo, en los puntos del interior de $M$, podemos aplicar el teorema
del rango constante, de cumplir $F$ con las condiciones del mismo, para
obtener una representaci\'{o}n de $F$ alrededor de cada punto del interior
como una proyecci\'{o}n. Pero, en un punto $p\in\borde[M]$ del borde de
$M$, esto no es cierto \emph{a priori}.

\begin{teoInmersionConBorde}[Inmersi\'{o}n de variedades con borde]%
	\label{thm:inmersionconborde}
	Sea $M$ una variedad con borde $\borde[M]\not =\varnothing$ y sea
	$m=\dim\,M$. Sea $N$ una variedad \emph{sin} borde de dimensi\'{o}n
	$n$ y sea $F:\,M\rightarrow N$ una inmersi\'{o}n suave. Si
	$p\in\borde[M]$, existe una carta $(U,\varphi)$ para $M$ en $p$
	y existe una carta $(V,\psi)$ para $N$ en $F(p)$ tales que
	$F(U)\subset V$ y
	\begin{align*}
		\widehat{F}(\lista*{x}{m}) & \,=\,
			(\lista*{x}{m},\,0,\,\dots,\,0)
		\text{ ,}
	\end{align*}
	%
	donde $\widehat{F}=\psi\circ F\circ\varphi^{-1}$.
\end{teoInmersionConBorde}

\begin{proof}
	Podemos suponer que $N=V\subset\bb{R}^{n}$ y que
	$M=U\subset\hemi[n]$ son abiertos y que $p=0$ y que $F(p)=0$.
	Por definici\'{o}n, existe $\widetilde{F}:\,W\rightarrow V$ definida
	en un abierto de $\bb{R}^{m}$ con $p\in W$ y
	$\widetilde{F}|_{W\cap U}=F$. En particular,
	\begin{align*}
		\diferencial[0]{\widetilde{F}} & \,=\,\diferencial[0]{F}
		\text{ .}
	\end{align*}
	%
	Como $F$ tiene rango m\'{a}ximo en $0$, la extensi\'{o}n
	$\widetilde{F}$ debe tener rango m\'{a}ximo en $0$. Achicando $W$,
	podemos suponer que $\widetilde{F}$ tiene rango m\'{a}ximo en
	todo el abierto $W$. Por el teorema del rango, existen cartas
	$(W_{0},\gamma_{0})$ para $\bb{R}^{m}$ centrada en $0$ y
	$(V_{0},\psi_{0})$ para $\bb{R}^{n}$ centrada en $0$ tales que
	\begin{align*}
		\psi_{0}\circ\widetilde{F}\circ\gamma_{0}^{-1}
			(\lista*{x}{m}) & \,=\,
			(\lista*{x}{m},\,0,\,\dots,\,0)
		\text{ .}
	\end{align*}
	%
	Pero, al elegir estas cartas, no hay control sobre lo que pasa con
	el borde de $M$: la imagen $\gamma_{0}(\borde[M]\cap W_{0})$
	podr\'{\i}a ser cualquier cosa --casi. Como
	$\gamma_{0}:\,W_{0}\rightarrow\widehat{W}_{0}=\gamma_{0}(W_{0})%
	\subset\bb{R}^{m}$ es un difeomorfismo, el producto
	$\gamma_{0}\times\id[\bb{R}^{n-m}]:\,W_{0}\times\bb{R}^{n-m}%
	\rightarrow\widehat{W}_{0}\times\bb{R}^{n-m}$ es un difeomorfismo,
	tambi\'{e}n. La composici\'{o}n
	$\psi=(\gamma_{0}^{-1}\times\id[\bb{R}^{n-m}])\circ\psi_{0}$ es un
	difeomorfismo definido en un entorno $V_{1}\subset V_{0}$ de $0$.
	% $V_{0}\cap(\widehat{W}_{0}\times\bb{R}^{n-m})$ de $0\in\bb{R}^{n}$.
	As\'{\i}, $(V_{1},\psi)$ es una carta para $V$ centrada en $0$ y
	\begin{align*}
		\psi\circ F (x) & \,=\,
			(\gamma_{0}^{-1}\times\id[\bb{R}^{n-m}])\circ
			(\psi_{0}\circ F\circ\gamma_{0}^{-1})\circ
			\gamma_{0}(x) \\
		& \,=\,(\gamma_{0}^{-1}\times\id[\bb{R}^{n-m}])
			(\gamma_{0}^{1}(x),\,\dots,\,\gamma_{0}^{m}(x),\,0,\,
				\dots,\,0) \,=\,(x,0)
		\text{ .}
	\end{align*}
	%
	En definitiva, usando las coordenadas originales de $M$ y $\psi$
	se obtiene una erpresentaci\'{o}n de $F$ de la forma deseada.
\end{proof}

\begin{obsInmersionConBorde}\label{obs:inmersionconborde}
	Supongamos que $F$ es submersi\'{o}n y $\borde[M]\not =\varnothing$
	como en el teorema \ref{thm:inmersionconborde}. Entonces argumentando
	de manera similar, podemos hallar cartas tales que
	\begin{align*}
		& F\circ (\gamma_{0}^{-1}\circ
			(\psi_{0}\times\id[\bb{R}^{m-n}]))(x) \\
		& \qquad\qquad
			\,=\,\psi_{0}^{-1}(\psi_{0}\circ F\circ\gamma_{0}^{-1})
				(\psi_{0}(\lista*{x}{n}),\,x^{n+1},\,\dots,\,
					x^{m}) \\
		& \qquad\qquad
			\,=\, (\lista*{x}{n})
		\text{ .}
	\end{align*}
	%
	Pero ?` es $(\psi_{0}^{-1}\times\id[\bb{R}^{m-n}])\circ\gamma_{0}$
	una carta de borde? En general, no, pues ser\'{a} de borde, si
	y s\'{o}lo si $\gamma_{0}$ lo es\dots

	?`Qu\'{e} pasa en el caso en que $\borde[N]\not=\varnothing$ y
	$\borde[M]=\varnothing$ y $F:\,M\rightarrow N$ sea submersi\'{o}n?
	En vez de considerar una extensi\'{o}n $\widetilde{F}$, habr\'{a}
	que considerar la composici\'{o}n $F_{0}=\inc[{\hemi[n]}]\circ F$.
	Como $N$ tiene borde, no est\'{a} garantizado que existan secciones
	locales suaves para $F$ y que, por lo tanto, $F$ sea abierta. Pero
	$F_{0}$ es submersi\'{o}n (una cuesti\'{o}n de n\'{u}meros y regla
	de la cadena) entre variedades \emph{sin} borde. Sea $U$ un entorno
	coordenado en $M$ centrado en $p$ y sea $V$ un entorno coordenado
	en $N$ centrado en $F(p)$. Entonces, el problema queda reducido
	al caso en que $M$ es un abierto de $\bb{R}^{m}$ y $N$ es un
	abierto de $\hemi[n]$ ($n\leq m$), $p=0$ y $F(p)=0$. Componiendo
	con la inclusi\'{o}n de $\hemi[n]$ en $\bb{R}^{n}$, la funci\'{o}n
	$F_{0}$ es submersi\'{o}n (su rango es igual al rango de $F$).
	Por el teorema del rango, existen cartas $(U_{0},\varphi_{0})$ para
	$\bb{R}^{m}$ centrada en $p=0$ y $(V_{0},\psi_{0})$ para
	$\bb{R}^{n}$ centrada en $F(p)=0$ de manera que
	\begin{align*}
		\psi_{0}\circ (\inc[{\hemi[n]}]\circ F)\circ\varphi_{0}^{-1}
			(x^{1},\,\dots,\,x^{n},\,x^{n+1},\,\dots,\,x^{m}) &
			\,=\,(\lista*{x}{n})
		\text{ .}
	\end{align*}
	%
	Completamos $\psi_{0}$ a un difeo definido en un abierto de
	$\bb{R}^{m}$: concretamente, $\psi_{0}\times\id[\bb{R}^{m-n}]$ es
	un difeomorfismo de $V_{0}\times\bb{R}^{m-n}$ en
	$\widehat{V}_{0}\times\bb{R}^{m-n}$, digamos, que es un entorno de
	$0\in\bb{R}^{m}$. Entonces la composici\'{o}n
	$(\psi_{0}^{-1}\times\id[\bb{R}^{m-n}])\circ\varphi_{0}^{-1}$ es un
	difeomorfismo definido en alg\'{u}n entorno de $0$ y vale que
	\begin{align*}
		& \inc[{\hemi[n]}]\circ F\circ
			(\varphi_{0}^{-1}\circ
				(\psi_{0}\times\id[\bb{R}^{m-n}]))
			(x^{1},\,\dots,\,x^{n},\,x^{n+1},\,\dots,\,x^{m}) \\
		&\qquad\qquad
			\,=\,\psi_{0}^{-1}\circ
				(\psi_{0}\circ F_{0}\circ\varphi_{0}^{-1})
				(\psi_{0}(x^{1},\,\dots,\,x^{n}),\,x^{n+1},\,
				\dots,\,x^{m}) \\
		& \qquad\qquad
			\,=\,\psi_{0}^{-1}(\psi_{0}^{1}(x),\,
				\dots,\,\psi_{0}^{n}(x)) \\
		& \qquad\qquad
			\,=\,(\lista*{x}{n})
		\text{ .}
	\end{align*}
	%

	Del hecho de que $F_{0}$ es submersi\'{o}n, podemos deducir que
	$F_{0}:\,U\rightarrow \widetilde{V}$ es abierta (porque existen
	secciones locales suaves), donde $\widetilde{V}\subset\bb{R}^{n}$ es
	abierto tal que $V=\hemi[n]\cap\widetilde{V}$ (podemos tomar una
	semibola y la bola correspondiente). Pero, como su imagen est\'{a}
	contenida en $\hemi[n]$, los puntos de $F_{0}(U)$ deben pertenecer al
	interior de $V$, a $\interior{\hemi[n]}\cap V$. Sabiendo que
	$F$ se correstringe a una submersi\'{o}n entre variedades sin borde,
	podemos aplicar el teorema del rango sin problema.
\end{obsInmersionConBorde}

%

%--------

\chapter{Subvariedades}
\section{Embeddings}
\theoremstyle{plain}
\newtheorem{teoEmbeddingLocal}{Teorema}[section]
\newtheorem{propoCuandoInmersionEsEmbedding}[teoEmbeddingLocal]%
	{Proposici\'{o}n}

\theoremstyle{remark}
\newtheorem{obsEmbeddingLocal}{Observaci\'{o}n}[section]

%-------------

Un \emph{embedding (suave)} es una inmersi\'{o}n (suave) $F:\,M\rightarrow N$
que es, adem\'{a}s, un \emph{embedding topol\'{o}gico}, es decir, un
subespacio. Un embedding suave es un subespacio (embedding topol\'{o}gico)
que es suave \emph{y que, adem\'{a}s,} tiene diferencial inyectivo,
de rango m\'{a}ximo en toda $M$.

\begin{propoCuandoInmersionEsEmbedding}\label{thm:cuandoinmersionesembedding}
	Sean $M$ y $N$ variedades diferenciales y $F:\,M\rightarrow N$
	una inmersi\'{o}n suave. Si $F$ es inyectiva y cumple con
	cualquiera de las siguientess propiedades, entonces $F$ es
	(subespacio y, por lo tanto,) embedding: \emph{(a)} $F$ es abierta
	o cerrada; \emph{(b)} $F$ es propia; \emph{(c)} $M$ es compacta;
	\emph{(d)} $\borde[M]=\varnothing$ y $\dim\,M=\dim\,N$.
\end{propoCuandoInmersionEsEmbedding}

\begin{proof}
	Si $F$ es abierta o cerrada, entonces es subespacio. Si $F$ es
	propia, entonces es cerrada. Si $M$ es compacta, como $F$ es
	continua, todo cerrado de $M$ es compacto y todo compacto de $N$
	es cerrado, entonces $F$ es propia y cerrada. Si
	$\borde[M]=\varnothing$ y $\dim\,M=\dim\,N$, sabemos que
	$F=\inc[\interior{N}]\circ F|$, donde
	$F|:\,M\rightarrow\interior{N}$ es la correstricci\'{o}n de $F$.
	Como $F|$ es difeomorfismo local (por dimensi\'{o}n), resulta ser
	abierta. La inclusi\'{o}n $\inc[\interior{N}]$ tambi\'{e}n es
	abierta. Entonces $F$ es abierta y luego embedding.
\end{proof}

Hay embeddings que no son ni abiertos ni cerrados: por ejemplo,
$\hemi[n]=\{x^{n}\geq 0\}\hookrightarrow\bb{R}^{n}$ no es abierta,
$\{x^{n}<1\}\hookrightarrow\bb{R}^{n}$ no es cerrada y
$\{x^{n}\geq 0\}\cap\{x^{n}<1\}\hookrightarrow\bb{R}^{n}$ no es ni
abierta, ni cerrada, pero todas son embeddings.

Toda inmersi\'{o}n es localmente un embedding.

\begin{teoEmbeddingLocal}\label{thm:embeddinglocal}
	Sean $M$ y $N$ variedades diferenciales. Sea $F:\,M\rightarrow N$
	una transformaci\'{o}n suave. Entonces $F$ es una inmersi\'{o}n,
	si y s\'{o}lo si, para todo punto $p\in M$, existe un entorno
	$p\in U\subset M$ tal que $F|_{U}:\,U\rightarrow N$ es embedding.
\end{teoEmbeddingLocal}

\begin{proof}
	Supongamos que $F$ tiene rango completo. Si $F(p)\not\in\borde[N]$,
	entonces $F$ no toma valores en $\borde[N]$ en un entorno de $p$.
	Podemos aplicar, o bien el teorema del rango constante, o el
	teorema de inmersi\'{o}n de una variedad con borde (dependiendo de
	$M$) para concluir que
	$\widehat{F}(\lista*{x}{m})=(\lista*{x}{m},\,0,\,\dots,\,0)$
	en un entorno de $p$. En particular, $F$ es inyectiva en dicho
	entorno. Sabiendo que $F$ es inyectiva en un entorno
	$U_{1}\subset M$ de $p$, podemos elegir $U\subset M$ abierto,
	con $p\in U$ y $\clos{U}$ compacta y contenida en $U_{1}$.
	Entonces $F|_{\clos{U}}:\,\clos{U}\rightarrow N$ es continua
	e inyectiva con dominio compacto y codominio Hausdorff. La
	restricci\'{o}n $F|_{\clos{U}}$ es subespacio. En particular,
	$F|_{U}:\,U\rightarrow N$ es subespacio, es suave y es inmersi\'{o}n,
	es decir, $F|_{U}$ es embedding.

	Si $F(p)\in\borde[N]$, podemos hallar un entorno de $p$ en donde $F$
	es inyectiva y, con el mismo argumente que en el caso anterior,
	probar que $F$ es un embedding restringida a un entorno de $p$.
	Si $(W,\psi)$ es una carta para $N$ centrada en $F(p)$, una
	carta de borde, como $F:\,M\rightarrow N$ es continua, la preimagen
	$U=F^{-1}(W)$ es abierta y contiene a $p$. La composici\'{o}n
	\begin{align*}
		\inc[{\hemi[n]}]\circ\psi\circ F & \,:\,
			U\,\rightarrow\,\psi(V)\subset\bb{R}^{n}
	\end{align*}
	%
	es suave y su diferencial,
	$\diferencial[p]{(\inc[{\hemi[n]}]\circ\psi\circ F)}$, es inyectivo.
	En definitiva, $\inc[{\hemi[n]}]\circ\psi\circ F$ es una
	inmersi\'{o}n de $U$ en $\bb{R}^{n}$, por lo que, aplicando el
	teorema del rango \ref{thm:} (si $p\not\in\borde[M]$) o el teorema
	de la inmersi\'{o}n \ref{thm:}, existe un entorno $U_{1}$ de $p$
	contenido en $U$ tal que $\inc[{\hemi[n]}]\circ\psi\circ F|_{U_{1}}$
	es inyectiva. Pero entonces $F|_{U_{1}}$ es inyectiva.

	Rec\'{\i}procamente, si $F$ es localmente embeddingm entonces
	en todo punto $p\in M$ el difernecial $\diferencial[p]{F}$ es
	inyectivo.
\end{proof}

\begin{obsEmbeddingLocal}\label{obs:embeddinglocal}
	Notemos que este argumento es bastante similar al usado en la
	demostraci\'{o}n de la proposici\'{o}n \ref{thm:}.
\end{obsEmbeddingLocal}

Podemos definir una \emph{inmersi\'{o}n topol\'{o}gica} como una
funci\'{o}n $f:\,X\rightarrow Y$ tal que, para todo punto $p\in X$,
existe un entorno $U\subset X$ tal que $f|_{U}:\,U\rightarrow Y$ es
un \emph{embedding topol\'{o}gico}, es decir, un subespacio. Una inmersi\'{o}n
suave es una inmersi\'{o}n topol\'{o}gica que es suave y que, adem\'{a}s,
tiene rango m\'{a}ximo.


%
\section{Submersiones}
\theoremstyle{plain}
\newtheorem{teoSeccionesLocales}{Teorema}[section]
\newtheorem{propoSubmersionEsAbierta}[teoSeccionesLocales]{Proposici\'{o}n}
\newtheorem{coroSubmersionSobreEsCociente}[teoSeccionesLocales]{Corolario}
\newtheorem{propoSubmersionesExtra}[teoSeccionesLocales]{Proposici\'{o}n}
\newtheorem{coroUnicidadDelCociente}[teoSeccionesLocales]{Corolario}

\theoremstyle{remark}
\newtheorem{obsCaracterizaSubmersionesSobre}{Observaci\'{o}n}[section]

%-------------

Empecemos con la siguiente definici\'{o}n: una \emph{secci\'{o}n} de un
morfismo $\pi:\,M\rightarrow N$ es un morfismo $\sigma:\,N\rightarrow M$
tal que $\pi\circ\sigma =\id[N]$, El siguiente diagrama es ilustrativo
en este sentido:
\begin{center}
\begin{tikzcd}
	M\arrow[d,"\pi"'] \\
	N\arrow[u, bend right=50, "\sigma"']
\end{tikzcd}
\end{center}

Si $\pi$ es una funci\'{o}n continua, una secci\'{o}n $\sigma$ de $\pi$ es
una funci\'{o}n \emph{continua} tal que $\pi\circ\sigma=\id[N]$; si
$\pi$ es suave, una secci\'{o}n $\sigma$ de $\pi$ es una funci\'{o}n
\emph{suave} tal que $\pi\circ\sigma=\id[N]$. En general, usaremos los
t\'{e}rminos \emph{secci\'{o}n continua} o, respectivamente,
\emph{secci\'{o}n suave} para evitar ambig\"{u}edades. una
\emph{secci\'{o}n local} de una funci\'{o}n (\emph{continua}) $\pi$ es
una funci\'{o}n (continua) $\sigma:\,U\rightarrow M$ definida en un abierto
$U\subset N$ y tal que $\pi\circ\sigma=\id[U]$.

\begin{teoSeccionesLocales}\label{thm:seccioneslocales}
	Sean $M$ y $N$ variedades diferenciales (\emph{sin} borde) y sea
	$\pi:\,M\rightarrow N$ una transformaci\'{o}n suave. Entonces
	$\pi$ es una submersi\'{o}n, si y s\'{o}lo si, para cada punto
	$p\in M$, existe una seccci\'{o}n (suave) local
	$\sigma:\,U\rightarrow M$ de $\pi$ tal que $p\in\sigma(U)$.
\end{teoSeccionesLocales}

Es decir, $\pi$ es una submersi\'{o}n, si todo punto del dominio pertenece
a la imagen de una secci\'{o}n local.

\begin{proof}
	Supongamos que $\pi:\,M\rightarrow N$  es una submersi\'{o}n y sea
	$p\in M$. Por el teorema del rango \ref{thm:delrango}, como estamos
	suponiendo que las variedades no tienen borde, existen coordenadas
	$(U,\varphi)$ en $p$ y $(V,\psi)$ en $\pi(p)=:q$ tales que
	$\pi(U)\subset V$ y
	\begin{align*}
		\widehat{\pi}(\lista*{x}{m}) & \,=\,
			\psi\circ\pi\circ\varphi^{-1}(\lista*{x}{n},\,
				x^{n+1},\,\dots,\,x^{m})
			\,=\,(\lista*{x}{n})
		\text{ .}
	\end{align*}
	%
	Achicando $U$ de ser necesario, podemos asumir que
	$\varphi(U)=\cubo[m]{\epsilon}{0}$. La imagen por $\widehat{\pi}$
	de este cubo es exactamente el cubo $\cubo[n]{\epsilon}{0}=%
	\{(\lista*{y}{n})\,:\,|y^{j}|<\epsilon\}$. Definimos entonces
	\begin{align*}
		\widehat{\sigma} & \,:\,\cubo[n]{\epsilon}{0}\,\rightarrow\,
			\cubo[m]{\epsilon}{0}\text{ ,}\quad
			\widehat{\sigma}(\lista*{y}{n})\,=\,
			(\lista*{y}{n},\,0,\,\dots,\,0)
			\quad\text{y}\\
		\sigma & \,:\,\psi^{-1}(\cubo[n]{\epsilon}{0}) \,\rightarrow\,
			\varphi^{-1}(\cubo[m]{\epsilon}{0})
				\text{ ,}\quad
			\sigma\,=\,\varphi^{-1}\circ\widehat{\sigma}\circ\psi
		\text{ .}
	\end{align*}
	%
	Entonces $\pi\circ\sigma=\id[\psi^{-1}({\cubo[n]{\epsilon}{0}})]$ y
	$\sigma$ es una secci\'{o}n suave de $\pi$ defininida en el
	abierto $\psi^{-1}(\cubo[n]{\epsilon}{0})\subset V$ y
	$\sigma(\pi(p))=p$.

	Rec\'{\i}procamente, si $p\in M$ y existe una transformaci\'{o}n
	suave $\sigma:\,U\rightarrow M$ definida en un abierto $U\subset N$
	tal que $p\in\sigma(U)$ y tal que $\pi\circ\sigma=\id[U]$, entonces,
	tomando diferencial en $q\in U$ tal que $\sigma(q)=p$ y aplicando
	la regla de la cadena,
	\begin{align*}
		\id[T_{q}N] & \,=\,\diferencial[q]{(\pi\circ\sigma)} \,=\,
			\diferencial[p]{\pi}\cdot\diferencial[q]{\sigma}
		\text{ ,}
	\end{align*}
	%
	de lo que se deduce que $\diferencial[p]{\pi}$ es sobreyectivo
	(y que $\diferencial[q]{\sigma}$ es inyectivo).
\end{proof}

Como en el caso de las inmersiones, podemos usar esta equivalencia para
definir una submersi\'{o}n topol\'{o}gica. Una funci\'{o}n continua
$\pi:\,X\rightarrow Y$ se dice \emph{submersi\'{o}n}, si admite una
cantidad suficiente de secciones locales, es decir, es tal que cada punto
$p\in X$ pertenece a la imagen de alguna secci\'{o}n local de $\pi$.

\begin{propoSubmersionEsAbierta}\label{thm:submersionesabierta}
	Una submersi\'{o}n $\pi:\,M\rightarrow N$ con $M$ y $N$ \emph{sin}
	borde, es abierta.
\end{propoSubmersionEsAbierta}

\begin{proof}
	Sea $U\subset M$ un subconjunto abierto y sea $q\in \pi(U)$. Sea
	$p\in U$ tal que $\pi(p)=q$ cualquier punto en la preimagen. Por
	el teorema \ref{thm:seccioneslocales}, existe una secci\'{o}n
	local suave $\sigma:\,W\rightarrow M$ de $\pi$ tal que $\sigma(q)=p$.
	El conjunto $\sigma^{-1}(U)$ es abierto, est\'{a} contenido en $W$
	y contiene a $q$. Pero, adem\'{a}s, si $y\in\sigma^{-1}(U)$, vale
	que $y=\pi(\sigma(y))\in\pi(U)$. Entonces $q$ pertenece a
	$\sigma^{-1}(U)$ que est\'{a} contenido en $\pi(U)$, de lo que se
	deduce que $\pi(U)$ es abierto.
\end{proof}

\begin{coroSubmersionSobreEsCociente}\label{thm:submersionsobreescociente}
	Si $\pi:\,M\rightarrow N$ es una submersi\'{o}n sobreyectiva
	($M$ y $N$ sin borde), entonces es cociente.
\end{coroSubmersionSobreEsCociente}

\begin{proof}
	Por \ref{thm:submersionesabierta}, $\pi:\,M\rightarrow N$ es
	sobreyectiva y abierta. En particular, es cociente.
\end{proof}

Las submersiones presentan algunas propiedades similares a las de los
difeomorfismos locales (c.~f. \ref{thm:propisextradifeoslocales}).

\begin{propoSubmersionesExtra}\label{thm:propisextrasubmersiones}
	Sean $M$ y $N$ variedades sin borde y sea $\pi:\,M\rightarrow N$ una
	submersi\'{o}n suave y sobreyectiva. Si $P$ es una variedad
	(con o sin borde) entonces \emph{(a)} una transformaci\'{o}n
	(no necesariamente continua) $F:\,N\rightarrow P$ es suave, si y
	s\'{o}lo si $F\circ\pi:\,M\rightarrow P$ lo es; \emph{(b)} si
	$G:\,M\rightarrow P$ es una funci\'{o}n constante en las fibras de
	$\pi$, existe una \'{u}nica funci\'{o}n $\tilde{G}:\,N\rightarrow P$
	tal que $G=\tilde{G}\circ\pi$ y $G$ es suave, si y s\'{o}lo si
	$\tilde{G}$ lo es.
\end{propoSubmersionesExtra}

Estas propiedades caracterizan a las submersiones sobreyectivas. Las
submersiones sobreyectivas son la versi\'{o}n an\'{a}loga a las
funciones cociente en espacios topol\'{o}gicos. M\'{a}s adelante veremos
qu\'{e}clase de transformaciones suaves cumplen un rol similar al de
los subespacios.

\begin{obsCaracterizaSubmersionesSobre}\label{obs:caracterizasubmersionessobre}
	Sea $\pi:\,M\rightarrow N$ como en \ref{thm:propisextrasubmersiones}.
	Si $\tilde{N}$ representa al mismo conjunto subyacente a $N$ con
	una topolog\'{\i}a posiblemente distinta (con respecto a la cual
	es una variedad topol\'{o}gica) y una estructura diferencial
	(posiblemente) distinta, pero de manera que el \'{\i}tem
	\emph{(a)} siga siendo v\'{a}lido reemplazando $N$ por $\tilde{N}$,
	entonces $\id:\,N\rightarrow\tilde{N}$ es difeomorfismo.

	Para demostrar esta afirmaci\'{o}n, consideramos los siguientes
	diagramas:
	\begin{center}
		\begin{tikzcd}
			M \arrow[r,equal] \arrow[d,"\tilde{\pi}"'] &
				M \arrow[d,"\pi"] \\
			\tilde{N} \arrow[r,"\id", shift left] &
			N \arrow[l, shift left]
		\end{tikzcd}
		%
		\begin{tikzcd}
			M \arrow[dr,dashed,"\tilde{\pi}"]
				\arrow[d,"\tilde{\pi}"'] & \\
			\tilde{N} \arrow[r,"{\id[\tilde{N}]}"'] & \tilde{N}
		\end{tikzcd}
	\end{center}

	Por hip\'{o}tesis, $\pi=\id[\rightarrow]\circ\tilde{\pi}$ es suave, si
	y s\'{o}lo si $\id[\rightarrow]$ es suave. Pero, tambi\'{e}n sabemos
	que $\pi$ es suave. Entonces $\id[\rightarrow]$ es suave. Por otro
	lado, por \emph{(a)}, $\tilde{\pi}=\id[\leftarrow]\circ\pi$ es suave,
	si y s\'{o}lo si $\id[\leftarrow]$ es suave. Pero $\tilde{\pi}$ es
	suave, pues: $\tilde{\pi}$ es suave, si y s\'{o}lo si $\id[\tilde{N}]$
	lo es, e $\id[\tilde{N}]$ siempre es suave. En definitiva,
	$\id[\leftarrow]$ tambi\'{e}n es suave, con lo que la identidad $\id$
	en el conjunto subyacente a $N$ determina un difeomorfismo entre las
	estructuras diferenciales de $N$ y de $\tilde{N}$. La conclusi\'{o}n
	es igual en el contexto de variedades topol\'{o}gicas.
\end{obsCaracterizaSubmersionesSobre}

\begin{coroUnicidadDelCociente}\label{thm:unicidaddelcociente}
	Sean $M$, $N_{1}$ y $N_{2}$ variedades \emph{sin} borde y sean
	$\pi_{1}:\,M\rightarrow N_{1}$ y $\pi_{2}:\,M\rightarrow N_{2}$
	submersiones suaves y sobreyectivas. Supongamos, adem\'{a}s, que
	$\pi_{1}(q)=\pi_{1}(q')\Leftrightarrow\pi_{2}(q)=\pi(q')$, es decir,
	cada una de las submersiones es constante en las fibras de la otra
	--dicho de otra manera, realizan las mismas identificaciones. Entonces
	existe un \'{u}nico difeomorfismo $F:\,N_{1}\rightarrow N_{2}$
	tal que $F\circ\pi_{1}=\pi_{2}$.
\end{coroUnicidadDelCociente}


%
\section{Subvariedades}
\theoremstyle{plain}
\newtheorem{teoDeLasFetas}{Teorema}[section]
\newtheorem{propoEmbeddingEsEmbedding}[teoDeLasFetas]{Proposici\'{o}n}
\newtheorem{propoRegularCerradaSiiPropia}[teoDeLasFetas]{Proposici\'{o}n}
\newtheorem{lemaContinuaPropia}[teoDeLasFetas]{Lema}
\newtheorem{lemaPropiaEsCerrada}[teoDeLasFetas]{Lema}
\newtheorem{teoDeLosConjuntosDeNivel}[teoDeLasFetas]{Teorema}
\newtheorem{coroDelValorRegular}[teoDeLasFetas]{Corolario}
\newtheorem{propoRegularEsLocalmenteDeNivel}[teoDeLasFetas]{Proposici\'{o}n}
\newtheorem{propoInmersaEsInmersa}[teoDeLasFetas]{Proposici\'{o}n}
\newtheorem{propoDeLasParametrizaciones}[teoDeLasFetas]{Proposici\'{o}n}

\theoremstyle{remark}
\newtheorem{obsCuandoInmersaEsRegular}{Observaci\'{o}n}[section]

%-------------

Dicho r\'{a}pidamente, una \emph{subvariedad} es un subconjunto de una
variedad que posee, a su vez, una topolog\'{\i}a y una estructura diferencial
(de manera que la inclusi\'{o}n sea suave). En esta secci\'{o}n indagamos un
poco m\'{a}s en las propiedades de los embeddings y de las inmersiones
con el objetivo de poder estudiar los distintos tipos de subvariedades.

\subsection{Subvariedades regulares}
En primer lugarm dada una variedad $M$, un subconjunto $S\subset M$ se dice
\emph{subvariedad regular} (o, a veces, \emph{subvariedad} a secas), si,
en tanto subespacio topol\'{o}gico es una variedad topol\'{o}gica
\emph{sin borde} y posee una estructura diferencial respecto de la cual
$\inc[S]:\,S\rightarrow M$ es un embedding (suave). Las subvariedades con
borde se definen de manera an\'{a}loga, pero algunos resultados v\'{a}lidos
para subvariedades sin borde no lo son para subvariedades con borde; las
definiremos m\'{a}s adelante.

La \'{u}ltima condici\'{o}n en la definici\'{o}n de subvariedad regular
se puede reemplazar por la condici\'{o}n de que $\inc[S]$ sea suave y que
el rango de $\inc[S]$ sea $\dim\,S$ (que, necesariamente, es
$\dim\,S\leq\dim\,M$), pues la condici\'{o}n de ser subespacio fue
incluida expl\'{\i}citamente antes.

Aunque parezca inmediato de las definiciones, demostramos la siguiente
proposici\'{o}n, pues, en alg\'{u}n sentido, introduce la idea de estructura
natural en una subvariedad.

\begin{propoEmbeddingEsEmbedding}\label{thm:embeddingesembedding}
	Sea $M$ una variedad diferencial y sea $N$ una variedad diferencial
	sin borde. Sea $F:\,N\rightarrow M$ un embedding suave. Sea
	$S=F(N)$ la imagen de $F$ en $M$. Entonces \emph{(a)} con la
	topolog\'{\i}a de subespacio de $M$, el subconjunto $S$ es una
	variedad topol\'{o}gica y \emph{(b)} admite una estructura
	diferencial de manera que $S$ sea una subvariedad regular de $M$ y
	$F|:\,N\rightarrow S$ sea un difeomorfismo. Esta topolog\'{\i}a
	y esta estructura diferencial en $S$ son las \'{u}nicas con
	estas propiedades.
\end{propoEmbeddingEsEmbedding}

\begin{proof}
	Le damos a $S=F(N)$ la topolog\'{\i}a de subespacio de $M$.
	De la definici\'{o}n de subvariedad regular, se deduce que esta
	es la \'{u}nica topolog\'{\i}a posible con la que $S$ puede ser
	una subvariedad regular. Como $F$ es un embedding, es, en
	particular, un subespacio y un homeomorfismo entre $N$ y $S$.
	Notemos tambi\'{e}n que, si deseamos que $F|:\,N\rightarrow S$
	sea un \emph{homeomorfismo}, como $F$ es subespacio, $S$ \emph{debe}
	tener la topolog\'{\i}a de subespacio de $M$. En particular, como
	$N$ es una variedad topol\'{o}gica, $S$ tambi\'{e}n lo es.
	
	De la misma manera, pasando a la estructura suave, si deseamos que
	$F|:\,N\rightarrow S$ sea un difeomorfismo, entonces los pares
	$(F(U),\varphi\circ F^{-1})$, donde $(U,\varphi)$ es una carta
	para $N$, deben ser cartas compatibles con la estructura de $S$.
	Notemos que la colecci\'{o}n
	\begin{align*}
		\cal{A}_{S} & \,=\,\left\lbrace (F(U),\varphi\circ F^{-1})\,:\,
			(U,\varphi)\text{ carta para } N\right\rbrace
	\end{align*}
	%
	es un atlas suavemente compatible (pues $N$ tiene una estructura
	diferencial y las cartas $(U,\varphi)$ consideradas son las
	cartas compatibles con dicha estructura). Con lo cual, la
	condici\'{o}n de que $F|:\,N\rightarrow S$ sea difeomorfismo
	impone una estructura suave sobre $S$. Finalmente, notemos que
	la inclusi\'{o}n $\inc[S]:\,S\rightarrow M$ se descompone de la
	siguiente manera:
	\begin{align*}
		\inc[S] & \,=\,F\circ (F|^{-1})
		\text{ ,}
	\end{align*}
	%
	donde $F|^{-1}$ es un difeomorfismo y $F$ es un embedding. En
	particular, $\inc[S]$ es embedding.
\end{proof}

Una subvariedad regular $S\subset M$ se dice \emph{propia}, si la
inclusi\'{o}n $\inc[S]:\,S\rightarrow M$ es una funci\'{o}n propia. Las
subvariedades propias son, exactamente, las subvariedades \emph{regulares} que
son cerradas en $M$.

\begin{propoRegularCerradaSiiPropia}\label{thm:regularcerradasiipropia}
	Sea $M$ una variedad diferencial y sea $S\subset M$ una
	subvariedad regular. Entonces $S$ es una subvariedad regular
	propia, si y s\'{o}lo si $S\subset M$ es una subconjunto cerrado.
\end{propoRegularCerradaSiiPropia}

Como toda funci\'{o}n continua y propia entre
variedades topol\'{o}gicas es cerrada, dichas funciones son, adem\'{a}s,
subespacio, con lo cual, si $\inc[S]:\,S\rightarrow M$ es continua y
propia, $\inc[S]$ es embedding (topol\'{o}gico). Es decir, no existen
subvariedades propias (tales que la inclusi\'{o}n sea propia) que no sean
subvariedades regulares (es decir, subespacios topol\'{o}gicos). Aun as\'{\i},
existen subvariedades \emph{cerradas} pero que no son regulares, es decir,
subespacios.

Antes de demostrar la proposici\'{o}n recordamos algunos lemas acerca de
funciones continuas propias.

\begin{lemaContinuaPropia}\label{thm:continuapropia}
	\emph{(a)} Sea $f:\,X\rightarrow Y$ una funci\'{o}n continua. Si
	$f$ es cerrada y las fibras $f^{-1}(y)$ con $y\in Y$ son compactas,
	entonces $f$ es propia; \emph{(b)} en particular, si $f$ es un
	embedding en un subespacio cerrado, entonces $f$ es propia;
	\emph{(c)} si $Y:T_{2}$ y $f$ admite una rettracci\'{o}n
	$g:\,Y\rightarrow X$ con $g\circ f=\id[X]$, entonces $f$ es
	propia; \emph{(d)} si $f$ es propia y $A\subset X$ es saturado
	($f^{-1}(f(A))=A$), entonces $f|_{A}:\,A\rightarrow f(A)$ es propia.
\end{lemaContinuaPropia}

\begin{proof}
	\emph{(a)}: Sea $K\subset Y$ un compacto. Sea $\{A_{i}\}_{i}$ una
	familia de subconjuntos cerrados de $f^{-1}(K)$ con la propiedad de
	la intersecci\'{o}n finita. Sea $I=\{i_{1},\,\dots,\,i_{k}\}$ un
	subconjunto finito de \'{\i}ndices. La familia
	\begin{align*}
		& \left\lbrace A_{I}:=A_{i_{1}}\cap \,\cdots\,\cap A_{i_{k}}
			\,:\, I=\{i_{1},\,\dots,\,i_{k}\}\right\rbrace
	\end{align*}
	%
	donde $I$ recorre todos los subconjuntos finitos de \'{\i}ndices
	constituye una familia de cerrados de $f^{-1}(K)$ con la propiedad
	de la intersecci\'{o}n finita, tambi\'{e}n. Como $f$ es cerrada,
	$f(A_{I})$ es cerrada en $K$ para cada $I$ (tomamos
	$\widetilde{A_{I}}$ cerrado en $X$ tal que
	$A_{I}=f^{-1}(K)\cap\widetilde{A_{I}}$, entonces $f(\widetilde{A_{I}}$
	es cerrado en $Y$ y $K\cap f(\widetilde{A_{I}})$ es igual $f(A_{I})$).
	La familia $\{f(A_{I})\}_{I}$ est\'{a} compuesta por cerrados (de $K$)
	y tales que
	\begin{align*}
		f(A_{I_{1}})\cap\,\cdots\,\cap f(A_{I_{t}}) & \,\supset\,
			f(A_{I_{1}}\cap\,\cdots\,\cap A_{I_{t}})\,\not=\,
			\varnothing
		\text{ .}
	\end{align*}
	%
	Concluimos entonces que $\{f(A_{I})\}_{I}$ tiene la propiedad de
	la intersecci\'{o}n finita, tambi\'{e}n. Como $K$ es compacto,
	existe $y\in K$ tal que $y\in\bigcap_{I}\,f(A_{I})$. Pero esto
	quiere decir que existen, para cada $I$, elementos $x_{I}\in A_{I}$
	tales que $f(x_{I})=y$, es decir, $x_{I}\in A_{I}\cap f^{-1}(y)$.
	Por hip\'{o}tesis, $f^{-1}(y)$ es compacta. Adem\'{a}s, por lo
	anterior,
	\begin{align*}
		\varnothing & \,\not=\,\big(A_{i_{1}}\cap\,\cdots\,\cap
			A_{i_{k}}\big)\cap f^{-1}(y) \,=\,
			\big(A_{i_{1}}\cap f^{-1}(y)\big) \cap\,\cdots\,\cap
			\big(A_{i_{k}}\cap f^{-1}(y)\big)
		\text{ .}
	\end{align*}
	%
	Entonces $\{A_{i}\cap f^{-1}(y)\}_{i}$ es una familia de cerrados
	en el compacto $f^{-1}(y)$ con la propiedad de intersecci\'{o}n
	finita. Por lo tanto,
	\begin{align*}
		\varnothing & \,\not=\,\bigcap_{i}\,
			\big(A_{i}\cap f^{-1}(y)\big) \,=\,
			\big(\bigcap_{i}\,A_{i}\big)\cap f^{-1}(y)
			\,\subset\, \bigcap_{i}\,A_{i}
		\text{ .}
	\end{align*}
	%

	\emph{(b)} Si asumimos que $f:\, X\rightarrow Y$ es subespacio y
	que $f(X)\subset Y$ es cerrada, entonces $f^{-1}(y)$ es vac\'{\i}a
	o consiste en un \'{u}nico punto. En cualquier caso,
	$f^{-1}(y)$ es compacta para todo $y\in Y$ y, como
	$f|:\,X\rightarrow f(X)$ es homeomorfismo, si $A\subset X$ es cerrado,
	entonces $f(A)$ es cerrada en $f(X)$ y, por lo tanto, en $Y$.

	\emph{(c)} Si $Y$ es Hausdorff y $f$ admite una retracci\'{o}n
	continua $g:\,Y\rightarrow X$, entonces, si $K\subset Y$ es compacto,
	entonces $K$ es cerrado en $Y$, $f^{-1}(K)$ es cerrado en $X$ y
	$g(K)$ es compacto (pues $g$ es continua). Pero, si $f(x)\in K$,
	entonces $x=g(f(x))$ y
	\begin{align*}
		f^{-1}(K) & \,\subset\, g(K)
		\text{ .}
	\end{align*}
	%
	Por lo tanto, $f^{-1}(K)$ es compacto.

	\emph{(d)} Si $K\subset f(A)$ es compacto, entonces $K\subset Y$
	es compacto y, por, hip\'{o}tesis, $f^{-1}(K)$ es compacto.
	Pero $f^{-1}(K)\subset f^{-1}(f(A))=A$.
\end{proof}

\begin{lemaPropiaEsCerrada}\label{thm:propiaescerrada}
	Si $f:\,x\rightarrow Y$ es continua, $Y$ es localmente compacto
	Hausdorff y, adem\'{a}s, $f$ es propia, entonces $f$ es cerrada.
\end{lemaPropiaEsCerrada}

\begin{proof}
	Sea $A\subset X$ cerrado. Sea $y\in\clos{f(A)}$ y sea $U\subset Y$
	un entorno de $y$ con clausura compacta $C$. Entonces
	$y\in\clos{f(A)\cap C}$ (si $V$ es entorno de $y$, entonces
	$V\cap U$ es abierta, $V\cap U\subset U\subset C$ y
	$(V\cap U)\cap f(A)\not=\varnothing$), con lo cual
	$(V\cap U)\cap (f(A)\cap C)\not=\varnothing$). Como $f$ es propia,
	$f^{-1}(C)$ es compacto y $A\cap f^{-1}(C)$ tambi\'{e}n, por ser
	cerrado contenido en un compacto. Como $f$ es continua,
	$f(A\cap f^{-1}(C))=f(A)\cap C$ es compacto. Como $Y$ es Hausdorff,
	$f(A)\cap C$ es cerrada e $y\in f(A)\cap C\subset f(A)$. Con lo
	cual $f(A)$ es cerrada.
\end{proof}

Ahora s\'{\i}, pasamos a la demostraci\'{o}n de la proposici\'{o}n.

\begin{proof}[Demostraci\'{o}n de \ref{thm:regularcerradasiipropia}]
	Si $S\hookrightarrow M$ es propia, $S\subset M$ es cerrado,
	por \ref{thm:propiaescerrada}. Si $S\subset M$ es cerrado,
	$\inc[S]:\,S\rightarrow M$ es subespacio cerrado y, por lo tanto,
	propia por \ref{thm:continuapropia}.
\end{proof}

\subsection{Cartas preferenciales}
Localmente, las subvariedadesregulares se pueden identificar con una
subvariedad lineal de un espacio euclideo ambiente. Para ser precisos,
llamamos, dado un abierto $U\subset\bb{R}^{n}$, \emph{feta de dimensi\'{o}n %
$k$}, o \emph{$k$-feta}, de $U$ a los subconjuntos de la forma
\begin{align*}
	S & \,=\,\left\lbrace (\lista*{x}{k},\,x^{k+1},\,\dots,\,x^{n})\in U
		\,:\, x^{k+1}=c^{k+1},\,\dots,\,x^{n}=c^{n}\right\rbrace
\end{align*}
%
para ciertas constantes reales $c^{k+1},\,\dots,\,c^{n}$. Es decir,
$S$ es igual a la intersecci\'{o}n de $U$ con un subesapcio lineal de
$\bb{R}^{n}$. En general, una $k$-feta $S$ es homeomorfa a un subconjunto
abierto de $\bb{R}^{k}$.

Dada una variedad (topol\'{o}gica o diferencial) $M$, definimos una
\emph{feta de dimensi\'{o}n $k$} de manera an\'{a}loga, tomando cartas.
Si $(U,\varphi)$ es una carta para $M$ y $S\subset U$ es un subconjunto
tal que $\varphi(S)\subset\varphi(U)\subset\bb{R}^{n}$ es una feta de
dimensi\'{o}n $k$ de $\varphi(U)$, decimos que $S$ es una feta de
dimensi\'{o}n $k$ de $U$. En ese caso, $S$ debe ser de la forma
$S=\{\varphi^{-1}(\lista*{x}{n})\,:\, x^{k+1}=c^{k+1},\,\dots,\,x^{n}=c^{n}\}$.
Como la aplicaci\'{o}n
\begin{align*}
	(\lista*{x}{n}) & \,\mapsto\, (\lista*{x}{k},\,x^{k+1}-c^{k+1},\,
		\dots,\,x^{n}-c^{n})
\end{align*}
%
es un difeomorfismo (homeomorfismo), toda $k$-feta de un abierto coordenado
$U$ es igual a la preimagen de los puntos tales que las \'{u}ltimas
coordenadas son iguales a cero.

Si, ahora, $S\subset M$ es un subconjunto arbitrario de la variedad, una
\emph{carta preferencial para $S$ en $M$} es una carta $(U,\varphi)$
tal que $S\cap U$ es, v\'{\i}a $\varphi$, una $k$-feta de $U$. Las
funciones coordenadas $(\lista*{x}{n})$ de una carta preferencial
para $S$ en $M$ se denominar\'{a}n \emph{coordenadas preferenciales}.
Si existe $k\geq 0$ tal que para todo punto del subconjunto $S$ existe un
entorno coordenado $U$ tal que $S\cap U$ es una $k$-feta de $U$, decimos que
\emph{$S$ verifica localmente (en $M$) la condici\'{o}n de ser $k$-feta} o
que \emph{es localmente una $k$-feta}. Dicho de otra manera, $S$ es localmente
una feta de dimensi\'{o}n $k$, si existe un cubrimiento de $S$ por entornos
coordenados $U$ tales que $S\cap U$ es una feta de dimensi\'{o}n $k$
de $U$. Con el nombre de carta preferencial la dimensi\'{o}n de la feta
resultante no queda especificada.

Las $k$-fetas de un abierto de $\bb{R}^{n}$ son homeomorfos a abiertos de
$\bb{R}^{k}$. Por lo tanto, es de esperar que un subconjunto de una variedad
que verifica localmente la condici\'{o}n de ser $k$-feta tenga, naturalmente,
una estructura de variedad. Por otro lado, si bien ser una subvariedad regular
es una propiedad global de un subconjunto $S\subset M$ (o, mejor dicho, de
la inclusi\'{o}n $\inc[S]$), es posible formular un criterio local para
determinar si un subconjunto admite una estructura de subvariedad regular
(es decir, determinar si $\inc[S]$ es un embedding).

\begin{teoDeLasFetas}\label{thm:delasfetas}
	Sea $M$ una variedad diferencial (sin borde) y sea $S\subset M$ una
	subvariedad regular de $M$ de dimensi\'{o}n $k$, Entonces
	para todo punto $p\in S$ existe un carta $(U,\varphi)$ para $M$
	en $p=\inc[S](p)$ tal que $S\cap U$ es una $k$-feta de $U$.
	Rec\'{\i}procamente, si $S\subset M$ es un subconjunto que
	verifica localmente la condici\'{o}n de ser una $k$-feta, entonces,
	con la topolog\'{\i}a de subespacio de $M$, $S$ es una variedad
	topol\'{o}gica de dimensi\'{o}n $k$ y, adem\'{a}s, admite una
	estructura diferencial de manera que resulta ser una subvariedad
	regular de $M$ de dimensi\'{o}n $k$.
\end{teoDeLasFetas}

\begin{proof}
	Supongamos, en primer lugar, que $S\subset M$ es una subvariedad
	regular de dimensi\'{o}n $k$. Por el teorema del rango
	\ref{thm:delrango}, como $\inc[S]:\,S\rightarrow M$ es una
	inmersi\'{o}n, si $p\in S$, existen coordenadas $(V,\psi)$ para
	$M$ en $p$ y $(U,\varphi)$ para $S$ tales que
	\begin{align*}
		\psi\circ\inc[S]\circ\varphi^{-1}(\lista*{x}{k}) &
			\,=\,(\lista*{x}{k},\,0,\,\dots,\,0)
		\text{ .}
	\end{align*}
	%
	Notemos que $\psi\circ\inc[S]\circ\varphi^{-1}=%
	\inc[U]:\,U\rightarrow V$. Es decir, $U$ se incluye en $V$ como
	una $k$-feta. El problema es que no podemos afirmar que $U$ sea
	exactamente $V\cap\{x^{k+1}=\,\cdots\,=x^{n}=0\}$.

	Tomemos $\epsilon>0$ tal que
	$U_{0}=\varphi^{-1}(\bola[k]{\epsilon}{0})\subset U$ y
	$V_{0}=\psi^{-1}(\bola[n]{\epsilon}{0})\subset V$. Como
	$U_{0}\subset S$ es abierto, existe $W\subset M$ tal que
	$U_{0}=S\cap W$. Sea $V_{1}=W\cap V_{0}$ y sea
	$\psi_{1}=\psi|_{V_{1}}$. Entonces $(V_{1},\psi_{1})$ es una carta
	para $M$ en $p$ y $S\cap V_{1}=U_{0}\cap V_{0}=U_{0}$. Entonces
	$V_{0}\cap\{x^{k+1}=\,\cdots\,=x^{n}=0\}=U_{0}$, $U_{0}$ es una
	$k$-feta de $V_{1}$ y $(V_{1},\psi_{1})$ es una carta preferencial
	para $S$ en $M$ en $p$.

	Rec\'{\i}procamente, supongamos que $S\subset M$ es un subconjunto
	que verifica localmente la condici\'{o}n de ser una feta de
	dimensi\'{o}n $k$ de $M$. En la topolog\'{\i}a de subespacio,
	$S$ es $T_{2}$ y $N_{2}$. Veamos que es localmente euclideo
	y que su dimensi\'{o}n es $k$.
	
	Sea $\pi:\,\bb{R}^{n}\rightarrow\bb{R}^{k}$ la proyecci\'{o}n en
	las primeras $k$ coordenadas. Sea $p\in S$ y sea $(U,\varphi)$ una
	carta preferencial para $S$ en $M$ en $p$. Sea $V=S\cap U$ la
	$k$-feta propio, sea $\widehat{V}=\pi\circ\varphi(V)$ lo que
	deber\'{\i}a ser la imaen de la carta para $S$ cuando \'{e}sta sea
	definida y sea $\psi=\pi\circ\varphi|_{V}:\,V\rightarrow\widehat{V}$
	la funci\'{o}n que ser\'{a} la carta. Por definici\'{o}n,
	$\varphi(V)=\varphi(U)\cap A$ para cierta subvariedad lineal
	$A\subset\bb{R}^{n}$. Supongamos que $A$ est\'{a} dada como
	el conjunto de ceros de las siguientes ecuaciones:
	\begin{align*}
		x^{k+1} \,=\,c^{k+1} \text{ ,}\dots\text{ ,}
		x^{n} \,=\,c^{n}
		\text{ .}
	\end{align*}
	%
	Como $\varphi(U)$ es abierto en $\bb{R}^{n}$, la $k$-feta
	$\varphi(V)$ es abierta en $A$. Como
	$\pi|_{A}:\,A\rightarrow\bb{R}^{k}$ es un difeomorfismo,
	$\widehat{V}=\pi(\varphi(V))$ es un subconjunto abierto de
	$\bb{R}^{k}$. Resta ver que $\psi$ es un homeomorfismo. La
	funci\'{o}n $\psi$ es continua por ser composici\'{o}n de
	funciones continuas: $\psi=\pi\circ\varphi\circ\inc[V]$; tiene
	inversa dada por
	\begin{align*}
		\varphi^{-1}\circ j\circ\inc[\widehat{V}]
			(\lista*{x}{k}) & \,=\,
		\varphi^{-1}(\lista*{x}{k},\,c^{k+1},\,\dots,\,c^{n})
		\text{ ,}
	\end{align*}
	%
	que es continua.

	En cuanto a la estructura diferencial en $S$, usamos las cartas
	reci\'{e}n definidas: dadas $(U,\varphi)$ y $(U',\varphi')$ cartas
	para $M$ tales que $V=S\cap U$ y $V'=U'\cap S$ son $k$-fetas
	de $U$ y de $U'$, respectivamente, tenemos cartas
	$(v,\psi)$ y $(V',\psi')$ en $S$ que verifican
	\begin{align*}
		\psi'\circ\psi^{-1} & \,=\, (\pi\circ\varphi'\circ\inc[V'])
			\circ (\pi\circ\varphi\circ\inc[V])^{-1} \\
		& \,=\, \pi\circ\varphi'\circ\inc[V']\circ\varphi^{-1}
			\circ j\circ\inc[\widehat{V}](\lista*{x}{k}) \\
		& \,=\,\pi\circ (\varphi'\circ\varphi^{-1})\circ
			j(\lista*{x}{k})
		\text{ .}
	\end{align*}
	%
	Pero $j$ y $\pi$ son suaves y $\varphi'$ y $\varphi$ son suavemente
	compatibles. Entonces $(V,\psi)$ y $(V',\psi')$ son compatibles.
	Con respecto a la estructura determinada por estas cartas en $S$,
	$\widehat{\inc[S]}(\lista*{x}{k})=%
	(\lista*{x}{k},\,c^{k+1},\,\dots,\,c^{n})$ localmente, que resulta
	pues una inmersi\'{o}n suave. Como $\inc[S]$ es, por definici\'{o}n
	de la topolog\'{\i}a de $S$, subespacio, la transformaci\'{o}n
	$\inc[S]$ es embedding suave.
\end{proof}

Demostraremos luego que esta estructura en $S$ es la \'{u}nica posible que
hace que $S$ sea una subvariedad de $M$. Con lo cual, si imponemos que
$S$ sea subespacio, admite una \'{u}nica estructura de subvariedad
(necesariamente regular). Pero $S$ podr\'{\i}a admitir otras topolog\'{\i}as
respecto a las cuales resulte variedad topol\'{o}gica y, por lo tanto,
podr\'{\i}a ser subvariedad (aunque no subvariedad regular). Para que
esto quede claro, habr\'{a} que introducir una noci\'{o}n m\'{a}s general
de subvariedad.

El borde de una variedad con borde es una subvariedad regular.

\subsection{Conjuntos de nivel}
Las subvariedades regulares se pueden expresar localmente como el
gr\'{a}fico de una funci\'{o}n, como tambi\'{e}n como el conjunto de ceros
de una funci\'{o}n (cartas preferenciales). Dada una transformaci\'{o}n
$\Phi:\,M\rightarrow N$ y un punto $c\in N$, decimos que
$\Phi^{-1}(c)\subset M$ es un \emph{conjunto de nivel de $\Phi$}. Todo
subconjunto cerrado de una variedad $M$ es el conjunto de ceros
($N=\bb{R}$, $c=0$) de alguna funci\'{o}n (suave). Para poder relacionar
los conceptos de conjunto de nivel y de subvariedad es necesario imponer
alguna condici\'{o}n adicional.

\begin{teoDeLosConjuntosDeNivel}
	[el conjunto de nivel de una funci\'{o}n de %
	rango constante]\label{thm:denivel}
	Sean $M$ y $N$ variedades diferenciales. Sea $\Phi:\,M\rightarrow N$
	una transformaci\'{o}n suave de rango constante $r$. Cada conjunto
	de nivel $\Phi^{-1}(c)\subset M$ es una subvariedad propia de
	codimensi\'{o}n $r$ en $M$.
\end{teoDeLosConjuntosDeNivel}

\begin{proof}
	Sea $m=\dim\,M$ y sea $n=\dim\,N$. Por el teorema del rango constante,
	para cada punto $p\in M$ existe una carta $(U,\varphi)$ para $M$
	en $p$ y existe una carta $(V,\psi)$ para $N$ en $\Phi(p)=c$ tales
	que $\Phi(U)\subset V$ y
	\begin{align*}
		\widehat{\Phi}(\lista*{x}{m}) & \,=\,
			(\lista*{x}{r},\,0,\,\dots,\,0)
		\text{ .}
	\end{align*}
	%
	En particular, el \emph{conjunto} $S=\Phi^{-1}(c)$ verifica la
	condici\'{o}n de ser localmente un $k$-feta con $k=m-r$, es
	decir, una feta de codimensi\'{o}n $r$. Por el teorema
	\ref{thm:delasfetas}, $S$ tiene estructura de subvariedad regular
	de $M$. Como $\Phi$ es continua, $S$ es cerrada en $M$ y, por lo
	tanto, subvariedad regular propia.
\end{proof}

Un caso particular de esto se da cuando $\Phi$ es una submersi\'{o}n.
En concordancia con el hecho de que las transformaciones de rango constante
son localmente como transformaciones lineales, el resultado anterior es
an\'{a}logo al resultado de \'{A}lgebra lineal que dice que, dada una
transformaci\'{o}n lineal $L:\,\bb{R}^{m}\rightarrow\bb{R}^{r}$ sobreyectiva
(o $\bb{R}^{m}\rightarrow\bb{R}^{n}$ de rango $r$), el n\'{u}cleo $\ker(L)$
es un subespacio de codimensi\'{o}n $r$, determinado por un sistema de $r$
ecuaciones lineales independientes igualadas a cero. En el caso de
variedades, si $\Phi:\,M\rightarrow N$ es una submersi\'{o}n, entonces
$\Phi^{-1}(c)$ --el an\'{a}logo del n\'{u}cleo de una t.l.-- es una
subvariedad de codimensi\'{o}n $n=\dim\,N$.

\subsection{Puntos y valores regulares}
Dada una transformaci\'{o}n suave $\Phi:\,M\rightarrow N$, decimos que
$p\in M$ es un \emph{punto regular}, si la transformaci\'{o}n lineal
$\diferencial[p]{\Phi}:\,\tangente[p]{M}\rightarrow\tangente[\Phi(p)]{N}$
es sobreyectiva. En otro caso, decimos que $p$ es un \emph{punto cr\'{\i}tico}.
El subconjunto de $M$ de puntos regulares es igual al subconjunto en donde
$\Phi$ tiene rango m\'{a}ximo. Este subconjunto es abierto en $M$ y, por lo
tanto, una subvariedad regular. Un punto $c\in N$ se dice \emph{valor %
regular} de $\Phi$, si todo punto perteneciente a $\Phi^{-1}(c)$ es un punto
regular. Si al menos un punto de la preimagen es punto cr\'{\i}tico, entonces
decimos que $c$ es un valor cr\'{\i}tico de $\Phi$. Un \emph{conjunto de %
nivel regular} de $\Phi$ es el conjunto de nivel correspondiente a un valor
regular de $\Phi$.

\begin{coroDelValorRegular}[teorema del valor regular]%
	\label{thm:delvalorregular}
	Sea $\Phi:\,M\rightarrow N$ una transformaci\'{o}n suave entre
	variedades diferenciales \emph{sin} borde. Todo conjunto de nivel
	regular de $\Phi$ es una subavariedad regular propia de $M$.
	Su codimensi\'{o}n es igual a la dimensi\'{o}n del codominio $N$.
\end{coroDelValorRegular}

\begin{proof}
	Si $c\in N$ es un valor regular y $S=\Phi^{-1}(c)$ es el conjunto de
	nivel correspondiente, entonces $S\subset M'$, donde
	$M'\subset M$ es la subvariedad regular abierta comformada por los
	puntos $p$ en los que $\diferencial[p]{\Phi}$ es sobreyectivo.
	La restricci\'{o}n $\Phi|_{M'}:\,M'\rightarrow N$ es una
	submersi\'{o}n, con lo cual, por el teorema \ref{thm:denivel},
	$\Phi^{-1}(c)\subset M'$ es una subvariedad regular. Pero
	$\inc[S]:\,\Phi^{-1}(c)\rightarrow M$ es igual a la composici\'{o}n
	$\inc[M']\circ\inc[S]':\,S\rightarrow M'\rightarrow M$. Tanto
	$\inc[S]'$ como $\inc[M']$ son embeddings. Entonces la inclusi\'{o}n
	de $S$ en $M$ lo es, tambi\'{e}n. Como $\Phi^{-1}(c)\subset M$ es
	cerrada, por continuidad, $S=\Phi^{-1}(c)$ es una subvariedad regular
	propia de $M$. Como $\codim(S,M')=\codim(S,M)$, porque
	$M'\subset M$ es abierto, se deduce que $\codim(S,M)=\dim(N)$.
\end{proof}

\begin{propoRegularEsLocalmenteDeNivel}\label{thm:regulardenivel}
	Sea $S\subset M$ un subconjunto de una variedad diferencial $M$ de
	dimensi\'{o}n $m$. Entonces $S$ es una subvariedad regular de
	dimensi\'{o}n $k$, si y s\'{o}lo si, para cada punto $p\in S$,
	es posible hallar una submersi\'{o}n $\Phi:\,U\rightarrow\bb{R}^{m-k}$
	definida en un abierto $U$ \emph{de $M$} tal que
	$\Phi^{-1}(c)=S\cap U$ para alg\'{u}n valor $c\in\bb{R}^{m-k}$.
\end{propoRegularEsLocalmenteDeNivel}

Esto es casi una reformulaci\'{o}n del criterio local de las fetas.

\begin{proof}
	Si $S\subset M$ es una subvariedad regular de dimensi\'{o}n $k$ de $M$
	y $p\in S$, podemos hallar una carta $(U,\varphi)$ para $M$ tal que
	$S\cap U$ es una feta de dimensi\'{o}n $k$ en $U$. Si
	$\varphi=(\lista*{x}{m})$, entonces $S\cap U=%
	\{x^{k+1}=\,\cdots\,=x^{n}=0\}$. Si definimos
	$\Phi:\,U\rightarrow\bb{R}^{m-k}$ por
	\begin{align*}
		\Phi(\lista*{x}{m}) & \,=\,(x^{k+1},\,\dots,\,x^{m})
		\text{ ,}
	\end{align*}
	%
	la proyecci\'{o}n en las \emph{\'{u}ltiimas} coordenadas, entonces
	$S\cap U=\Phi^{-1}(0)$. Rec\'{\i}procamente, si $S\subset M$ es un
	subconjunto de $M$ que verifica que, para cada punto, existe una
	funci\'{o}n $\Phi$ como en el enunciado y $p\in S$, entonces
	podemos elegir un abierto $U\subset M$ y una submersi\'{o}n
	$\Phi:\,U\rightarrow\bb{R}^{m-k}$, de manera que
	$S\cap U=\Phi^{-1}(c)$ para cierto valor $c\in \bb{R}^{m-k}$.
	Hay dos maneras de concluir: o bien usamos esto para definir una
	carta preferencial para $S$ en $M$ en $p$, o bien, por el teorema
	\ref{thm:delasfetas}, concluimos que $S\cap U$ es una subvariedad
	regular de $U$. En cualquier caso, $S\cap U$ es localmente en $U$
 	una feta de codimensi\'{o}n $m-k$. Como $U\subset M$ es abierto,
	concluimos que $S$ es localmente en $M$ una feta de codimensi\'{o}n
	$m-k$ (pues $\codim(U,M)=0$), o sea que $S$ es una subvariedad
	regular de dimensi\'{o}n $k$ en $M$.
\end{proof}

No sabemos, \textit{a priori}, que existan cartas preferenciales para $S$ en
$M$, pero, por el teorema del conjunto de nivel de una submersi\'{o}n, sabemos
que, para cada punto $p\in S$, existe un abierto $U$ de $M$ tal que
$p\in U$ y que $S\cap U$ es subvariedad regular de $U$. Por lo tanto,
existen cartas preferenciales para $S\cap U$ en $U$. Como $U$ es abierto
en $M$, podemos concluir que existen cartas preferenciales para $S$ en $M$
cerca de cada punto $p\in S$ y que, por lo tanto, $S$ es una subvariedad
regular de $M$.

\subsection{Subvariedades inmersas}
Una \emph{subvariedad inmersa} de una variedad (diferencial, topol\'{o}gica)
$M$ es un subconjunto $S$ con una topolog\'{\i}a con respecto a la cual
resulta una variedad topol\'{o}gica y una estructura diferencial con respecto
a la cual $\inc[S]:\,S\rightarrow M$ es una inmersi\'{o}n (suave, continua).

\begin{propoInmersaEsInmersa}\label{thm:inmersaesinmersa}
	Sea $M$ una variedad diferencial y sea $N$ una variedad diferencial
	sin borde. Sea $F:\,N\rightarrow M$ una inmersi\'{o}n suave e
	inyectiva y sea $S=F(N)\subset M$. Existen una \'{u}nica
	topolog\'{\i}a y una \'{u}nica estructura diferencial en $S$ con
	respecto a las cuales $S\subset M$ es una subvariedad inmersa y
	$F|:\,N\rightarrow S$ es un difeomorfismo.
\end{propoInmersaEsInmersa}

\begin{proof}
	Como queremos que $F$ sea un difeorfismo entre $N$ y $S$, en
	particular, la topolog\'{\i}a en $S$ deber\'{a} ser tal que
	$N$ y $S$ sean homeomorfos v\'{\i}a $F$. Definimos una topolog\'{\i}a
	en $S$ trasladando la topolog\'{\i}a de $N$: un subconjunto
	$U\subset S$ es abierto, si y s\'{o}lo si $F^{-1}(U)$ es abierto
	en $N$. Esto determina una topolog\'{\i}a en $S$ y es la \'{u}nica
	con respecto a la cual $F$ puede ser un homeomorfismo. De la misma
	manera, con respecto a la estructura diferencial, cubrimos a $S$
	con los pares $(F(U),\varphi\circ F^{-1})$, donde $(U,\varphi)$
	es una carta compatible con la estructura de $N$. Si queremos que $F$
	sea un difeomorfismo, todas estas cartas deben pertenecer a la
	estructura de $S$. Estas cartas en $N$ forman un atlas compatible y,
	por lo tanto, determinan una estructura diferencial en $S$. Finalmente,
	resta ver que, con esta topolog\'{\i}a y esta estructura suave,
	$\inc[S]:\,S\rightarrow M$ es una inmersi\'{o}n. Pero
	$\inc[S]=F\circ (F|^{-1})$, $F|$ es un difeomorfismo y $F$ es una
	inmersi\'{o}n. En particular, la inclusi\'{o}n es composici\'{o}n
	de inmersiones suaves y, en consecuencia, inmersi\'{o}n suave,
	tambi\'{e}n.
\end{proof}

\begin{obsCuandoInmersaEsRegular}\label{obs:cuandoinmersionesembedding}
	De la proposici\'{o}n \ref{thm:cuandoinmersionesembedding} podemos
	deducir los siguientes criterios para determinar si una subvariedad
	inmersa es regular: \emph{(a)} $\codim(S,M)=0$; \emph{(b)}
	$\inc[S]:\,S\hookrightarrow M$ es propia; \emph{(c)} $S$ es compacta.
	En cualquiera de estos casos, $S$ es una subvariedad regular.
	Adem\'{a}s, usando el teorema \ref{thm:embeddinglocal},
	que dice que las inmersiones son localmente embeddings, deducimos que
	toda subvariedad inmersa es localmente regular. Precisamente,
	si $M$ es una variedad diferencial y $S\subset M$ es una subvariedad
	inmersa, entonces, para cada punto $p\in S$, existe un abierto
	$V$ \emph{de $S$}, tal que $p\in V$ y $V\subset M$ es una subvariedad
	regular.
\end{obsCuandoInmersaEsRegular}

Terminamos esta secci\'{o}n definiendo el concepto de
\emph{parametrizaci\'{o}n} de una subvariedad. Sea $M$ una variedad
diferencial. Sea $S\subset M$ una subvariedad inmersa de dimensi\'{o}n $k$.
Una funci\'{o}n
\begin{align*}
	X & \,:\,U\subset\bb{R}^{k}\,\rightarrow M
\end{align*}
%
se dice \emph{parametrizaci\'{o}n (local)} de $S$, si $X:\,U\rightarrow M$
es continua en $M$, $U\subset\bb{R}^{k}$ es abierto, la imagen $X(U)\subset S$
es abierta en $S$ y $X|:\,U\rightarrow X(U)$ es un homeomorfismo
($X|:\,U\rightarrow S$ es subespacio abierta). Si $X|$ es difeomorfismo,
decimos que $X$ es \emph{parametrizaci\'{o}n local suave}.

Todo punto de una subvariedad inmersa est\'{a} en la imagen de una
parametrizaci\'{o}n local suave y toda parametrizaci\'{o}n da lugar a una
carta.

\begin{propoDeLasParametrizaciones}\label{thm:delasparametrizaciones}
	Sea $M$ una variedad diferencial y sea $S\subset M$ una
	subvariedad inmersa de dimensi\'{o}n $k$. Sea $U\subset\bb{R}^{k}$
	un subconjunto abierto. Entonces una funci\'{o}n $X:\,U\rightarrow M$
	es una parametrizaci\'{o}n local suave de $S$, si y s\'{o}lo si
	existe una carta $(V,\psi)$ para $S$ tal que $X=\inc[S]\circ\psi^{-1}$.
\end{propoDeLasParametrizaciones}


%
\section{El tangente a una subvariedad}
\theoremstyle{plain}
\newtheorem{teoRestringirElDominio}{Teorema}[section]
\newtheorem{teoCorrestringirElCodominio}[teoRestringirElDominio]{Teorema}
\newtheorem{teoUnicidadDeLaEstructuraRegular}[teoRestringirElDominio]{Teorema}
\newtheorem{teoUnicidadDeLaEstructuraInmersa}[teoRestringirElDominio]{Teorema}
\newtheorem{teoUnicidadDebilmenteRegular}[teoRestringirElDominio]{Teorema}
\newtheorem{lemaDeExtensiones}[teoRestringirElDominio]{Lema}
\newtheorem{propoDeNoExtensiones}[teoRestringirElDominio]{Proposici\'{o}n}
\newtheorem{propoTangenteSubvarCoordenadas}[teoRestringirElDominio]%
	{Proposici\'{o}n}
\newtheorem{propoTangenteSubvarRegularI}[teoRestringirElDominio]%
	{Proposici\'{o}n}
\newtheorem{propoTangenteSubvarRegularII}[teoRestringirElDominio]%
	{Proposici\'{o}n}
\newtheorem{propoTangenteBordeCurvas}[teoRestringirElDominio]{Proposici\'{o}n}
\newtheorem{propoBordeDeNivel}[teoRestringirElDominio]{Proposici\'{o}n}

\theoremstyle{remark}
\newtheorem{obsUnicidadDeLaEstructuraRegular}{Observaci\'{o}n}[section]
\newtheorem{obsTangenteValorRegular}[obsUnicidadDeLaEstructuraRegular]%
	{Observaci\'{o}n}
\newtheorem{obsTangenteConjuntoDeNivel}[obsUnicidadDeLaEstructuraRegular]%
	{Observaci\'{o}n}
\newtheorem{obsBordeDeNivel}[obsUnicidadDeLaEstructuraRegular]%
	{Observaci\'{o}n}
\newtheorem{obsDiferencialComoFuncional}[obsUnicidadDeLaEstructuraRegular]%
	{Observaci\'{o}n}

%-------------

\subsection{Restricci\'{o}n y correstricci\'{o}n de transformaciones}
Ha surgido, en algunas ocasiones, la necesidad de restringir o de
correstringir una transformaci\'{o}n suave. Empecemos, pues, esta secci\'{o}n
estudiando el comportamiento de una transformaci\'{o}n suave al restringir
su dominio o su codominio. En particular, veamos algunos casos en los que
la suavidad de la transformaci\'{o}n se preserva.

\begin{teoRestringirElDominio}[restricci\'{o}n de dominio]%
	\label{thm:restringireldominio}
	Sea $F:\,M\rightarrow N$ una transformaci\'{o}n suave. Sea $S\subset M$
	una subvariedad (inmersa o regular). Entonces $F|_{S}:\,S\rightarrow N$
	es suave.
\end{teoRestringirElDominio}

\begin{proof}
	La restricci\'{o}n de $F$ a la subvariedad $S$ es una composici\'{o}n
	de funciones suaves: $F|_{S}=F\circ\inc[S]$.
\end{proof}

\begin{teoCorrestringirElCodominio}[correstricci\'{o}n del codominio]%
	\label{thm:correstringirelcodominio}
	Sea $M$ una variedad diferencial \emph{sin} borde y sea
	$F:\,N\rightarrow M$ una transformaci\'{o}n suave. Sea $S\subset M$
	una subvariedad (inmersa o regular) tal que $F(N)\subset S$.
	Si $F|^{S}:\,N\rightarrow S$ es continua, entonces es suave.
\end{teoCorrestringirElCodominio}

\begin{proof}
	Sean $p\in N$ y $q=F(p)\in S$. Sea $V\subset S$ un abierto tal que
	$q\in V$ y tal que $\inc[V]:\,V\rightarrow M$ sea embedding. Como
	$V\subset M$ es una subvariedad regular, existe una carta preferencial
	$(W,\psi)$ para $V$ en $M$ en $q$. A partir de esta carta preferencial,
	definimos una carta para $V$: sea $V_{0}=V\cap W$ y sea
	$\psi_{0}=\pi\circ\psi|_{V}=\pi\circ\psi\circ\inc[V]$, donde
	$\pi:\,\bb{R}^{n}\rightarrow\bb{R}^{k}$ es la proyecci\'{o}n en
	las primeras $k$ coordenadas. El par $(V_{0},\psi_{0})$ es una
	carta para $V$ en $q$. Como $\inc[V]$ es continua,
	$V_{0}=\inc[V]^{-1}(W)$ es abierta en $V$. Como $V$ es abierto en $S$,
	$V_{0}$ es abierto en $S$. Asumiendo que la correstricci\'{o}n
	$F|:\,N\rightarrow S$ es continua, la preimagen $U=F^{-1}(V_{0})$ es
	abierta en $N$. Adem\'{a}s, $p\in U$. Sea $(U_{0},\varphi)$ una
	carta para $N$ en $p$ con $U_{0}\subset U$. Entonces
	$\psi_{0}\circ F|\circ\varphi^{-1}:%
	\,\varphi(U_{0})\rightarrow\psi_{0}(v_{0})$ es igual a
	\begin{align*}
		\psi_{0}\circ F|\circ\varphi^{-1} & \,=\,
			\pi\circ (\psi\circ F\circ\varphi^{-1})
		\text{ .}
	\end{align*}
	%
	Como $F:\,N\rightarrow M$ es suave, $\psi\circ F\circ\varphi^{-1}$ es
	suave y, por lo tanto, $\psi_{0}\circ F|\circ\varphi^{-1}$ es suave.
	En definitiva, la correstricci\'{o}n $F|:\,N\rightarrow S$ es suave.
\end{proof}

Una subvariedad inmersa $S\subset M$ se dice \emph{d\'{e}bilmente regular}
(?), si, para toda transformaci\'{o}n suave $F:\,N\rightarrow M$ tal que
$F(N)\subset S$, la correstricci\'{o}n $F|^{S}:\,N\rightarrow S$ es suave.
Vale que

\begin{center}
\begin{tikzcd}
	\left\lbrace
		\begin{array}{c}
			\text{subvariedades} \\
			\text{regulares}
		\end{array}
	\right\rbrace \arrow[r, phantom,"\supsetneq"] &
	\left\lbrace
		\begin{array}{c}
			\text{subvariedades} \\
			\text{d\'{e}bilmente regulares}
		\end{array}
	\right\rbrace \arrow[r, phantom, "\supsetneq"] &
	\left\lbrace
		\begin{array}{c}
			\text{subvariedades} \\
			\text{inmersas}
		\end{array}
	\right\rbrace
\end{tikzcd}
\end{center}
Por ejemplo, la lemniscata es inmersa pero no d\'{e}bilmente regular y
la curva irracional en el toro es d\'{e}bilmente regular pero no
regular.

\begin{teoUnicidadDeLaEstructuraRegular}[unicidad de la estructura en una %
	subvariedad regular]\label{thm:unicidaddelaestructuraregular}
	Sea $M$ una variedad diferencial y sea $S\subset M$ una subvariedad
	regular. La topolog\'{\i}a de subespacio y la estructura diferencial
	dad son las \'{u}nica con respecto a las cuales $S$ es una
	subvariedad regular de $M$.
\end{teoUnicidadDeLaEstructuraRegular}

\begin{proof}
	Sea $\tilde{S}$ una variedad diferencial obtenida a partir del
	conjunto $S$ imponiendo una topolog\'{\i}a y una estructura diferencial
	posiblemente distintas. Sean $\iota=\inc[S]:\,S\rightarrow M$ e
	$\tilde{\iota}=\inc[\tilde{S}]:\,\tilde{S}\rightarrow M$ las
	respectivas inclusiones. Supongamos, adem\'{a}s, que $\tilde{\iota}$
	es una inmersi\'{o}n. Como $\iota$ es un embedding e
	$\tilde{\iota}(\tilde{S})\subset S$, podemos concluir que la
	correstricci\'{o}n $\tilde{\iota}|:\,\tilde{S}\rightarrow S$ es
	suave (porque es continua). Entonces,
	\begin{align*}
		\iota\circ(\tilde{\iota}|) & \,=\,\tilde{\iota}
			\quad\text{y} \\
		\diferencial[p]{\iota}\cdot\diferencial[p]{(\tilde{\iota}|)}
			& \,=\,\diferencial[p]{\tilde{\iota}}
			\,:\,\tangente[p]{\tilde{S}}\rightarrow
				\tangente[p]{S}\rightarrow\tangente[p]{M}
		\text{ .}
	\end{align*}
	%
	Como $\diferencial[p]{\tilde{\iota}}$ es, por hip\'{o}tesis, inyectivo,
	el diferencial $\diferencial[p]{(\tilde{\iota}|)}$ debe serlo,
	tambi\'{e}n. Pero esto es cierto para todo punto $p\in\tilde{S}$,
	con lo que $\tilde{\iota}|:\,\tilde{S}\rightarrow S$ es de rango
	constante. Como, adem\'{a}s, $\tilde{\iota}|$ es biyectiva, el
	teorema global del rango \ref{thm:} implica que $\tilde{\iota}|$
	es difeomorfismo. Pero $\tilde{\iota}|=\id[S]$ como funci\'{o}n
	de conjuntos, con lo cual, concluimos que $\tilde{S}$ tiene
	exactamente \emph{la misma} topolog\'{\i}a y \emph{la misma}
	estructura diferencial que $S$.
\end{proof}

\begin{obsUnicidadDeLaEstructuraRegular}%
	\label{obs:unicidaddelaestructuraregular}
	Por el teorema \ref{thm:delasfetas} toda subvariedad regular verifica
	localmente la propiedad de ser $k$-feta para alg\'{u}n entero
	$k\leq n$. A su vez, esta \emph{estructura de $k$-feta} en el
	\emph{subconjunto} $S$ determina una estructura de subvariedad
	regular de $M$. Podemos concluir, por el teorema anterior, que estas
	dos estructuras coinciden. Adem\'{a}s, si un subconjunto verifica la
	propiedad de ser $k$-feta localmente, entonces tiene estructura de
	subvariedad regular y dicha estructura es \'{u}nica.
\end{obsUnicidadDeLaEstructuraRegular}

\begin{teoUnicidadDeLaEstructuraInmersa}%
	\label{thm:unicidaddelaestructurainmersa}
	Sea $M$ una variedad diferencial y sea $S\subset M$ una
	subvariedad (inmersa). Fijada la topolog\'{\i}a de $S$, existe una
	\'{u}nica estructura diferencial con respecto a la cual
	$S\subset M$ es subvariedad.
\end{teoUnicidadDeLaEstructuraInmersa}

\begin{proof}
	El \'{u}nico obst\'{a}culo en la demostraci\'{o}n en el caso
	en que $S$ fuere una subvariedad meramente inmersa es que no queda
	garantizado que la correstricci\'{o}n
	$\tilde{\iota}|:\,\tilde{S}\rightarrow S$ sea suave. Pero, si
	asumimos que las topolog\'{\i}as coinciden, como
	$\tilde{\iota}|=\id[S]$, concluimos que es continua y, por lo tanto,
	suave.
\end{proof}

\begin{teoUnicidadDebilmenteRegular}\label{thm:unicidaddebilmenteregular}
	Sea $M$ una variedad diferencial y sea $S\subset M$ una
	subvariedad d\'{e}bilmente regular. Entonces la topolog\'{\i}a y la
	estructura suave en $S$ dadas son las \'{u}nicas respecto de las
	cuales $S\subset M$ es una subvariedad (inmersa).
\end{teoUnicidadDebilmenteRegular}

\begin{proof}
	La demostraci\'{o}n de este resultado es an\'{a}loga a la de
	\ref{thm:unicidaddelaestructuraregular}, la \'{u}nica diferencia
	est\'{a} en la justificaci\'{o}n de que la correstricci\'{o}n
	$\tilde{\iota}|:\,\tilde{S}\rightarrow S$ es suave. En aquel caso,
	esto se deb\'{\i}a a que $S$ era regular e $\tilde{\iota}|$ era la
	correstricci\'{o}n a $S$, un subespacio, de una transformaci\'{o}n
	suave que resulta continua y, en consecuencia, suave. En este caso,
	esto queda garantizado porque $S$ se supone d\'{e}bilmente regular.
\end{proof}

\subsection{Extensi\'{o}n de funciones}
Dada una subvariedad $S\subset M$, hemos dado dos definiciones de lo que
significa que una funci\'{o}n definida en $S$ sea suave: por un lado,
podemos decir que una funci\'{o}n $f:\,S\rightarrow\bb{R}$ es suave, si es
suave respecto de la estructura diferencial en $S$, es decir, si es suave en
sentido usual tomando coordenadas; por otro lado, podemos decir que $f$ es
suave, si para cada punto $p\in S$ existe un abierto $U$ de $M$ tal que
$p\in U$ y una extensi\'{o}n suave $\tilde{f}:\,U\rightarrow\bb{R}$, es decir,
si $f$ es suave en $S$ como subconjunto de $M$. Usaremos la notaci\'{o}n
$C^{\infty}(S)$ para referirnos a las funciones suaves en $S$ en tanto
variedad diferencial. En el caso de subvariedades regulares, las dos
nociones coinciden. Notemos que, en el caso de las subvariedades que son
meramente inmersas, tambi\'{e}n es posible hacer coincidir ambas nociones,
si nos restringimos a un entorno \emph{de la subvariedad} suficientemente
peque\~{n}o alrededor de cada punto.

\begin{lemaDeExtensiones}\label{thm:deextensiones}
	Sea $M$ una variedad diferencial. Sea $S\subset M$ una subvariedad
	y sea $f\in C^{\infty}(S)$. \emph{(a)} Si $S$ es subvariedad regular,
	entonces existen un entorno $S\subset U\subset M$ y una funci\'{o}n
	suave $\tilde{f}\in C^{\infty}(U)$ tal que $\tilde{f}|_{S}=f$.
	\emph{(b)} Si, adem\'{a}s, $S$ es cerrada (regular propia), entonces
	se puede tomar $U=M$.
\end{lemaDeExtensiones}

\begin{proof}
	Supongamos que $\dim\,S=k$. Para cada $p\in S$ existe una carta
	$(U_{p},\varphi_{p})$ para $M$ centrada en $p$ tal que
	$S\cap U_{p}=\{x^{k+1}=0,\,\dots,\,x^{n}=0\}$. Sea
	$U=\bigcup_{p}\,U_{p}$. Entonces $S\subset U$ y $U$ es abierto en
	$M$. Para cada punto $p\in S$, definimos una funci\'{o}n
	$f_{p}:\,U_{p}\rightarrow\bb{R}$:
	\begin{align*}
		f_{p}(\lista*{x}{k},\,x^{k+1},\,\dots,\,x^{n}) & \,=\,
			f(\lista*{x}{k})
		\text{ .}
	\end{align*}
	%
	Expl\'{\i}citamente, $f_{p}(q)=f(\varphi^{-1}(\pi\circ\varphi(q)))$,
	donde $\pi:\,\bb{R}^{n}\rightarrow\bb{R}^{k}$ es la proyecci\'{o}n en
	las primeras $k$ coordenadas. Cada funci\'{o}n $f_{p}$ es suave y,
	si $q\in S\cap U_{p}$, entonces $f_{p}(q)=f(q)$.
	Para pegar estas funciones y as\'{\i} definir una funci\'{o}n
	en $U$, usamos particiones de la unidad. El abierto $U$ es una
	variedad (diferencial) y $\{U_{p}\}_{p\in S}$ es un cubrimiento por
	abiertos. Entonces existe una partici\'{o}n suave de la unidad
	subordinada al cubrimiento, $\{\psi_{p}\}_{p}$. Para cada punto
	$p\in S$, el producto $\psi_{p}f_{p}$ est\'{a} definido \'{u}nicamente
	en $U_{p}$, pero $V_{p}=\{\psi_{p}\not =0\}$ es abierto
	y $\clos{V_{p}}\subset U_{p}$. No es cierto \emph{a priori} que
	$\psi_{p}(p)\not =0$, pero podr\'{\i}amos tener un poco m\'{a}s
	de cuidado en la definici\'{o}n del cubrimiento como para que
	as\'{\i} sea (por ejemplo, tomando bolas coordenadas regulares
	dentro de cada $U_{p}$ y una partici\'{o}n subordinada al cubrimiento
	formado por las bolas coordenadas que contienen a las bolas
	coordenadas regulares). De todas maneras, podemos extender el
	producto $\psi_{p}f_{p}$ a todo $U$ por cero fuera de
	$\soporte{\psi_{p}}$. Esta funci\'{o}n sigue siendo suave. Sea,
	entonces
	\begin{align*}
		\tilde{f} & \,=\,\sum_{p\in S}\,\psi_{p}f_{p}
		\text{ .}
	\end{align*}
	%
	La funci\'{o}n $\tilde{f}$ est\'{a} definida en todo el abierto
	$U$ y es suave. Si $q\in S$,
	\begin{align*}
		\tilde{f}(q) & \,=\,\sum_{p\in S}\,\psi_{p}(q)f(q) \,=\,f(q)
		\text{ .}
	\end{align*}
	%
	Entonces $\tilde{f}\in C^{\infty}(U)$ y $\tilde{f}|_{S}=f$.

	Si asumimos que $S$ es subvariedad regular propia, entonces
	$U_{0}=S\subset M$ es un subconjunto cerrado y
	$\{U_{p}\}_{p\in S}\cup\{U_{0}\}$ es un cubrimiento de $M$ por
	abiertos. Si $\{\psi_{p}\}_{p}\cup\{\psi_{0}\}$ es una partici\'{o}n
	suave subordinada a este cubrimiento, entonces, definiendo
	$\tilde{f}=\sum_{p\in S}\psi_{p}f_{p}$, como antes, pero sin
	incluir la funci\'{o}n $\psi_{0}$, obtenemos una funci\'{o}n
	suave definida en $U=U_{0}\cup\bigcup_{p\in S}\,U_{p}=M$ que
	extiende a $f$.
\end{proof}

En general, si $S$ no es cerrada en $M$, no es posible extender una
funci\'{o}n arbitraria $f\in C^{\infty}(S)$ a toda la variedad: por ejemplo,
si $S=(-1,0)\cup (0,1)$ y $M=\bb{R}^{1}$, entonces la funci\'{o}n
$f$ que toma el valor $-1$ en $(-1,0)$ y el valor $1$ en $(0,1)$, es suave,
pero no se puede extender de manera suave a $\bb{R}^{1}$. De hecho, no se
puede extender de manera continua. Si existe una extensi\'{o}n continua,
?`existe una extensi\'{o}n suave? Por otro lado, si $S$ es una subvariedad
inmersa, las funciones suaves en $S$, en tanto variedad, no son, en general,
suaves como funciones definidas en el subconjunto $S\subset M$.
Si $f\in C^{\infty}(S)$ y $f:\,S\rightarrow\bb{R}$ es continua en $S$ como
subespacio de $M$, ?`es $f$ suave en $S$ como subconjunto de $M$?

Dejamos de lado estas preguntas. Las propiedades enunciadas en el lema
\ref{thm:deextensiones} son, en realidad, equivalencias.

\begin{propoDeNoExtensiones}\label{thm:denoextensiones}
	Sea $M$ una variedad diferencial y sea $S\subset M$ una subvariedad.
	\emph{(a)} Si toda funci\'{o}n $f\in C^{\infty}(S)$ se puede extender
	a una funci\'{o}n suave definida en alg\'{u}n abierto
	$U\subset M$ tal que $S\subset U$, entonces $S$ es subvariedad
	regular. \emph{(b)} Si toda funci\'{o}n $f\in C^{\infty}(S)$ se puede
	extender de manera suave a toda la variedad $M$, entonces $S$
	es una subvariedad propia.
\end{propoDeNoExtensiones}

\begin{proof}
	Supongamos que $S$ no es regular. Sea $p\in S$ un punto para
	el cual no existe una carta preferencial en $M$. Aun as\'{\i},
	existe un abierto $V\subset S$ tal que $V$ es una subvariedad
	regular de $M$. Como $S$ es una variedad diferencial, existe
	una partici\'{o}n de la unidad $\{\psi_{0},\psi_{1}\}$ para
	$S$ subordinada al cubrimiento $\{S\setmin\{p\},V\}$. En particular,
	$f=\psi_{1}|_{V}$ es suave en $V$ y $f(p)=1$. Como $V$ es una
	subvariedad regular de $M$, existe un abierto $U\subset M$ tal que
	$V\subset U$ y existe una funci\'{o}n $\tilde{f}:\,U\rightarrow\bb{R}$
	suave tal que $\tilde{f}|_{V}=f$. Pero, en $U\cap S$, no es cierto
	que $\tilde{f}$ coincida con $\psi_{1}$: como $p$ no admite una
	carta preferencial, si $\{U_{n}\}_{n\geq 1}$ es una sucesi\'{o}n
	de entornos de $p$ en $M$ con $U_{n+1}\subset U_{n}\subset U$ y
	$\bigcap_{n}\,U_{n}=\{p\}$, entonces existe
	\begin{align*}
		y_{n} & \,\in\, (S\cap U_{n})\setmin V
		\text{ .}
	\end{align*}
	%
	Evaluando $\tilde{f}$ en estos puntos, por continuidad de $\tilde{f}$
	en $M$, debe valer $\tilde{f}(y_{n})\to \tilde{f}(p)=f(p)=1$, pero
	$\psi_{1}(y_{n})=0$ para todo $n\geq 1$. En definitiva, la funci\'{o}n
	$\psi_{1}$ no se puede extender de manera suave (continua) a un
	abierto que contenga a $S$, pues, si se pudiese, dicha extensi\'{o}n
	deber\'{\i}a tomar valor $1$ en $p$ y $0$ en puntos arbitrariamente
	cerca (en $M$) de $p$.

	Supongamos que $S\subset M$ es una subvariedad (inmersa o regular),
	pero que no es cerrada como subconjunto de $M$. Entonces existe una
	sucesi\'{o}n de puntos $y_{n}\in S$ y un punto $y\in M\setmin S$
	tales que $y_{n}\to y$ en la topolog\'{\i}a de $M$ y tales que
	$\{y_{n}\}_{n}$ es un subconjunto cerrado en $S$. Para cada
	entero $n\geq 1$ sea $f(y_{n})=n$. Como $\{y_{n}\}_{n}$ es cerrado
	en $S$, podemos extender $f$ de manera suave a $S$ (por ejemplo,
	tomando entornos $V_{n}$ de cada punto $y_{n}$ de manera que
	$y_{m}\not\in V_{n}$, si $m\not =n$, y una partici\'{o}n
	de la unidad subordinada a estos conjuntos y a
	$S\setmin\{y_{n}\}_{n}$). Pero no hay manera de extender $f$ a una
	funci\'{o}n suave en $y$ (ni siquiera de manera continua).
\end{proof}

\subsection{El espacio tangente a una subvariedad}
Sea $M$ una variedad diferencial y sea $S\subset M$ una subvariedad.
Dado que la inclusi\'{o}n $\inc[S]:\,S\rightarrow M$ es una inmersi\'{o}n,
en cada punto $p\in S$, el diferencial $\diferencial[p]{\inc}:%
\,\tangente[p]{S}\rightarrow\tangente[p]{M}$ es una transformaci\'{o}n
lineal inyectiva. Ya vimos en alguna ocasi\'{o}n c\'{o}mo el diferencial
permite identificar el espacio tangente a un abierto en un punto dado.
Aquella identificaci\'{o}n funciona de manera general en el contexto de
subvariedades. Sea $v\in\tangente[p]{S}$ un vector tangente a $S$ en $p$ y
sea $\tilde{v}=\diferencial[p]{\inc}$ su imagen en $\tangente[p]{M}$.
En tanto derivaci\'{o}n, dada $f\in C^{\infty}(M)$,
\begin{align*}
	\tilde{v}\,f & \,=\,\diferencial[p]{\inc}(v)\,f \,=\,
		v(f\circ\inc) \,=\,v(f|_{S})
	\text{ .}
\end{align*}
%

Una forma alternativa de caracterizar el espacio tangente $\tangente[p]{S}$
cuando $S$ es una subvariedad arbitraria como subespacio de
$\tangente[p]{M}$ es en t\'{e}rminos de curvas suaves. Sabemos que
todo vector tangente a una variedad es igual a la velocidad de alguna curva.
Supongamos, entonces, en primer lugar, que $\gamma:\,J\rightarrow M$ es una
curva suave con origen en $p$ tal que $\gamma(J)\subset S$ y
$\eta=\gamma|:\,J\rightarrow S$ es suave, tambi\'{e}n. Entonces, tomando
diferenciales,
\begin{align*}
	\dot{\gamma}(0) & \,=\,\diferencial[p]{\inc}(\dot{\eta}(0))
	\text{ ,}
\end{align*}
%
de lo que se deduce que el vector tangente $\dot{\gamma}(0)\in\tangente[p]{M}$
pertenece a la imagen del diferencial $\diferencial[p]{\inc}$.
Rec\'{\i}procamente, si $v\in\tangente[p]{S}$ y $\eta:\,J\rightarrow S$ es
una curva suave con origen en $p$ y velocidad $\dot{\eta}(0)=v$, entonces
$\gamma=\inc\circ\eta:\,J\rightarrow M$ es suave y
$\dot{\gamma}(0)=\diferencial[p]{\inc}(v)=\tilde{v}$. En definitiva,
en t\'{e}rminos de curvas, un vector $v$ tangente a $M$ en $p$
\emph{es tangente a la subvariedad $S$}, si y s\'{o}lo si existe una curva
suave en $M$ cuya imagen est\'{e} contenida en $S$, que sea suave como
curva en $S$ y con velocidad $v$.

\begin{propoTangenteSubvarCoordenadas}\label{thm:tangentesubvarcoordenadas}
	Sea $M$ una variedad diferencial y sea $S\subset M$ una subvariedad.
	Sea $p\in S$ y sea $V\subset S$ un entorno de $p$ que es subvariedad
	regular de $M$. Sea $(U,\varphi)$ una carta para $M$ en $p$ tal que
	$V\cap U=\{x^{k+1}=0,\,\dots,\,x^{n}=0\}$. Entonces
	\begin{align*}
		\tangente[p]{S} & \,=\,\tangente[p]{V} \,=\,
		\generado{\gancho[p]{x^{1}},\,\dots,\,\gancho[p]{x^{k}}}
		\,\subset\,\tangente[p]{M}
		\text{ .}
	\end{align*}
	%
\end{propoTangenteSubvarCoordenadas}

\begin{proof}
	Sea $v\in\tangente[p]{S}$. Entonces
	\begin{align*}
		v & \,\equiv\,\diferencial[p]{\inc}(v) \,=\,
			\diferencial[p]{v}(x^{i})\gancho[p]{x^{i}} \\
		& \,=\, v(x^{i}|_{V\cap U})\gancho[p]{x^{i}} \,=\,
			\sum_{i=1}^{k}\,v(x^{i}|_{V\cap U})\gancho[p]{x^{i}}
		\text{ .}
	\end{align*}
	%
	Con lo cual
	$\tangente[p]{S}\subset\generado{\gancho[p]{x^{1}},\,\dots,\,%
	\gancho[p]{x^{k}}}$. Podemos concluir sabiendo que
	$\dim\,\tangente[p]{S}=k$. Otra manera de concluir que los
	subespacios coinciden es usando curvas: si $w=w^{i}\gancho[p]{x^{i}}$
	con $w^{k+1}=\,\cdot\,w^{n}=0$, entonces, tomamos
	$\gamma:(-\epsilon,\epsilon)\rightarrow M$ tal que
	$\gamma(0)=p$, $\dot{\gamma}(0)=w$ y $\epsilon>0$ suficientemente
	peque\~{n}o de manera que $\gamma(J)\subset U$. Precisamente,
	elegimos
	$\gamma(t)=\varphi^{-1}(w^{1}t,\,\dots,\,w^{k}t,\,0,\,\dots,\,0)$.
	En particular, $\gamma$ tiene imagen en $V$ y, como $V$ es una
	subvariedad regular de $M$, $\gamma$ es suave como curva en $V$.
	En definitiva, $w\in\tangente[p]{S}$.
\end{proof}

En el caso en que $S\subset M$ sea una subvariedad regular, el subespacio
$\tangente[p]{S}\subset\tangente[p]{M}$ est\'{a} caracterizado como el
conjunto de ceros de ciertas ecuaciones. Hay, al menos, dos maneras de
interpretar esta afirmaci\'{o}n.

\begin{propoTangenteSubvarRegularI}\label{thm:tangentesubvarregulari}
	Se $M$ una variedad diferencial (sin borde) y sea $S\subset M$ una
	subvariedad regular. Sea $p\in S$. En tanto subespacio de
	$\tangente[p]{M}$,
	\begin{align*}
		\tangente[p]{S} & \,=\,\left\lbrace v\in\tangente[p]{M}\,:\,
			v\,f=0\,\forall f\in C^{\infty}(M),\, f|_{S}=0
			\right\rbrace
		\text{ .}
	\end{align*}
	%
\end{propoTangenteSubvarRegularI}

\begin{propoTangenteSubvarRegularII}\label{thm:tangentesubvarregularii}
	Sea $M$ una variedad diferencial (sin borde) y sea $S\subset M$ una
	subvariedad regular. Si $\Phi:\,U\rightarrow N$ es una
	transformaci\'{o}n suave definida en un abierto $U$ de $M$ tal que
	$S\cap U=\Phi^{-1}(c)$, para cierto valor regular $c\in N$, entonces,
	para todo punto $p\in S\cap U$,
	\begin{align*}
		\tangente[p]{S} & \,=\, \ker\,\big(\diferencial[p]{\Phi}:\,
			\tangente[p]{M}\rightarrow\tangente[\Phi(p)]{N}\big)
		\text{ .}
	\end{align*}
	%
\end{propoTangenteSubvarRegularII}

\begin{proof}[Demostraci\'{o}n de \ref{thm:tangentesubvarregulari}]
	Sea $v\in\tangente[p]{S}$. Si $f\in C^{\infty}(M)$ es una funci\'{o}n
	suave que se anula en $S$, entonces
	\begin{align*}
		v\,f & \,\equiv\,v(f|_{S}) \,=\, 0
		\text{ .}
	\end{align*}
	%
	Rec\'{\i}procamente, dado un vector tangente $w\in\tangente[p]{M}$
	que verifica $w\,f=0$ para toda funci\'{o}n suave $f\in C^{\infty}(M)$
	tal que $f|_{S}=0$, entonces podemos demostrar que
	$w\in\tangente[p]{S}$. Para ver que esto es as\'{\i}, elegimos
	una carta $(U,\varphi)$ para $M$ en $p$ tal que
	$S\cap U=\{x^{k+1}=0,\,\dots,\,x^{n}=0\}$. (Aqu\'{i} podr\'{\i}amos
	tomar, en lugar de $S$, un entorno $V$ de $p$ en $S$ que sea
	subvariedad regular de $M$ y una carta $(U,\varphi)$ tal que
	$V\cap U$ sea el conjunto en donde las \'{u}ltimas coordenadas se
	anulan. Esto es v\'{a}lido incluso en el caso de una subvariedad
	meramente inmersa, pero m\'{a}s adelante se requerir\'{a} que $S$
	sea subvariedad regular). Por \ref{thm:tangentesubvarcoordenadas},
	sabemos que $w\in\tangente[p]{S}$, si y s\'{o}lo si, con
	respecto a las funciones coordenadas,
	$w(x^{k+1})=\,\cdots\,=w(x^{n})=0$. Vamos a ver que esta \'{u}ltima
	condici\'{o}n se cumple. Sea $\psi$ una funci\'{o}n chich\'{o}n
	con soporte en $U$ y que vale $1$ en un entorno (de $M$) de $p$
	(por ejemplo, una de las dos funciones de alguna partici\'{o}n
	suave de la unidad subordinada al cubrimiento
	$\{U,M\setmin\clos{U'}\}$, donde $U'$ es un entorno de $p$ en $M$
	cuya clausura est\'{a} contenida en $U$). Para cada $j>k$, la
	funci\'{o}n $\psi(x)x^{j}\equiv\psi\cdot\varphi^{j}$ es suave en
	$U$ y tiene soporte contenido en $U$. Si llamamos $f$ a la 
	extensi\'{o}n de esta funci\'{o}n por cero fuera de $U$, entonces
	$f\in C^{\infty}(M)$ y $f|_{S}=0$ (esto no es cierto, en general,
	si $S$ no es regular). As\'{\i},
	\begin{align*}
		0 & \,=\, w\,f \,=\,w^{i}\derivada{f}{x^{i}} \,=\, v^{j}
	\end{align*}
	%
	y concluimos que $w\in\tangente[p]{S}$.
\end{proof}

\begin{proof}[Demostraci\'{o}n de \ref{thm:tangentesubvarregularii}]
	Si $S\cap U=\Phi^{-1}(c)$ para alg\'{u}n punto $c\in N$, entonces
	$\Phi|_{S\cap U}$ toma constantemente el valor $c$. As\'{\i},
	\begin{align*}
		\diferencial[p]{\Phi}\cdot\diferencial[p]{\inc} & \,=\,0
	\end{align*}
	%
	en tanto transformaci\'{o}n lineal de $\tangente[p]{S}$ en
	$\tangente[\Phi(p)]{N}$. Es decir,
	\begin{align*}
		\img\,\diferencial[p]{\inc} & \,\subset\,
		\ker\,\diferencial[p]{\Phi}
		\text{ .}
	\end{align*}
	%
	Pero, como $\diferencial[p]{\Phi}$ es suryectivo,
	\begin{align*}
		\dim\big(\ker\,\diferencial[p]{\Phi}\big) & \,=\,
			\dim\,\tangente[p]{M}\,-\,\tangente[\Phi(p)]{N}
			\,=\,\dim\,M\,-\,\dim\,N \\
		& \,=\,\dim\,S \,=\, \dim\,\tangente[p]{S}
		\,=\,\dim\big(\img\,\diferencial[p]{\inc}\big)
		\text{ .}
	\end{align*}
	%
	En definitiva,
	$\ker\,\diferencial[p]{\Phi}=\img\,\diferencial[p]{\inc}$.
\end{proof}

\begin{obsTangenteValorRegular}\label{obs:tangentevalorregular}
Si la funci\'{o}n que define localmente a $S$ como conjunto de nivel regular
es de la forma $\Phi=(\lista*{\Phi}{k}):\,U\rightarrow\bb{R}^{k}$, entonces
$w\in\tangente[p]{M}$ pertenece al n\'{u}cleo $\ker\,\diferencial[p]{\Phi}$,
si y s\'{o}lo si $w\,\Phi^{j}=0$ para cada una de las funionces que
componen a $\Phi$: si $\diferencial[p]{\Phi}(w)=0$, entonces
\begin{align*}
	w\,\Phi^{j} & \,=\,w(y^{j}\circ\Phi) \,=\,
		\diferencial[p]{\Phi}(w)\,y^{j} \,=\, 0
\end{align*}
%
para todo $j\in [\![1,k]\!]$ y, si $w$ se anula en las funciones
$\Phi^{j}=y^{j}\circ\Phi$, entonces
\begin{align*}
	\diferencial[p]{\Phi}(w) & \,=\,
		\diferencial[p]{\Phi}(w)(y^{j})\gancho[\Phi(p)]{y^{j}}
		\,=\,w(y^{j}\circ\Phi)\gancho[\Phi(p)]{y^{j}}
		\,=\, 0
	\text{ .}
\end{align*}
%
\end{obsTangenteValorRegular}

\begin{obsTangenteConjuntoDeNivel}\label{obs:tangeteconjuntodenivel}
	Supongamos que $S\subset M$ est\'{a} dada como el conjunto de nivel de
	una transformaci\'{o}n $\Phi:\,M\rightarrow N$ de rango constante,
	no necesariamente una submersi\'{o}n. Aun as\'{\i}, sabemos, por el
	teorema del reango \ref{thm:delrango}, que
	$\codim\,S=\rango{\Phi}$. Pero $\codim(\ker\,\diferencial[p]{\Phi})$
	es igual al rango de $\diferencial[p]{\Phi}$. Entonces
	$\tangente[p]{S}$ y $\ker\,\diferencial[p]{\Phi}$ tienen la misma
	dimensi\'{o}n. Por otro lado, sigue siendo cierto, en este caso
	tambi\'{e}n, que $\tangente[p]{S}\subset\ker\,\diferencial[p]{\Phi}$.
	En definitiva, deben coincidir.
\end{obsTangenteConjuntoDeNivel}

Ya hemos mencionado que el borde de una variedad con borde es una
subvariedad regular. Sea $M$ una variedad con borde
$\borde[M]\not=\varnothing$. De acuerdo con las observaciones realizadas al
comienzo de esta parte, podemos, en esta situaci\'{o}n tambi\'{e}n,
identificar el tangente a $\borde[M]$ en un punto $p$ con un subespacio
de $\tangente[p]{M}$ v\'{\i}a el diferencial de la inclusi\'{o}n
$\inc[{\borde[M]}]:\,\borde[M]\rightarrow M$ o en t\'{e}rminos de curvas
suaves. Las identificaciones que hacen uso de coordenadas, por otro lado,
funcionaron porque ten\'{\i}amos a nuestra disposici\'{o}n las cartas
preferenciales, que, en el caso de una variedad con borde no est\'{a}n
definidas. En lugar de cartas preferenciales, contamos con las cartas de
borde. Usando estas cartas, podemos intentar repetir los argumentos
de las demostraciones de los resultados de esta parte.

Sea $M$ una variedad con borde $\borde[M]\not =\varnothing$ y sea
$p\in\borde[M]$. Sea $\inc=\inc[{\borde[M]}]$ la includi\'{o}n del borde
en la variedad. Si $v\in\tangente[p]{\borde[M]}$, entonces, como antes,
si llamamos $\tilde{v}=\diferencial[p]{\inc}(v)$, vale que
\begin{align*}
	\tilde{v}\,f & \,=\,v(f\circ\inc) \,=\,v(f|_{\borde[M]})
\end{align*}
%
para toda funci\'{o}n $f\in C^{\infty}(M)$. En coordenadas,
\begin{align*}
	\tilde{v} & \,=\,v^{i}\gancho[p]{x^{i}}
\end{align*}
%
donde $v^{i}=\tilde{v}(x^{i})=v(x^{i}|_{\borde[M]})$. En particular,
como $x^{n}|_{\borde[M]}=0$, debe valer que $v^{n}=0$. Entonces
\begin{align*}
	\tangente[p]{\borde[M]} & \,\subset\,
		\generado{\gancho[p]{x^{1}},\,\dots,\,\gancho[p]{x^{n-1}}}
		\,=\,\left\lbrace w=w^{i}\gancho[p]{x^{i}}\in\tangente[p]{M}
		\,:\,w^{n}=0\right\rbrace
		\text{ .}
\end{align*}
%
Ahora, como en la demostraci\'{o}n de \ref{thm:tangentesubvarcoordenadas},
hay varias maneras de concluir que estos subespacios son iguales. Podemos
usar la dimensi\'{o}n: $\dim(\tangente[p]{\borde[M]})=n-1$; podemos
usar la curva $\eta(t)=(w^{1}t,\,\dots,\,w^{n-1}t)$ en $\borde[M]$ con
origen en $p$, que es suave y, al componer con la inclusi\'{o}n,
$\gamma=\inc\circ\eta$ es suave en $M$, tiene origen en $p$ y velocidad
$\dot{\gamma}(0)=w$. Tambi\'{e}n podemos recurrir al hecho de que,
como $\varphi(\borde[M]\cap U)=\borde[{\hemi[n]}]\cap\varphi(U)$, localmente
contamos con una descripci\'{o}n de $\borde[M]$ como conjunto de nivel
regular de una funci\'{o}n suave: $x^{n}:\,U\rightarrow\bb{R}$ y
$\borde[M]\cap U=\{x^{n}=0\}$.

Esta descripci\'{o}n del tangente al borde de una variedad nos permite,
tambi\'{e}n, descomponer el tangente a la variedad misma. Sea $M$ una
variedad diferencial con borde $\borde[M]\not=\varnothing$ y sea $p$
un punto del borde. Sea $(U,\varphi)$ una carta de borde centrada en $p$.
En coordenadas, $w\in\tangente[p]{M}$ se escribe como
$w=w^{i}\gancho[p]{x^{i}}$ y $w\in\tangente[p]{\borde[M]}$, si y s\'{o}lo
si $w^{n}=0$. El resto de los vectores tangentes, aquellos con
$w^{n}\not =0$, se dividen en dos clases: aquellos con $w^{n}>0$ y
aquellos con $w^{n}<0$. Para ver que esta divisi\'{o}n no depende de la
carta de borde elegida, usamos la descripci\'{o}n de los vectores
tangentes como velocidades de curvas contenidas en la variedad. Supongamos
que $w\in\tangente[p]{M}$ es tal que $w^{n}<0$, entonces la curva
$\gamma(t)=(w^{1}t,\,\dots,\,w^{n-1}t,\,w^{n}t)$ definida en $(-\epsilon,0]$
es una curva suave en $M$, $\gamma(0)=p$ y $\dot{\gamma}(0)=w$.
Rec\'{\i}procamente, si $w\in\tangente[p]{M}$ es tal que existe una
curva suave $\gamma:\,(-\epsilon,0]\rightarrow M$ con origen en $p$ y
velocidad $w$, entonces en las coordenadas dadas, como en cualquier otra
carta,
\begin{align*}
	w^{n} & \,=\,\dot{\gamma}(0)(x^{n}) \,=\,
		\gancho[0]{t}(x^{n}\circ\gamma) \\
	& \,=\,\lim_{t\to0,\,t<0}\,\frac{\gamma^{n}(t)-\gamma^{n}(0)}{t}
		\,<\,0
	\text{ ,}
\end{align*}
%
pues $\gamma^{n}(0)=0$, $t<0$ y $\gamma^{n}(t)>0$. An\'{a}logamente,
$w^{n}>0$, si y s\'{o}lo si existe una curva suave
$\gamma:\,[0,\epsilon)\rightarrow M$ con origen en $p$ y velocidad $w$.

\begin{propoTangenteBordeCurvas}\label{thm:tangentebordecurvas}
	Sea $M$ una variedad diferencial y sea $p\in\borde[M]$. Sea
	$(U,\varphi)$ una carta de borde para $M$ centrada en $p$ y
	sea $w\in\tangente[p]{M}$ un vector tangente a $M$ en $p$.
	Entonces \emph{(i)} $w$ es la velocidad de una curva suave
	$\gamma:\,(-\epsilon,0]\rightarrow M$ con origen en $p$, si y
	s\'{o}lo si, en coordenadas, $w^{n}=w(x^{n})<0$; \emph{(ii)} $w$
	es la velocidad de una curva suave
	$\gamma:\,[0,\epsilon)\rightarrow M$ con origen en $p$, si y
	s\'{o}lo si $w^{n}>0$; \emph{(iii)} $w\in\tangente[p]{\borde[M]}$,
	si y s\'{o}lo si $w^{n}=0$, si y s\'{o}lo si $w$ es la velocidad
	de una curva suave con origen en $p$ e imagen contenida en
	$\borde[M]$.
\end{propoTangenteBordeCurvas}

Notemos que los tres conjuntos de la proposici\'{o}n anterior son disjuntos
y que, adem\'{a}s, si $w$ es tal que $w^{n}>0$ para alguna --y, por lo
tanto, para todas-- carta, entones $-w$ verifica la desigualdad opuesta.
Aquellos vectores tangentes con $w^{n}<0$ se dice que \emph{apuntan hacia %
afuera} y aquellos con $w^{n}>0$ que \emph{apuntan hacia adentro}. Si
$w$ apunta hacia adentro, entonces $-w$ apunta hacia afuera.

Hemos mencionado, m\'{a}s arriba que $\borde[M]$ se puede ver localmente
como el conjunto de ceros de la funci\'{o}n coordenada $x^{n}$ (dada una
carta de borde para la variedad). En general, decimos que una funci\'{o}n
$f:\,M\rightarrow [0,\infty)$ es una \emph{funci\'{o}n de definici\'{o}n %
del borde}, si $f^{-1}(0)=\borde[M]$ y $\diferencial[p]{f}\not=0$ para
todo $p\in\borde[M]$. Estas funciones son an\'{a}logas a las funciones de
definici\'{o}n de un conjunto de nivel.

\begin{propoBordeDeNivel}\label{thm:bordedenivel}
	Sea $M$ una variedad diferencial con $\borde[M]\not =\varnothing$.
	Existe una funci\'{o}n de definici\'{o}n del borde de $M$.
\end{propoBordeDeNivel}

\begin{proof}
	Sea $\{(U_{\alpha},\varphi_{\alpha})\}_{\alpha}$ un atlas compatible
	para $M$. Sea $\{\psi_{\alpha}\}_{\alpha}$ una partici\'{o}n
	de la unidad subordinada al cubrimiento $\{U_{\alpha}\}_{\alpha}$.
	Para definir la funci\'{o}n de definici\'{o}n de borde, definimos
	primero funciones en cada abierto $U_{\alpha}$ del cubrimiento.
	Si $U_{\alpha}$ es el dominio de una carta del interior,
	es decir, si $U_{\alpha}\cap\borde[M]=\varnothing$, entonces
	definimos $f_{\alpha}:\,U_{\alpha}\rightarrow\bb{R}$ como la
	funci\'{o}n suave que toma constantemente el valor $1$. Si, en cambio,
	$U_{\alpha}\cap\borde[M]\not=\varnothing$, entonces
	$(U_{\alpha},\varphi_{\alpha})$ es una carta de borde. En este caso,
	definimos $f_{\alpha}=x^{n}$. En particular,
	\begin{align*}
		f_{\alpha}^{-1}(0) & \,=\,\{x^{n}=0\} \,=\,
		\borde[M]\cap U_{\alpha}
		\text{ .}
	\end{align*}
	%
	La funci\'{o}n $\psi_{\alpha}$, por otro lado, tiene soporte en
	un cerrado contenido en $U_{\alpha}$, entonces, si extendemos
	$f_{\alpha}\psi_{\alpha}$ por cero fuera de $U_{\alpha}$, queda
	definida una funci\'{o}n suave en $M$ que se anula fuera
	de $\soporte{\psi_{\alpha}}$. Sea
	$f=\sum_{\alpha}\,\psi_{\alpha}f_{\alpha}$. Entonces $f$ est\'{a}
	bien definida y es suave en $M$, porque los sumandos lo son y porque
	los soportes de los sumandos forman una familia localmente
	finita.

	Si $p\in\borde[M]$, entonces $\psi_{\alpha}(p)\not=0$ s\'{o}lo si
	$U_{\alpha}$ interseca el borde, y, en ese caso, $f_{\alpha}(p)=0$.
	Entonces $f(p)=0$, si $p$ es un punto del borde. Rec\'{\i}procamente,
	si $p\in\interior{M}$, entonces $f_{\alpha}(p)>0$ para todo $\alpha$.
	Como $\sum_{\alpha}\,\psi_{\alpha}=1$, existe $\alpha$ tal que
	$\psi_{\alpha}(p)\not =0$. Entonces $f(p)>0$, si $p$ es un punto
	del interior.

	Resta verificar que el diferencial $\diferencial[p]{f}$ no es la
	transformaci\'{o}n lineal (funcional lineal) nula para ning\'{u}n
	punto $p\in\borde[M]$. Sea $w\in\tangente[p]{M}$ un vector tangente
	que apunta hacia afuera. Como $p$ es un punto del borde de $M$,
	si $\alpha$ es tal que $p\in U_{\alpha}$ y
	$\varphi_{\alpha}=(\lista*{x}{n})$, entonces, por definici\'{o}n,
	$f_{\alpha}(p)=0$ y
	\begin{align*}
		\diferencial[p]{f_{\alpha}}(w) & \,=\,
			\diferencial[p]{x^{n}}(w) \,=\,w(x^{n})\,<\,0
		\text{ .}
	\end{align*}
	%
	En particular, como $p$ pertenece a alg\'{u}n $U_{\alpha}$,
	vale que $\diferencial[p]{f_{\alpha}}(w)\leq 0$ para todo
	$\alpha$ y que $\diferencial[p]{f_{\alpha}}(W)<0$ estrictamente
	para, al menos, un $\alpha$. En definitiva,
	\begin{align*}
		\diferencial[p]{f}(w) & \,=\,\sum_{\alpha}\,\big(
			f_{\alpha}(p)\diferencial[p]{\psi_{\alpha}}(w) +
			\psi_{\alpha}(p)\diferencial[p]{f_{\alpha}}(w)
			\big)
			\,<\,0
		\text{ ,}
	\end{align*}
	%
	de lo que se deduce que $\diferencial[p]{f}$ no es la
	transformaci\'{o}n cero.
\end{proof}

\begin{obsBordeDeNivel}\label{obs:bordedenivel}
	De la \'{u}ltima parte de la demostraci\'{o}n anterior, se ve
	que, si, en lugar de tomar un vector que apunta hacia afuera,
	hubi\'{e}semos elegido un vector $w$ que apunta hacia adentro,
	entonces $\diferencial[p]{f}(w)>0$ estrictamente, y, si
	$w\in\tangente[p]{\borde[M]}$, entonces $\diferencial[p]{f}(w)=0$.
	En general, si $f$ es cualquier funci\'{o}n de definici\'{o}n de
	borde para $M$, entonces estas afirmaciones siguen siendo ciertas.
	Por ejemplo, si $w\in\tangente[p]{M}$ apunta hacia afuera, entonces,
	dada una carta de borde $(U,\varphi)$ centrada en $p$,
	\begin{align*}
		\diferencial[p]{f}(w) & \,=\,w^{i}\derivada{f}{x^{i}}
			\,=\, w^{n}\derivada{f}{x^{n}}
		\text{ ,}
	\end{align*}
	%
	pues $f$ es constantemente $0$ en $\{x^{n}=0\}=\borde[M]\cap U$.
	Pero $\derivada{f}{x^{n}}>0$ y $w^{n}<0$. El mismo argumento,
	cambiando $w^{n}<0$ por $w^{n}>0$ o por $w^{n}=0$ sigue valiendo,
	si, en lugar de tomar un vector que apunta hacia afuera, elegimos
	$w$ apuntando hacia adentro o tangente al borde de la variedad.
\end{obsBordeDeNivel}

\begin{obsDiferencialComoFuncional}\label{obs:diferencialcomofuncional}
	Tanto en la demostraci\'{o}n de \ref{thm:bordedenivel}, como en la
	observaci\'{o}n \ref{obs:bordedenivel} abusamos de la siguiente
	identificaci\'{o}n entre elementos de $\bb{R}$ y elementos del
	tangente $\bb{R}$. Esta identificaci\'{o}n consiste en identificar,
	paralelamente, el diferencial de una funci\'{o}n
	$f:\,M\rightarrow\bb{R}$ con un elemento del dual de
	$\tangente[p]{M}$. Concretamente, si $f:\,M\rightarrow\bb{R}$ es una
	funci\'{o}n suave y $w\in\tangente[p]{M}$ es un vector tangente,
	entonces $w\,f\in\bb{R}$ y
	$\diferencial[p]{f}(w)\in\tangente[f(p)]{\bb{R}}$. La relaci\'{o}n
	entre ambos est\'{a} dada por evaluar el vector tangente a $\bb{R}$
	en la funci\'{o}n identidad $\id[\bb{R}]:\,\bb{R}\rightarrow\bb{R}$.
	Es decir, identificamos el vector $\diferencial[p]{f}(w)$ con el
	n\'{u}mero real
	\begin{align*}
		\diferencial[p]{f}(w)(\id[\bb{R}]) & \,=\,
			w(\id[\bb{R}]\circ f) \,=\, w\,f
		\text{ .}
	\end{align*}
	%
	Con esta identificaci\'{o}n, tiene sentido decir que
	$\diferencial[p]{f}(w)$ es mayor, menor o igual a cero.
\end{obsDiferencialComoFuncional}

%
\section{Subvariedades con borde}
\theoremstyle{plain}

\theoremstyle{remark}

%-------------

Completamos esta introducci\'{o}n a subvariedades considerando la posibilidad
de la existencia de borde. En primer lugar, ampliamos la definici\'{o}n
dada en la secci\'{o}n \ref{sec:subvars}. Sea $M$ una variedad diferencial
(topol\'{o}gica). Una \emph{subvariedad con borde} de $M$ es un subconjunto
$S\subset M$ con una topolog\'{\i}a y una estructura diferencial de manera
que $S$ sea una variedad diferencial y que la inclusi\'{o}n
$\inc[S]:\,S\rightarrow M$ sea una inmersi\'{o}n (suave, topol\'{o}gica).
Si la inclusi\'{o}n es un embedding, decimos que $S$ es una \emph{subvariedad %
regular}.

\subsection{Dominios regulares}
Un \emph{dominio} o \emph{dominio regular} en $M$ es una subvariedad regular
propia de codimensi\'{o}n $0$ (por lo tanto, cerrada, a diferencia de un
``dominio'' en Ecuaciones diferenciales). Los dominios tienen la propiedad de
que su interior y borde en tanto subespacio topol\'{o}gico de la variedad
ambiente coinciden con su interior y borde, respectivamente, en tanto
variedad.

Si $f\in C^{\infty}(M)$, entonces los conjuntos $\{f\leq b\}$ y
$\{a\leq f\leq b\}$ son dominios regulares en $M$, siempre que $a$ y $b$ sean
valores regulares de $f$. Si $D$ es un dominio regular en $M$, existen una
funci\'{o}n suave $f\in C^{\infty}(M)$ y valores regulares $a,b$ tales que
$D=\{a\leq f\leq b\}$, o bien $D=\{f\leq b\}$. Una funci\'{o}n con esta
propiedad se dice \emph{funci\'{o}n de definici\'{o}n} para el dominio $D$.

\subsection{Algunas propiedades de las subvariedades con borde}
Enunciamos versiones an\'{a}logas de algunas de las propiedades estudiadas
anteriormente en el caso de subvariedades sin borde. Muchas de aquellas
propiedades siguen siendo v\'{a}lidas en este contexto, pero no todas.

Sea $M$ una variedad diferencial. Todo abierto de $M$ es una subvariedad
regular de codimensi\'{o}n $0$ en $M$, pero no toda subvariedad de
codimensi\'{o}n $0$ es un subespacio abierto de $M$.

La imagen de un embedding $F:\,N\rightarrow M$ es una subvariedad regular
con borde.

Una subvariedad regular se dice \emph{propia}, si la inclusi\'{o}n es propia.
Una subvariedad regular es propia, si y s\'{o}lo si es subespacio cerrado.

Toda subvariedad con borde es localmente regular, es decir, para todo
punto de la subvariedad, existe un entorno del punto que tiene estructura
de subvariedad regular.

\subsection{Fetas de borde}
Sea $M$ una variedad diferencial sin borde. Sea $(U,\varphi)$ una carta
de $M$. Una \emph{media $k$-feta de $U$} es un subconjunto de la forma
\begin{align*}
	\left\lbrace (\lista*{x}{n})\in U\,:\,
		x^{k+1}=c^{k+1},\,\dots,\,x^{n}=c^{n}\text{ y }
		x^{k}\geq 0\right\rbrace
	\text{ .}
\end{align*}
%
De un subconjunto $S\subset M$ se dice que \emph{verifica localmente la %
condici\'{o}n de ser $k$-feta}, si para todo punto $p\in S$ existe una
carta $(U,\varphi)$ para $M$ tal que $p\in U$ y que $S\cap U$ sea, o bien
una $k$-feta de $U$, o bien una media $k$-feta de $U$. Dependiendo del caso,
decimos que $(U,\varphi)$ es una \emph{carta preferencial} para $S$ en $M$,
o que es una \emph{carta preferencial de borde} para $S$ en $M$.

Toda subvariedad regular con borde (de una variedad diferencial sin borde)
verifica localmente la condici\'{o}n de ser $k$-feta. Rec\'{\i}procamente,
si un subconjunto $S\subset M$ verifica la condici\'{o}n, entonces, con
la topolog\'{\i}a de subespacio es una variedad topol\'{o}gica con borde
de dimensi\'{o}n $k$ y admite una estructura diferencial de manera que sea
subvariedad regular de $M$.

%

%--------

\chapter{Campos y $1$-formas}
\section{Campos vectoriales}
\theoremstyle{plain}
\newtheorem{teoDerivacionesYCampos}{Teorema}[section]
\newtheorem{propoCamposEnCoordenadas}[teoDerivacionesYCampos]{Poposici\'{o}n}
\newtheorem{propoExtenderUnCampo}[teoDerivacionesYCampos]{Proposici\'{o}n}
\newtheorem{lemaAplicarCamposLocalmenteDeterminado}%
	[teoDerivacionesYCampos]{Lema}
\newtheorem{propoEquivCampoSuave}[teoDerivacionesYCampos]{Proposici\'{o}n}
\newtheorem{propoFRelacionados}[teoDerivacionesYCampos]{Proposici\'{o}n}
\newtheorem{teoPushforwardDifeo}[teoDerivacionesYCampos]{Teorema}
\newtheorem{propoCamposTangentes}[teoDerivacionesYCampos]{Proposici\'{o}n}
\newtheorem{propoTangenteSubvarSubvar}[teoDerivacionesYCampos]{Proposici\'{o}n}
\newtheorem{propoCamposTangentesSuaves}[teoDerivacionesYCampos]%
	{Proposici\'{o}n}
\newtheorem{lemaElCorcheteEsSuave}[teoDerivacionesYCampos]{Lema}
\newtheorem{propoCorcheteFRelacionados}[teoDerivacionesYCampos]%
	{Proposici\'{o}n}
\newtheorem{coroTangentesCorcheteCerrados}[teoDerivacionesYCampos]%
	{Proposici\'{o}n}
\newtheorem{lemaExtenderCampos}[teoDerivacionesYCampos]{Lema}

\theoremstyle{remark}
\newtheorem{obsCamposConstantes}{Observaci\'{o}n}[section]
\newtheorem{obsDerivacionEsDerivacion}[obsCamposConstantes]{Observaci\'{o}n}
\newtheorem{obsFRelacionados}[obsCamposConstantes]{Observaci\'{o}n}
\newtheorem{obsTangenteSubvarReg}[obsCamposConstantes]{Observaci\'{o}n}
\newtheorem{obsCorcheteEnCoords}[obsCamposConstantes]{Observaci\'{o}n}

%-------------

Empecemos con algunas definiciones generales y poco claras. Dada una
variedad $M$, un \emph{campo en $M$} es una funci\'{o}n
$X:\,M\rightarrow\tangente{M}$ tal que $\pi\circ X=\id[M]$, donde
$\pi:\,\tangente{M}\rightarrow M$ es la proyecci\'{o}n can\'{o}nica.
Un poco m\'{a}s en general, si $U\subset M$ es un abierto, un campo en
$U$ es una funci\'{o}n $X:\,U\rightarrow\tangente{M}$ tal que
$\pi\circ X=\id[U]$. Ser\'{a} \'{u}til poder extender esta noci\'{o}n a
otros subconjuntos de $M$. Dada una variedad $N$ y una aplicaci\'{o}n
$\psi:\,Q\rightarrow N$, un \emph{campo sobre/en/a lo largo de $\psi$} es una
funci\'{o}n $X:\,Q\rightarrow\tangente{N}$ tal que $\pi\circ X=\psi$. De
particular importancia ser\'{a}n los campos a lo largo de curvas
$\sigma:\,[a,b]\rightarrow N$. Si $X$ es, adem\'{a}s, una funci\'{o}n
continua, entonces se dir\'{a} que $X$ es un \emph{campo continuo}.

Los campos en una variedad $M$ son, en otras palabras, secciones del
fibrado tangente a la variedad. Entre todas las posibles secciones
del fibrado hay un \emph{elemento distinguido}: como puntualmente,
arriba de cada punto se levanta un espacio vectorial, el espacio tangente
a la variedad en el punto, y cada fibra tiene un elemento distinguido,
el vector cero, al fibrado tangente le corresponde la secci\'{o}n nula
como elemento distinguido, es decir, $0:\,M\rightarrow\tangente{M}$
que a cada punto $p\in M$ le asigna el vector $0\in\tangente[p]{M}$.
Nos interesar\'{a}n particularmente las secciones con alguna propiedad
extra, como continuidad o suavidad, pero una secci\'{o}n no tiene por qu\'{e}
tener ninguna de estas propiedades. Es de esperar que la secci\'{o}n
cero sea suave.

Un campo $X:\,Q\rightarrow\tangente{N}$ se dice \emph{suave}, si $X$
es una transformaci\'{o}n suave. Esta definici\'{o}n presupone que $Q$ tiene
estructura de variedad. El caso de mayor inter\'{e}s, por el momento
ser\'{a} el de un campo en una variedad $M$ o en un abierto $U\subset M$.
En este caso, $U$ tiene naturalmente estructura de variedad diferencial y
un campo $X:\,U\rightarrow\tangente{M}$ es, por definici\'{o}n, suave, si
es una secci\'{o}n local suave del fibrado $\pi:\,\tangente{M}\rightarrow M$,
o, lo que es lo mismo, una secci\'{o}n suave de $\pi:\,%
\tangente{U}\rightarrow U$. El \emph{soporte} de un campo
$X:\,Q\rightarrow\tangente{N}$ se define como la clausura del subconjunto
$\{q\in Q\,:\,X_{q}\not =0\}$. La igualdad $X_{q}=0$ se entiende en
$\tangente[p]{N}$, donde $p\in N$ es tal que $p=\pi(X_{q})=\psi(q)$.

\subsection{El espacio $\champs{M}$ de campos suaves en una variedad}
Sea $M$ una variedad diferencial y sea $X:\,M\rightarrow\tangente{M}$
un campo no necesariamente continuo. Si $(U,\varphi)$ es una carta
compatible con la estructura de $M$, entonces, para cualquier punto
$p\in U$, $X_{p}\in\tangente[p]{M}$ y
\begin{align*}
	X_{p} & \,=\,X^{i}(p)\cdot\gancho[p]{x^{i}}
\end{align*}
%
para ciertos coeficientes $X^{i}(p)\in\bb{R}$. Esto determina funciones
$X^{i}:\,U\rightarrow\bb{R}$ denominadas \emph{las componentes de $X$ en/%
con respecto a la carta $(U,\varphi)$}. La aplicaci\'{o}n $X$, originalmente
definida en $M$, se restringe a un campo $X|_{U}:\,%
U\rightarrow\tangente{M}$ en $U$. Tanto $U$ como $\tangente{M}$
tienen estructura de variedad diferencial.

\begin{propoCamposEnCoordenadas}\label{thm:camposencoordenadas}
	Con las definiciones anteriores, $X|_{U}:\,U\rightarrow\tangente{M}$
	es una transformaci\'{o}n suave, si y s\'{o}lo si
	las funciones $X^{i}:\,U\rightarrow\bb{R}$ lo son.
\end{propoCamposEnCoordenadas}

\begin{proof}
	Si $(\widetilde{U},\widetilde{\varphi})$ denota la carta
	correspondiente a $(U,\varphi)$ en $\tangente{M}$, entonces, en
	coordenadas,
	\begin{align*}
		\widehat{X}(x) & \,=\,
			\widetilde{\varphi}\circ X\circ\varphi(x)\,=\,
			\widetilde{\varphi}
				(\varphi^{-1}(x),\,X_{\varphi^{-1}(x)}) \\
		& \,=\,(\lista*{x}{n},\,X^{1}(\varphi^{-1}(x)),\,\dots,\,
			X^{n}(\varphi^{-1}(x)))
		\text{ .}
	\end{align*}
	%
	La expresi\'{o}n en coordenadas $\widehat{X}:\,%
	\varphi(U)\rightarrow\varphi(U)\times\bb{R}^{n}$ es suave, si y
	s\'{o}lo si $\widehat{X^{i}}=X^{i}\circ\varphi^{-1}$ son suaves,
	si y s\'{o}lo si las funciones $X^{i}$ son suaves.
\end{proof}

La demostraci\'{o}n consiste, simplemente, en volver sobre la definici\'{o}n
de lo que significa que una funci\'{o}n entre variedades sea una
transformaci\'{o}n suave. Como ya se hab\'{\i}a mencionado, si
$U\subset M$ es un subconjunto abierto de una variedad diferencial $M$,
un campo $X$ en $U$ es una \emph{campo suave}, si, en tanto funci\'{o}n
entre variedades, es una transformaci\'{o}n suave. Un poco m\'{a}s en general,
dado un subconjunto arbitrario $A\subset M$ de la variedad y un campo
$X:\,A\rightarrow\tangente{M}$ sobre $A$, se dice que $X$ es \emph{suave},
si, para cada punto $p\in A$, existe un entorno $V\subset M$ de $p$ y un
campo suave $\tilde{X}:\,V\rightarrow\tangente{M}$ tal que $X=\tilde{X}$
en $V\cap A$.

Dada una variedad $M$ denotaremos por $\champs{M}$ al espacio de
campos suaves definidos en $M$. El conjunto de campos en $M$ constituye un
espacio vectorial. M\'{a}s aun, $\champs{M}$ es un $C^{\infty}(M)$-m\'{o}dulo.
Dados campos arbitrarios $X,Y:\,M\rightarrow\tangente{M}$ y funciones
$f,g:\,M\rightarrow\bb{R}$, para cada punto $p\in M$,
\begin{align*}
	\left.\big(f\cdot X+g\cdot Y\big)\right|_{p} & \,=\,
		f(p)X|_{p}+g(p)Y|_{p}
	\text{ ,}
\end{align*}
%
que pertenece a la fibra $\tangente[p]{M}$ de $\pi$. Si $X,Y\in\champs{M}$ y
$f,g\in C^{\infty}(M)$, entonces, en coordenadas,
\begin{align*}
	\left.\big(f\cdot X+g\cdot Y\big)\right|_{p} & \,=\,
		\big(f(p)X^{i}(p)+g(p)Y^{i}(p)\big)\cdot\gancho[p]{x^{i}}
	\text{ .}
\end{align*}
%
Pero las funciones $fX^{i}+gY^{i}$, en el abierto coordenado donde est\'{e}n
definidas, son \phantom{continuas} suaves. Dado un campo arbitrario
$X$, la escritura de $X_{p}$ en la base can\'{o}nica
$\Big\{\gancho[p]{x^{i}}\Big\}_{i}$ se puede expresar como una igualdad
en el espacio de todos los campos en $M$ (o, mejor dicho, en el abierto
coordenado):
\begin{align*}
	X|_{U} & \,=\, X^{i}\cdot\gancho{x^{i}}
	\text{ .}
\end{align*}
%
En particular, si $X$ es suave, la igualdad es en $\champs{U}$. Los
\emph{campos coordenados} $\gancho{x^{i}}:\,U\rightarrow\tangente{M}$
son suaves, pues sus componentes con respecto a la carta $(U,\varphi)$ son
constantes y, en particular, suaves. La imagen de estos campos est\'{a}
contenida en $\tangente{U}$. La expersi\'{o}n en coordenadas del campo
$\gancho{x^{i}}$ con respecto a otra carta compatible $(V,\tilde{\varphi})$
est\'{a} dada por
\begin{equation}
	\label{eq:cambiodeganchos}
	\gancho{x^{i}} \,=\,\derivada{\tilde{x}^{j}}{x^{i}}\cdot
		\gancho{\tilde{x}^{j}}
	\text{ .}
\end{equation}
%

\subsection{Extensi\'{o}n de campos vectoriales}
Sea $M$ una variedad diferencial y sea $A\subset M$ un subconjunto
arbitrario. Por definici\'{o}n, un campo (suave) sobre $A$ es una secci\'{o}n
$X:\,A\rightarrow\tangente{M}$ que admite, en un entorno de cada punto,
una extensi\'{o}n suave. Si $A$ es un subconjunto cerrado, es posible
extender el campo a toda la variedad.

\begin{propoExtenderUnCampo}\label{thm:extenderuncampo}
	Sea $M$ una variedad diferencial y sea $A\subset M$ un subconjunto
	cerrado. Si $U\subset M$ es abierto y $U\supset A$ y
	$X:\,A\rightarrow\tangente{M}$ es un campo suave sobre $A$, existe
	un campo $\tilde{X}\in\champs{M}$ tal que $\tilde{X}|_{A}=X$ y
	$\soporte{\tilde{X}}\subset U$.
\end{propoExtenderUnCampo}

\begin{proof}
	Dado $p\in A$, existe un abierto $U_{p}\subset M$ tal que
	$p\in U_{p}$ y un campo $X^{p}:\,U_{p}\rightarrow\tangente{M}$
	tal que $X^{p}=X$ en $U_{p}\cap A$. Sea
	$\{\psi_{p}\}_{p\in A}\cup\{\psi_{0}\}$ una partici\'{o}n de la unidad
	subordinada al cubrimiento $\{U_{p}\}_{p\in A}\cup\{U_{0}\}$, donde
	$U_{0}=M\setmin A$. En particular, vale que
	$\sum_{p\in A}\,\psi_{p}(p')=1$ para todo $p'\in A$, pues $\psi_{0}=0$
	all\'{\i}. Sea $\psi_{p}X^{p}$ el campo dado por
	\begin{align*}
		\left.\big(\psi_{p}X^{p}\big)\right|_{q} & \,=\,
			\begin{cases}
				\psi_{p}(q)X^{p}|_{q} & \quad\text{si }
							q\in U_{p} \\
				0 & \quad\text{si }q\not\in\soporte{\psi_{p}}
			\end{cases}
		\text{ .}
	\end{align*}
	%
	Este campo est\'{a} bien definido, pues ambas definiciones
	coinciden en la intersecci\'{o}n de los abiertos $U_{p}$ y
	$M\setmin\soporte{\psi_{p}}$, y es suave, pues es suave
	restringida a $U_{p}$ y restringida a $M\setmin\soporte{\psi_{p}}$.
	Finalmente, sea $\tilde{X}$ el campo
	\begin{align*}
		\tilde{X} & \,=\,\sum_{p}\,\psi_{p}X^{p}
		\text{ .}
	\end{align*}
	%
	Como cada uno de los sumandos es suave y los soportes forman una
	familia localmente finita, dado $p'\in M$, existe un abierto
	$V\subset M$ tal que $p'\in V$ en donde la suma
	$\sum_{p\in A}\,\psi_{p}X^{p}$ es efectivamente finita y, por lo
	tanto, $\tilde{X}$ es $C^{\infty}$ en $p'$. Si $p'\in A$,
	\begin{align*}
		\tilde{X}|_{p'} & \,=\, \sum_{p\in A}\,\psi_{p}(p')X^{p}|_{p'}
			\,=\,\sum_{p\in A}\,\psi_{p}(p')X|_{p'}
			\,=\, X|_{p'}
		\text{ ,}
	\end{align*}
	%
	ya que $\psi_{p}(p')$ es igual a cero, o bien $X^{p}|_{p'}=X|_{p'}$.
\end{proof}

Si $A=\{p\}$ consiste en un \'{u}nico punto, un campo $X$ sobre $A$ es
exactamente un vector del tangente $v_{p}\in\tangente[p]{M}$. Todo campo
sobre $\{p\}$ es suave, es decir, se extiende a un campo en un entorno de
$p$, al \emph{campo constante}: si $(U,\varphi)$ es una carta en $p$,
\begin{align*}
	v_{p} & \,=\, v^{i}\gancho[p]{x^{i}}
	\text{ .}
\end{align*}
%
Sea $X:\,U\rightarrow\tangente{M}$ el campo dado por
\begin{align*}
	X & \,=\, v^{i}\gancho{x^{i}}
	\text{ ,}
\end{align*}
%
donde $v^{i}\in\bb{R}$ son \emph{constantes} iguales a los coeficientes
de $v_{p}$ en la base can\'{o}nica $\Big\{\gancho[p]{x^{i}}\Big\}_{i}$.
Por la proposici\'{o}n \ref{thm:extenderuncampo}, el vector tangente
$v_{p}\in\tangente[p]{M}$ se extiende a un campo $C^{\infty}$ en $M$,
de manera que $X_{p}=v_{p}$. Con un poco m\'{a}s de cuidado, se puede
tomar $X$ constante en todo un entorno de $p$, usando una funci\'{o}n
chich\'{o}n en $p$.

\begin{obsCamposConstantes}\label{obs:camposconstantes}
	De acuerdo con la expresi\'{o}n \eqref{eq:cambiodeganchos} para el
	cambio de base en t\'{e}rminos de los ganchos un ``campo constante''
	no es en verdad constante \emph{en s\'{\i}}, sino respecto de una
	carta en particular; al cambiar de carta, un campo vectorial con
	coeficientes constantes en una carta pasa (en general, a menos que
	sea un cambio ``lineal'', una homotecia) a verse como un campo con
	componentes variables.
\end{obsCamposConstantes}

\subsection{Propiedades equivalentes a la suavidad de un campo}
Dado un campo $X:\,M\rightarrow\tangente{M}$ y dada una funci\'{o}n
\emph{suave} $f\in C^{\infty}(M)$, queda determinada una funci\'{o}n
$Xf:\,M\rightarrow\bb{R}$ por $(Xf)(p)=X_{p}f$. Llamaremos a esta
operaci\'{o}n \emph{aplicar el campo $X$ a la funci\'{o}n $f$}. Esta
nueva funci\'{o}n est\'{a} \emph{localmente determinada}.

\begin{lemaAplicarCamposLocalmenteDeterminado}%
	\label{thm:aplicarcamposlocalmentedeterminado}
	Sea $X:\,M\rightarrow\tangente{M}$ un campo arbitrario, no
	necesariamente suave ni continuo. Sean $f,g\in C^{\infty}(M)$ dos
	funciones suaves. Si existe un abierto $V\subset M$ tal que
	$f|_{V}=g|_{V}$, entonces $(Xf)|_{V}=(Xg)|_{V}$.
\end{lemaAplicarCamposLocalmenteDeterminado}

\begin{proof}
	Dado $p\in V$, $(Xf)(p)\equiv X_{p}f$ y $(Xg)(p)\equiv X_{p}g$.
	Pero $f|_{V}=g|_{V}$, es decir, $f$ y $g$ coicinden en un
	entorno de $p$. Como el vector tangente $X_{p}\in\tangente[p]{M}$
	est\'{a} determinado localmente, $X_{p}f=X_{p}g$. Como $p$ es
	un punto arbitrario de $V$, se concluye que $Xf=Xg$ en $V$.
\end{proof}

\begin{propoEquivCampoSuave}\label{thm:equivalenciascamposuave}
	Sea $M$ una variedad diferencial. Sea $X:\,M\rightarrow\tangente{M}$
	un campo no necesariamente continuo. Las siguientes propiedades
	son equivalentes:
	\begin{itemize}
		\item[(\i)] $X\in\champs{M}$;
		\item[(\i\i)] dada $f\in C^{\infty}(M)$, la funci\'{o}n
			$Xf:\,p\mapsto X_{p}f$ es suave en $M$;
		\item[(\i\i\i)] dado un abierto arbitrario $U\subset M$
			y una funci\'{o}n suave $f\in C^{\infty}(U)$,
			la funci\'{o}n $Xf\equiv X|_{U}f$ es suave en $U$.
	\end{itemize}
	%
\end{propoEquivCampoSuave}

\begin{proof}
	Si $X$ es un campo suave y $f\in C^{\infty}(M)$, entonces, tomando
	coordenadas,
	\begin{align*}
		\widehat{Xf}(x) & \,=\,X^{i}(\varphi^{-1}(x))
			\derivada{(f\circ\varphi^{-1})}{x^{i}}
				(\varphi(\varphi^{-1}(x))) \\
		& \,=\, \widehat{X^{i}}(x)\derivada{\widehat{f}}{x^{i}}(x)
		\text{ .}
	\end{align*}
	%
	Como $X$ es suave, las funciones $X^{i}$ son suaves en $M$, es
	decir, las expresiones en coordenadas $\widehat{X^{i}}$ son suaves
	en sentido usual, en el codominio de la carta (cualquiera sea la
	carta, siempre que sea compatible). Pero tambi\'{e}n $\widehat{f}$
	es suave en sentido usual, porque $f$ suave en $M$. En particular,
	las derivadas $\derivada{\widehat{f}}{x^{i}}:\,%
	\widehat{U}\rightarrow\bb{R}$ son suaves. Entonces $Xf$ es suave
	en $M$.

	Asumiendo que $Xf\in C^{\infty}(M)$ para toda funci\'{o}n suave
	$f\in C^{\infty}(M)$, dado $U\subset M$ abierto y
	$f\in C^{\infty}(U)$, se ver\'{a} que $X|_{U}f$ es diferenciable.
	Sea $p\in U$ un punto arbitrario del abierto. Sea $V\subset U$
	abierto tal que $p\in V$ y $\clos{V}\subset U$. Sea
	$\{\psi_{0},\psi_{1}\}$ una partici\'{o}n de la unidad
	subordinada al cubrimiento $\{U_{0},U_{1}\}$ de $M$, donde
	$U_{0}=U$ y $U_{1}=M\setmin\clos{V}$. Sea $\tilde{f}:\,%
	M\rightarrow\bb{R}$ la funci\'{o}n definida por
	\begin{align*}
		\tilde{f} & \,=\,
			\begin{cases}
				\psi_{0} f & \quad\text{en } U_{0} \\
				0 & \quad\text{en } U_{1}
			\end{cases}
		\text{ .}
	\end{align*}
	%
	Esta funci\'{o}n pertenece a $C^{\infty}(M)$ y $f|_{V}=\tilde{f}|_{V}$.
	Entonces
	\begin{align*}
		\left.\big(X|_{U}f\big)\right|_{V} & \,=\,
			\left.\big(X\tilde{f}\big)\right|_{V}
		\text{ .}
	\end{align*}
	%
	Pero $X\tilde{f}:\,M\rightarrow\bb{R}$ es suave. Entonces
	$X|_{U}f:\,U\rightarrow\bb{R}$ coincide con una funci\'{o}n
	suave en un entorno de $p$. Como $p$ era arbitrario,
	$X|_{U}f\in C^{\infty}(U)$.

	Finalmente, si $Xf\in C^{\infty}(U)$ para toda funci\'{o}n
	suave $f$ definida en alg\'{u}n abierto $U\subset M$, entonces,
	dado $p\in M$ y una carta $(U,\varphi)$ en $p$,
	\begin{align*}
		X|_{U} & \,=\, X^{i}\gancho{x^{i}}
		\text{ .}
	\end{align*}
	%
	Las funciones coordenadas $x^{i}:\,U\rightarrow\bb{R}$ asociadas
	a la carta $\varphi$, son funciones suaves para cada
	$i\in[\![1,n]\!]$. Aplicando $X|_{U}$ a $x^{j}$, se deduce que
	$X^{j}=X x^{j}\in C^{\infty}(U)$ y que $X\in\champs{M}$,
	por \ref{thm:camposencoordenadas}.
\end{proof}

\subsection{Campos como derivaciones}
Un campo $X\in\champs{M}$ define una derivaci\'{o}n en el \'{a}lgebra
de funciones $C^{\infty}(M)$: dadas $f,g\in C^{\infty}(M)$, dados
$a,b\in\bb{R}$ y dado $p\in M$,
\begin{align*}
	\big(X(af+bg)\big)(p) & \,=\,X|_{p}(af+bg) \\
	& aX|_{p}f+bX|_{p}g \,=\,(aXf+bXg)(p)
		\quad\text{y} \\
	\big(X(fg)\big)(p) & \,=\,X|_{p}(fg) \\
	& \,=\,f(p)X|_{p}g + (X|_{p}f)g(p)
		\,=\,\big(fXg+(Xf)g\big)(p)
	\text{ .}
\end{align*}
%
Es decir, $X:\,C^{\infty}(M)\rightarrow C^{\infty}(M)$ es $\bb{R}$-lineal
y cumple con la regla de Leibniz: $X(fg)=fX(g)+X(f)g$. La linealidad
y la regla de Leibniz vienen de que $X_{p}$ es un vector tangente en $p$
para cada punto $p$, no de que $X$ es suave. En realidad, lo que no depende
de que $X$ sea un campo suave es que la regla de Leibniz y la linealidad
valen puntualmente, bajando a cada punto. Al ser $X$ un campo suave en $M$
aplicar el campo a una funci\'{o}n, a una combinaci\'{o}n de funciones o
a un producto de funciones suaves devuelve funciones suaves.
Rec\'{\i}procamente, toda derivaci\'{o}n en $C^{\infty}(M)$ viene de un
campo suave.

\begin{obsDerivacionEsDerivacion}\label{obs:derivacionesderivacion}
	Antes de pasar a demostrar esta afirmaci\'{o}n, es importante
	notar que toda derivaci\'{o}n \emph{global}, es decir,
	toda derivaci\'{o}n $D:\,C^{\infty}(M)\rightarrow C^{\infty}(M)$
	determina, para cada punto $p\in M$, una \emph{derivaci\'{o}n en %
	en $p$}. Si $X_{p}:\,f\mapsto (Df)(p)$, entonces
	$X_{p}\in\tangente[p]{M}$, pues, recorriendo las igualdades
	anteriores en sentido contrario,
	\begin{align*}
		X_{p}(af+bg) & \,\equiv\, \big(D(af+bg)\big)(p) \\
		& \,=\, (aDf+bDg)(p) \,=\,aX_{p}f+bX_{p}g
		\quad\text{y} \\
		X_{p}(fg) & \,\equiv\,D(fg)(p) \\
		& \,=\,\big(fDg+(Df)g\big)(p) \,=\,
			f(p)X_{p}g +(X_{p}f)g(p)
		\text{ .}
	\end{align*}
	%
\end{obsDerivacionEsDerivacion}

\begin{teoDerivacionesYCampos}\label{thm:derivacionesycampos}
	Sea $D:\,C^{\infty}(M)\rightarrow C^{\infty}(M)$ una derivaci\'{o}n.
	Existe un campo $X\in\champs{M}$ tal que $Df=Xf$ para toda
	$f\in C^{\infty}(M)$.
\end{teoDerivacionesYCampos}

\begin{proof}
	Si existiese un campo (no necesariamente continuo)
	$X:\,M\rightarrow\tangente{M}$ tal que $Df=Xf$, entonces
	\begin{align*}
		X_{p}f & \,\equiv\,(Xf)(p) \,=\,(Df)(p)
	\end{align*}
	%
	para todo punto $p\in M$ y toda funci\'{o}n diferenciable $f$.
	Sea $X:\,M\rightarrow\tangente{M}$ el campo $p\mapsto X_{p}$,
	donde $X_{p}\in\tangente[p]{M}$ es la derivaci\'{o}n \emph{en $p$}
	dada por $X_{p}f=(Df)(p)$ (c.~f. la observaci\'{o}n
	\ref{obs:derivacionesderivacion}). Para ver que $X\in\champs{M}$,
	tomamos $f\in C^{\infty}(M)$ y aplicamos $X$ a $f$: puntualmente,
	\begin{align*}
		(Xf)(p) & \,\equiv\, X_{p}f\,\equiv\,(Df)(p)
		\text{ .}
	\end{align*}
	%
	Pero $Df\in C^{\infty}(M)$. Entonces $Xf\in C^{\infty}(M)$ y
	$X\in\champs{M}$.
\end{proof}

\subsection{El \emph{pushforward} de un campo}
Sea $F:\,M\rightarrow N$ una transformaci\'{o}n suave. Sabemos que $F$
determina una transformaci\'{o}n lineal $\diferencial[p]{F}:\,%
\tangente[p]{M}\rightarrow\tangente[F(p)]{N}$ para cada punto $p\in M$ y
que tambi\'{e}n determina una transformaci\'{o}n suave $\diferencial{F}$
entre los fibrados tangentes. Fijado un punto $p\in M$, el diferencial
asocia a cada vector tangente $v_{p}\in\tangente[p]{M}$ un vector en
el espacio tangente $\tangente[F(p)]{N}$ que, en t\'{e}rminos de derivaciones,
est\'{a} determinado por
\begin{align*}
	\diferencial[p]{F}(v_{p})(g) & \,=\,v_{p}(g\circ F)
\end{align*}
%
para toda funci\'{o}n suave $g\in C^{\infty}(N)$.

Dado un campo $X:\,M\rightarrow\tangente{M}$, para cada punto $p$ de $M$,
$X_{p}$ es un vector tangente a $M$ en $p$ y $\diferencial[p]{F}(X_{p})$
es un vector tangente a $N$ en $F(p)$. Esto determina un campo a lo largo
de $F$ v\'{\i}a el diferencial global de $F$, pues
$\diferencial{F}\circ X:\,M\rightarrow\tangente{N}$ y
$\pi(\diferencial[p]{F}(X_{p}))=F(p)$. Si $X\in\champs{M}$, entonces
$\diferencial{F}\circ X$ es un campo suave a lo largo de $F$, es decir, es
una transformaci\'{o}n suave.

Si $Y:\,N\rightarrow\tangente{N}$ es un campo en $N$, entonces tambi\'{e}n
se obtiene un campo a lo largo de $F$, si se toma la composici\'{o}n
$Y\circ F:\,p\mapsto Y_{F(p)}\in\tangente[F(p)]{N}$. Este es suave, si
$Y\in\champs{N}$ y si $F$ es suave.

Pero, en general, $F$ no determina un campo en $N$ a partir de un campo en
$M$: por ejemplo, si $F$ no es suryectiva, no hay, en principio, una manera
natural de asignarle un vector tangente a puntos $q\in N$ que no est\'{a}n
en la imagen de $F$; si, por otra parte, $F$ no es inyectiva, entonces
dos puntos distintos $p,p'\in M$ pueden dar lugar a vectores tangentes
distintos en $F(p)=F(p')$ y no habr\'{\i}a manera natural de elegir uno en
lugar del otro.

Hay, aun as\'{\i}, un caso en que s\'{\i} tiene sentido hablar de un campo
en $N$ determinado por $F$ y por un campo en $M$. Este es el caso en que
$F:\,M\rightarrow N$ es un difeomorfismo (c.~f. el teorema
\ref{thm:pushforwarddifeo}). Es de esperar que, si $M$ y $N$
son indistinguibles desde el punto de vista de su estructura diferencial,
entonces los espacios de campos (suaves) sean, tambi\'{e}n, en alg\'{u}n
sentido, indistinguibles, ya que esta noci\'{o}n fue definida, en \'{u}ltima
instancia, a partir de la noci\'{o}n de estructura diferencial de una
variedad.

Sea entonces $F:\,M\rightarrow N$ una transformaci\'{o}n suave entre
variedades diferenciales con o sin borde y sean $X:\,M\rightarrow\tangente{M}$
e $Y:\,N\rightarrow\tangente{N}$ campos (no necesariamente continuos).
Se dice que $X$ e $Y$ \emph{est\'{a}n $F$-relacionados}, si, para cada punto
$p\in M$, vale que
\begin{equation}
	\label{eq:frelacionadosi}
	\diferencial[p]{F}\big(X_{p}\big) \,=\, Y_{F(p)}
	\text{ .}
\end{equation}
%
En otras palabras, dado un campo $X$, existe un campo en $N$ que est\'{e}
$F$-relacionado con $X$, si y s\'{o}lo si existe un campo en $N$ que da
lugar al campo $p\mapsto \diferencial[p]{F}X_{p}$ a lo largo de $F$. Esta
noci\'{o}n est\'{a} definida para campos $X$ e $Y$ no necesariamente
continuos. En particular, aunque $X$ sea continuo, no es necesariamente
cierto que $Y$, si existiese, sea continuo.

\begin{propoFRelacionados}\label{thm:frelacionados}
	Sean $M$ y $N$ variedades diferenciales y sea $F:\,M\rightarrow N$
	una transformaci\'{o}n suave. Sean $X\in\champs{M}$ e $Y\in\champs{N}$
	campos suaves. Entonces $X$ e $Y$ est\'{a}n $F$-relacionados,
	si y s\'{o}lo si, para todo abierto $V\subset N$ y toda
	funci\'{o}n suave $g\in C^{\infty}(V)$, vale que
	\begin{equation}
		\label{eq:frelacionadosii}
		X|_{F^{-1}(V)}(g\circ F) \,=\, (Y|_{V}g)\circ F
		\text{ .}
	\end{equation}
	%
\end{propoFRelacionados}

\begin{proof}
	Sea $p\in M$ y sea $g$ una funci\'{o}n suave definida en un entorno
	de $F(p)$. Por un lado,
	\begin{align*}
		X(g\circ F) (p) & \,\equiv\, X_{p}(g\circ F) \,\equiv\,
			\diferencial[p]{F}\big(X_{p}\big)(g)
	\end{align*}
	%
	y, por otro,
	\begin{align*}
		(Yg)\circ F(p) & \,\equiv\, Y_{F(p)}g
		\text{ .}
	\end{align*}
	%
	Rigurosamente, $X$ e $Y$ est\'{a}n siendo restringidos a los
	abiertos en donde $g\circ F$ y $g$ est\'{e} definidas,
	respectivamente. Ahora bien $X(g\circ F)(p)=(Yg)\circ F(p)$
	para toda $g$ (y todo punto $p$), si y s\'{o}lo si
	$\diferencial[p]{F}(X_{p})=Y_{F(p)}$ para todo $p\in M$.
\end{proof}

\begin{obsFRelacionados}\label{obs:frelacionados}
	Este resultado y \ref{thm:derivacionesycampos} nos permiten deducir
	propiedades de regularidad de campos, a partir de propiedades
	algebraicas de los mismos: por un lado, podemos deducir
	que un campo $X$ en una variedad $M$ es un campo suave, verificando
	que es derivaci\'{o}n en $C^{\infty}(M)$; por otro lado, dados
	campos $X$ en $M$ e $Y$ en $N$ y una transformaci\'{o}n suave
	$F:\,M\rightarrow N$, sabiendo que son campos suaves, podemos
	concluir que est\'{a}n $F$-relacionados verificando que se
	cumple la igualdad \eqref{eq:frelacionadosii} para toda
	funci\'{o}n suave $g$ definida en un abierto de $N$ (usando
	particiones de la unidad, se deduce que es suficiente verificar la
	igualdad para funciones $g\in C^{\infty}(N)$ (?)).
	Denotaremos que $X\in\champs{M}$ e $Y\in\champs{N}$ est\'{a}n
	$F$-relacionados por $X\sim_{F}Y$.
\end{obsFRelacionados}

\begin{teoPushforwardDifeo}\label{thm:pushforwarddifeo}
	Sean $M$ y $N$ variedades diferenciales y sea $F:\,M\rightarrow N$
	un difeomorfismo. Dado $X\in\champs{M}$, existe un \'{u}nico
	campo $Y$ en $N$ que est\'{a} $F$-relacionado con $X$. Adem\'{a}s,
	$Y\in\champs{N}$.
\end{teoPushforwardDifeo}

\begin{proof}
	Como $F$ es una biyecci\'{o}n, $Y:\,N\rightarrow\tangente{N}$
	dado por $q\mapsto \diferencial[F^{-1}(q)]{F}(X_{F^{-1}(q)})$ es
	un campo en $N$ (no necesariamente continuo) y est\'{a}
	$F$-relacionado con $X$. Por otra parte, todo campo $Y$ que
	est\'{e} $F$-relacionado con $X$ debe verificar que
	$\diferencial[p]{F}(X_{p})=Y_{F(p)}$. Con lo cual $Y$ definido
	de esta manera es el \'{u}nico campo posible.
	Dado que, en tanto transformaci\'{o}n,
	$Y=\diferencial{F}\circ X\circ F^{-1}$, se deduce que $Y$ es
	suave, por ser composici\'{o}n de transformaciones suaves.
\end{proof}

El campo $Y\in\champs{X}$ determinado por el difeomorfismo $F$ y el
campo suave $X\in\champs{M}$, se denomina \emph{pushorward de $X$ por $F$}.
Denotamos a este campo por $F_{*}X$. Expl\'{\i}citamente, dado $q\in N$,
\begin{equation}
	\label{eq:pushforwarddifeo}
	(F_{*}X)_{q} \,=\,\diferencial[F^{-1}(q)]{F}\big(X_{F^{-1}(q)}\big)
	\text{ .}
\end{equation}
%
Para que esta expresi\'{o}n tenga sentido, es suficiente que $F$ sea
diferenciable e invertible, pero para que defina un campo suave, se
requiere (en general) que $F^{-1}$ tambi\'{e}n sea diferenciable.
Aunque parezca una construcci\'{o}n trivial, el pushforward $F_{*}$
ser\'{a} de utilidad en el contexto de grupos de Lie.

\subsection{Campos tangentes a una subvariedad}
De la misma manera en que es posible identificar el espacio tangente a
una subvariedad en un punto con un subespacio del espacio tangente a toda
la variedad, nos podemos preguntar si es posible identificar los
campos en una subvariedad con campos en la variedad ambiente. En general,
como se mencion\'{o} m\'{a}s arriba, esto no es posible, pero se pueden
dar algunos criterios para determinar cu\'{a}ndo es posible o, mejor
dicho, dar una interpretaci\'{o}n de lo que significa que esto sea posible.

Sea $M$ una variedad diferencial y sea $S\subset M$ una subvariedad.
Dado un campo vectorial $X:\,M\rightarrow\tangente{M}$, se puede obtener un
campo a lo largo de la inclusi\'{o}n $\inc[S]:\,S\rightarrow M$ simplemente
restringiendo el campo a $S$, es decir,
$Y|_{S}=Y\circ\inc[S]:\,S\rightarrow\tangente{M}$. Si $Y\in\champs{M}$,
entonces $Y|_{S}$ es suave, pero no es necesariamente cierto que sea
un campo en $S$. Si, por otro lado, $X:\,S\rightarrow\tangente{S}$ es un
campo en $S$, entonces $\diferencial{\inc[S]}\circ X:\,%
S\rightarrow\tangente{M}$ define un campo a lo largo de $\inc[S]$. Como antes,
si $X\in\champs{S}$ es suave, entonces $Y=\diferencial{\inc[S]}\circ X$
tambi\'{e}n lo es, pero, en general, no define un campo en $M$.

Sean $S$ y $M$ como antes y sea $p\in S$ un punto arbitrario. Dado un campo
$Y:\,M\rightarrow\tangente{M}$, se dice que $Y$ es \emph{tangente a $S$ en %
$p$}, si $Y_{p}\in\tangente[p]{S}\equiv%
\diferencial[p]{\inc[S]}(\tangente[p]{S})$. Se dice que $Y$ es
\emph{tangente a $S$}, si es tangente a $S$ en todo punto de la subvariedad.
Como $\inc[S]$ es una inmersi\'{o}n, el diferencial
$\diferencial[p]{\inc[S]}:\,\tangente[p]{S}\rightarrow\tangente[p]{M}$ es
inyectivo. Si $Y:\,M\rightarrow\tangente{M}$ es un campo tangente a $S$,
entonces, para cada $p\in S$, existe un \'{u}nico vector tangente
$X_{p}\in\tangente[p]{S}$ tal que
$Y_{p}=\diferencial[p]{\inc[S]}\big(X_{p}\big)$. Esto determina un campo
$X:\,S\rightarrow\tangente{S}$ que, por su definici\'{o}n, est\'{a}
$\inc[S]$-relacionado con $Y$. Rec\'{\i}procamente, si $X$ es un campo
en $S$, $Y$ es un campo en $M$ y $X$ est\'{a} $\inc[S]$-relacionado con
$Y$, entonces, para cada $p\in S$,
$Y_{p}=\diferencial[p]{\inc[S]}\big(X_{p}\big)$, de lo que se deduce que
$Y$ es tangente a $S$.

\begin{propoCamposTangentes}\label{thm:campostangentes}
	Sea $M$ una variedad diferencial y sea $S\subset M$ una subvariedad.
	Sea $Y:\,M\rightarrow\tangente{M}$ un campo (no necesariamente
	continuo). Entonces $Y$ es tangente a $S$, si y s\'{o}lo si
	existe un campo $X:\,S\rightarrow\tangente{S}$ que est\'{e}
	$\inc[S]$-relacionado con $Y$. En tal caso, $X=Y|_{S}$,
	la restricci\'{o}n de $Y$ a $S$.
\end{propoCamposTangentes}

Este resultado dice que hay una correspondencia entre campos tangentes
y campos $\inc[S]$-relacionados. La pregunta es cu\'{a}ndo estos
campos son suaves: si $Y|_{S}$ cuando $Y$ es suave, o si $Y$ es suave
cuando $Y|_{S}$ es suave. Antes de pasar a responder estas preguntas hacemos
un breve comentario con respecto al fibrado tangente de una subvariedad.

La identificaci\'{o}n $X=Y|_{S}$ viene de identificar $\tangente[p]{S}$
con un subespacio de $\tangente[p]{M}$ para cada punto $p\in S$ v\'{\i}a
$\diferencial[p]{\inc[S]}$. M\'{a}s aun, como $\inc[S]$ es inyectiva e
inmersi\'{o}n, el diferencial $\diferencial[p]{\inc[S]}$ es inyectivo y
el diferencial \emph{global}
$\diferencial{\inc[S]}:\,\tangente{S}\rightarrow\tangente{M}$ es inyectivo,
tambi\'{e}n. Entonces $\tangente{S}$ se puede identificar, al menos, con un
subconjunto de $\tangente{M}$.

\begin{propoTangenteSubvarSubvar}\label{thm:tangentesubvarsubvar}
	Sea $M$ una variedad diferencial y sea $S\subset M$ una subvariedad.
	Sea $\inc[S]:\,S\rightarrow M$ la inclusi\'{o}n y sea
	$\diferencial{\inc[S]}:\,\tangente{S}\rightarrow\tangente{M}$ su
	diferencial. Entonces $\tangente{S}$ es una subvariedad del fibrado
	$\tangente{M}$ y $\diferencial{\inc[S]}$ es una inmersi\'{o}n.
\end{propoTangenteSubvarSubvar}

\begin{proof}
	En un par $(p,v_{p})\in\{p\}\times\tangente[p]{S}\subset\tangente{S}$,
	el diferencial $\diferencial{\inc[S]}$ est\'{a} dado por
	\begin{align*}
		\diferencial{\inc[S]}(p,v_{p}) & \,=\,
			(\inc[S](p),\diferencial[p]{\inc[S]}(v_{p}))
		\,=\,(p,v_{p})\,\in\,\{p\}\times\tangente[p]{M}\,\subset\,
			\tangente{M}
		\text{ ,}
	\end{align*}
	%
	haciendo las identificaciones usuales. Para ver que es una
	inmersi\'{o}n usamos su expresi\'{o}n en coordenadas. Sea $q\in S$
	y sea $V\subset S$ un abierto que es subvariedad regular de $M$
	y tal que $q\in V$. Sea $(U,\varphi)$ una carta preferencial para
	$V$ en $M$ centrada en $q$, de manera que,
	$V\cap U=\{x^{k+1}=0,\,\dots,\,x^{n}=0\}$, donde $n=\dim\,M$ y
	$k=\dim\,S$ (si $q$ es un punto del borde de $S$, se puede elegir
	una carta preferencial de borde de manera que tambi\'{e}n se cumpla
	$x^{k}\geq 0$ en la intersecci\'{o}n). Sea
	$(\widetilde{U},\widetilde{\varphi})$ la carta correspondiente en
	$\tangente{M}$. Por definici\'{o}n, esto significa que
	\begin{align*}
		\widetilde{U} & \,=\,\tangente{U} \\
		& \,=\,\left\lbrace (p,v_{p})\in\tangente{M}\,:\,
			p\in U,\,v_{p}\in\tangente[p]{U}=\tangente[p]{M}
			\right\rbrace \quad\text{y} \\
		\widetilde{\varphi}\Big(p,v^{i}\gancho[p]{x^{i}}\Big) &
			\,=\,(\varphi^{1}(p),\,\dots,\,\varphi^{n}(p),\,
				v^{1},\,\dots,\,v^{n})
		\text{ .}
	\end{align*}
	%
	Por otro lado, si $p\in V\cap U$,
	\begin{align*}
		\tangente[p]{S} & \,=\,\left\lbrace
			v^{k+1}=0,\,\dots,\,v^{n}=0\right\rbrace
			\,=\,\generado{\gancho[p]{x^{1}},\,\dots,\,
				\gancho[p]{x^{k}}}
		\text{ .}
	\end{align*}
	%
	Ahora bien, dado que $V\subset S$ es abierto,
	$\tangente{V}$ es abierto en el fibrado $\tangente{S}$. Adem\'{a}s,
	\begin{align*}
		\tangente{V}\cap\tangente{U} & \,=\,
			\left\lbrace (p,v_{p})\in\tangente{M}\,:\,
			p\in V\cap U,\,v_{p}\in\tangente[p]{V}=\tangente[p]{S}
			\right\rbrace \\
		& \,=\,\left\lbrace x^{k+1}=0,\,\dots,\,x^{n}=0,\,
			v^{k+1}=0,\,\dots,\,v^{n}=0\right\rbrace
		\text{ .}
	\end{align*}
	%
	Pero entonces $\tangente{V}$ es un abierto de $\tangente{S}$
	que es, adem\'{a}s, una subvariedad regular de $\tangente{M}$,
	pues, para punto $(q,v_{q})$ existe una carta preferencial
	$\tangente{U}$ para $\tangente{V}$ en $\tangente{M}$.
	En definitiva $\tangente{S}$ es una subvariedad inmersa de
	$\tangente{M}$.
\end{proof}

En particular, se deduce que el rango de $\diferencial{\inc[S]}:\,%
\tangente{S}\rightarrow\tangente{M}$ es constante e igual a $k^{2}$, donde
$k=\dim\,S$.

En realidad, el fibrado tangente a una subvariedad es m\'{a}s que una
subvariedad inmersa del fibrado tangente de $M$. Esto tiene que ver
con campos tangentes. Sea $S$ una subvariedad de $M$ y sea
$Y:\,M\rightarrow\tangente{M}$ un campo (no necesariamente continuo).
Si $q\in S$ y $V\subset S$ es un entorno de $q$ en $S$ que es subvariedad
regular de $M$, entonces existe una carta preferencial $(U,\varphi)$
para $V$ en $M$ tal que $V\cap U$ coincide con la feta $k$-dimensional
$\{x^{k+1}=0,\,\dots,\,x^{n}=0\}$ (agregando la condici\'{o}n $x^{k}\geq 0$
si $q$ es un punto del borde de $S$). Con respecto a las coordenadas
asociadas a esta carta,
\begin{align*}
	Y & \,=\,Y^{1}\gancho{x^{1}}\,+\,\cdots\,+\,
		Y^{n}\gancho{x^{n}}
	\text{ .}
\end{align*}
%
En particular, $Y$ es un campo tangente a $V$, si y s\'{o}lo si
$Y^{k+1}=0,\,\dots,\,Y^{n}=0$ en $V\cap U$. De esto se deduce que,
si $Y$ es suave en $U$, entonces las funciones $\lista*{Y}{n}$ son suaves
en $U$ y, en particular, como $\inc:\,V\cap U\rightarrow U$ es suave, que
$Y^{1}|_{V\cap U},\,\dots,\,Y^{k}|_{V\cap U}$ son suaves en $V\cap U$.
Es decir, si $Y$ es tangente a $V$ y es suave en $U$, el campo
$\inc[V]$-relacionado $Y|_{V\cap U}:\,V\cap U\rightarrow\tangente{V}$
se expresa en coordenadas como
\begin{align*}
	Y|_{V\cap U} & \,=\,Y^{1}\gancho{x^{1}}\,+\,\cdots\,+\,
		Y^{k}\gancho{x^{k}}
	\text{ ,}
\end{align*}
%
donde $Y^{1},\,\dots,\,Y^{k}:\,V\cap U\rightarrow\bb{R}$ son funciones
suaves. En definitiva, si $Y\in\champs{M}$ es tangente a $V$, entonces
$Y|_{V}\in\champs{V}$. Esto implica que si $Y\in\champs{M}$ es tangente a $S$,
entonces se restringe a un campo suave $Y|_{V}:\,V\rightarrow\tangente{V}$ y,
en particular, el campo $\inc[S]$-relacionado $Y|_{S}$ en $S$ es suave
como campo en $S$. Esto demuestra el siguiente resultado.

\begin{propoCamposTangentesSuaves}\label{thm:campostangentessuaves}
	Sea $M$ una variedad diferencial y sea $S\subset M$ una subvariedad.
	Sea $Y\in\champs{M}$ (un campo \emph{suave}). Entonces $Y$ es
	tangente a $S$, si y s\'{o}lo si existe un campo suave
	$X\in\champs{S}$ que est\'{e} $\inc[S]$-relacionado con $Y$.
\end{propoCamposTangentesSuaves}

Esto no quiere decir que, si $Y:\,M\rightarrow\tangente{M}$ es un campo
que est\'{a} $\inc[S]$-relacionado con un campo suave $X=Y|_{S}\in\champs{S}$.
entonces $Y$ sea suave: un campo $X\in\champs{S}$ define un
campo en la imagen de $S$ en $M$ v\'{\i}a la inclusi\'{o}n como
cualquier otra transformaci\'{o}n inyectiva y diferenciable, pero el hecho
de que sea subvariedad no es lo suficientemente fuerte como para determinar
que cualquier extensi\'{o}n a $M$ sea suave. Lo que dicen los
resultados \ref{thm:campostangentes} y \ref{thm:campostangentessuaves} es que
un campo $Y$ en una variedad $M$ es tangente a una subvariedad $S$, si y
s\'{o}lo si su restricci\'{o}n $Y|_{S}$ tiene imagen en el fibrado
$\tangente{S}$ (como $Y$ es una secci\'{o}n de
$\pi:\,\tangente{M}\rightarrow M$, $Y|_{S}$ es autom\'{a}ticamente una
secci\'{o}n de $\pi|_{\tangente{S}}$); la restricci\'{o}n de $Y$ a $S$ es
simplemente el campo $X$ en $S$ que a cada punto $q\in S$ le asigna el
\'{u}nico elemento de $\tangente[p]{S}$ cuya imagen v\'{\i}a
$\diferencial[p]{\inc[S]}$ es $Y_{{\inc[S]}(p)}$. Por otro lado, si se
empieza con un campo suave $Y\in\champs{M}$, entonces el \'{u}nico campo
$\inc[S]$-relacionado $X:\,S\rightarrow\tangente{S}$ debe ser suave,
tambi\'{e}n.

\begin{obsTangenteSubvarReg}\label{obs:tangentesubvarreg}
	En el caso de una subvariedad regular es posible dar otra
	descripci\'{o}n de los campos tangentes. Sea $M$ una variedad
	diferencial y sea $S\subset M$ una subvariedad regular. Sea
	$X\in\champs{M}$. Puntualmente, para cada $p\in S$,
	por \ref{thm:}
	\begin{align*}
		\tangente[p]{S} & \,=\,\left\lbrace
			v_{p}\in\tangente[p]{M}\,:\,
			v_{p}f=0\,\forall f\in C^{\infty}(M),\,f|_{S}=0
			\right\rbrace
		\text{ .}
	\end{align*}
	%
	En particular, se deduce de esto que $X$ es tangente a $S$, si
	y s\'{o}lo si $(Xf)|_{S}$ es la funci\'{o}n cero, para toda
	funci\'{o}n suave $f\in C^{\infty}(M)$ tal que $f|_{S}=0$.
\end{obsTangenteSubvarReg}


\subsection{El corchete de Lie}
Sea $M$ una variedad diferencial y sea $X\in\champs{M}$ un campo suave.
Para cada funci\'{o}n $f\in C^{\infty}(M)$, aplicar $X$ a $f$ define una
nueva funci\'{o}n $Xf$ y la misma pertenece al espacio $C^{\infty}(M)$ de
funciones suaves, tambi\'{e}n. Dado otro campo $Y\in\champs{M}$, podemos
aplicar $Y$ a $Xf$ y obtener, as\'{\i}, una tercera funci\'{o}n
$YXf\equiv Y(Xf)=\in C^{\infty}$. En general, esto no define un campo, con
lo que no tiene sentido hablar de la composici\'{o}n $YX$ en t\'{e}rminos
de campos, aunque, en t\'{e}rminos de derivaciones s\'{\i} tenga sentido
la composici\'{o}n. La composici\'{o}n $Y\circ X:\,f\mapsto Y(Xf)$ de
derivaciones de $C^{\infty}(M)$ no define, en general, una nueva
derivaci\'{o}n. Lo que s\'{\i} define una nueva derivaci\'{o}n es el
\emph{corchete} de dos derivaciones. Dadas dos derivaciones
$D_{1},D_{2}:\,C^{\infty}(M)\rightarrow C^{\infty}(M)$, definimos el
corchete de $D_{1}$ contra $D_{2}$ como la aplicaci\'{o}n dada en
una funci\'{o}n suave $f\in C^{\infty}(M)$ por
\begin{align*}
	[D_{1},D_{2}]f & \,=\,\big(D_{1}\circ D_{2} - D_{2}\circ D_{1}\big)f
	\text{ .}
\end{align*}
%
En t\'{e}rminos de campos, si $X,Y\in\champs{M}$, el corchete est\'{a}
dado puntualmenten por
\begin{align*}
	[X,Y]_{p}f & \,=\,X_{p}(Yf) -Y_{p}(Xf)
	\text{ .}
\end{align*}
%
Hay que verificar que esto define un campo y que, este campo es suave.

\begin{lemaElCorcheteEsSuave}\label{thm:elcorcheteessuave}
	Sean $X,Y\in\champs{M}$ campos suaves en una variedad diferencial
	$M$. Entonces el corchete de Lie de $X$ contra $Y$ es un
	campo suave en $M$.
\end{lemaElCorcheteEsSuave}

\begin{proof}
	Por \ref{thm:derivacionesycampos}, alcanzar\'{a} con verificar que
	\begin{align*}
		[X,Y]f & \,=\,X(Yf)-Y(Xf)
	\end{align*}
	%
	define una derivaci\'{o}n en $C^{\infty}(M)$. Dados $a,b\in\bb{R}$
	y dadas $f,g\in C^{\infty}(M)$, como $X$ e $Y$ son derivaciones,
	en particular son transformaciones lineales y
	\begin{align*}
		X(Y(af+bg))-Y(X(af+bg)) & \,=\,aX(Yf)+bX(Yg)-aY(Xf)-bY(Xg) \\
		& \,=\,a[X,Y]f+b[X,Y]g
		\text{ .}
	\end{align*}
	%
	Usando que aplicar $X$ y aplicar $Y$ son operaciones que cumplen
	la regla de Leibniz y que $C^{\infty}(M)$ es conmutativa,
	\begin{align*}
		X(Y(fg))-Y(X(fg)) & \,=\,X(fYg+(Yf)g)-Y(fXg+(Xf)g) \\
		& \,=\,fX(Yg)+X(Yf)g-fY(Xg)-Y(Xf)g \\
		& \,=\,f[X,Y]g + g[X,Y]f
		\text{ .}
	\end{align*}
	%
\end{proof}

\begin{propoCorcheteFRelacionados}\label{thm:corchetefrelacionados}
	Sea $F:\,M\rightarrow N$ una transformaci\'{o}n suave. Sean
	$X,Y\in\champs{M}$ y $U,V\in\champs{N}$ campos suaves tales que
	$X\sim_{F}U$ e $Y\sim_{F}V$. Entonces la composici\'{o}n
	$XY$ est\'{a} $F$-relacionada con $UV$. En particular,
	$[X,Y]\sim_{F}[U,V]$.
\end{propoCorcheteFRelacionados}

\begin{proof}
	Sea $g\in C^{\infty}(N)$ --una funci\'{o}n suave definida en
	alg\'{u}n abierto de $N$. Entonces, por hip\'{o}tesis y por
	\ref{thm:frelacionados},
	\begin{align*}
		X(Y(g\circ F)) & \,=\,X((Vg)\circ F) \,=\,
			U(Vg)\circ F
		\text{ .}
	\end{align*}
	%
	Esto muestra que $XY$ est\'{a} $F$-relacionada con $UV$, en tanto
	transformaciones de $C^{\infty}(M)$ y de $C^{\infty}(N)$,
	respectivamente, aunque no sean necesariamente derivaciones. Por
	otro lado, como, por el mismo argumento, tambi\'{e}n es cierto que
	$Y(X(g\circ F))=V(Ug)\circ F$, se deduce que
	\begin{align*}
		[X,Y](g\circ F) & \,=\,U(Vg)\circ F - V(Ug)\circ F \,=\,
			([U,V]g)\circ F
	\end{align*}
	%
	y, por lo tanto, $[X,Y]\sim_{F}[U,V]$.
\end{proof}

\begin{obsCorcheteEnCoords}\label{obs:corcheteencoords}
	Sabiendo que todo par de campos suaves $X$ e $Y$ determina un nuevo
	campo $[X,Y]$ y que el mismo es suave, es importante contar con una
	manera de calcular el corchete $[X,Y]$ sabiendo c\'{o}mo calcular los
	campos originales $X$ e $Y$. Como $X$ e $Y$ son suaves, podemos
	recurrir a sus expresiones en coordenadas. Sea $(U,\varphi)$ una
	carta compatible con la estructura de $M$ y sea $f\in C^{\infty}(U)$
	una funci\'{o}n suave. Entonces
	\begin{align*}
		[X,Y]f & \,\equiv\,X(Yf)-Y(Xf) \\
		& \,=\,	X\Big(Y^{i}\gancho{x^{i}}f\Big)-
			Y\Big(X^{j}\gancho{x^{j}}f\Big)
		\,=\,X\Big(Y^{i}\derivada{f}{x^{i}}\Big)-
			Y\Big(X^{j}\derivada{f}{x^{j}}\Big) \\
		& \,=\,\Big(X^{j}\derivada{Y^{i}}{x^{j}}\derivada{f}{x^{i}}+
			X^{j}Y^{i}
			\frac{\partial^{2} f}{\partial x^{j}\partial x^{i}}
			\Big) - \Big(
			Y^{i}\derivada{X^{j}}{x^{i}}\derivada{f}{x^{j}}+
			Y^{i}X^{j}
			\frac{\partial^{2} f}{\partial x^{i}\partial x^{j}}
			\Big)
		\text{ .}
	\end{align*}
	%
	En el caso particular en que $X=\gancho{x^{k}}$ e $Y=\gancho{x^{l}}$,
	como las componentes de $X$ y de $Y$ son constantes y, m\'{a}s aun,
	son o iguales a $0$ o iguales a $1$, se obtiene
	\begin{align*}
		\left[\gancho{x^{k}},\gancho{x^{l}}\right]f & \,=\,
			\frac{\partial^{2} f}{\partial x^{k}\partial x^{l}}-
			\frac{\partial^{2} f}{\partial x^{l}\partial x^{k}}
		\,=\, 0\text{ .}
	\end{align*}
	%
	En general, entonces,
	\begin{align*}
		[X,Y]f & \,=\,\Big(
			X^{j}\derivada{Y^{i}}{x^{j}}\derivada{f}{x^{i}} -
			Y^{i}\derivada{X^{j}}{x^{i}}\derivada{f}{x^{j}}\Big)
			+X^{j}Y^{i}\left[\gancho{x^{j}},\gancho{x^{i}}\right]
			\\
		& \,=\,\Big(X^{j}\derivada{Y^{i}}{x^{j}}\gancho{x^{i}}-
			Y^{i}\derivada{X^{j}}{x^{i}}\gancho{x^{j}}\Big) f
		\text{ .}
	\end{align*}
	%
	Como esto es cierto para toda $f$ suave en el entorno coordenado $U$,
	\begin{align*}
		[X,Y] & \,=\,\Big(X^{j}\derivada{Y^{i}}{x^{j}}\gancho{x^{i}}-
			Y^{i}\derivada{X^{j}}{x^{i}}\gancho{x^{j}}\Big) \,=\,
			\Big(XY^{i}\gancho{x^{i}}-YX^{j}\gancho{x^{j}}\Big) \\
		& \,=\, \Big(X^{i}\derivada{Y^{j}}{x^{i}}-
				Y^{i}\derivada{X^{j}}{x^{i}}\Big)
			\gancho{x^{j}} \,=\,
			\big(XY^{j}-YX^{j}\big)\gancho{x^{j}}
		\text{ .}
	\end{align*}
	%
	Esta es la expresi\'{o}n del corchete en coordenadas.
\end{obsCorcheteEnCoords}

\subsection{El corchete como derivaci\'{o}n}
El espacio $\champs{M}$ de campos vectoriales suaves en una variedad $M$
se transforma en un \'{a}lgebra de Lie con el corchete de Lie de campos.
Todo campo $X\in\champs{M}$ se puede ver como una derivaci\'{o}n
$X:\,C^{\infty}(M)\rightarrow C^{\infty}(M)$. En particular, todo
campo suave $X$ en $M$ da lugar a un operador lineal
$X\in\End{C^{\infty}(M)}$. Como fue mencionado, la composici\'{o}n de dos
campos, si bien no es necesariamente un campo, una derivaci\'{o}n, sigue
siendo un operador lineal en el \'{a}lgebra $C^{\infty}(M)$ (o anillo de
funciones en $M$).

El espacio $\End{C^{\infty}(M)}$ es un \'{a}lgebra asociativa junto con la
composici\'{o}n usual de operadores. Con lo cual existe una manera
can\'{o}nica de transformarla en un \'{a}lgebra de Lie v\'{\i}a el
``conmutador'' de dos operadores: $(S,T)\mapsto S\circ T-T\circ S$.
El hecho de que el corchete de dos campos siga siendo un campo se puede
expresar diciendo que $\champs{M}\subset\End{C^{\infty}(M)}$, el conjunto
de derivaciones de $C^{\infty}(M)$, es una sub\'{a}lgebra de Lie del
\'{a}lgebra de Lie de endomorfismos del anillo de funciones suaves de la
variedad $M$.

Sean $X$, $Y$ y $Z$ tres campos suaves en $M$. Tomar el corchete contra
el campo $Z$ define un operador en $\champs{M}$ que es lineal, por
la bilinealidad del corchete. Por otro lado, por la \emph{identidad de %
Jacobi},
\begin{align*}
	[Z,[X,Y]] & \,=\,[X,[Z,Y]]+[[Z,X],Y]
	\text{ ,}
\end{align*}
%
es decir, $[Z,\cdot]:\,\champs{M}\rightarrow\champs{M}$ se comporta como
una derivaci\'{o}n en $\champs{M}$, donde el producto de dos elementos
se define como el corchete de Lie de los campos correspondientes. Esta
es una propiedad general de \'{a}lgebras de Lie. Esto no quiere decir que
toda derivaci\'{o}n de $\champs{M}$ se tomar el corchete contra alg\'{u}n
campo.

\subsection{M\'{a}s sobre campos tangentes}
Sea $M$ una variedad diferencial y sea $S\subset M$ una subvariedad.
Sean $Y_{1},Y_{2}\in\champs{M}$ campos suaves en $M$ tangentes a la
subvariedad $S$. Por la proposici\'{o}n \ref{thm:campostangentessuaves}
existen campos $X_{1},X_{2}\in\champs{S}$ tales que
$Y_{1}|_{S}=X_{1}$ e $Y_{2}|_{S}=X_{2}$, es decir, tales que $X_{i}$
est\'{e} $\inc[S]$-relacionado con $Y_{i}$ para cada $i\in\{1,2\}$.
Por la proposici\'{o}n \ref{thm:corchetefrelacionados},
\begin{align*}
	[X_{1},X_{2}] &\,\sim_{\inc[S]}\,[Y_{1},Y_{2}]
	\text{ .}
\end{align*}
%
En consecuencia, el campo $[Y_{1},Y_{2}]\in\champs{M}$ resulta ser,
tambi\'{e}n, un campo tangente a $S$.

\begin{coroTangentesCorcheteCerrados}\label{thm:tangentescorchetecerrados}
	Sea $M$ una variedad diferencial y sea $S$ una subvariedad. Sean
	$Y_{1}$ e $Y_{2}$ campos suaves en $M$ tangentes a la subvariedad
	$S$. Entonces el corchete $[Y_{1},Y_{2}]$ es tangente a $S$,
	tambi\'{e}n.
\end{coroTangentesCorcheteCerrados}

Finalmente, enunciamos un resultado an\'{a}logo a \ref{thm:deextensiones}
para campos vectoriales suaves.

\begin{lemaExtenderCampos}\label{thm:extendercampos}
	Sea $M$ una variedad diferencial y sea $S\subset M$ una subvariedad.
	Si $S$ es subvariedad regular, entonces, dado $X\in\champs{S}$,
	existe un campo $Y\in\champs{U}$, donde $U\subset M$ es un
	subconjunto abierto que contiene a $S$, tal que $Y$ es tangente
	a $S$ e $Y|_{S}=X$. M\'{a}s aun, si $S$ es subvariedad propia,
	entonces se puede tomar $U=M$.
\end{lemaExtenderCampos}


%
\section{El fibrado cotangente}
\theoremstyle{plain}

\theoremstyle{remark}

%-------------

\subsection{El espacio cotangente}
Sea $M$ una variedad diferencial. Dado $p\in M$, el
\emph{espacio cotangente a $M$ en $p$} se define como el espacio vectorial
dual del espacio tangente a $M$ en $p$. Denotamos este espacio por
$\tangente*[p]{M}$. Es decir, el espacio cotangente a $M$ en $p$ est\'{a}
dado por
\begin{align*}
	\tangente*[p]{M} & \,=\,\dual{\big(\tangente[p]{M}\big)}
	\text{ .}
\end{align*}
%
Los elementos del espacio cotangente a una variedad en un punto $p$
se denominar\'{a}n \emph{covectores tangentes (o vectores cotangentes) en %
$p$}
Sea $(U,\varphi)$ una carta en $p$ compatible con la estructura de $M$.
La base $\Big\{\gancho[p]{x^{i}}\Big\}_{i}$ del espacio tangente en $p$
determina una base $\{\lambda^{i}|_{p}\}_{i}$ del espacio cotangente,
la base dual. Dado un vector cotangente $\omega\in\tangente*[p]{M}$,
este vector se puede escribir de manera \'{u}nica como combinaci\'{o}n
lineal de los elementos de la base $\lambda^{i}|_{p}$,
\begin{align*}
	\omega & \,=\,\omega_{i}\lambda^{i}|_{p}
	\text{ ,}
\end{align*}
%
donde
\begin{align*}
	\omega_{i} & \,=\, \omega\Big(\gancho[p]{x^{i}}\Big)
	\text{ .}
\end{align*}
%

Dada otra carta $(\tilde{U},\tilde{\varphi})$ en $p$, vale que
$\omega=\tilde{\omega}_{i}\tilde{\lambda}^{i}|_{p}$, de manera
an\'{a}loga, respecto de la base dual asociada a la base
$\Big\{\gancho[p]{\tilde{x}^{i}}\Big\}_{i}$ del tangente, correspondiente
a la nueva carta. Para hacer expl\'{\i}cita la relaci\'{o}n entre las
dos escrituras, es suficiente escribir los vectores $\gancho[p]{x^{i}}$
en t\'{e}rminos de los vectores $\gancho[p]{\tilde{x}^{i}}$. Se sabe que
\begin{align*}
	\gancho[p]{x^{i}} & \,=\,\derivada{\tilde{x}^{j}}{x^{i}}(p)
		\gancho[p]{\tilde{x}^{j}}
	\text{ .}
\end{align*}
%
Entonces se deduce que
\begin{align*}
	\omega_{i} & \,=\,\omega\Big(\gancho[p]{x^{i}}\Big)
		\,=\,\derivada{\tilde{x}^{j}}{x^{i}}(p)\,
			\omega\Big(\gancho[p]{\tilde{x}^{j}}\Big)
		\,=\,\derivada{\tilde{x}^{j}}{x^{i}}(p)\,\tilde{\omega}_{j}
	\text{ .}
\end{align*}
%

\subsection{La transpuesta del diferencial}
Sean $M$ y $N$ variedades diferenciales y sea $F:\,M\rightarrow N$ una
transformaci\'{o}n suave. Dado $p\in M$, el diferencial de la
transformaci\'{o}n $F$, la transformaci\'{o}n lineal
$\diferencial[p]{F}:\,\tangente[p]{M}\rightarrow\tangente[F(p)]{N}$, est\'{a}
definida en derivaciones por
\begin{align*}
	\diferencial[p]{F}(v)g & \,=\,v(g\circ F)
\end{align*}
%
para toda funci\'{o}n suave $g$ definida en un entorno de $F(p)$ y todo
vector tangente $v\in\tangente[p]{M}$. La transpuesta de esta
transformaci\'{o}n lineal es la transformaci\'{o}n lineal
$\diferencial*[p]{F}:\,\tangente*[F(p)]{N}\rightarrow\tangente*[p]{M}$
dada por
\begin{align*}
	\diferencial*[p]{F}(\omega)(v) & \,=\,
		\omega\big(\diferencial[p]{F}(v)\big)
	\text{ .}
\end{align*}
%
Usando la notaci\'{o}n de los corchetes esta igualdad se expresa como
\begin{align*}
	\langle v|\diferencial*[p]{F}(\omega)\rangle & \,=\,
		\langle \diferencial[p]{F}(v)|\omega\rangle
	\text{ .}
\end{align*}
%
Esta transformaci\'{o}n transformaci\'{o}n lineal asociada al diferencial
de una transformaci\'{o}n suave $F$ por transposici\'{o}n se denomina
\emph{pullback de $F$ en $p$}.

\subsection{El diferencial de una funci\'{o}n suave}
Sea $M$ una variedad, sea $p\in M$ un punto arbitrario y sea
$v\in\tangente[p]{M}$ un vector tangente a $M$ en $p$. En tanto derivaci\'{o}n,
dada una funci\'{o}n suave $f:\,M\rightarrow\bb{R}$ definida cerca de $p$,
el vector tangente $v$ act\'{u}a en $f$, tomando un determinado valor
$vf\in\bb{R}$. Si, en cambio, se fija una funci\'{o}n suave $f$ definida
en $p$ y se deja variar $v$ entre los vectores tangentes en $p$, se
obtiene una aplicaci\'{o}n $\derext[p]{f}:\,%
\tangente[p]{M}\rightarrow\bb{R}$ dada por
\begin{align*}
	\derext[p]{f}(v) & \,=\,v(f)
	\text{ .}
\end{align*}
%
De la definici\'{o}n del espacio de derivaciones, se deduce que
$\derext[p]{f}$ es una funcional lineal en $\tangente[p]{M}$. Esta
funcional se denomina \emph{diferencial de $f$ en $p$}. En tanto
transformaci\'{o}n suave, de la variedad $M$ en la variedad $\bb{R}$,
el diferencial de $f$ en $p$ hab\'{\i}a sido definido como la
transformaci\'{o}n lineal $\diferencial[p]{f}:\,%
\tangente[p]{M}\rightarrow\tangente[f(p)]{\bb{R}}$ que, a un vector
tangente $v\in\tangente[p]{M}$ le asocia la derivaci\'{o}n que, en
una funci\'{o}n suave $g$ definida en $\bb{R}$, toma el valor
\begin{align*}
	\diferencial[p]{f}(v)g & \,=\,v(g\circ f)
	\text{ .}
\end{align*}
%
Por el momento, para distinguir estas dos nociones, denominaremos
\emph{derivada de $f$ en $p$} al diferencial $\derext[p]{f}$ de $f$ en $p$.

El diferencial $\diferencial[p]{f}$ de una funci\'{o}n suave $f$ vista
como transformaci\'{o}n suave en $M$ se puede interpretar como una
funcional. En primer lugar, dado un punto $x\in\bb{R}$, en tanto variedad
diferencial, se puede identificar can\'{o}nicamente el espacio tangente
$\tangente[x]{\bb{R}}$ con el espacio vectorial $\bb{R}$, asociando a
un elemento $y\in\bb{R}$ la derivaci\'{o}n en $x$ dada por
\begin{align*}
	y\,f & \,=\,\lim_{t\to 0}\,\frac{f(x+ty)-f(x)}{t}
	\text{ ,}
\end{align*}
%
es decir, tomar derivada direccional en la direcci\'{o}n de $y$. En
coordenadas, tomando la carta global $(\bb{R},\id[\bb{R}])$, esta
identificaci\'{o}n est\'{a} dada por $\upsilon =y\gancho[x]{t}%
\leftrightarrow y$. Es decir, el elemento del espacio vectorial $\bb{R}$
determinado por la derivaci\'{o}n $\upsilon\in\tangente[x]{\bb{R}}$ est\'{a}
dado por evaluar la derivaci\'{o}n en la funci\'{o}n $\id[\bb{R}]$:
\begin{align*}
	\upsilon\,\id[\bb{R}] & \,=\,y
	\text{ .}
\end{align*}
%
Si ahora $v\in\tangente[p]{M}$ es un vector tangente a la variedad $M$ en el
punto $p$, entonces la derivaci\'{o}n $\diferencial[p]{f}(v)\in%
\tangente[f(p)]{\bb{R}}$ se corresponde con el elemento
\begin{align*}
	\diferencial[p]{f}(v)(\id[\bb{R}]) & \,=\,v(\id[\bb{R}]\circ f)
		\,=\,v(f)
	\text{ .}
\end{align*}
%
Es decir, identificando el espacio tangente $\tangente[f(p)]{\bb{R}}$
con el mismo espacio $\bb{R}$, la derivaci\'{o}n $\diferencial[p]{f}(v)$
se corresponde con
\begin{align*}
	\diferencial[p]{f}(v)(\id[\bb{R}]) & \,=\,\derext[p]{f}(v)
	\text{ .}
\end{align*}
%
De esta manera, se puede interpretar que el diferencial $\diferencial[p]{f}$
en tanto transformaci\'{o}n suave de $f$ en $p$ se corresponde con la
funcional lineal dada por la derivada de $f$ en $p$, el diferencial en tanto
funci\'{o}n $\derext[p]{f}$. Haciendo esta identificaci\'{o}n, nos referiremos
indistintamente usando el nombre ``diferencial de $f$ en $p$'' a
cualquiera de estos dos objetos.

Por otro lado, el diferencial de $f$ como funci\'{o}n (la derivada de $f$)
se puede relacionar con la adjunta del diferencial de $f$ en tanto
transformaci\'{o}n: si $\gancho[x]{t}$ es el \'{u}nico elemento de la
base de $\tangente[x]{\bb{R}}$ dado por la carta global usual, entonces,
tomando la base dual, se obtiene un covector $\omega|_{x}%
\in\tangente*[x]{\bb{R}}$. Si $x=f(p)$, tomando el pullback de $\omega$ por
$f$ en $p$, se obtiene un vector cotangente
$\diferencial*[p]{f}(\omega|_{f(p)})\in\tangente*[p]{M}$. Este covector
est\'{a} determinado por su valor en los vectores del espacio tangente
a $M$ en $p$. Si $v\in\tangente[p]{M}$, entonces
\begin{align*}
	\diferencial*[p]{f}(\omega|_{f(p)})(v) & \,=\,
		\omega|_{f(p)}\big(\diferencial[p]{f}(v)\big)
	\text{ .}
\end{align*}
%
Dado que
\begin{align*}
	\diferencial[p]{f}(v) & \,=\,y\gancho[f(p)]{t}
	\text{ ,}
\end{align*}
%
donde
\begin{align*}
	y & \,=\,\diferencial[p]{f}(v)\,(\id[\bb{R}]) \,=\,
		v(\id[\bb{R}]\circ f) \,=\, v(f)
	\text{ ,}
\end{align*}
%
se deduce que
\begin{align*}
	\omega|_{f(p)}\big(\diferencial[p]{f}(v)\big)
	& \,=\,v(f) \,=\,\derext[p]{f}(v)
\end{align*}
%
y, por lo tanto, que
\begin{align*}
	\diferencial*[p]{f}(\omega|_{f(p)})(v) & \,=\,
		\derext[p]{f}(v)
\end{align*}
%
para toda derivaci\'{o}n $v\in\tangente[p]{M}$. Es decir, en definitiva,
el pullback por $f$ del covector de la base $\omega|_{f(p)}$ es igual a
\begin{align*}
	\diferencial*[p]{f}(\omega|_{f(p)}) & \,=\,
		\derext[p]{f}
	\text{ .}
\end{align*}
%

\subsection{Bases coordenadas del espacio cotangente}
Sea $M$ una variedad diferencial, sea $(U,\varphi)$ una carta compatible
con la estructura de $M$ y sea $p\in U$. Sea
$\Big\{\gancho[p]{x^{i}}\Big\}_{i}$ la base del espacio tangente
$\tangente[p]{M}$ asociada a las funciones coordenadas $\varphi=(x^{i})$ y
sea $\{\lambda^{i}|_{p}\}_{i}$ la base dual en el espacio cotangente
$\tangente*[p]{M}$. Hemos visto que es posible asociarle a toda funci\'{o}n
suave $f$ en la variedad $M$ definida cerca de $p$ una funcional
$\derext[p]{f}:\,\tangente[p]{M}\rightarrow\bb{R}$, un elemento del
espacio cotangente a $M$ en $p$, v\'{\i}a
\begin{align*}
	\derext[p]{f}(v) & \,=\,v(f)
\end{align*}
%
para todo vector tangente $v\in\tangente[p]{M}$. Con respecto a la base
$\Big\{\gancho[p]{x^{i}}\Big\}_{i}$ del espacio tangente en $p$,
\begin{align*}
	\derext[p]{f}(v) & \,=\,v^{i}\,\gancho[p]{x^{i}}f \,=\,
		v^{i}\derivada{f}{x^{i}}(p)
	\text{ ,}
\end{align*}
%
con lo cual, en la base $\{\lambda^{i}|_{p}\}_{i}$ del espacio cotangente
en $p$,
\begin{align*}
	\derext[p]{f} & \,=\,\derivada{f}{x^{i}}(p)\lambda^{i}|_{p}
	\text{ .}
\end{align*}
%
Vale la pena mencionar que en funci\'{o}n del punto $p$ las componentes
de $\derext[p]{f}$ en esta base, es decir, sus derivadas respecto de
las derivaciones $\gancho[p]{x^{i}}$, son funciones suaves.

En el caso particular en que $f=x^{i}$ es igual a alguna de las funciones
coordenadas asociadas a la carta $(U,\varphi)$, dado que
$x^{i}:\,U\rightarrow\bb{R}$ es una funci\'{o}n suave para cada \'{\i}ndice
$i$, es posible determinar el diferencial $\derext[p]{x^{i}}$ en $p$.
Esta funcional est\'{a} dada, en coordenadas, por
\begin{align*}
	\derext[p]{x^{i}} & \,=\, \derivada{x^{i}}{x^{j}}(p)\,\lambda^{j}|_{p}
		\,=\,\lambda^{i}|_{p}
	\text{ .}
\end{align*}
%
Es decir, los diferenciales $\{\derext[p]{x^{1}},\,\dots,\,\derext[p]{x^{n}}\}$
de las funciones coordenadas en $p$ son precisamente los elementos de
la base dual asociada a la base del espacio tangente en $p$ formada
por las derivaciones $\gancho[p]{x^{i}}$. Podemos reescribir, entonces,
dada una funci\'{o}n suave $f$ definida cerca de $p$, la expresi\'{o}n
de su diferencial en coordenadas:
\begin{align*}
	\derext[p]{f} & \,=\,\derivada{f}{x^{i}}(p)\,\derext[p]{x^{i}}
	\text{ .}
\end{align*}
%
En ocasiones, escribiremos $\de[p]{x^{i}}$ en lugar de $\derext[p]{x^{i}}$,
para distinguirlos en tanto elementos de la base asociada a una carta
coordenada, pero ambas expresiones har\'{a}n referencia al mismo objeto.

Para terminar esta secci\'{o}n, introducimos un objeto an\'{a}logo
al fibrado tangente de una variedad, pero que re\'{u}ne coherentemente los
espacios cotangentes. Sea $M$ una variedad diferencial y sea
\begin{align*}
	\tangente*{M} & \,=\,\bigsqcup_{p\in M}\,\tangente*[p]{M}
\end{align*}
%
la uni\'{o}n disjunta de los espacios cotangentes a $M$ en cada punto $p$.
Sea $\pi:\,\tangente*{M}\rightarrow M$ la proyecci\'{o}n can\'{o}nica
$\pi(p,\omega)=p$. Para dar una estructura de variedad diferencial a
esta uni\'{o}n, partimos, como en el caso del fibrado tangente, de una
carta compatible $(U,\varphi)$ en $M$. Sean
\begin{align*}
	\widetilde{U} & \,=\, \left\lbrace (p,\omega_{p})\in\tangente*{M}
		\,:\,p\in U,\,\omega_{p}\in\tangente*[p]{M}
		\right\rbrace
	\quad\text{y} \\
	\widetilde{\varphi}\big(p,\omega_{i}\de[p]{x^{i}}\big) & \,=\,\big(
		x^{1}(p),\,\dots,\,x^{n}(p),\,\omega_{1},\,\dots,\,\omega_{n}
		\big)
	\text{ ,}
\end{align*}
%
siempre que $p\in U$. Si $(V,\psi)$ y $\psi=(y^{i})$ es otra carta
compatible que se solapa con $(U,\varphi)$, entonces
\begin{align*}
	\widetilde{\varphi}\circ\widetilde{\psi}^{-1}(
		\lista*{y}{n},\,\lista{\omega}{n}) & \,=\,
		\widetilde{\varphi}\big(\psi^{-1}(y),
		\omega_{i}\de[\psi^{-1}(y)]{y^{i}}\big) \\
	& \,=\, \Big(x^{1}(y),\,\dots,\,x^{n}(y),\,
		\derivada{y^{j}}{x^{1}}(y)\,\omega_{j},\,\dots,\,
		\derivada{y^{j}}{x^{n}}(y)\,\omega_{j}\Big)
	\text{ .}
\end{align*}
%
Dado que las cartas $(U,\varphi)$ y $(V,\psi)$ son suavemente compatibles,
la composici\'{o}n 
\begin{align*}
	\widetilde{\varphi}\circ\widetilde{\psi}^{-1} & \,:\,
		\psi(U\cap V)\times\bb{R}^{2n}\,\rightarrow\,
		\varphi(U\cap V)\times\bb{R}^{2n}
\end{align*}
%
es suave en el sentido usual. Los pares $(\widetilde{U},\widetilde{\varphi})$
definidos a partir de cartas compatibles $(U,\varphi)$ para $M$ dan lugar a
un atlas suavemente compatible en $\tangente*{M}$ y determinan una
estructura diferencial en la uni\'{o}n disjunta de los espacios cotangentes.
La variedad diferencial determinada a partir de $\tangente*{M}$ de esta
manera se denomina \emph{fibrado cotangente en/a/de $M$}.


%
\section{$1$-formas}
\theoremstyle{plain}
\newtheorem{propoUnoFormasSuaves}{Proposici\'{o}n}[section]
\newtheorem{lemaElPullbackEnUnoFormas}[propoUnoFormasSuaves]{Lema}
\newtheorem{propoElPullbackDeUnaSuave}[propoUnoFormasSuaves]{Proposici\'{o}n}
\newtheorem{teoUnoFormasSuavesComoMorfismos}[propoUnoFormasSuaves]{Teorema}
\newtheorem{propoIntegrarElPullDeUnaCurva}%
	[propoUnoFormasSuaves]{Proposici\'{o}n}
\newtheorem{propoIntegrarConcretamenteSobreUnaCurva}%
	[propoUnoFormasSuaves]{Proposici\'{o}n}
\newtheorem{teoUnoFormaConservativaSiiExacta}[propoUnoFormasSuaves]{Teorema}

\theoremstyle{remark}
\newtheorem{obsDiferencialDerivacion}{Observaci\'{o}n}[section]
\newtheorem{obsUnoFormasYCampos}[obsDiferencialDerivacion]{Observaci\'{o}n}
\newtheorem{obsVariedadSuaveATrozosConexa}%
	[obsDiferencialDerivacion]{Observaci\'{o}n}
\newtheorem{obsIntegralEsLinealEnFormas}%
	[obsDiferencialDerivacion]{Observaci\'{o}n}
\newtheorem{obsIntegralEnCaminoInverso}%
	[obsDiferencialDerivacion]{Observaci\'{o}n}
\newtheorem{obsIntegralEsLinealEnCaminos}%
	[obsDiferencialDerivacion]{Observaci\'{o}n}
\newtheorem{obsIntegrarFormasExactas}%
	[obsDiferencialDerivacion]{Observaci\'{o}n}
\newtheorem{obsDiferencialCeroConstante}%
	[obsDiferencialDerivacion]{Observaci\'{o}n}
\newtheorem{obsIntegrarExactasEnCaminosCerrados}%
	[obsDiferencialDerivacion]{Observaci\'{o}n}
\newtheorem{obsConservativaSiiIntegralIndependienteDelCamino}%
	[obsDiferencialDerivacion]{Observaci\'{o}n}

%-------------

Un campo vectorial se define como una secci\'{o}n local del fibrado tangente.
Las secciones locales del fibrado cotangente reciben el nombre de
\emph{$1$-formas}.

Sea $(U,\varphi)$ una carta en $M$. Sean $\varphi=(x^{i})$ las funciones
coordenadas, $\Big\{\gancho[p]{x^{i}}\Big\}_{i}$ la base del espacio tangente
a $M$ en un punto $p\in U$ y sea $\big\{\de[p]{x^{i}}\big\}_{i}$ la base
dual en el espacio cotangente, la cual est\'{a} compuesta por los
diferenciales de las funciones coordenadas $x^{i}$ en $p$. Sea
$\omega:\,U\rightarrow\tangente*{M}$ una secci\'{o}n de la proyecci\'{o}n
can\'{o}nica $\pi:\,\tangente*{M}\rightarrow M$ definida, localmente,
en el abierto $U$. En particular, $\pi\circ\omega=\id[U]$. Si $p\in U$,
denotamos al elemento en el espacio $\tangente*[p]{M}$ dado por la
secci\'{o}n $\omega$ en $p$ por $\omega_{p}$, o bien $\omega|_{p}$.
Es decir, $\omega:\,p\in U\mapsto\omega_{p}\in\tangente*{M}$ es una
apliaci\'{o}n que a cada elemento $p$ en el abierto coordenado $U$ le
asigna una funcional en el espacio cotangente $\tangente*[p]{M}$.
En coordenadas, esta aplicaci\'{o}n est\'{a} dada de la siguiente manera:
cada covector $\omega_{p}\in\tangente*[p]{M}$ se escribe de manera \'{u}nica
como combinaci\'{o}n lineal de los elementos de la base
$\big\{\de[p]{x^{i}}\big\}_{i}$ dada por los diferenciales de las funciones
coordenadas, es decir, existen n\'{u}meros reales
$\omega_{i}(p)\in\bb{R}$ \'{u}nicos tales que
\begin{equation}
	\label{eq:covectencoordes}
	\omega_{p} \,=\,\omega_{i}(p)\,\de[p]{x^{i}}
	\text{ .}
\end{equation}
%
Cada uno de los coeficientes $\omega_{i}(p)$ en la escritura anterior
se denomina \emph{componente de $\omega$ en $p$} y su valor est\'{a} dado,
recurriendo a las funciones coordenadas, por
\begin{align*}
	\omega_{i}(p) & \,=\,\omega_{p}\Big(\gancho[p]{x^{i}}\Big)
	\text{ .}
\end{align*}
%
Esto determina un\'{\i}vocamente funciones $\omega_{i}:\,U\rightarrow\bb{R}$
tales que vale \eqref{eq:covectencoordes} para todo punto $p\in U$. Estas
funciones se denominan \emph{componentes de $\omega$} con respecto a
las coordenadas $\varphi=(x^{i})$ en el abierto coordenado $U$.

\subsection{$1$-formas suaves}
Recordando que todo abierto $U\subset M$ tiene estructura de variedad
diferencial en tanto subespacio abierto de la variedad $M$ y que a
$\tangente*{M}$ tambi\'{e}n se le dio una estructura de variedad diferencial,
diremos que una $1$-forma $\omega:\,U\rightarrow\tangente*{M}$ definida
en un abierto $U$ de $M$ es \emph{suave}, si es suave en tanto
transformaci\'{o}n entre las variedades $U$ y $\tangente*{M}$.
La siguiente proposici\'{o}n da una forma alternativa de caracterizar la
suavidad de una $1$-forma.

\begin{propoUnoFormasSuaves}\label{thm:unoformassuavescomponentes}
	Sea $M$ una variedad diferencial y sea $\omega:\,%
	M\rightarrow\tangente*{M}$ una $1$-forma. Las siguientes afirmaciones
	son equivalentes.
	\begin{itemize}
		\item[(\i)] $\omega$ es suave;
		\item[(\i\i)] si $(U,\varphi)$ es una carta compatible
			con la estructura de $M$, entonces las componentes
			de $\omega$ con respecto a $\varphi$ son funciones
			suaves en $U$;
		\item[(\i\i\i)] todo punto $p\in M$ est\'{a} contenido en
			el dominio de alguna carta compatible
			$(U,\varphi)$ tal que las componentes de $\omega$
			en $U$ son funciones suaves.
	\end{itemize}
	%
\end{propoUnoFormasSuaves}

Dada una carta $(U,\varphi)$, los diferenciales $\de[p]{x^{i}}$ dan
lugar a $1$-formas definidas en el abierto $U$. Dado que
los covectores $\de[p]{x^{i}}$ constituyen una base de los espacios
cotangentes en cada punto $p$ de $U$, toda $1$-forma definida en (alg\'{u}n
lugar de) $U$ se puede escribir como combinaci\'{o}n lineal de las
$1$-formas $\de{x^{i}}:\,p\mapsto\de[p]{x^{i}}$ por ciertas funciones
definidas en $U$ a valores reales. Es decir, la igualdad
\eqref{eq:covectoresencoordes} v\'{a}lida para todo $p\in U$ se puede
escribir como una igualdad entre $1$-formas: si
$\omega:\,U\rightarrow\tangente*{M}$ es una $1$-forma, entonces
existen funciones $\omega_{i}:\,U\rightarrow\bb{R}$ \'{u}nicas tales que
\begin{align*}
	\omega & \,=\,\omega_{i}\de{x^{i}}
	\text{ .}
\end{align*}
%
En particular, si $f:\,U\rightarrow\bb{R}$ es una funci\'{o}n suave,
su \emph{diferencial (derivada)} es la $1$-forma dada por
\begin{align*}
	\derext{f} & \,=\,\derivada{f}{x^{i}}\,\de{x^{i}}
	\text{ .}
\end{align*}
%

Sea $\omega$ una $1$-forma y sea $X$ un campo tangente, es decir, secciones
localmente definidas de los fibrados cotantegente y tangente, respectivamente.
Sean o no suaves o continuas, estas aplicaciones determinan, all\'{\i}
donde ambas est\'{a}n definidas, una funci\'{o}n a valores reales:
\begin{align*}
	\big(\omega X\big)(p) & \,=\,\omega_{p}\big(X_{p}\big)
\end{align*}
%
dada por evaluar el covector $\omega_{p}\in\tangente*[p]{M}$ en el vector
tangente $X_{p}\in\tangente[p]{M}$. En coordenadas, si $(U,\varphi)$ es una
carta compatible tal que ambos $\omega$ y $X$ est\'{a}n definidos en $U$,
entonces existen funciones (no necesariamente continuas) $X^{i}$ y
$\omega_{i}$ definidas en el abierto $U$ tales que
\begin{align*}
	X \,=\,X^{i}\gancho{x^{i}} & \quad\text{, }\quad
	\omega \,=\,\omega_{i}\de{x^{i}}\quad\text{y} \\
	\omega X & \,=\,\omega_{i}X^{i}
	\text{ .}
\end{align*}
%

\begin{propoUnoFormasSuaves}\label{thm:unoformassuavescampos}
	Sea $M$ una variedad diferencial y sea $\omega$ una $1$-forma
	definida en $M$. Entonces que $\omega$ sea suave equivale a que
	se cumpla cualquiera de las siguientes afirmaciones.
	\begin{itemize}
		\item[(\j)] Si $X\in\champs{M}$ es un campo (global) suave,
			entonces la funci\'{o}n $\omega X:%
			\,M\rightarrow\bb{R}$ es suave;
		\item[(\j\j)] si $U\subset M$ es abierto y $X\in\champs{U}$
			es un campo suave definido en $U$, entonces
			la funci\'{o}n $\omega X:\,U\rightarrow\bb{R}$ es
			suave.
	\end{itemize}
	%
\end{propoUnoFormasSuaves}

\begin{proof}
	Si $p\in M$ es un punto arbitrario de la variedad, entonces,
	tomando coordenadas $(U,\varphi)$ en $p$, vale que la funci\'{o}n
	$\omega X$ es igual a $\omega_{i}X^{i}$ en $U$. Si $X$ y $\omega$
	son suaves, entonces, por el \'{\i}tem \textit{(\i\i)} de
	la proposici\'{o}n \ref{thm:unoformassuavescomponentes}, las funciones
	componentes $\omega_{i}$ son suaves y, como las componentes
	$X^{i}$ tambi\'{e}n son suaves, $\omega_{i}X^{i}$ es suave en $U$
	y, por lo tanto, $\omega X$ es suave en $U$. Como $p$ es arbitrario,
	se deduce que $\omega X\in C^{\infty}(M)$.

	Supongamos que vale \textit{(\j)} y que $X:\,U\rightarrow\tangente{M}$
	es un campo suave definido en un abierto $U\subset M$. Sea $p\in U$
	un punto arbitrario del abierto. Es necesario definir un campo global
	suave que coincida con $X$ cerca de $p$. Sea
	$\psi:\,M\rightarrow\bb{R}$ una funci\'{o}n chich\'{o}n en $p$ y sea
	$V\subset U$ un entorno de $p$ contenido en $U$ tal que
	$\clos{V}\subset U$ y $\psi=1$ en $V$ y $\soporte{\psi}\subset U$.
	Sea $\tilde{X}:\,M\rightarrow\tangente{M}$ el campo global dado
	por $\psi X$ en $U$ y por $0$ en $M\setmin\soporte{\psi}$. Como
	$\psi$ es una funci\'{o}n chich\'{o}n suave, $\tilde{X}\in\champs{M}$
	y $\tilde{X}|_{V}=X|_{V}$. Entonces, por \textit{(\j)}, la funci\'{o}n
	$\omega\tilde{X}:\,M\rightarrow\bb{R}$ es suave y, adem\'{a}s, por
	construcci\'{o}n,
	\begin{align*}
		\omega\tilde{X}|_{V} & \,=\,\omega X|_{V}
		\text{ .}
	\end{align*}
	%
	Pero, entonces, $\omega X$ coincide en el abierto $V$ con una
	funci\'{o}n suave. Como $p\in U$ es arbitrario, se deduce que
	$\omega X:\,U\rightarrow\bb{R}$ es suave en $U$.

	Supongamos ahora que la afirmaci\'{o}n \textit{(\j\j)} es cierta
	para $\omega$. Sea $p\in M$ y sea $(U,\varphi)$ una carta compatible
	en $p$. Los campos locales $\gancho{x^{i}}:\,U\rightarrow\tangente{M}$
	son suaves (son constantes). Por hip\'{o}tesis, evaluar $\omega$ en
	cada uno de estos campos define una funci\'{o}n suave
	$\omega\gancho{x^{i}}$ en $U$. Pero
	\begin{align*}
		\omega\gancho{x^{i}} & \,=\,\omega_{i}
		\text{ ,}
	\end{align*}
	%
	donde $\omega_{i}$ es la componente correspondiente a
	$\de{x^{i}}$. En definitiva, todo punto de la variedad admite una
	carta en donde la forma $\omega$ tiene componentes suaves.
	Con lo cual, por \textit{(\i\i\i)}, se concluye que $\omega$ es una
	$1$-forma suave en $M$.
\end{proof}

Usaremos la notaci\'{o}n $\formes[1]{M}$ para denotar el conjunto de
$1$-formas suaves en una variedad diferencial $M$. Definiendo la suma y el
producto por escalares puntualmente:
\begin{align*}
	(a\omega+b\eta)|_{p} & \,=\,a\omega_{p}+b\eta_{p}
	\text{ ,}
\end{align*}
%
se le da una estructura de espacio vectorial real al conjunto de $1$-formas.
M\'{a}s aun, al igual que con los campos vectoriales, dada una
funci\'{o}n $f$ y una $1$-forma $\omega$ (no necesariamente continuas),
el \emph{producto} de $f$ por $\omega$,
\begin{align*}
	\big(f\omega\big)|_{p} & \,=\,f(p)\omega_{p}
	\text{ ,}
\end{align*}
%
define una nueva $1$-forma en donde ambas aplicaciones est\'{e}n definidas.
En particular, si $f\in C^{\infty}(M)$ y $\omega\in\formes[1]{M}$, se
deduce que $f\omega\in\formes[1]{M}$, tambi\'{e}n. Es decir, el espacio
vectorial $\formes[1]{M}$ tiene estructura de $C^{\infty}(M)$-m\'{o}dulo.

\begin{obsDiferencialDerivacion}\label{obs:diferencialderivacion}
	Dada una variedad $M$ y funciones suaves $f,g$, la aplicaci\'{o}n
	$f\mapsto\derext{f}\in\formes[1]{M}$ dada por tomar diferencial
	es lineal:
	\begin{align*}
		\derext{(af+bg)} & \,=\,a\derext{f}+b\derext{g}
	\end{align*}
	%
	y se comporta, tambi\'{e}n, como una derivaci\'{o}n:
	\begin{align*}
		\derext{(fg)} & \,=\,f\derext{g} + g\derext{f}
		\text{ .}
	\end{align*}
	%
\end{obsDiferencialDerivacion}

\begin{obsUnoFormasYCampos}\label{obs:unoformasycampos}
	Sea $M$ una variedad diferencial y sea
	$\omega:\,M\rightarrow\tangente*{M}$ una $1$-forma. Como se
	mencion\'{o} anteriormente, dado un campo tangente
	$X:\,M\rightarrow\tangente{M}$, se obtiene una funci\'{o}n
	$\omega X:\,M\rightarrow\bb{R}$ dada por $p\mapsto\omega_{p}X_{p}$.
	Adem\'{a}s, si $X\in\champs{M}$ y $\omega\in\formes[1]{M}$ son
	suaves, entonces $\omega X\in C^{\infty}(M)$. Por otro lado,
	$C^{\infty}(M)$ es una $\bb{R}$-\'{a}lgebra y $\champs{M}$
	y $\formes[1]{M}$ son $C^{\infty}(M)$-m\'{o}dulos. Cada $1$-forma
	suave $\omega$ define una aplicaci\'{o}n
	\begin{align*}
		\omega & \,:\,\champs{M}\,\rightarrow\,C^{\infty}(M)
		\text{ .}
	\end{align*}
	%
	Si $f\in C^{\infty}(M)$ es una funci\'{o}n suave,
	\begin{align*}
		\omega\big( fX\big) & \,:\,p\,\mapsto\,
			\omega_{p}\big(f(p)X_{p}\big) \,=\,
			f(p)\omega_{p}X_{p}
		\text{ .}
	\end{align*}
	%
	En definitiva, $\omega\big(fX\big)=f\omega X$, es decir, $\omega$
	respeta la estrctura de $C^{\infty}(M)$-m\'{o}dulos.

	Rec\'{\i}procamente, si $T:\,\champs{M}\rightarrow C^{\infty}(M)$
	es morfismo de $C^{\infty}(M)$-m\'{o}dulos, entonces se puede
	definir una $1$-forma que da lugar a morfismo $T$. En primer lugar,
	la aplicaci\'{o}n $p\mapsto (TX)(p)$, es decir, la
	funci\'{o}n suave $TX:\,M\rightarrow\bb{R}$ depende de $X$
	\'{u}nicamente en su valor en cada punto. Dicho de otra manera,
	entodo punto $p\in M$, el valor $TX(p)$ depende \'{u}nicamente del
	vector tangente $X_{p}$ y no del campo $X$. Si $Y\in\champs{M}$ es
	otro campo tal que $Y_{p}=X_{p}$ entonces $TY(p)=TX(p)$. Para
	demostrar esta afirmaci\'{o}n es suficiente demostrarlo que,
	si $X_{p}=0$, entonces $TX(p)=0$, pues $TX-TY=T(X-Y)$ para todo
	par de campos suaves $X$ e $Y$, dado que $T$ es, en particular,
	$\bb{R}$-lineal. En segundo lugar, una $1$-forma es, puntualmente,
	un covector y un covector en un punto $p$ es una funcional lineal
	en el espacio tangente a $M$ en $p$. As\'{\i}, dado que $TX$ depende
	\emph{puntualmente} de $X$, es posible definir una $1$-forma
	por $\omega_{p}v=TX(p)$, donde $X$ es cualquier extensi\'{o}n
	suave a $M$ del vector tangente $v\in\tangente[p]{M}$.

	En cuanto a la primera afirmaci\'{o}n, sea $p\in M$ un punto
	de la variedad y sea $X\in\champs{M}$ un campo suave tal que $X_{p}$
	es el vector cero. Si, en coordenadas, $X = X^{i}\gancho{x^{i}}$,
	cerca de $p$, en $p$ vale que $X_{p}=0$ y que, por independencia,
	$X^{i}(p)=0$. Sea $U$ el entorno coordenado de $p$ en donde las
	funciones $X^{i}$ y los campos $\gancho{x^{i}}$ est\'{a}n definidos.
	Sea $\psi:\,M\rightarrow\bb{R}$ una funci\'{o}n chich\'{o}n en $p$
	con soporte contenido en $U$. Para demostrar que $TX(p)=0$, ser\'{a}
	necesario poder realizar una c\'{a}lculo puntual en $p$. Para
	ello se extender\'{a}n los campos coordenados y las funciones
	componentes de $X$. Para cada \'{\i}ndice $i$, sea $\tilde{X}^{i}$
	la funci\'{o}n que toma el valor $\psi\cdot X^{i}$ en $U$ y el valor
	$0$ fuera del soporte de $\psi$. Sean, tambi\'{e}n $Y_{i}$ los campos
	dados por $\psi\cdot\gancho{x^{i}}$ en $U$ y por $0$ fuera del
	soporte de $\psi$. Entonces, en toda $M$,
	\begin{align*}
		X & \,=\, \tilde{X}^{i}Y_{i} + (1-(\psi)^{2})X
		\text{ .}
	\end{align*}
	%
	Aplicando $T$,
	\begin{align*}
		TX & \,=\, \tilde{X}^{i}TY_{i} + (1-(\psi)^{2})TX
		\text{ .}
	\end{align*}
	%
	En el punto $p$, la funci\'{o}n $\psi$ vale $\psi(p)=1$, por lo que
	\begin{align*}
		TX(p) & \,=\,\tilde{X}^{i}(p)TY_{i}(p) +
			(1-\psi(p)^{2})TX(p) \,=\,
			X^{i}(p)TY_{i}(p)
		\text{ .}
	\end{align*}
	%
	Entonces, asumiendo que $X_{p}=0$, las funciones componentes
	$X^{i}$ toman el valor $0$ en $p$ y $TX(p)=0$.

	Definir una $1$-forma a partir del morfismo $T$, implica dar, para
	cada $p\in M$, una funcional $\omega_{p}\in\tangente*[p]{M}$.
	Sea $p\in M$ y sea $v_{p}\in\tangente[p]{M}$ un vector tangente
	en $p$. Sea $X\in\champs{M}$ un campo suave tal que $X_{p}=v_{p}$.
	Un campo con esta propiedad se puede definir usando una funci\'{o}n
	chich\'{o}n en $p$. Sea $\omega_{p}(v_{p})=TX(p)$. El valor
	de $\omega_{p}$ en el vector $v_{p}$ est\'{a} bien definido pues
	el valor de la funci\'{o}n $TX$ en $p$ no depende del campo $X$
	particular elegido como extensi\'{o}n de $v_{p}$. Adem\'{a}s,
	$\omega_{p}:\,\tangente[p]{M}\rightarrow\bb{R}$ es lineal, ya que,
	si $v_{p},v'_{p}\in\tangente[p]{M}$ y $a\in\bb{R}$, y si
	$X,X'\in\champs{M}$ son campos suaves tales que $X_{p}=v_{p}$ y
	$X'_{p}=v'_{p}$, entonces $X+aX'\in\champs{M}$ es un campo suave
	tal que $(X+aX')|_{p}=v_{p}+av'_{p}$. En particular, vale que
	\begin{align*}
		\omega_{p}(v_{p}+av'_{p}) & \,=\,T(X+aX')(p) \,=\,
			TX(p)+aTX'(p) \,=\,
			\omega_{p}(v_{p})+a\omega_{p}(v'_{p})
		\text{ .}
	\end{align*}
	%
	As\'{\i}, se deduce que $\omega:\,M\rightarrow\tangente*{M}$ es una
	$1$-forma. Por otro lado, dado un campo suave $X\in\champs{M}$,
	\begin{align*}
		\omega X (p) & \,=\,\omega_{p}X_{p} \,=\, TX(p)
		\text{ .}
	\end{align*}
	%
	Esta igualdad muestra dos cosas: en primer lugar, muestra que
	$\omega X=TX$ para todo campo suave $X$ y, en segundo lugar, que
	$\omega X=TX\in C^{\infty}(M)$ para todo campo suave
	$X\in\champs{M}$. Por la proposici\'{o}n
	\ref{thm:unoformassuavescampos}, la forma $\omega$ es suave.
\end{obsUnoFormasYCampos}

De la misma manera que el teorema \ref{thm:derivacionesycampos} dice que
los campos suaves en una variedad $M$ se corresponden con las derivaciones de
$C^{\infty}(M)$, la observaci\'{o}n anterior muestra que las $1$-formas
suaves se pueden caracterizar de manera similar.

\begin{teoUnoFormasSuavesComoMorfismos}%
	\label{thm:unoformassuavescomomorfismos}
	Sea $M$ una variedad diferencial. Toda $1$-forma suave
	$\omega\in\formes[1]{M}$ define un morfismo de
	$C^{\infty}(M)$-m\'{o}dulos
	\begin{align*}
		\omega & \,:\,\champs{M}\,\rightarrow\,C^{\infty}(M)
	\end{align*}
	%
	por $\omega:\,X\mapsto (\omega X:\,p\mapsto\omega_{p}X_{p})$.
	Rec\'{\i}procamente, todo morfismo $T:\,%
	\champs{M}\rightarrow C^{\infty}(M)$ de $C^{\infty}(M)$-m\'{o}dulos
	determina una \'{u}nica $1$-forma suave $\omega\in\formes[1]{M}$
	tal que
	\begin{align*}
		\omega X & \,=\, TX
	\end{align*}
	%
	para todo campo suave $X\in\champs{M}$.
\end{teoUnoFormasSuavesComoMorfismos}

\subsection{El \emph{pullback} de una $1$-forma}
Sean $M$ y $N$ variedades diferenciales y sea $F:\,M\rightarrow N$ una
transformaci\'{o}n suave. Sea $\omega:\,N\rightarrow\tangente*{N}$ una
$1$-forma (no necesariamente continua) en $N$. Puntualmente, dado $p\in M$,
se defin\'{\i}a el pullback de $\omega_{F(p)}$ por $F$ en $p$ como el
covector en $\tangente*[p]{M}$ dado por
\begin{align*}
	\diferencial*[p]{F}(\omega_{F(p)})(v_{p}) & \,=\,
		\omega_{F(p)}(\diferencial[p]{F}(v_{p}))
	\text{ .}
\end{align*}
%
Esta construcci\'{o}n da lugar a una $1$-forma $\pull{F}\omega$ definida
en $M$ (si $\omega$ est\'{a} definida en un abierto $V\subset N$, entonces
$\pull{F}\omega$ queda definida en el abierto $F^{-1}(V)\subset M$):
dado un punto $p\in M$, $(\pull{F}\omega)|_{p}\in\tangente*[p]{M}$ es el
covector dado por
\begin{align*}
	(\pull{F}\omega)|_{p}(v_{p}) & \,=\,
		\omega_{F(p)}(\diferencial[p]{F}(v_{p})) \,=\,
	\text{ ,}
\end{align*}
%
es decir,
\begin{align*}
	(\pull{F}\omega)|_{p} & \,=\,\diferencial*[p]{F}(\omega_{F(p)})
	\text{ .}
\end{align*}
%

En el caso de un campo, no es posible definir el \emph{pushforward} del mismo,
salvo casos muy particulares. En cuanto a $1$-formas, no hay inconveniente
en definir el \emph{pullback} de una $1$-forma por una transformaci\'{o}n
suave. En definitiva, dada una transformaci\'{o}n suave
$F:\,M\rightarrow N$, queda definida una transformaci\'{o}n
$\pull{F}$ que toma una $1$-forma en $N$ y devuelve una $1$-forma en $M$
(en el sentido opuesto). Esta aplicaci\'{o}n es una transformaci\'{o}n lineal
ya que, bajando a un punto $p\in M$, se ve que
\begin{align*}
	(\pull{F}(a\omega+b\eta))|_{p} & \,=\,
		\diferencial*[p]{F}((a\omega+b\eta)|_{F(p)}) \,=\,
		a\diferencial*[p]{F}(\omega_{F(p)}) +
			b\diferencial*[p]{F}(\eta_{F(p)}) \,=\,
		a(\pull{F}\omega)|_{p} +b(\pull{F}\eta)|_{p}
	\text{ .}
\end{align*}
%
En definitiva,
\begin{align*}
	\pull{F}(a\omega+b\eta)=a\pull{F}\omega+b\pull{F}\eta
\end{align*}
%
para todo par de $1$-formas $\omega$ y $\eta$ y n\'{u}meros reales
$a$ y $b$.

\begin{lemaElPullbackEnUnoFormas}\label{thm:elpullbackenunoformas}
	Sea $F:\,M\rightarrow N$ una transformaci\'{o}n suave.
	Sea $u:\,N\rightarrow\bb{R}$ una funci\'{o}n y sea
	$\omega:\,N\rightarrow\tangente*{N}$ una $1$-forma en $N$.
	Entonces
	\begin{align*}
		\pull{F}(u\omega) & \,=\,(u\circ F)\,\pull{F}\omega
		\text{ .}
	\end{align*}
	%
	Si $u\in C^{\infty}(N)$, entonces
	\begin{align*}
		\pull{F}\derext{u} & \,=\,\derext{(u\circ F)}
		\text{ .}
	\end{align*}
	%
\end{lemaElPullbackEnUnoFormas}

\begin{proof}
	En las hip\'{o}tesis del enunciado del lema, puntualmente,
	\begin{align*}
		\big(\pull{F}(u\omega)\big)|_{p} & \,=\,
			u(F(p))\diferencial*[p]{F}(\omega_{F(p)}) \,=\,
			\big((u\circ F)\,\pull{F}\omega\big)|_{p}
		\text{ .}
	\end{align*}
	%
	Asumiendo que $u$ es suave, su diferencial est\'{a} bien definido y,
	dado $v\in\tangente[p]{M}$, vale que
	\begin{align*}
		(\pull{F}\derext{u})|_{p} & \,=\,
			\derext[F(p)]{u}\big(\diferencial[p]{F}(v)\big) \,=\
			\diferencial[p]{F}(v)u \\
		& \,=\,v(u\circ F) \,=\,\derext[p]{(u\circ F)}(v)
		\text{ .}
	\end{align*}
	%
\end{proof}

\begin{propoElPullbackDeUnaSuave}\label{thm:elpullbackdeunasuave}
	Sea $F:\,M\rightarrow N$ una transformaci\'{o}n suave y
	sea $\omega:\,N\rightarrow\tangente*{N}$ una $1$-forma.
	Entonces, si $\omega$ es continua, $\pull{F}\omega$ es una
	$1$-forma continua en $M$. Si $\omega$ es suave, entonces
	$\pull{F}\omega$ es una $1$-forma suave en $M$.
\end{propoElPullbackDeUnaSuave}

\begin{proof}
	Sea $p\in M$ un punto arbitrario de la variedad. Sea $(V,\psi)$
	una carta en $F(p)$ compatible con la estructura de $N$ y sea
	$U=F^{-1}(V)$. Por continuidad de $F$, la preimagen $U$ es abierta
	en $M$ y contiene al punto $p$. En coordenadas, y usando
	el lema \ref{thm:elpullbackenunoformas}, se deduce que
	\begin{align*}
		\pull{F}\omega & \,=\,\pull{F}\big(\omega_{j}\de{y^{j}}\big)
			\,=\,(\omega_{j}\circ F)\,\derext{(y^{j}\circ F)}
			\text{ .}
	\end{align*}
	%
	Ahora bien, $\omega$ es continua, si y s\'{o}lo si las componentes
	$\omega_{i}$ son continuas. Por otro lado, las funciones
	$y^{j}\circ F:\,U\rightarrow\bb{R}$ son suaves, por que la carta
	$(V,\psi)$ es compatible y $F$ es suave. Esto implica que las
	$1$-formas $\derext{(y^{j}\circ F)}$ (definidas en $U$) son
	suaves. En coordenadas,
	\begin{align*}
		\derext{(y^{j}\circ F)} & \,=\,
			\derivada{(y^{j}\circ F)}{x^{i}}\de{x^{i}}
		\text{ .}
	\end{align*}
	%
	Pero las funciones $\derivada{(y^{j}\circ F)}{x^{i}}$ no son otra
	cosa que las componentes de la matriz jacobiana de $F$ con respecto
	a estas coordenadas (donde est\'{e}n definidas) y, por lo tanto,
	son funciones suaves.
	
	En definitiva, si $\omega$ es continua, las componentes $\omega_{i}$
	son continuas y la $1$-forma en $M$ dada por
	$\pull{F}\omega$ se expresa localmente como combinaci\'{o}n
	de formas suaves por funciones continuas, con lo que resulta ser
	continua. Y si $\omega$ es suave, las componentes son suaves y
	el pullback resulta ser suave, tambi\'{e}n.
\end{proof}

Sabiendo que $\pull{F}\omega$ es una $1$-forma continua, si $\omega$ lo
es y que es suave, si $\omega$ lo es, se deduce que el pullback de $F$
es una transformaci\'{o}n lineal $\pull{F}:\,%
\formes[1]{N}\rightarrow\formes[1]{M}$. Adem\'{a}s, por
\ref{thm:elpullbackenunoformas}, respeta en cierto sentido las estructuras
de m\'{o}dulos sobre $C^{\infty}(M)$ y sobre $C^{\infty}(N)$, asignando, a
una funci\'{o}n (suave, continua, nada) $g$ definida en $N$, la funci\'{o}n
(suave, continua, nada, respectivamente) $f=g\circ F$ en $M$
(el pullback de $g$ por $F$).

\subsection{Integrar sobre segmentos}
Sea $[a,b]\subset\bb{R}$ un intervalo real compacto ($a$ y $b$ finitos).
Sea $\omega\in\formes[1]{[a,b]}$ una $1$-forma suave. Definiremos lo que
significa integrar la $1$-forma $\omega$ en/sobre el intervalo $[a,b]$ --o,
mejor dicho, mostraremos que la noci\'{o}n intuitiva tiene sentido.

Antes de pasar a integrar sobre el intervalo, aplicamos los resultados
generales sobre $1$-formas al caso particular de formas sobre esta
variedad. El intervalo, como variedad con borde se puede cubrir usando
dos cartas: $(\left[a,b\right),r:\,x\mapsto x-a)$ y
$(\left(a,b\right],s:\,x\mapsto -(x-b))$. Estas dos cartas son compatibles.
El cambio de coordenadas en $(a,b)$ est\'{a}, en uno de los dos sentidos,
dado por
\begin{align*}
	& x\,\mapsto\, x+a\,\mapsto\,-(x+a-b)= (b-a) - x
	\text{ ,}
\end{align*}
%
es decir, por invertir el intervalo $(0,b-a)$. Sea
$\omega\in\formes[1]{[a,b]}$. En cada una de estas cartas, $\omega$ se puede
escribir como un m\'{u}ltiplo, por una funci\'{o}n suave, de los
diferenciales de las funciones coordenadas respectivas:
\begin{align*}
	\omega(r+a) & \,=\, f(r)\,\derext{r}\quad\text{, si }a\leq r<b
	\quad\text{y} \\
	\omega(b-s) & \,=\,g(s)\,\derext{s}\quad\text{, si }a<s\leq b
	\text{ .}
\end{align*}
%
El cambio de coordenadas $s\circ r^{-1}(x)=F(x)=(b-a)-x$ es la
representaci\'{o}n de la funci\'{o}n identidad en $(a,b)$. El pullback
de $\omega$ por la identidad expresa la relaci\'{o}n entre las dos
representaciones de $\omega$. Por un lado,
\begin{align*}
	\pull{\id[(a,b)]}(g\,\derext{s}) & \,=\,
		g\circ\id[(a,b)]\cdot\big(\pull{\id[(a,b)]}\derext{s}\big)
	\text{ .}
\end{align*}
%
Para expresar $\pull{\id[(a,b)]}\derext{s}$ con respecto a la
$1$-forma $\derext{r}$, recurrimos a la definici\'{o}n:
\begin{align*}
	(\pull{\id}\derext{s})|_{x}\gancho[x]{r} & \,=\,
		\derext[\id(x)]{s}\cdot
		\diferencial[x]{\id}\Big(\gancho[x]{r}\Big)\\
	& \,=\,\derext[\id(x)]{s}\Big(v\cdot\gancho[\id(x)]{s}\Big) \,=\,v
	\text{ ,}
\end{align*}
%
donde
\begin{align*}
	v & \,=\,\diferencial[x]{\id}\Big(\gancho[x]{r}\Big)(s) \,=\,
		\gancho[x]{r}(s\circ\id)\,=\,(s\circ\id\circ r^{-1})'(r(x)) \\
	& \,=\,F'(r(x)) \,=\,-1
	\text{ .}
\end{align*}
%
En definitiva,
\begin{align*}
	f\,\derext{r} \,=\,\omega & \,=\,g\,\derext{s} \,=\,
		\pull{\id[(a,b)]} \big(g\,\derext{s}\big) \\
	& \,=\,	g\circ\id[(a,b)]\cdot\big(\pull{\id[(a,b)]}\derext{s}\big)
		\,=\,-g\,\derext{r}
\end{align*}
%
de lo que se deduce que $f=-g$ 

Volviendo a la cuesti\'{o}n de la integral, sea $\omega=f\,\derext{r}$ la
expresi\'{o}n de una $1$-forma suave en el abierto coordenado $[a,b)$.
El punto $b$ en el borde del intervalo es un \emph{conjunto de medida nula}.
Si bien esta noci\'{o}n no fue definida, pero es una propiedad deseable.
Entonces, para integrar $\omega$ en $[a,b]$, deber\'{\i}a, intuitivamente,
ser suficiente considerar $\omega$ restringida a $[a,b)$, en donde
la forma se puede representar como una funci\'{o}n por la forma
can\'{o}nica $\derext{r}$ proveniente de la carta $([a,b),r)$. Se define
\emph{la integral de $\omega$ en el inervalo $[a,b]$} como
\begin{align*}
	\int_{[a,b]}\,\omega & \,=\,\int_{0}^{b-a}\,f\circ r^{-1}(x)\,dx
	\text{ .}
\end{align*}
%
La integral as\'{\i} definida viene con un \emph{sentido de integraci\'{o}n},
se integra sobre $[a,b]$ de $a$ hacia $b$. Tal vez de manera un poco m\'{a}s
precisa \'{e}sta es la integral sobre un \emph{intervalo orientado}.
La integral $\int_{[a,b]}\,\omega$ est\'{a} dada por la integral de Riemann
de la representaci\'{o}n en coordenadas $f\circ r^{-1}$ de la funci\'{o}n
$f$ que representa a la forma $\omega$ en t\'{e}rminos de la forma
$\derext{r}$. En particular, $f\circ r^{-1}$ est\'{a} definida en
el intervalo abierto cerrado $[0,b-a)$.

Una \emph{reparametrizaci\'{o}n de $[a,b]$} es un difeomorfismo
$\varphi:\,[c,d]\rightarrow [a,b]$. Usando la f\'{o}rmula del pullback,
\begin{align*}
	\pull{\varphi}\omega & \,=\,\pull{\varphi}\big(f\,\derext{r}\big)
		\,=\,(f\circ\varphi)\cdot\pull{\varphi}\derext{r}
	\text{ .}
\end{align*}
%
Si $\varphi(c)=a$, $\varphi(d)=b$ (es decir, $\varphi$ es creciente, es
una \emph{reparametrizaci\'{o}n}) y en $[c,d)$ la coordenada est\'{a}
dada por $s:\,[c,d)\rightarrow [0,d-c)$ con $s(x)=x-c$, entonces
\begin{align*}
	\pull{(\varphi\circ s^{-1})}\derext{r} & \,=\,
		(r\circ\varphi\circ s^{-1})'\cdot\derext{x} \,=\,
		(\varphi(x+c)-a)'(x)\,\derext{x} \\
	& \,=\,\varphi'(x+c)\,\derext{x} \,=\,\varphi'\circ s^{-1}\,\derext{x}
	\text{ .}
\end{align*}
%
Expresado de manera m\'{a}s concisa,
\begin{align*}
	\pull{\varphi}\derext{r} & \,=\,\varphi'\,\derext{s}
	\quad\text{y} \\
	\pull{\varphi}\omega & \,=\,\pull{\varphi}\big(f\,\derext{r}\big) \,=\,
		(f\circ\varphi)\cdot\varphi'\,\derext{s}
	\text{ .}
\end{align*}
%
En particular, en $[c,d)$, la integral del pullback es igual a
\begin{align*}
	\int_{[c,d]}\,\pull{\varphi}\omega & \,=\,
		\int_{[c,d]}\,(f\circ\varphi)\cdot\varphi'\,\derext{s} \,=\,
		\int_{0}^{d-c}\,(f\circ\varphi\circ s^{-1})(x)\,
			(\varphi'\circ s^{-1})(x)\,dx
		\text{ .}
\end{align*}
%
Dado que $\frac{\varphi'\circ s^{-1}}{|\varphi'\circ s^{-1}|}=+1$ en
$[c,d]$, se deduce que
\begin{align*}
	\int_{[c,d]}\,\pull{\varphi}\omega & \,=\,
		\int_{0}^{b-a}\,(f\circ r^{-1})(y)\,dy \,=\,
		\int_{[a,b]}\,\omega
	\text{ .}
\end{align*}
%
Si, en cambio, $\varphi(c)=b$ y $\varphi(d)=a$ ($\varphi$ es decreciente),
entonces
\begin{align*}
	\int_{[c,d]}\,\pull{\varphi}\omega & \,=\,
		\int_{0}^{d-c}\,(f\circ\varphi\circ s^{-1})(x)\,
			(\varphi'\circ s^{-1})(x)\,dx \\
	& \,=\,-\int_{0}^{b-a}\,(f\circ r^{-1})(y)\,dy \,=\,
		-\int_{[a,b]}\,\omega
	\text{ ,}
\end{align*}
%
pues $\frac{\varphi'\circ s^{-1}}{|\varphi'\circ s^{-1}|}=-1$ en $[c,d]$,
en ese caso.

Todo esto es para dar sentido a una identificaci\'{o}n entre
$1$-formas diferenciables en un intervalo compacto $[a,b]\subset\bb{R}$ y
funciones de manera de que se pueda hablar de la integral de una forma
en t\'{e}rminos de la noci\'{o}n conocida de integral de una funci\'{o}n.
Tal vez haya una manera m\'{a}s obvia de hacer esta identificaci\'{o}n o
alguna manera natural.

Una vez definida la noci\'{o}n de integral sobre un intervalo, integrar en
variedades es algo m\'{a}s o menos directo.

\begin{obsVariedadSuaveATrozosConexa}\label{obs:variedadsuaveatrozosconexa}
	En una variedad conexa, todo par de puntos se puede conectar por
	una curva $C^{1}$ (suave) a trozos.
\end{obsVariedadSuaveATrozosConexa}

Sea $M$ una varieda diferencial y sea $\gamma:[a,b]\rightarrow M$ una
curva suave ($C^{1}$), es decir, una transformaci\'{o}n suave entre
variedades, o bien una funci\'{o}n que se extiende a un entorno del
intervalo en $\bb{R}$ de manera que se obtenga una transformaci\'{o}n
suave. Dada $\omega\in\formes[1]{M}$, la \emph{integral de $\omega$ sobre %
$\gamma$} se define como
\begin{align*}
	\int_{\gamma}\,\omega & \,\equiv\,\int_{[a,b]}\,\pull{\gamma}\omega
	\text{ .}
\end{align*}
%
Si $\gamma$ es suave a trozos, de manera que $\gamma$ sea suave en ciertos
subintervalos $[a_{i},a_{i+1}]$, entonces se define
\begin{align*}
	\int_{\gamma}\,\omega & \,\equiv\,\sum_{i}\,
		\int_{[a_{i},a_{i+1}]}\,\pull{\gamma_{i}}\omega
	\text{ ,}
\end{align*}
%
donde $\gamma_{i}=\gamma|_{[a_{i},a_{i+1}]}$.

\begin{obsIntegralEsLinealEnFormas}\label{obs:integraleslinealenformas}
	Dada $\gamma:\,[a,b]\rightarrow M$ suave a trozos, la integral
	sobre $\gamma$ define una funci\'{o}n lineal
	\begin{align*}
		\int_{\gamma} & \,:\,\formes[1]{M}\,\rightarrow\,\bb{R}
	\end{align*}
	%
	e $\int_{\gamma}=0$, si $\gamma$ es un camino constante.
\end{obsIntegralEsLinealEnFormas}

\begin{obsIntegralEnCaminoInverso}\label{obs:integralencaminoinverso}
	La relaci\'{o}n con reparametrizaciones se mantiene en este
	contexto, tambi\'{e}n: si $\tilde{\gamma}=\gamma\circ\varphi$
	es una reparametrizaci\'{o}n de $\gamma$, entonces
	\begin{align*}
		\int_{\tilde{\gamma}}\,\omega & \,=\,
			+\int_{\gamma}\,\omega\quad\text{o}\quad
			-\int_{\gamma}\,\omega
		\text{ ,}
	\end{align*}
	%
	de acuerdo a si $\tilde{\gamma}$ es una \emph{reparametrizaci\'{o}n %
	positiva}, es decir, si $\varphi$ es creciente, o si,
	respectivamente, $\tilde{\gamma}$ es una \emph{reparametrizaci\'{o}n %
	negativa}, es decir, $\varphi$ es decreciente.
\end{obsIntegralEnCaminoInverso}

\begin{obsIntegralEsLinealEnCaminos}\label{obs:integraleslinealencaminos}
	Extendiendo por linealidad, la aplicaci\'{o}n
	$\gamma\mapsto\int_{\gamma}\,\omega$ dada en curvas suaves a
	trozos, fijada la forma $\omega\in\formes[1]{M}$, determina una
	funcional lineal
	\begin{align*}
		\omega & \,:\,\cadenas[1]{M}\,\rightarrow\,\bb{R}
	\end{align*}
	%
	en el espacio de \emph{$1$-cadenas suaves}. M\'{a}s all\'{a} de esta
	extensi\'{o}n formal, integrar una forma respeta subdivisiones del
	intervalo: si $\gamma:\,[a,b]\rightarrow M$ y
	$\gamma_{1}=\gamma|_{[a,c]}$ y $\gamma_{2}=\gamma|_{[c,b]}$, entonces
	\begin{align*}
		\int_{\gamma}\,\omega & \,=\,\int_{\gamma_{1}}\,\omega
			+\int_{\gamma_{2}}\,\omega
		\text{ .}
	\end{align*}
	%
\end{obsIntegralEsLinealEnCaminos}

\begin{propoIntegrarElPullDeUnaCurva}\label{thm:integrarenelpulldeunacurva}
	Si $F:\,M\rightarrow N$ es suave y $\eta\in\formes[1]{N}$, dada
	$\gamma:\,[a,b]\rightarrow M$,
	\begin{align*}
		\int_{\gamma}\,\pull{F}\eta & \,=\,\int_{F\circ\gamma}\,\eta
		\text{ .}
	\end{align*}
	%
\end{propoIntegrarElPullDeUnaCurva}

\begin{proof}
	\begin{align*}
		\int_{\gamma}\,\pull{F}\eta & \,=\,
			\int_{[a,b]}\,\pull{\gamma}(\pull{F}\eta) \,=\,
			\int_{[a,b]}\,\pull{(F\circ\gamma)}\eta \,=\,
			\int_{F\circ\gamma}\,\eta
		\text{ .}
	\end{align*}
	%
\end{proof}

\begin{propoIntegrarConcretamenteSobreUnaCurva}%
	\label{thm:integrarconcretamentesobreunacurva}
	Si $\gamma:\,[a,b]\rightarrow M$ es una curva suave a trozos y
	$\omega\in\formes[1]{M}$,
	\begin{align*}
		\int_{\gamma}\,\omega & \,=\,
			\int_{a}^{b}\,\omega_{\gamma(t)}
				\big(\dot{\gamma}(t)\big)\,dt
	\end{align*}
	%
\end{propoIntegrarConcretamenteSobreUnaCurva}

\begin{proof}
	Por definici\'{o}n,
	\begin{align*}
		\int_{\gamma}\,\omega & \,=\,\int_{[a,b]}\,\pull{\gamma}\,=\,
			\int_{0}^{b-a}\,f\circ r^{-1}(x)\,dx
		\text{ ,}
	\end{align*}
	%
	donde $r:\,[a,b)\rightarrow\bb{R}$ es la funci\'{o}n coordenada
	$y\mapsto y-a$ y la integral es la integral de Riemann.
	Componiendo con la traslaci\'{o}n, $x\mapsto x+a$, se obtiene
	la igualdad
	\begin{align*}
		\int_{0}^{b-a}\,f\circ r^{-1}(x)\,dx & \,=\,
			\int_{a}^{b}\,f(t)\,dt
		\text{ .}
	\end{align*}
	%
	Esto corresponde a tomar la ``carta de borde''
	$\id:\,[a,b)\rightarrow[a,b)$, en lugar de
	$r:\,[a,b)\rightarrow[0,b-a)$ que, si bien es lo correcto por
	definici\'{o}n, no es muy intuitivo.

	Ahora bien, como no hay, propiamente, una carta global en
	$[a,b]$ --aunque $\gamma$ se podr\'{\i}a extender a un intervalo
	abierto, de forma que s\'{\i} la haya-- no es posible argumentar de
	la siguiente manera:
	\begin{align*}
		(\pull{\gamma}\omega)_{t}\gancho[t]{t} & \,=\,
			\omega_{\gamma(t)}\Big(\diferencial[t]{\gamma}
				\gancho[t]{t}\Big) \,=\,
			\omega_{\gamma(t)}\dot{\gamma}(t)
		\text{ .}
	\end{align*}
	%
	De todas maneras, tomando coordenadas en $M$,
	\begin{align*}
		\pull{\gamma}\omega & \,=\,(\omega_{j}\circ\gamma)\cdot
			\derext{\gamma^{j}} \,=\,\omega_{j}\circ\gamma\cdot
			(\gamma^{j})'\,\derext{t} \\
		& \,=\,\omega_{\gamma(t)}\,\dot{\gamma}(t)\,\derext{t}
		\text{ .}
	\end{align*}
	%
	Por otro lado, porque $[a,b]$ es compacto, se puede subdividir el
	intervalo de manera que cada subintervalo de la divisi\'{o}n est\'{e}
	contenido en la preimagen del dominio de una carta. Aplicando el
	argumento anterior a cada subintervalo y usando la propiedad
	de que $\omega$ respeta subdivisiones,
	\begin{align*}
		\int_{\gamma}\,\omega & \,=\,\sum_{i}\,
			\int_{\gamma_{i}}\,\omega \,=\,\sum_{i}\,
			\int_{a_{i}}^{a_{i+1}}\,\omega_{\gamma_{i}(t)}\,
				\dot{\gamma_{i}}(t)\,dt
		\text{ .}
	\end{align*}
	%
	Si $\gamma$ es suave a trozos, se aplica el argumento en cada
	subintervalo en donde $\gamma$ es suave y se suma como antes.
\end{proof}

\begin{obsIntegrarFormasExactas}\label{obs:integrarformasexactas}
	Si $\omega$ es el diferencial de una funci\'{o}n suave,
	$\omega=\derext{f}$, entonces
	\begin{align*}
		\int_{\gamma}\,\derext{f} & \,=\,\int_{[a,b]}\,
			\pull{\gamma}(\derext{f}) \,=\,
			\int_{a}^{b}\,\dot{\gamma}(t)\,f\,dt \\
		& \,=\,\int_{a}^{b}\,(f\circ\gamma)'(t)\,dt \,=\,
			\sum_{i}\,f(\gamma(a_{i+1}))-f(\gamma(a_{i})) \\
		& \,=\,f(\gamma(b))-f(\gamma(a))
		\text{ .}
	\end{align*}
	%
\end{obsIntegrarFormasExactas}

\begin{obsDiferencialCeroConstante}\label{obs:diferencialceroconstante}
	Si $M$ es una variedad conexa y $f\in C^{\infty}(M)$,
	entonces $\derext{f}=0$ implica que $f$ es constante.
\end{obsDiferencialCeroConstante}

Una $1$-forma $\omega\in\formes[1]{M}$ se dice \emph{exacta}, si existe
$f\in C^{\infty}(M)$ tal que $\omega=\derext{f}$. Si $f,g\in C^{\infty}(M)$
son funciones tales que $\derext{f}=\derext{g}$, entonces $\derext{(f-g)}=0$,
por linealidad, y $f-g$ es constante en cada componente conexa de la variedad.

\begin{obsIntegrarExactasEnCaminosCerrados}%
	\label{obs:integrarexactasencaminoscerrados}
	Dada una funci\'{o}n suave $f\in C^{\infty}(M)$, la integral
	$\int_{\gamma}\,\derext{f}$ es cero para toda curva suave a trozos
	$\gamma$ tal que $\gamma(a)=\gamma(b)$. Tales curvas se denominan
	\emph{cerradas}.
\end{obsIntegrarExactasEnCaminosCerrados}

Una $1$-forma se dice \emph{conservativa}, si $\int_{\gamma}\,\omega=0$ para
todo camino suave a trozos cerrado $\gamma$. Un camino cerrado suave a trozos,
no necesariamente es un ``loop suave'' (una transformaci\'{o}n suave
$\esfera[1]\rightarrow M$).

\begin{obsConservativaSiiIntegralIndependienteDelCamino}%
	\label{obs:conservativasiiintegralindependientedelcamino}
	Una $1$-forma $\omega\in\formes[1]{M}$ es conservativa, si y
	s\'{o}lo si $\int_{\gamma}\,\omega=\int_{\delta}\,\omega$ para todo
	par de curvas suaves a trozos tales que empiecen en el mismo
	punto y terminen en el mismo punto en $M$.
\end{obsConservativaSiiIntegralIndependienteDelCamino}

\begin{teoUnoFormaConservativaSiiExacta}%
	\label{thm:unoformaconservativasiiexacta}
	Sea $M$ una variedad diferencial. Sea $\omega\in\formes[1]{M}$ una
	$1$-forma suave. Entonces $\omega$ es conservativa, si y s\'{o}lo
	si es exacta.
\end{teoUnoFormaConservativaSiiExacta}


%

%--------

\chapter{Ejemplos}
\section{Generalidades}
%



\begin{ejemplo}\nom{El gr\'{a}fico de una funci\'{o}n}
	Sea $U\subset\bb{R}^{n}$ un subconjunto abierto y sea
	$f:\,U\rightarrow\bb{R}^{k}$ una funci\'{o}n \emph{continua}. El
	\emph{gr\'{a}fico de $f$} es el subconjunto de
	$\bb{R}^{n}\times\bb{R}^{k}$ definido por
	\begin{align*}
		\graf{f} & \,=\,\big\lbrace (x,y)\in\bb{R}^{n}\times\bb{R}^{k}
			\,:\,x\in U,\,y=f(x)\big\rbrace
		\text{ .}
	\end{align*}
	%
	A este subconjunto se le da la topolog\'{\i}a de subespacio del
	producto. Sea $\pi_{1}:\,\bb{R}^{n}\times\bb{R}^{k}\rightarrow%
	\bb{R}^{n}$ la proyecci\'{o}n en el primer factor y sea
	$\varphi:\,\graf{f}\rightarrow U$ la restricci\'{o}n de $\pi_{1}$
	al gr\'{a}fico de $f$. La aplicaci\'{o}n $\varphi$ es continua,
	siendo la restricci\'{o}n de una funci\'{o}n continua a un subespacio;
	su imagen es el abierto $U$ de $\bb{R}^{n}$ y
	$\varphi^{-1}(x)=(x,f(x))$ es una inversa para $\varphi$ definida en
	$U$. En definitiva, $\graf{f}$ es homeomorfo a $U$ v\'{\i}a
	$\varphi$ y el par $(\graf{f},\varphi)$ es una carta global
	para $\graf{f}$ que hace del mismo un espacio localmente euclideo
	de dimensi\'{o}n $n$. Como $\graf{f}$ es $T_{2}$ y $N_{2}$, por
	ser subespacio de $\bb{R}^{n}\times\bb{R}^{k}$, resulta ser una
	variedad topol\'{o}gica de dimensi\'{o}n $n$, tambi\'{e}n.
\end{ejemplo}

\begin{ejemplo}\nom{Espacios euclideos}
	Para cada entero $n\geq 0$, $\bb{R}^{n}$ es una variedad diferencial
	de dimensi\'{o}n $n$. Su estructura suave est\'{a} determinada por el
	atlas trivial $\{(\bb{R}^{n},\id[\bb{R}^{n}])\}$ que consta de una
	\'{u}nica carta. Esta estructura diferencial en $\bb{R}^{n}$ se
	denominar\'{a} \emph{estructura usual} de $\bb{R}^{n}$ o
	\emph{coordenadas usuales}. Vale la pena observar que las cartas
	compatibles con esta estructura son, precisamente, los pares
	$(U,\varphi)$ con $U$ es abierto (en la topolog\'{\i}a usual) y
	$\varphi:\,U\rightarrow\bb{R}^{n}$ es difrenciable en el sentido
	usual, tambi\'{e}n.
\end{ejemplo}

\begin{ejemplo}\nom{Subvariedades abiertas}
	Sea $U$ un subconjunto abierto de $\bb{R}^{n}$. Entonces $U$ es una
	variedad topol\'{o}gica de dimensi\'{o}n $n$ y la carta
	$(U,\id[U])$ define una estructura suave en $U$. En general, sea
	$M$ es una variedad diferencial y sea $U$ un abierto de $M$.
	Sea $\cal{A}$ la colecci\'{o}n de todas las cartas suaves para
	$M$, es decir, el atlas maximal que define la estructura de $M$
	como variedad diferencial. La colecci\'{o}n
	\begin{align*}
		\cal{A}_{U} & \,=\,\left\lbrace
			(V,\varphi)\in\cal{A}\,:\,V\subset U\right\rbrace
	\end{align*}
	%
	es un atlas de $U$ que es, adem\'{a}s, suavemente compatible.
	Este atlas determina naturalmente una estructura diferencial
	sobre $U$. Los subconjuntos abiertos de una variedad $M$ son
	de manera natural variedades diferenciales que denominaremos
	\emph{subvariedades abierta} de $M$.
\end{ejemplo}

\begin{ejemplo}\nom{El gr\'{a}fico de una funci\'{o}n diferenciable}
	Sea $U\subset\bb{R}^{n}$ un abierto y sea $f:\,U\rightarrow\bb{R}^{k}$
	una funci\'{o}n diferenciable. El gr\'{a}fico de $f$ es una variedad
	topol\'{o}gica, dado que $f$ es continua. Dado que, adem\'{a}s, el
	gr\'{a}fico de $f$, $\graf{f}$, se puede cubrir con una \'{u}nica
	carta, $(\graf{f},\varphi)$, donde $\varphi:\,\graf{f}\rightarrow U$
	es la restricci\'{o}n de la proyecci\'{o}n en la primer coordenada,
	el gr\'{a}fico de una funci\'{o}n suave tiene una estructura suave
	de manera can\'{o}nica.
\end{ejemplo}

\begin{ejemplo}\nom{Las proyecciones son submersiones}
	Sean $M_{1},\,\dots,\,M_{k}$ variedades diferenciales y sea
	$M$ el producto de todas ellas. Cada una de las proyecciones
	$\pi_{i}:\,M\rightarrow M_{i}$ es una submeris\'{o}n. Si tomamos
	cartas $(U_{i},\varphi_{i})$ para cada \'{\i}ndice $i$, entonces
	\begin{align*}
		\widehat{\pi_{i}} & \,=\,
		\varphi_{i}\circ\pi_{i}\circ
			(\varphi_{1}\times\,\cdots\,\times\varphi_{k})^{-1}
			(\lista{x}{k})
			\,=\,x_{i}
		\text{ .}
	\end{align*}
	%
	La matriz jacobiana de la representaci\'{o}n en coordenadas
	$\widehat{\pi_{i}}$ es
	\begin{align*}
		\jacobiana[x_{i}]{\widehat{\pi_{i}}} & \,=\,
		\sbox0{$
			\begin{matrix}
				1 & & \\
				& \ddots & \\
				& & 1
			\end{matrix}
		$}
		\sbox1{$
			\begin{matrix}
				\ddots
			\end{matrix}
		$}
		\left[
		\begin{array}{ccc}
			\vphantom{\usebox{0}}\makebox[\wd0]{} &
			\usebox{0} &
			\vphantom{\usebox{0}}\makebox[\wd0]{}
		\end{array}
		\right]
	\end{align*}
	%
	que es sobreyectiva.
\end{ejemplo}

\begin{ejemplo}\nom{Una curva con velocidad nunca nula es inmersi\'{o}n}
	Sea $\gamma:\,J\rightarrow M$ una curva suave en una variedad $M$.
	Una condici\'{o}n necesaria y suficiente para que $\gamma$
	esa una inmersi\'{o}n es que su velocidad $\diferencial[t]{\gamma}=%
	\dot{\gamma}(t)$ sea distinta de cero en todo instante $t\in J$.
\end{ejemplo}

\begin{ejemplo}\nom{La proyecci\'{o}n desde el fibrado tangente}
	Sea $M$ una variedad diferencial y sea $TM$ su fibrado tangente.
	Sea $\pi:\,TM\rightarrow M$ la proyecci\'{o}n can\'{o}nica
	$\pi:\,(p,v_{p})\mapsto p$. Tomando coordenadas $(U,\varphi)$ en $M$ y
	las coordenadas correspondientes $(\widetilde{U},\widetilde{\varphi})$
	en $TM$,
	\begin{align*}
		\widehat{\pi}(\lista*{x}{n},\,\lista*{v}{n}) & \,=\,
			(\lista*{x}{n})
	\end{align*}
	%
	y su matriz jacobiana es igual a
	\begin{align*}
		\jacobiana[(x^{i},v^{i})]{\widehat{\pi}} & \,=\,
		\sbox0{$
			\begin{matrix}
				1 & & \\
				& \ddots & \\
				& & 1
			\end{matrix}
		$}
		\left[
		\begin{array}{cc}
			\usebox{0} & \makebox[\wd0]{}
		\end{array}
		\right]
		\text{ .}
	\end{align*}
	%
\end{ejemplo}

\begin{ejemplo}\nom{El toro parametrizado}
	Sea $X:\,\bb{R}^{2}\rightarrow\bb{R}^{3}$ la funci\'{o}n dada por
	\begin{align*}
		X(u,v) & \,=\,\big(
		(2+\cos 2\pi u)\,\cos 2\pi v,\,
		(2+\cos 2\pi u)\,\sin 2\pi v,\,
		\sin 2\pi u\big)
		\text{ .}
	\end{align*}
	%
	Su matriz jacobiana est\'{a} dada por
	\begin{align*}
		\jacobiana[(u,v)]{X} & \,=\,
			\begin{bmatrix}
				-2\pi\,(\sin 2\pi u)\,(\cos 2\pi v) &
					-2\pi\,(2+\cos 2\pi u)\,\sin 2\pi v \\
				-2\pi\,(\sin 2\pi u)\,(\sin 2\pi v) &
					2\pi\,(2+\cos 2\pi u)\,\cos 2\pi v \\
				2\pi\,\cos 2\pi u & 0
			\end{bmatrix}
		\text{ .}
	\end{align*}
	%
	El determinante de la submatriz superior es igual a
	$(-4\pi^{2})(\sin 2\pi u)(2+\cos 2\pi u)$ que es distinto de cero,
	salvo en $u\in\frac{1}{2}\bb{Z}$. Pero para estos valores de $u$,
	las columnas de $\jacobiana[(u,v)]{X}$ son linealmente
	independientes. Entonces $X$ es una inmersi\'{o}n.
\end{ejemplo}


\section{Espacios vectoriales y grupos de matrices}
%



\begin{ejemplo}\nom{Espacios vectoriales}
	Sea $E$ un espacio vectorial topol\'{o}gico real. Si $E$ es de
	dimensi\'{o}n finita $n$, equivalentemente, si $E$ es
	localmente compacto, entonces $E$ es isomorfo a $\bb{R}^{n}$ y
	todo isomorfismo \emph{lineal} de $E$ en $\bb{R}^{n}$ es un
	homeomorfismo. Es decir, $E$ tiene una \'{u}nica estructura
	de espacio topol\'{o}gico que hace que las operaciones en tanto
	espacio vectorial real sean continuas. Todo isomorfismo
	$E\rightarrow\bb{R}^{n}$ est\'{a} dado por elegir una base: dada
	una base $\{\lista{\varepsilon}{n}\}$ de $E$, sea
	$\varepsilon:\,\bb{R}^{n}\rightarrow E$ la aplicaci\'{o}n
	\begin{align*}
		\varepsilon(x) & \,=\,\sum_{i=1}^{n}\,x^{i}\varepsilon_{i}
			\,\equiv\,x^{i}\varepsilon_{i}
		\text{ .}
	\end{align*}
	%
	Seg\'{u}n lo mencionado anteriormente, $\varepsilon$ es un
	homeomorfismo y, entonces, el par $(E,\varepsilon^{-1})$ es una
	carta (continua) para $E$. Si ahora $\tilde{\varepsilon}$ denota
	el isomorfismo correspondiente a otra base
	$\{\lista{\tilde{\varepsilon}}{n}\}$, entonces existe una matriz
	invertible $\left[A_{i}^{j}\right]^{i}_{j}$ tal que
	\begin{align*}
		\varepsilon_{i} & \,=\,A_{i}^{j}\tilde{\varepsilon}_{j}
	\end{align*}
	%
	para cada $i$. El cambio de cartas (cambio de coordenadas)
	correspondiente est\'{a} dado por
	$\tilde{\varepsilon}^{-1}\circ\varepsilon(x)=\tilde{x}$,
	donde $\tilde{x}=(\lista*{\tilde{x}}{n})$ es el punto de
	$\bb{R}^{n}$ dado por
	\begin{align*}
		\tilde{x}^{j}\tilde{\varepsilon}_{j} & \,=\,
			x^{i}\varepsilon_{i}
			\,=\,x^{i}A_{i}^{j}\tilde{\varepsilon}_{j}
		\text{ .}
	\end{align*}
	%
	Es decir, para cada $j$, $\tilde{x}^{j}=A_{i}^{j}x^{i}$. En
	particular, la aplicaci\'{o}n $x\mapsto\tilde{x}$, el cambio de
	coordenadas (definido globalmente), es lineal e invertible y,
	en particular, un difeomorfismo. En definitiva, las cartas de la
	forma $(E,\varepsilon^{-1})$, donde
	$\varepsilon:\,\bb{R}^{n}\rightarrow E$ es el isomorfismo
	determinado por la base $\{\lista{\varepsilon}{n}\}$ de $E$,
	son todas (suavemente) compatibles. La estructura que estas cartas
	determinan en $E$ se denominar\'{a} la \emph{estructura usual} en $E$.
\end{ejemplo}

\begin{ejemplo}\nom{El espacio de matrices}
	El espacio de matrices $\MM{m\times n,\bb{R}}$ de tama\~{n}o
	$m\times n$ con coeficientes reales es un espacio vectorial de
	dimensi\'{o}n $mn$ y, por lo tanto, una variedad diferencial con
	su estructura usual. El espacio de matrices $\MM{m\times n,\bb{C}}$
	complejas constituye una variedad diferencial de dimensi\'{o}n $2mn$,
	pues su dimensi\'{o}n como espacio vectorial sobre $\bb{R}$ es $2mn$.
	En el caso de matrices cuadradas de tama\~{n}o $n\times n$ usaremos
	la notaci\'{o}n $\MM{n,\cdot}$.
\end{ejemplo}

\begin{ejemplo}\nom{El grupo general lineal}
	El \emph{grupo general lineal (real)}, denotado $\GL{n,\bb{R}}$
	es el conjunto de matrices invertibles con coeficientes reales.
	Dado que la funci\'{o}n $\det:\,\MM{n,\bb{R}}\rightarrow\bb{R}$
	es continua y que $\GL{n,\bb{R}}$ es el subconjunto de matrices
	con determinante no nulo, $\GL{n,\bb{R}}\subset\MM{n,\bb{R}}$ es
	un subconjunto abierto y, por lo tanto, una variedad diferencial
	de dimensi\'{o}n $\dim(\MM{n,\bb{R}})=n^{2}$.
\end{ejemplo}

\begin{ejemplo}\nom{Matrices de rango $\geq k$}
	Sea $k\geq 0$. De manera an\'{a}loga al ejemplo anterior, en
	$\MM{m\times n,\bb{R}}$, las matrices de rango mayor o igual a $k$
	forman un subconjunto abierto: una matriz tiene rango al menos $k$,
	si el determinante de alguna submatriz es distinto de cero. Por
	continuidad del determinante, existe un entorno de una tal matriz
	que verifica que, para todas las matrices de dicho abierto, el
	determinante de la misma submatriz no se anula. En definitiva, todas
	las matrices del abierto tienen rango al menos $k$. Un caso particular
	de esto es cuando $k$ es m\'{a}ximo, es decir, $k=\min\{m,n\}$.
\end{ejemplo}

\begin{ejemplo}\nom{Transformaciones lineales}
	Sean $E$ y $F$ dos espacios vectoriales de dimensi\'{o}n finita
	y sea $\lineal{E,F}$ el conjunto de transformaciones lineales
	de $E$ en $F$. Si a $E$ y $F$ se los considera espacios vectoriales
	topol\'{o}gicos reales, entonces $\lineal{E,F}$ coincide con
	el conjunto de transformaciones lineales y continuas. En general,
	como $\lineal{E,F}$ es un espacio vectorial (real) de dimensi\'{o}n
	finita, es una variedad diferencial. Eligiendo bases de $E$ y de $F$,
	se puede representar un elemento $T\in\lineal{E,F}$ como una matriz,
	lo cual determina un isomorfismo
	$\lineal{E,F}\simeq\MM{m\times n,\bb{R}}$.
\end{ejemplo}

\begin{ejemplo}\nom{El grupo lineal especial}
	El subgrupo $\SL{n,\bb{R}}$ de matrices reales de tama\~{n}o
	$n\times n$ invertibles de determinante $1$ es una variedad
	topol\'{o}gica con la topolog\'{\i}a de subespacio de
	$\GL{n,\bb{R}}$. Su dimensi\'{o}n es $n^{2}-1$. Se le puede
	dar una estructura de variedad diferencial de manera que
	la inclusi\'{o}n $\SL{n,\bb{R}}\hookrightarrow\GL{n,\bb{R}}$
	sea una inmersi\'{o}n. En consecuencia, el grupo lineal especial
	resulta una subvariedad regular de $\GL{n,\bb{R}}$.
	De manera an\'{a}loga, $\SL{n,\bb{C}}$ es una variedad topol\'{o}gica,
	si se le da la topolog\'{\i}a de subespacio de $\GL{n,\bb{C}}$ y
	se le puede dar una estructura de variedad diferencial de manera
	que sea una subvariedad regular del grupo general lineal.
\end{ejemplo}

\begin{ejemplo}\nom{El grupo de matrices ortogonales}
	El grupo $\ortogonal{n}$ de matrices \emph{ortogonales} en
	$\MM{n,\bb{R}}$ tambi\'{e}n es una variedad topol\'{o}gica,
	visto como subespacio de $\MM{n,\bb{R}}$. Se le puede dar una
	estructura de variedad diferencial de forma que resulte una
	subvariedad regular.
\end{ejemplo}


\section{Esferas y espacios proyectivos}
%



\begin{ejemplo}\nom{La esfera}
	Sea $n\geq 0$ y sea $\esfera{n}$ la esfera de dimensi\'{o}n $n$ en
	$\bb{R}^{n+1}$. Por ser subespacio de$\bb{R}^{n+1}$, es un espacio
	$T_{2}$ y $N_{2}$. Para cada $i\in[\![1,n+1]\!]$ sea $U_{i}^{+}$ el
	subconjunto de $\bb{R}^{n+1}$ definido por
	\begin{align*}
		U_{i}^{+} & \,=\,\big\lbrace (\lista*{x}{n+1})\,:\,x^{i}>0
			\big\rbrace
		\text{ .}
	\end{align*}
	%
	De manera an\'{a}loga, se define $U_{i}^{-}$ como el subconjunto en
	donde la coordenada $x^{i}$ es negativa (estrictamente). Si se
	define una funci\'{o}n $f:\,\bola{1}{0}\rightarrow\bb{R}$ por
	\begin{align*}
		f(u) & \,=\,\sqrt{1-|u|^{2}}\text{ ,}
	\end{align*}
	%
	entonces $f$ es continua y, para cada $i$, el subconjunto
	$U_{i}^{+}\cap\esfera{n}$ de la esfera es igual al gr\'{a}fico de
	la funci\'{o}n
	\begin{align*}
		x^{i} & \,=\,f(x^{1},\,\dots,\,\widehat{x^{i}},\,\dots,\,
			x^{n+1})
	\end{align*}
	%
	Similarmente, $U_{i}^{-}\cap\esfera{n}$ es el gr\'{a}fico de
	$x^{i}=-f(x^{1},\,\dots,\,\widehat{x^{i}},\,\dots,\,x^{n+1})$. De
	esta manera, se ve cada abierto $U_{i}^{\pm}\cap\esfera{n}$ de la
	esfera es localmente euclideo de dimensi\'{o}n $n$, por ser
	gr\'{a}ficos de funciones continuas, y que las coordenadas
	$\varphi_{i}^{\pm}:\,U_{i}^{\pm}\cap\esfera\rightarrow\bola{1}{0}$
	dadas por
	\begin{align*}
		\varphi_{i}^{\pm}(\lista*{x}{n+1}) & \,=\,
			(x^{1},\,\dots,\,\widehat{x^{i}},\,\dots,\,x^{n+1})
	\end{align*}
	%
	son las coordenadas correspondientes. Dado que los dominios de estas
	cartas $(U_{i}^{\pm}\cap\esfera{n},\varphi_{i}^{\pm})$ cubren a
	$\esfera{n}$, se deduce que $\esfera{n}$ es una variedad
	topol\'{o}gica de dimensi\'{o}n $n$.

	Otro juego de cartas que tambi\'{e}n aparece con $\esfera{n}$ es
	el formado por las proyecciones estereogr\'{a}ficas. Sea $N$ el
	punto $(0,\,\dots,\,0,\,1)\in\esfera{n}\subset\bb{R}^{n+1}$ y
	sea $S=(0,\,\dots,\,0,\,-1)$ su ant\'{\i}poda. Sea
	$\sigma:\,\esfera{n}\setmin\{N\}\rightarrow\bb{R}^{n}$ la
	funci\'{o}n definida por
	\begin{align*}
		\sigma(\lista*{x}{n+1}) & \,=\,
			\frac{(\lista*{x}{n})}{1-x^{n+1}}
	\end{align*}
	%
	y sea $\tilde{\sigma}:\,\esfera{n}\setmin\{S\}\rightarrow\bb{R}^{n}$
	la funci\'{o}n
	\begin{align*}
		\tilde{\sigma}(x) & \,=\,-\sigma(-x) \,=\,
			\frac{(\lista*{x}{n})}{1+x^{n+1}}
		\text{ .}
	\end{align*}
	%
	Llamemos $H=\{x^{n+1}=0\}$. Entonces $\sigma(x)=u$, donde
	$(u,0)$ es el punto en que la recta que pasa por $N$ y por $x$
	interseca a $H$. De manera an\'{a}loga, $\tilde{\sigma}(x)=u$, donde
	$(u,0)$ es el punto en donde la recta que pasa por $S$ y por $x$
	interseca a $H$. La funci\'{o}n $\sigma$ es invertible con inversa
	\begin{align*}
		\sigma^{-1}(\lista*{u}{n}) & \,=\,
			\frac{(\lista*{2u}{n},\,|u|^{2}-1)}{|u|^{2}+1}
		\text{ .}
	\end{align*}
	%
	La composici\'{o}n $\tilde{\sigma}\circ\sigma^{-1}$ en un punto
	$u=(\lista*{u}{n})$ es igual a
	\begin{align*}
		\tilde{\sigma}(\sigma^{-1}(u)) & \,=\,
			-\sigma(-\sigma^{-1}(u)) \\
		&\,=\,\frac{(\lista*{2u}{n})}{|u|^{2}+1}\cdot
			\frac{1}{1+(\frac{|u|^{2}-1}{|u|^{2}+1})} \\
		& \,=\, \frac{1}{|u|^{2}}u
		\text{ .}
	\end{align*}
	%
\end{ejemplo}

\begin{ejemplo}\nom{El espacio proyectivo real}
	El espacio proyectivo real $\proyectivo{\bb{R}}{n}$ es una variedad
	de dimensi\'{o}n $n$. Una realizaci\'{o}n posible de este espacio es
	en tanto el conjunto de subespacios vectoriales de dimensi\'{o}n $1$
	(subvariedades lineales de dimensi\'{o}n $1$) en $\bb{R}^{n+1}$. La
	topolog\'{\i}a de este espacio es la topolog\'{\i}a cociente
	determinada por la aplicaci\'{o}n sobreyectiva
	$\pi:\,\bb{R}^{n+1}\setmin\{0\}\rightarrow\proyectivo{\bb{R}}{n}$
	que a un punto $x\not =0$ le asigna el subespacio que genera,
	denotado $[x]$.

	Para cada $i\in[\![1,n+1]\!]$, sea $\widetilde{U}_{i}$ el
	subconjunto de $\bb{R}^{n+1}\setmin\{0\}$ donde la coordenada
	$x^{i}$ es no nula. Sea $U_{i}=\pi(\widetilde{U}_{i})$ la
	proyecci\'{o}n correspondiente en el espacio proyectivo. Como
	$\widetilde{U}_{i}$ es un abierto saturado, el subconjunto $U_{i}$
	es abierto y la restricci\'{o}n de $\pi$ a $\widetilde{U}_{i}$ es
	una aplicaci\'{o}n cociente sobre $U_{i}$. Sea
	$\varphi_{i}:\,U_{i}\rightarrow\bb{R}^{n}$ la aplicaci\'{o}n dada por
	\begin{align*}
		\varphi_{i}[x^{1}\,:\cdots:\,x^{n+1}] & \,=\,
			\Big(\frac{x^{1}}{x^{i}},\,\dots,\,
			\frac{x^{i-1}}{x^{i}},\,\frac{x^{i+1}}{x^{i}},
			\,\dots,\,\frac{x^{n+1}}{x^{i}}
			\Big)
		\text{ .}
	\end{align*}
	%
	Dado que la composici\'{o}n $\varphi\circ\pi$ es continua (de
	hecho podemos ir adelantando que es suave entre un abierto de
	$\bb{R}^{n+1}$ y un abierto de $\bb{R}^{n}$), la funci\'{o}n
	$\varphi_{i}$ es continua, por la propiedad caracter\'{\i}stica
	del cociente (de $\pi|_{\widetilde{U}_{i}}$). Esta funci\'{o}n
	tiene una inversa dada por
	\begin{align*}
		\varphi_{i}^{-1}(\lista*{u}{n}) & \,=\,
			[u^{1}\,:\dots:\,u^{i-1}\,:\,1\,:\,
			u^{i}\,:\dots:\,u^{n}]
		\text{ .}
	\end{align*}
	%
	Esta inversa tambi\'{e}n es continua y $\varphi$ es un
	homeomorfismo. La interpretaci\'{o}n geom\'{e}trica de $\varphi[x]=u$
	es que el punto $(u,1)$ es aquel por donde $[x]$ cruza al hiperplano
	$x^{i}=1$. Dado que los abiertos $U_{1},\,\dots,\,U_{n+1}$ cubre al
	espacio proyectivo, $\proyectivo{\bb{R}}{n}$ es localmente euclideo
	de dimensi\'{o}n $n$.

	Para ver que los espacios proyectivos reales son espacios Hausdorff,
	dados dos subespacios lineales de dimensi\'{o}n $1$ distintos,
	$\xi,\upsilon$, en $\bb{R}^{n+1}$, existen abiertos
	\emph{c\'{o}nicos} disjuntos (el punto $\{0\}$ se omite) cada uno
	de los cuales contiene a uno de ellos, es decir, a las
	l\'{\i}neas. Estos abiertos son saturados respecto de la
	suryecci\'{o}n natural $\pi$ y por lo tanto sus im\'{a}genes son
	abiertos disjuntos de $\proyectivo{\bb{R}}{n}$ que contienen a
	cada uno de las clases $\xi$ y $\upsilon$. Para que quede un poco
	m\'{a}s claro, podemos restringir $\pi$ a la esferera
	$\esfera{n}\subset\bb{R}^{n+1}\setmin\{0\}$. La restricci\'{o}n
	sigue siendo sobreyectiva y cociente. Dada una clase
	$\xi\in\proyectivo{\bb{R}}{n}$, podemos tomar como representante
	a cualquiera de los dos puntos de norma $1$ en $\xi\cap\esfera{n}$.
	Llamemos $x$ a dicho punto. De manera similar podr\'{\i}amos tomar
	el punto $-x$. Si $y\in\esfera{n}$ es un punto distinto de $x$ y de
	$-x$, existen abiertos disjuntos $\widetilde{U}$ y $\widetilde{V}$
	que contienen a $x$ y a $y$. Los abiertos $-\widetilde{U}$ y
	$-\widetilde{V}$ contienen a los puntos $-x$ y $-y$ y, tomando
	abiertos m\'{a}s peque\~{n}os de ser necesario, podemos suponera
	que los cuatro abiertos son disjuntos de a pares. De esta manera,
	se obtienen abiertos saturados
	$\widetilde{U}\cup\big(-\widetilde{U}\big)$ y
	$\widetilde{V}\cup\big(-\widetilde{V}\big)$ que contienen a $x$ y a
	$y$ y que, adem\'{a}s, son disjuntos. Proyectando, se obtienen
	abiertos disjuntos $U$ y $V$ de $\proyectivo{\bb{R}}{n}$ que
	contienen a $x$ y a $y$, respectivamente.

	Siendo imagen por una funci\'{o}n continua de un espacio compacto
	$\pi(\esfera{n})=\proyectivo{\bb{R}}{n}$, el espacio proyectivo
	es compacto para todo $n\geq 0$. Para ver que
	$\proyectivo{\bb{R}}{n}$ es $N_{2}$, alcanza con notar que se
	puede cubrir el espacio con numerables (finitas) cartas. En
	definitiva, los espacios $\proyectivo{\bb{R}}{n}$ son variedades
	topol\'{o}gicas compactas.
	
	Otra descripci\'{o}n --aunque esencialmente la misma-- del espacio
	proyectivo est\'{a} dada por dejar actuar al grupo $\bb{R}^{\times}$
	de reales distintos de cero sobre $\bb{R}^{n+1}\setmin\{0\}$ por
	multiplicaci\'{o}n por escalares. Las \'{o}rbitas de esta acci\'{o}n
	son, precisamente, los subespacios reales de dimensi\'{o}n $1$. Dado
	que el cociente $\bb{R}^{\times}\backslash(\bb{R}^{n+1}\setmin\{0\})$
	es homeomorfo a $\{\pm1\}\backslash\esfera{n}$ podemos describir a
	$\proyectivo{\bb{R}}{n}$ como el cociente de un espacio
	topol\'{o}gico (de una variedad compacta) por la acci\'{o}n de un
	grupo discreto (en particular actuando de manera propiamente
	discontinua). Es esto lo que nos permite deducir que los espacios
	proyectivos son variedades topol\'{o}gicas y, adem\'{a}s, compactas.
\end{ejemplo}

\begin{ejemplo}\nom{Una estructura de variedad diferencial en la esfera}
	Las esferas son variedades diferenciales. Usando las cartas
	$(U_{i}^{\pm},\varphi_{i}^{\pm})$ se obtiene un atlas compatible.
	S\'{o}lo hay que verificar que las composiciones
	$\varphi_{i}^{\pm}\circ(\varphi_{j}^{\pm})^{-1}$ sean diferenciables.
	Esta estructura en $\esfera{n}$ se denominar\'{a} la
	\emph{estructura usual} en $\esfera{n}$.

	Las cartas correspondientes a las proyecciones esterogr\'{a}ficas
	son compatibles entre s\'{\i} y, adem\'{a}s, son compatibles
	con la estructura usual de $\esfera{n}$. Veamos primero la
	compatibilidad de $\sigma$ con $\varphi_{i}^{\epsilon}$
	para $i\not =n+1$ y $\epsilon=+\,(>)\text{ o }-\,(<)$. Por un lado,
	en este	caso,
	\begin{align*}
		\big(U_{i}^{\epsilon}\cap\esfera{n}\big)\cap
			\big(\esfera{n}\setmin\{N\}\big) & \,=\,
			U_{i}^{\epsilon}\cap\esfera{n}
		\text{ .}
	\end{align*}
	%
	Entonces, dado que
	\begin{align*}
		\varphi_{i}^{\epsilon}(U_{i}^{\epsilon}\cap\esfera{n}) &
			\,=\,\bola{1}{0}\quad\text{y} \\
		\sigma(U_{i}^{\epsilon}\cap\esfera{n}) & \,=\,
			\left\lbrace u\in\bb{R}^{n}\,:\,u^{i}\epsilon 0
				\right\rbrace
		\text{ ,}
	\end{align*}
	%
	la composici\'{o}n $\varphi_{i}^{\epsilon}\circ\sigma^{-1}:\,%
	\{u^{i}\epsilon 0\}\rightarrow\bola{1}{0}$, dada por
	\begin{align*}
		\varphi_{i}^{\epsilon}\circ\sigma^{-1}(u) & \,=\,
			\frac{(2u^{1},\,\dots,\,\widehat{2u^{i}},\,2u^{n},\,%
				|u|^{2}-1)}{|u|^{2}+1}
		\text{ ,}
	\end{align*}
	%
	es suave (, inyectiva) y su matriz de derivadas parciales es
	no singular en todo punto, como se puede verificar, derivando la
	expresi\'{o}n anterior respecto de cada una de las variables
	$u^{k}$. Esto es suficiente para concluir que la composici\'{o}n
	en el sentido inverso tambi\'{e}n es suave. En todo caso, se
	puede verificar directamente que es suave: como $i\not =n+1$,
	vale que
	\begin{align*}
		\sigma\circ(\varphi_{i}^{\epsilon})^{-1}(v) & \,=\,
			\frac{(v^{1},\,\dots,\,\sqrt{1-|v|^{2}},%
				\,\dots,\,v^{n-1})}{1-v^{n}}
		\text{ .}
	\end{align*}
	%
	Como $|v|<1$, en particular $|v^{n}|<1$ y la composici\'{o}n es
	suave de $B_{1}(0)$ en $\{u^{i}\epsilon 0\}$.

	Si $i=n+1$, $U_{n+1}^{+}\cap\esfera{n}$ contiene a $N$ pero
	$U_{n+1}^{-}\cap\esfera{n}$ no lo contiene. Veamos primero que
	$\varphi_{n+1}^{-}$ es compatible con $\sigma$: la composici\'{o}n
	\begin{align*}
		\varphi_{n+1}^{-}\circ\sigma^{-1}(u) & \,=\,
			\frac{(\lista*{2u}{n})}{|u|^{2}+1}
	\end{align*}
	%
	es suave de $\sigma(U_{n+1}^{-}\cap\esfera{n})=\bola{1}{0}$ en
	$\varphi_{n+1}^{-}(U_{n+1}^{-}\cap\esfera{n})=\bola{1}{0}$ y, en el
	sentido contrario,
	\begin{align*}
		\sigma\circ(\varphi_{n+1}^{-})^{-1}(v) & \,=\,
			\frac{(\lista*{v}{n})}{1+\sqrt{1-|v|^{2}}}
	\end{align*}
	%
	que es suave tambi\'{e}n. De manera similar, se puede comprobar que
	la \emph{otra} proyecci\'{o}n estereogr\'{a}fica $\tilde{\sigma}$
	es compatible con $\varphi_{n+1}^{+}$ y, como $\sigma$ y
	$\tilde{\sigma}$ son compatibles, por tansitividad de la relaci\'{o}n
	de compatibilidad suave, $\sigma$ es compactible con
	$\varphi_{n+1}^{+}$, tambi\'{e}n. Tambi\'{e}n se puede verificar
	directamente: si llamamos $U^{+}$ al abierto
	\begin{align*}
		U^{+} & \,=\,\big(U_{n+1}^{+}\cap\esfera{n}\big)\cap
			\big(\esfera{n}\setmin\{N\}\big)
			\text{ ,}\quad\text{entonces} \\
		\sigma(U^{+}) & \,=\,\left\lbrace u\in\bb{R}^{n}\,:\,
					|u|>1\right\rbrace
			\quad\text{y} \\
		\varphi_{n+1}^{+}(U^{+}) & \,=\,\bola{1}{0}\setmin\{0\}
		\text{ .}
	\end{align*}
	%
	Ahora bien,
	\begin{math}
		\varphi_{n+1}^{+}\circ\sigma^{-1}(u) \,=\,
			\frac{(\lista*{2u}{n})}{|u|^{2}+1}
	\end{math}
	%
	que es suave y
	\begin{math}
		\sigma\circ(\varphi_{n+1}^{+})^{-1}(v) \,=\,
			\frac{(\lista*{v}{n})}{1-\sqrt{1-|v|^{2}}}
	\end{math}
	%
	que tambi\'{e}n es suave en su dominio de definici\'{o}n.
\end{ejemplo}

\begin{ejemplo}\nom{Una estructura diferencial en el espacio proyectivo}
	El espacio proyectivo $\proyectivo{n}{\bb{R}}$ tambi\'{e}n tiene
	estructura de variedad diferencial. Los cambios de coordenadas
	son diferenciables, pues, si $i>j$,
	\begin{align*}
		\varphi_{j}\circ\varphi_{i}^{-1}(\lista*{u}{n}) & \,=\,
		\Big(\frac{u^{1}}{u^{j}},\,\dots,\,\frac{u^{j-1}}{u^{j}},\,
			\frac{u^{j+1}}{u^{j}},\,\dots,\,\frac{u^{i-1}}{u^{j}},\,
			\frac{1}{u^{j}},\,\frac{u^{i}}{u^{j}},\,\dots,\,
				\frac{u^{n}}{u^{j}}\Big) \\
		\varphi_{i}\circ\varphi_{j}^{-1}(\lista*{u}{n}) & \,=\,
		\Big(\frac{u^{1}}{u^{i-1}},\,\dots,\,\frac{u^{j-1}}{u^{i-1}},\,
			\frac{1}{u^{i-1}},\,\frac{u^{j}}{u^{i-1}},\,\dots,\,
			\frac{u^{i-2}}{u^{i-1}},\,\frac{u^{i}}{u^{i-1}},\,
			\dots,\,\frac{u^{n}}{u^{i-1}}\Big)
		\text{ .}
	\end{align*}
	%
\end{ejemplo}

\begin{ejemplo}\nom{El espacio proyectivo complejo}
	De la misma manera en que se defini\'{o} el espacio proyectivo real,
	se puede definir el espacio proyectivo sobre el cuerpo de n\'{u}meros
	complejos, denotado $\proyectivo{\bb{C}}{n}$, como el conjunto de
	subespacios vectoriales \emph{complejos} de dimensi\'{o}n $1$ en
	$\bb{C}^{n+1}$. Es decir, $\proyectivo{\bb{C}}{n}$ es el
	cociente $(\bb{C}^{n+1}\setmin\{0\})/\sim$ donde dos puntos
	$z=(\lista*{z}{n+1})$ y $w=(\lista*{w}{n+1})$ est\'{a}n relacionados,
	si generan el mismo espacio \emph{sobre $\bb{C}$}, dicho de otra
	manera, si existe un n\'{u}mero complejo no nulo
	$\alpha\in\bb{C}^{\times}$ tal que $z\cdot\alpha=w$.

	En primer lugar, para ver que $\proyectivo{\bb{C}}{n}$ es una variedad
	topol\'{o}gica y para luego ver que se le puede dar una estructura
	diferencial, definimos una correspondencia. Sea
	$\Phi:\,\bb{C}^{n+1}\rightarrow\bb{R}^{2n+2}$ la funci\'{o}n dada por
	\begin{align*}
		\Phi(\lista*{z}{n+1}) & \,=\,
			(x^{1},\,y^{1},\,\dots,\,x^{n},\,y^{n})
		\text{ ,}
	\end{align*}
	%
	donde, para cada $i$, $z^{i}=x^{i}+\sqrt{-1}y^{i}$. Si $w$ es un
	punto de la forma $z\cdot\alpha$ con $\alpha=a+\sqrt{-1}b$ y
	$a,b\in\bb{R}$,
	\begin{align*}
		\Phi(w) & \,=\, \Phi(z^{1}\cdot\alpha,\,\dots,\,
					z^{n+1}\cdot\alpha) \\
		& \,=\,(x^{1}a-y^{1}b,\,x^{1}b+y^{1}a,\,\dots,\,
			x^{n+1}a-y^{n+1}b,\,x^{n+1}b+y^{n+1}a)
		\text{ .}
	\end{align*}
	%
	La acci\'{o}n de multiplicar (a derecha) por un n\'{u}mero complejo
	no nulo $\alpha$ se traduce v\'{\i}a $\Phi$ como el producto a derecha
	por una matriz:
	\begin{align*}
		\Phi(z\cdot\alpha) & \,=\,\Phi(z)\cdot m_{\alpha}
		\text{ ,}
	\end{align*}
	%
	donde $m_{\alpha}$ es la matriz \emph{real} dada por
	\begin{align*}
		m_{\alpha} & \,=\,
			\begin{bmatrix}
				\widehat{\alpha} & & \\
				& \ddots & \\
				& & \widehat{\alpha}
			\end{bmatrix}
		\quad\text{donde}\quad
		\widehat{\alpha} \,=\,
			\begin{bmatrix}
				a & b \\
				-b & a
			\end{bmatrix}
		\text{ .}
	\end{align*}
	%
	Es decir, para $\alpha\in\bb{C}^{\times}$,
	$\widehat{\alpha}\in\MM{2,\bb{R}}$ y $m_{\alpha}\in\MM{2n+2,\bb{R}}$.
	De hecho, como $\det(\widehat{\alpha})=a^{2}+b^{2}>0$,
	$\widehat{\alpha}\in\GL{2,\bb{R}}$ y $m_{\alpha}\in\GL{2n+2,\bb{R}}$.

	Dado $\alpha\in\bb{C}^{\times}$, sea
	$f_{\alpha}:\,\bb{R}^{2n+2}\setmin\{0\}\rightarrow%
		\bb{R}^{2n+2}\setmin\{0\}$ la funci\'{o}n correspondiente a
	multiplicar a derecha por la matriz $m_{\alpha}$.

	V\'{\i}a la correspondencia $\Phi$, la acci\'{o}n de
	$\bb{C}^{\times}$ en $\bb{C}^{n+1}$ est\'{a} dada por la acci\'{o}n
	diagonal del grupo de matrices
	\begin{align*}
		G & \,=\,\left\lbrace
			\begin{bmatrix}a & b\\ -b & a\end{bmatrix}\,:\,
			a^{2}+b^{2}>0\right\rbrace
		\,\simeq\,\bb{R}_{>0}\times\SO{2}
	\end{align*}
	%
	en el producto de $n+1$ copias de $\bb{R}^{2}$. Es decir,
	\begin{align*}
		\Phi((\lista*{z}{n+1})\cdot\alpha) & \,=\,
			((x^{1},y^{1})\cdot\widehat{\alpha},\,\dots,\,
			(x^{n+1},y^{n+1})\cdot\widehat{\alpha}) \\
		& \,=\,\Phi(z)\cdot m_{\alpha}
		\text{ .}
	\end{align*}
	%
	El espacio proyectivo complejo se obtiene como el espacio de
	\'{o}rbitas de $\bb{C}^{\times}$ actuando en $\bb{C}^{n+1}\setmin\{0\}$
	por multiplicaci\'{o}n por escalares, es decir, por homotecias.
	Si llamamos $M$ al conjunto de \'{o}rbitas de la acci\'{o}n
	correspondiente de $G$ en $\bb{R}^{2n+2}\setmin\{0\}$, entonces
	$\Phi$ induce una correspondencia entre $\proyectivo{\bb{C}}{n}$ y
	$M$. En definitiva, tenemos un diagrama conmutativo, en principio, de
	conjuntos:
	\begin{center}
	\begin{tikzcd}
		\bb{C}^{n+1}\setmin\{0\} \arrow[r,"\Phi"] \arrow[d,"\pi"] &
			\bb{R}^{2n+2}\setmin\{0\} \arrow[d,"q"] \\
		\proyectivo{\bb{C}}{n} \arrow[r,"F"] & M
	\end{tikzcd} .
	\end{center}

	En cuanto a la topolog\'{\i}a, $\bb{R}^{2n+2}\setmin\{0\}$
	tiene la estructura de variedad topol\'{o}gica heredada de
	$\bb{R}^{2n+2}$ en tanto subespacio abierto y a $M$ se le da la
	topolog\'{\i}a cociente. Usando las correspondencias $\Phi$ y $F$,
	damos a $\bb{C}^{n+1}$ y a $\proyectivo{\bb{C}}{n}$ las
	topolog\'{\i}as correspondientes de manera que $\Phi$ y $F$ sean
	homeomorfismos. As\'{\i}, el diagrama anterior pasa a ser un diagrama
	conmutativo de espacios topol\'{o}gicos y funciones continuas.

	El espacio topol\'{o}gico $M$ es una variedad topol\'{o}gica. Para
	mostrar esto busquemos primero cartas para $M$. Dado que las
	funciones $f_{\alpha}:\,\bb{R}^{2n+2}\setmin\{0\}%
		\rightarrow\bb{R}^{2n+2}\setmin\{0\}$ son restricciones de
	transformaciones lineales $m_{\alpha}$, son, en particular,
	continuas. Adem\'{a}s, como dichas transformaciones son invertibles,
	\begin{align*}
		f_{\alpha}\circ f_{\alpha^{-1}} & \,=\,
			f_{\alpha^{-1}}\circ f_{\alpha} \,=\,
			\id[\bb{R}^{2n+2}\setmin\{0\}]
		\text{ ,}
	\end{align*}
	%
	de lo que se deduce que el grupo $G$ act\'{u}a en
	$\bb{R}^{2n+2}\setmin\{0\}$ v\'{\i}a homeomorfismos. Una consecuencia
	de esto es que la funci\'{o}n cociente $q$ es abierta: si
	$V\subset\bb{R}^{2n+2}\setmin\{0\}$ es abierto, entonces
	\begin{align*}
		q^{-1}\big(q(V)\big) & \,=\,V\cdot G \\
		& \,=\,\bigcup_{m_{\alpha}\in G}\,V\cdot m_{\alpha}
			\,=\,\bigcup_{\alpha\in\bb{C}^{\times}}\,
				f_{\alpha}(V)
	\end{align*}
	%
	que es una uni\'{o}n de abiertos. Como $M$ tiene la topolog\'{\i}a
	cociente inducida por $q$, el conjunto $q(V)$ es abierto.
	Ahora bien, definimos, para cada $i$,
	\begin{align*}
		\widetilde{U}_{i} & \,=\,
			\left\lbrace((x^{1},y^{1}),\,\dots,\,
			(x^{n+1},y^{n+1}))\in\bb{R}^{2n+2}\,:\,
			(x^{i},y^{i})\not = 0\right\rbrace
		\text{ .}
	\end{align*}
	%
	Como las matrices $\widehat{\alpha}$ son invertibles, se ve que
	$q^{-1}(q(\widetilde{U}_{i}))=\widetilde{U}_{i}$ para todo $i$. En
	particular, $\widetilde{U}_{i}$ es un abierto saturado respecto de
	$q$ y, si llamamos $U_{i}=q(\widetilde{U}_{i})$, entonces la
	restricci\'{o}n de $q|_{i}:\,\widetilde{U}_{i}\rightarrow U_{i}$ es
	cociente.

	Para definir cartas en $M$, definimos funciones en
	$\proyectivo{\bb{C}}{n}$ an\'{a}logas a las cartas para los espacios
	proyectivos reales y las trasladamos, v\'{\i}a $F$ a $M$.
	Notemos que $\Phi^{-1}(\widetilde{U}_{i})$ es el subconjunto de
	$\bb{C}^{n+1}\setmin\{0\}$ conformado por los puntos cuya coordenada
	$i$ es distinta de cero. Si $z\in\bb{C}^{n+1}\setmin\{0\}$,
	denotaremos
	\begin{align*}
		\pi(z) & \,=\,\pi(\lista*{z}{n+1}) \,=\,
			\left[z^{1}:\,\cdots\,:z^{n+1}\right]
	\end{align*}
	%
	al punto correpondiente en el espacio proyectivo. De manera
	an\'{a}loga los puntos de $M$ los denotaremos de la siguiente manera:
	\begin{align*}
		q((x^{1},y^{1}),\,\dots,\,(x^{n+1},y^{n+1})) & \,=\,
			\left[(x^{1},y^{1}):\,\cdots\,:(x^{n+1},y^{n+1})\right]
		\text{ .}
	\end{align*}
	%
	Sea, entonces, $\varphi_{i}:\,F^{-1}(U_{i})\rightarrow\bb{C}^{n}$
	la funci\'{o}n dada por
	\begin{align*}
		\varphi_{i}\big(\left[z^{1}:\,\cdots\,:z^{n+1}\right]\big) &
			\,=\,\left(\frac{z^{1}}{z^{i}},\,\dots,\,
			\frac{z^{i-1}}{z^{i}},\,\frac{z^{i+1}}{z^{i}},
			\,\dots,\,\frac{z^{n+1}}{z^{i}}\right) \\
		& \,=\,(z^{1},\,\dots,\,z^{i-1},\,z^{i+1},\,\dots,\,z^{n+1})
			\cdot\frac{1}{z^{i}}
		\text{ .}
	\end{align*}
	%
	De la expresi\'{o}n anterior, se deduce que las funciones
	$\varphi_{i}$ son biyectivas. Veamos que son continuas. Para eso
	usamos las correspondencias $\Phi$ y $F$. En
	$\widetilde{U}_{i}\subset M$, entonces, definimos
	$\tilde{\varphi}_{i}=\Phi\circ\varphi_{i}\circ F^{-1}$. Donde
	$\Phi:\,\bb{C}^{n}\rightarrow\bb{R}^{2n}$ es an\'{a}loga a la
	funci\'{o}n $\Phi$ definida anteriormente. Cuando
	$i=n+1$, por ejemplo, si $z^{n+1}=x^{n+1}+\sqrt{-1}y^{n+1}$ y
	$\alpha=\frac{1}{z^{n+1}}$,
	\begin{align*}
		\tilde{\varphi}_{n+1}\big(\left[(x^{1},y^{1}):\,\cdots\,:
					(x^{n+1},y^{n+1})\right]\big) & \,=\,
			((x^{1},y^{1})\cdot\widehat{\alpha},\,\dots,\,
			(x^{n},y^{n})\cdot\widehat{\alpha})
		\text{ .}
	\end{align*}
	%
	Un poco m\'{a}s expl\'{\i}citamente, en el caso $i=n+1$ (para
	simplificar),
	\begin{align*}
		\tilde{\varphi}_{n+1}\big(\left[(x^{1},y^{1}):\,\cdots\,:
					(x^{n+1},y^{n+1})\right]\big) & \,=\,
		\left(ax^{1}-by^{1},\,bx^{1}+ay^{1},\right. \\
		& \left.\qquad\,\dots,\,ax^{n}-by^{n},\,bx^{n}+ay^{n}\right)
		\text{ ,}
	\end{align*}
	%
	donde $a=\frac{x^{n+1}}{(x^{n+1})^{2}+(y^{n+1})^{2}}$ y
	$b=\frac{y^{n+1}}{(x^{n+1})^{2}+(y^{n+1})^{2}}$.
	Llamamos, para simplificar, $\varphi_{i}$ a $\tilde{\varphi}_{i}$,
	tambi\'{e}n. De la expresi\'{o}n anterior, para cada $i$, la
	composici\'{o}n $\varphi_{i}\circ q$ es continua. Porque $q|_{i}$
	es cociente, $\varphi_{i}:\,U_{i}\rightarrow\bb{R}^{2n}$ es
	continua. 

	La inversa de $\varphi_{i}$ est\'{a} dada por
	$\varphi_{i}^{-1}:\,\bb{C}^{n}\rightarrow F^{-1}(U_{i})$ donde
	\begin{align*}
		\varphi_{i}^{-1}(\lista*{u}{n}) & \,=\,
			\left[u^{1}:\,\cdots\,:u^{i-1}:1:u^{i}:
				\,\cdots\,:u^{n}\right]
		\text{ .}
	\end{align*}
	%
	El elemento $1$ en la expresi\'{o}n anterior es el $1$ de $\bb{C}$.
	V\'{\i}a $\Phi$ y $F$, en t\'{e}rminos de puntos de $M$ y de
	$\bb{R}^{2n}$,
	\begin{align*}
		\varphi_{i}(\xi^{1},\upsilon^{1},\,\dots,\,
			\xi^{n},\,\upsilon^{n}) & \,=\,
			\left[(\xi^{1},\upsilon^{1}):\,\cdots\,:
			(\xi^{i-1},\upsilon^{i-1}):(1,0):
			(\xi^{i},\upsilon^{i}):\,\cdots\,
			(\xi^{n},\upsilon^{n})\right]
		\text{ .}
	\end{align*}
	%
	Notemos que $(1,0)$ es la representaci\'{o}n de $1\in\bb{C}$ en
	coordenadas reales. La funci\'{o}n $\varphi_{i}^{-1}$ es continua
	porque se puede escribir como composici\'{o}n de funciones continuas
	seg\'{u}n el siguiente diagrama:
	\begin{center}
	\begin{tikzcd}
		\bb{R}^{2n+2}\setmin\{0\}\supset\widetilde{U}_{i}
			\arrow[d,"q"'] & \\
		M\supset U_{i} & \bb{R}^{2n}\arrow[l,"\varphi_{i}^{-1}"]
					\arrow[ul,"\psi_{i}"']

	\end{tikzcd}
	\end{center}
	La funci\'{o}n $\psi_{i}$ es \'{u}nica tal que el diagrama
	conmuta y debe cumplir que
	\begin{align*}
		\psi_{i}(\xi^{1},\upsilon^{1},\,\dots,\,
			\xi^{n},\,\upsilon^{n}) & \,=\,
			(\xi^{1},\,\upsilon^{1},\,\dots,\,
			\xi^{i-1},\,\upsilon^{i-1},\,1,\,0,\,
			\xi^{i},\,\upsilon^{i},\,\dots,\,
			\xi^{n},\,\upsilon^{n})
		\text{ .}
	\end{align*}
	%
	De esta expresi\'{o}n, es inmediato que $\psi_{i}$ es continua y que
	$\varphi_{i}^{-1}=q\circ\psi_{i}$. En definitiva, las funciones
	$\varphi_{i}:\,U_{i}\rightarrow\bb{R}^{2n}$ son homeomorfismos
	y las correspondientes
	$\varphi_{i}:\,F^{-1}(U_{i})\rightarrow\bb{C}^{n}$ tambi\'{e}n, donde
	$\bb{C}^{n}$ tiene la topolog\'{\i}a inducida por la biyecci\'{o}n
	$\Phi:\,\bb{C}^{n}\rightarrow\bb{R}^{2n}$.

	% Una observaci\'{o}n importante es que las operaciones de espacio
	% vectorial complejo en $\bb{C}^{n}$ son continuas si se le da
	% la topolog\'{\i}a inducida por el homeomorfismo
	% $\Phi:\,\bb{C}^{n}\rightarrow\bb{R}^{2n}$ para todo $n\geq 0$.

	Hemos demostrado que $\proyectivo{\bb{C}}{n}$ con la topolg\'{\i}a
	cociente que se obtiene de considerar $\bb{C}^{n+1}$ como
	$\bb{R}^{2n+2}$ y a $\bb{C}^{n+1}\setmin\{0\}$ como subconjunto
	abierto es localmente euclidea de dimensi\'{o}n $2n$. De hecho,
	demostramos que $\proyectivo{\bb{C}}{n}$ es homeomorfo al
	cociente de $\bb{R}^{2n+2}\setmin\{0\}$ por la acci\'{o}n
	diagonal del grupo $\bb{R}_{>0}\times\SO{2}$.
\end{ejemplo}

\begin{ejemplo}\nom{Una versi\'{o}n m\'{a}s resumida del ejemplo anterior}
	El \emph{espacio proyectivo complejo} de dimensi\'{o}n $n$ es el
	conjunto de $\bb{C}$-subespacios vectoriales de $\bb{C}^{n+1}$. Se
	puede realizar como el cociente de $\bb{C}^{n+1}\setmin\{0\}$
	por la relaci\'{o}n de equivalencia $z\sim w$, si generan el mismo
	subespacio, o, lo que es equivalente, si existe un n\'{u}mero
	complejo $\alpha\in\bb{C}^{\times}$ no nulo tal que $\alpha\cdot z=w$.
	Si $z\in\bb{C}^{n+1}\setmin\{0\}$, denotamos su clase por
	$z=[z^{1}:\,\cdots\,:z^{n+1}]$. Al conjunto de clases
	$\big(\bb{C}^{n+1}\setmin\{0\}\big)/\sim$ lo denotamos por
	$\proyectivo{\bb{C}}{n}$.

	Para cada $i\in[\![1,n+1]\!]$, sea $\widetilde{U}_{i}$ el conjunto
	\begin{align*}
		\widetilde{U}_{i} & \,=\,\left\lbrace
			(\lista*{z}{n+1})\in\bb{C}^{n+1}\setmin\{0\}\,:\,
			z^{i}\not =0\right\rbrace
		\text{ .}
	\end{align*}
	%
	Todo vector no nulo $z$ pertenece a alguno de estos subconjuntos.
	Sea $q:\,\bb{C}^{n+1}\setmin\{0\}\rightarrow\proyectivo{\bb{C}}{n}$
	la funci\'{o}n que a cada elemento $z$ le asigna su clase:
	\begin{align*}
		q(\lista*{z}{n+1}) & \,=\,\left[z^{1}:\,\cdots\,:z^{n+1}\right]
		\text{ .}
	\end{align*}
	%
	Sea $U_{i}=q(\widetilde{U}_{i})$. Como $q$ es sobreyectiva, los
	conjuntos $U_{i}$ cubren $\proyectivo{\bb{C}}{n}$.

	La topolog\'{\i}a que daremos al espacio proyectivo ser\'{a} la
	topolog\'{\i}a cociente determinada por $q$. Para ello, es necesario,
	en primer lugar especificar una topolog\'{\i}a en
	$\bb{C}^{n+1}\setmin\{0\}$. Cada vector $z$ complejo se puede
	representar en coordenadas reales: si $z=(\lista*{z}{n+1})$, cada
	$z^{i}$ es igual a $x^{i}+\sqrt{-1}y^{i}$ para ciertos n\'{u}meros
	reales $x^{i}$ e $y^{i}$. Sea, para $k\geq 0$, $\Phi\equiv%
	\Phi_{k}:\,\bb{C}^{k}\rightarrow\bb{R}^{2k}$ la biyecci\'{o}n
	\begin{align*}
		\Phi(\lista*{z}{k}) & \,=\,\big( (x^{1},y^{1}),\,\dots,\,
			(x^{k},y^{k})\big)
		\text{ .}
	\end{align*}
	%
	Como $\bb{R}^{2n+2}\setmin\{0\}$ es un subconjunto abierto de
	$\bb{R}^{2n+2}$ (con la topolog\'{\i}a usual), es una variedad
	topol\'{o}gica con la topolog\'{\i}a subespacio. La topolog\'{\i}a
	en $\bb{C}^{n+1}\setmin\{0\}$ es la que se obtiene de tomar
	preim\'{a}genes v\'{\i}a $\Phi$ de los abiertos del codominio.
	Notemos que esta topolog\'{\i}a coincide con la topolog\'{\i}a de
	subespacio abierto de $\bb{C}^{n+1}$, donde este \'{u}ltimo tiene
	la topolog\'{\i}a de $\bb{R}^{2n+2}$ (o la topolog\'{\i}a
	producto de $n+1$ copias de $\bb{C}$, cada una de ellas con la
	topolog\'{\i}a del plano $\bb{R}^{2}$, pues son iguales).

	Demos, entonces, a $\proyectivo{\bb{C}}{n}$ la topolog\'{\i}a
	cociente determinada por la suryecci\'{o}n $q$. Las clases de
	equivalencia en $\bb{C}^{n+1}\setmin\{0\}$ no son otra cosa
	que las \'{o}rbitas por la acci\'{o}n diagonal del grupo
	$\bb{C}^{\times}$ en los vectores complejos distintos de cero.
	En otras palabras, como las identificaciones dadas por la
	relaci\'{o}n $\sim$ y por la acci\'{o}n de $\bb{C}^{\times}$ son
	iguales en $\bb{C}^{n+1}\setmin\{0\}$, hay un homeomorfismo
	\begin{align*}
		\bb{C}^{\times}\backslash\big(\bb{C}^{n+1}\setmin\{0\}\big)
			 & \,\simeq\, \big(\bb{C}^{n+1}\setmin\{0\}\big)/\sim
		 \text{ .}
	\end{align*}
	%
	As\'{\i}, como el espacio proyectivo es un cociente por la
	acci\'{o}n de un grupo, el epimorfismo can\'{o}nico $q$ es
	una funci\'{o}n abierta. Precisamente, si $\alpha\in\bb{C}^{\times}$,
	la funci\'{o}n $z\mapsto\alpha\cdot z$ es una funci\'{o}n continua.
	Vale la pena notar que, por ahora, no estamos considerando esta
	acci\'{o}n como la acci\'{o}n de un grupo topol\'{o}gico, es decir,
	no estamos considerando la topolog\'{\i}a de $\bb{C}^{\times}$,
	s\'{o}lo estamos haciendo uso del hecho de que, para cada $\alpha$,
	la funci\'{o}n inducida es continua.
	
	De las observaciones anteriores, deducimos que los conjuntos
	$U_{i}=q(\widetilde{U}_{i})$ son abiertos en $\proyectivo{\bb{C}}{n}$.
	M\'{a}s aun, como cada $\widetilde{U}_{i}$ es saturado respecto
	de $q$, las restricciones $q_{i}=q|_{\widetilde{U}_{i}}:\,%
	\widetilde{U}_{i}\rightarrow U_{i}$ son cocientes. Estos conjuntos,
	como en el caso real, servir\'{a}n de cartas. Para demostrar esta
	afirmaci\'{o}n, definimos
	$\varphi_{i}:\,U_{i}\rightarrow\bb{C}^{n}$ por
	\begin{align*}
		\varphi_{i}\big(\left[z^{1}:\,\cdots\,:z^{n+1}\right]\big) &
			\,=\,\big(\frac{z^{1}}{z^{i}},\,\dots,\,
			\frac{z^{i-1}}{z^{i}},\,\frac{z^{i+1}}{z^{i}},
			\,\dots,\,\frac{z^{n+1}}{z^{i}}\big)
		\text{ .}
	\end{align*}
	%
	Esta aplicaci\'{o}n est\'{a} bien definida, pues la expresi\'{o}n
	de la derecha no depende del representante elegido. Adem\'{a}s,
	como la composici\'{o}n $\varphi_{i}\circ q_{i}$ es continua en
	$\widetilde{U}_{i}$, se deduce que $\varphi_{i}$ es continua,
	por la propiedad (?`universal?) del cociente. Esta aplicaci\'{o}n
	tiene una inversa, dada por la composici\'{o}n de la inclusi\'{o}n
	de una tajada en la coordenada $i$:
	\begin{align*}
		\iota_{i,1} & \,:\,
			(\lista*{z}{n})\,\mapsto\,(z^{1},\,\dots,\,z^{i-1},\,
			1,\,z^{i},\,\dots,\,z^{n})
	\end{align*}
	%
	compuesta con $q_{i}$, es decir,
	$\varphi_{i}^{-1}=q_{i}\circ\iota_{i,1}$. Como esta funci\'{o}n es una
	composici\'{o}n de funciones continuas, es continua y $\varphi_{i}$ es
	un homeomorfismo. Para asegurarnos de que $\varphi_{i}\circ q_{i}$ y
	que $\iota_{i,1}$ son funciones continuas, podemos componer con las
	``coordenadas reales'' $\Phi$: si
	$a=\frac{x^{i}}{(x^{i})^{2}+(y^{i})^{2}}$ y
	$b=\frac{y^{i}}{(x^{i})^{2}+(y^{i})^{2}}$,
	\begin{align*}
		\Phi_{n}\circ(\varphi_{i}\circ q_{i})\circ\Phi_{n+1}^{-1}
		\big((x^{1},y^{1}),\,\dots,\,(x^{n+1},y^{n+1})\big) & \,=\, \\
		\big(ax^{1}-by^{1},\,bx^{1}+ay^{1}, \,\dots,\,
		& ax^{i-1}-by^{i-1},\,bx^{i-1}+ay^{i-1},\, \\
		ax^{i+1}-by^{i+1},\,bx^{i+1}+ay^{i+1},\,\dots,\,
		& ax^{n+1}-by^{n+1},\,bx^{n+1}+ay^{n+1}\big)
		\quad\text{y} \\
		\Phi_{n+1}\circ\iota_{i,1}\circ\Phi_{n}^{-1}
		\big((x^{1},y^{1}),\,\dots,\,(x^{n},y^{n})\big) & \,=\, \\
		\big((x^{1},y^{1}),\,\dots,\,(x^{i-1},y^{i-1}), & \,(1,0),\,
		(x^{i},y^{i}),\,\dots,\,(x^{n},y^{n})\big)
		\text{ .}
	\end{align*}
	%
	Estas funciones son continuas es sus dominios de definici\'{o}n.

	Hemos demostrado que $\proyectivo{\bb{C}}{n}$ es localmente euclideo
	de dimensi\'{o}n $2n$. Tambi\'{e}n hemos hallado expl\'{\i}citamente
	un atlas \emph{finito} para el espacio. Para poder que los espacios
	proyectivos complejos son variedades topol\'{o}gicas, s\'{o}lo resta
	verificar que la topolog\'{\i}a es Hausdorff (que hay una base
	numerable para la topolog\'{\i}a, se deduce de que
	$\bb{C}^{n+1}\setmin\{0\}$ es $N_{2}$ y de que $q$ es abierta, o
	bien de que el espacio proyectivo se puede cubrir por finitas
	``cartas'' (abiertos, homeomorfos a $\bb{R}^{2n}$)). Para demostrar
	esto, usaremos el hecho \ref{thm:},
	como la funci\'{o}n cociente $q$ es abierta, ser\'{a} suficiente
	verificar que la relaci\'{o}n $\sim$ es cerrada, es decir, el
	subconjunto
	\begin{align*}
		R & \,=\,\left\lbrace (z,w)\,:\,z\sim w\right\rbrace
	\end{align*}
	%
	es cerrado en el producto de $\bb{C}^{n+1}\setmin\{0\}$ consigo
	mismo.

	Si restringimos $q$ a la esfera compleja, la aplicaci\'{o}n
	\begin{align*}
		q| & \,:\,\esfera{n}(\bb{C})\,\equiv\,
			\left\lbrace z\in\bb{C}^{n+1}\setmin\{0\}\,:\,
			|z|=1\right\rbrace\,\rightarrow\,
			\proyectivo{\bb{C}}{n}
	\end{align*}
	%
	es continua, por ser restricci\'{o}n a un subespacio de una
	funci\'{o}n continua, y es sobreyectiva. Como el dominio es
	compacto y sabemos que el codominio es Hausdorff, la restricci\'{o}n
	$q|$ es cociente. Notemos, adem\'{a}s, que $\proyectivo{\bb{C}}{n}$
	se puede realizar, tambi\'{e}n, como el cociente de
	$\esfera{n}(\bb{C})$ por la acci\'{o}n del grupo compacto
	$S^{1}=\{|\alpha|=1\}\subset\bb{C}^{\times}$. Sabemos, tambi\'{e}n,
	que la acci\'{o}n
	\begin{align*}
		S^{1}\times\esfera{n}(\bb{C}) & \,\rightarrow\,
			\esfera{n}(\bb{C})
	\end{align*}
	%
	es continua: es la restricci\'{o}n de la acci\'{o}n
	\begin{align*}
		\bb{C}^{\times}\times\big(\bb{C}^{n+1}\setmin\{0\}\big) &
		\,\rightarrow\,\bb{C}^{n+1}\setmin\{0\}
	\end{align*}
	%
	y que esta \'{u}ltima es continua (en ambos factores) se puede ver
	tomando las cartas reales $\Phi$. Finalmente, para ver que la
	relaci\'{o}n $R$ es cerrada, tomamos una sucesi\'{o}n convergente
	$\{(z_{m},w_{m})\}_{m\geq 1}$ en
	$R\cap\big(\esfera{n}(\bb{C})\times\esfera{n}(\bb{C})\big)$.
	Cada $w_{m}$ es igual a $\alpha_{m}\cdot z_{m}$ para alg\'{u}n
	$\alpha_{m}\in S^{1}$. Como $S^{1}$ es compacto, existe una
	subsucesi\'{o}n convergente. Llamemos igualmente $\alpha_{m}$ a los
	elementos de esta subsucesi\'{o}n y $z_{m}$ a los correspondientes
	t\'{e}rminos en la sucesi\'{o}n de vectores. Sean
	$\alpha=\lim_{m\to\infty},\alpha_{m}$ y $z=\lim_{m\to\infty}\,z_{m}$
	como $\esfera{n}(\bb{C})$ es compacta, es cerrada y el l\'{\i}mite
	$z$ debe pertenecer a la esfera. Pero, entonces,
	\begin{align*}
		\lim_{m\to\infty}\,(z_{m},\alpha_{m}\cdot z_{m}) & \,=\,
		(z,\alpha\cdot z)\in
		R\cap\big(\esfera{n}(\bb{C})\times\esfera{n}(\bb{C})\big)
		\text{ .}
	\end{align*}
	%
\end{ejemplo}

\begin{ejemplo}\nom{La proyecci\'{o}n en el espacio proyectivo}
	Sea $M$ una variedad diferencial y sea $\pi:\,\esfera{n}\rightarrow%
	\proyectivo{\bb{R}}{n}$ la proyecci\'{o}n can\'{o}nica.
	Dada funci\'{o}n $f:\,\proyectivo{\bb{R}}{n}\rightarrow M$,
	entonces $f$ es suave, si y s\'{o}lo si
	$f\circ\pi:\,\esfera{n}\rightarrow M$ lo es. Esto se debe a que
	hay secciones locales suaves para $\pi$. Esto es una propiedad
	caracter\'{\i}stica de las submersiones. Expl\'{\i}citamente, en
	este caso, si $U_{i}^{+}=\{x^{i}>0\}\cap\esfera{n}$ y
	$U_{i}=\{[x]\,:\,x^{i}\not =0\}\subset\proyectivo{\bb{R}}{n}$,
	entonces podemos definir, dada una clase $\xi\in U_{i}$,
	\begin{align*}
		\sigma_{i}(\xi) & \,=\,(x^{1},\,\dots,\,x^{n+1})
	\end{align*}
	%
	donde $(x^{1},\,\dots,\,x^{n+1})\in\esfera{n}$ es el representante
	de $\xi$ con $x^{i}>0$. Esta funci\'{o}n es suave. Si
	denotamos con $\varphi_{i}:\,U_{i}\rightarrow\bb{R}^{n}$ a las
	coordenadas en $U_{i}$ y con
	$\varphi_{i}^{+}:\,U_{i}^{+}\rightarrow\bola{0}{1}$ a las coordenadas
	en $U_{i}^{+}$, entonces
	\begin{align*}
		\varphi_{i}^{+}\circ\sigma_{i}\circ\varphi_{i}^{-1}
		(\lista*{u}{n}) & \,=\,
			\varphi_{i}^{+}\circ\sigma_{i}\big(
			\left[u^{1}:\,\cdots\,:u^{i-1}:1:\,\cdots\,:u^{n}
				\right]\big) \\
		& \,=\,\varphi_{i}^{+}\Big(
			\tfrac{u^{1}}{\sqrt{|u|^{2}+1}},\,\dots,\,
			\tfrac{u^{i-1}}{\sqrt{|u|^{2}+1}},\,
			\tfrac{1}{\sqrt{|u|^{2}+1}},\,\dots,\,
			\tfrac{u^{n}}{\sqrt{|u|^{2}+1}}\Big) \\
		& \,=\,\Big(\tfrac{u^{1}}{\sqrt{|u|^{2}+1}},\,\dots,\,
			\tfrac{u^{n}}{\sqrt{|u|^{2}+1}}
			\Big)
		\text{ .}
	\end{align*}
	%
	Esto demuestra que $\sigma_{i}$ es suave. Adem\'{a}s, de la
	definici\'{o}n se deduce que $\pi\circ\sigma_{i}=\id[U_{i}]$,
	Ahora bien, si $f$ es suave, $f\circ\pi$ es suave, por ser
	composici\'{o}n de funciones suaves. Rec\'{\i}procamente, si
	la composici\'{o}n $f\circ\pi$ es suave, restringiendo a cada
	$U_{i}$, vale que
	\begin{align*}
		\left.f\right|_{U_{i}} & \,=\,(f\circ\pi)\circ\sigma_{i}
		\text{ ,}
	\end{align*}
	%
	que es, nuevamente, composici\'{o}n de funciones suaves. Entonces
	$f|_{U_{i}}$ es suave para cada $i$ y $f$ es suave en
	$\proyectivo{\bb{R}}{n}$.

	En cuanto al rango de $\pi$, localmente sabemos que vale una igualdad
	de la forma $\pi\circ\sigma_{i}=\id[U_{i}]$. Tomando diferenciales,
	$\diferencial{\pi}\cdot\diferencial{\sigma_{i}} =\id[TU_{i}]$.
	Con lo cual, $\diferencial{\pi}$ es sobreyectivo y $\pi$ es una
	submersi\'{o}n. Notemos que, adem\'{a}s, $\diferencial{\sigma_{i}}$
	es inyectivo. En particular, como sabemos que las dimensiones
	de $\esfera{n}$ y de $\proyectivo{\bb{R}}{n}$ son iguales, podemos
	concluir que $\diferencial{\sigma_{i}}$ es un isomorfismo lineal
	(estamos omitiendo el punto en la notaci\'{o}n del diferencial, todo
	ocurre a nivel de los tanentes en cada punto por separado).

	Veamos cu\'{a}l es la relaci\'{o}n
	entre el rango de $f$ y el de $f\circ\pi$. Sea $\xi\in U_{i}$ un
	punto de $\proyectivo{\bb{R}}{n}$ y sea $x=\sigma_{i}(x)$. La
	ecuaci\'{o}n $f|_{U_{i}}=f\circ\pi\circ\sigma_{i}$ implica que
	\begin{align*}
		\diferencial[\xi]{f}\cdot\diferencial[\xi]{\inc[U_{i}]} &
		\,=\,
		\diferencial[x]{(f\circ\pi)}\cdot\diferencial[\xi]{\sigma_{i}}
	\end{align*}
	%
	Pero $\diferencial[\xi]{\inc[U_{i}]}$ es un isomorfismo y,
	como vimos antes, $\diferencial[\xi]{\sigma_{i}}$ tambi\'{e}n lo es.
	Entonces $\rango{f}=\rango{f\circ\pi}$ en todo punto. Rigurosamente,
	dado $x\in\esfera{n}$, el rango de $f$ en $\pi(x)$ es igual al rango
	de $f\circ\pi$ en $x$.

	Las secciones $\sigma_{i}:\,U_{i}\rightarrow U_{i}^{+}$ son,
	en realidad, difeomorfismos. Las inversas est\'{a}n dadas por las
	restricciones de $\pi$ a los abiertos $U_{i}^{+}$. En particular,
	$\diferencial[\xi]{\sigma_{i}}$ es un isomorfismo lineal, no s\'{o}lo
	una transformaci\'{o}n inyectiva que resulta isomorfismo por
	dimensi\'{o}n.
\end{ejemplo}
%
\section{Variedades de dimensi\'{o}n $1$}
%\theoremstyle{plain}
\newtheorem{teoClasTop}{Teorema}[subsection]
\newtheorem{propoExisteUnAbiertoMaximalmenteHomeo}[teoClasTop]{Proposici\'{o}n}
\newtheorem{propoALoSumoDosComponentes}[teoClasTop]{Proposici\'{o}n}
\newtheorem{lemaEsSubintervaloExterno}[teoClasTop]{Lema}
\newtheorem{propoExtenderHomeoConIntervalo}[teoClasTop]{Proposici\'{o}n}
\newtheorem{teoClasTopConBorde}[teoClasTop]{Teorema}

\theoremstyle{remark}

%-------------


En esta secci\'{o}n presentamos algunas demostraciones de la clasificaci\'{o}n
de variedades de dimensi\'{o}n uno. Presentamos la clasificaci\'{o}n
en el contexto de variedades topol\'{o}gicas y de variedades diferenciales.
En el primer caso damos dos demostraciones, una elemental y otra usando
resultados de triangulaci\'{o}n de variedades.

\subsection{Variedades topol\'{o}gicas de dimensi\'{o}n $1$}
El teorema que pretendemos demostrar es el siguiente.

\begin{teoClasTop}\label{thm:clastop}
	Si $M$ es una variedad sin borde, conexa y de dimensi\'{o}n $1$,
	entonces $M=\esfera{1}$ o $M=\bb{R}$.
\end{teoClasTop}

Empezamos con una observaci\'{o}n general acerca de $M$ que nos va a
permitir demostrar algunas cosas y orientar la demostraci\'{o}n del
teorema.

La variedad $M$ se puede cubrir por abiertos homeomorfos a abiertos de
$\bb{R}$. Dejando de lado el caso trivial en que $M=\varnothing$,
hay abiertos de $M$ homeomorfos a intervalos abiertos de $\bb{R}$ o,
lo que es lo mismo, a $\bb{R}$. Veamos que existen abiertos maximales con
respecto a esta propiedad, es decir, existen abiertos de $M$ homeomorfos
a $\bb{R}$ que no est\'{a}n propiamente contenidos en otro abierto homeomorfo
a $\bb{R}$.

\begin{propoExisteUnAbiertoMaximalmenteHomeo}%
	\label{thm:existeunabiertomaximalmentehomeo}
	Existe un abierto conexo de $M$ homeomorfo a $\bb{R}$ que
	no est\'{a} propiamente incluido en otro abierto con estas
	propiedades.
\end{propoExisteUnAbiertoMaximalmenteHomeo}

Esto es una consecuencia del lema de Zorn, pero la hip\'{o}tesis
de que $M$ es $N_{2}$, en particular, de que es \emph{Lindel\"{o}f} y todo
subespacio de $M$ es Lindel\"{o}f. Sea $\cal{O}$ el poset formado por
la colecci\'{o}n de todos los abiertos de $M$ homeomorfos a $\bb{R}$
ordenados por la inclusi\'{o}n. Sea $\cal{C}$ una cadena $\cal{O}$,
queremos considerar la uni\'{o}n $X=\bigcup\,\cal{C}$. En general, no es
cierto que una uni\'{o}n arbitraria de subespacios homeomorfos a la recta
real sea homeomorfa a la recta, incluso en el caso en que los subconjuntos
est\'{e}n encajados. Pero $M$ es $N_{2}$. Esto implica que $X$ es $N_{2}$
y que el cubrimiento $\cal{C}$ admite un subcubrimiento \emph{numerable}.
Es decir, existe una subfamilia $\{U_{n}\}_{n\geq 1}$ tal que
$X=\bigcup_{n\geq 1}\,U_{n}$. Sea $V_{m}=U_{1}\cup\,\cdots\,\cup U_{m}$.
Entonces $\{V_{m}\}_{m\geq 1}$ es una sucesi\'{o}n de abiertos encajados.
Para cada $m\geq 1$, $V_{m}=U_{i}$ para alg\'{u}n $i\leq m$ y, entonces
$V_{m}$ es homeomorfo a $\bb{R}$. En definitiva, como
\begin{align*}
	X & \,=\,\bigcup\,\cal{C} \,=\,\bigcup_{n\geq 1}\,U_{n}
	\,=\,\bigcup_{m\geq 1}\,V_{m}
\end{align*}
%
es una uni\'{o}n numerable creciente de conjuntos homeomorfos a $\bb{R}$,
$X$ es homeomorfo a $\bb{R}$ con lo que pertenece a $\cal{O}$ y es
cota superior para $\cal{C}$. Por Zorn, existe un abierto conexo de $M$
homeomorfo a $\bb{R}$ y maximal con esta propiedad.

Sea $U\subset M$ un elemento maximal en $\cal{O}$ y sea
$\varphi:\,U\rightarrow\bb{R}$ un homeomorfismo. Si es el caso que
$U=M$, entonces $M=\bb{R}$ v\'{\i}a $\varphi$. Si no es as\'{\i},
supongamos primero que $M=U\cup\{\infty\}$, es decir que $M\setmin U$
consta de un \'{u}nico punto. Sea $(V,\psi)$ una casrta en $\infty$
con dominio conexo, supongamos que $\psi(V)=(a,b)$ es un intervalo abierto
de $\bb{R}$ y que $\psi(\infty)=x\in (a,b)$. Sea
$W=\psi^{-1}((a',b'))\subset V$ con $a<a'<x<b'<b$, de manera que
\begin{align*}
	\infty & \,\in\, W\,\subsetneq\,\clos{W}\,\subsetneq\,V
	\text{ .}
\end{align*}
%
Como $M$ es Hausdorff, la clausura de $W$ en $V$ coincide con la clausura
de $W$ en $M$. Intersecando con $U$ y tomando coordenadas con $\psi$,
obtenemos que, como $M\setmin U=\{\infty\}$,
\begin{align*}
	\psi(U\cap V) & \,=\,\psi(V\setmin\{\infty\}) \,=\,(a,x)\cup (x,b)
		\text{ ,} \\
	\psi(U\cap\clos{W}) & \,=\,[a',x)\cup (x,b']
		\quad\text{y} \\
	\psi(U\cap W) & \,=\,(a',x)\cup (x,b')
	\text{ .}
\end{align*}
%
En particular, $U\cap\clos{W}$ es homeomorfo (v\'{\i}a $\varphi$)
a una uni\'{o}n disjunta de dos intervalos abiertos y cerrados, pero una
uni\'{o}n que debe ser, tambi\'{e}n, cerrada en $\bb{R}$. Entonces
$\varphi(U\cap\clos{W})=\bb{R}\setmin (r,s)$, donde $r<s$ son n\'{u}meros
reales (finitos), es decir, $\varphi(U\cap\clos{W})$ debe ser una uni\'{o}n
disjunta de dos semiractas de extremo cerrado en $\bb{R}$. Pero
entonces $\varphi(U\cap W)=\bb{R}\setmin [r,s]$ de lo que se deduce que
$\infty$ contiene una base de entornos que son complementos de compactos
en $M$. Dicho de otra manera, $M=U\cup\{\infty\}$ es la compactificaci\'{o}n
de Alexandroff de $U=\bb{R}$, es decir, $M=\esfera{1}$.

Si $M\setmin U$ contiene m\'{a}s de un punto, ya no parece muy evidente
qu\'{e} hacer con los puntos que est\'{a}n por fuera del abierto. Podemos
ver qu\'{e} pasa con los puntos en la clausura y tomar entornos coordenados
de dichos puntos. Pero entonces tendr\'{\i}amos que saber c\'{o}mo se
relacionan estos entornos con el abierto $U$. Dejemos, por un momento, de
lado este argumento y veamos algunas otras propiedades de la variedad $M$.

Supongamos, ahora, que $(U,\varphi)$ y que $(V,\psi)$ son cartas en $M$.
Supongamos, adem\'{a}s, que $U$ y $V$ son conexos, es decir, homeomorfos
v\'{\i}a $\varphi$ y $\psi$, respectivamente, a intervalos abiertos en
$\bb{R}$. Sin p\'{e}rdida de generalidad, podemos asumir que
$\varphi(U)=(a,b)$ y $\psi(V)=(c,d)$ para ciertos n\'{u}meros reales
$a<b$ y $c<d$. La intersecci\'{o}n de estos abiertos es una uni\'{o}n
de sus componentes conexas, las cuales son a lo sumo numerables.

\begin{propoALoSumoDosComponentes}\label{thm:alosumodoscomponentes}
	La intersecci\'{o}n $U\cap V$ tiene cero, una o dos componentes
	conexas.
\end{propoALoSumoDosComponentes}

La cantidad de componentes es cero, si y s\'{o}lo si $U\cap V=\varnothing$.
Si $U\subset V$ o $V\subset U$, entonces la intersecci\'{o}n consiste en una
\'{u}nica componente. Supongamos que no estamos en ninguno de estos casos,
es decir, tanto $U\cap V$, como $U\setmin V$, como $V\setmin U$ son no
vac\'{\i}os y sea $W$ alguna de todas las componentes. El conjunto $W$ es
abierto dentro de $U\cap V$ porque $V$ y $U$ son abiertos y $M$ es localmente
conexa. En particular, $W$ es abierto en $M$.

\begin{lemaEsSubintervaloExterno}\label{thm:essubintervaloexterno}
	La imagen de $W$ por $\varphi$ es $\varphi(W)=(a,r)$, o bien
	$\varphi(W)=(s,b)$ para cierto $r>a$ o $s<b$. Por simetr\'{\i}a
	en los roles de $U$ y $V$, debe valer que $\psi(W)=(c,t)$, o bien
	$\psi(W)=(u,d)$ para cierto $t>c$ o $u<d$.
\end{lemaEsSubintervaloExterno}

Supongamos, para llegar a una contradicci\'{o}n, que no es este el caso,
es decir, como $W$ (y tambi\'{e}n $\varphi(W)$) es conexo, existe
$\epsilon>0$ tal que
\begin{align*}
	\sup\left\lbrace\varphi(x)\,:\,x\in W\right\rbrace\leq b-\epsilon
	& \quad\text{e}\quad
	\inf\left\lbrace\varphi(x)\,:\,x\in W\right\rbrace \geq a+\epsilon
	\text{ .}
\end{align*}
%
En ese caso, $\varphi(W)$ se un subintervalo propio de $(a,b)$ contenido
en $[a+\epsilon,b-\epsilon]$. Entonces $\varphi(W)=(\xi,\upsilon)$ para
ciertos $\xi,\upsilon$ pertenecientes a $[a+\epsilon,b-\epsilon]$. Sean
$x=\varphi^{-1}(\xi)$ e $y=\varphi^{-1}(\upsilon)$ los puntos
correspondientes en $I_{\epsilon}=\varphi^{-1}([a+\epsilon,b-\epsilon])%
\subset U$. Como todos estos conjuntos est\'{a}n contenidos dentro del
dominio de la carta $\varphi$ e $I_{\epsilon}$ es compacto y, por lo
tanto, cerrado en $M$, se deduce que
\begin{align*}
	I_{\epsilon} & \,\supset\, \clos{W}^{M}\,=\,\clos{W}^{U}
		\,=\,\varphi^{-1}([\xi,\upsilon])
	\text{ .}
\end{align*}
%

Veamos que los puntos $x$ e $y$ no pertenecen a $V$. Si $x\in V$, entonces
$W\cup\{x\}$ es conexo (porque $x\in\clos{W}$) y est\'{a} incluida en
$U\cap V$. Pero las componentes de $U\cap V$ son abiertas y, por lo tanto,
abiertas en $U$. Lo mismo podemos decir de los conjuntos
$W\cup\{y\}$ y $W\cup\{x,y\}$, si $y$ o ambos puntos pertenecen a $V$.
Pero ninguno de ellos es abierto en $U$, pues son, v\'{\i}a $\varphi$, o
bien intervalos cerrados y abiertos (si se agrega uno s\'{o}lo de $x$ e $y$),
o bien un intervalo cerrado (si se agregan ambos). En definitiva, en
cualquiera de estos tres casos, se deduce que $W$ no es una componente
conexa, un subconjunto conexo maximal en $U\cap V$. Concluimos, as\'{\i},
que $x\not\in V$ e $y\not\in V$.

Manteniendo las hip\'{o}tesis sobre $W$, entonces, sabemos que $W$ es
conexo, que $\psi(W)$ es conexo y, en particular, un intervalo abierto en
$\psi(V)=(c,d)$, y que $\clos{W}\not\subset V$. Pero, por el mismo
argumento que antes, si $W$ es un subintervalo propio de $V$,
$\clos{W}^{V}\setmin W$ es un conjunto no vac\'{\i}o. Ahora bien, como
la clausura en $V$, $\clos{W}^{V}$ est\'{a} contenida en la clausura en
$M$, $\clos{W}^{M}=W\cup\{x,y\}$, debe ser cierto que $x\in V$ o que
$y\in V$, lo cual contradice la conclusi\'{o}n del p\'{a}rrafo anterior.
En definitiva, llegamos a una contradicci\'{o}n suponiendo que
$\varphi(W)$ estaba contenido en un subintervalo compacto de $(a,b)$.
Entonces, $\varphi(W)$ debe tener puntos arbitrariamente cerca de alguno
(o ambos) de los extremos $a$ y $b$. Como $\varphi(W)$ es un intervalo,
porque $W$ es conexo, concluimos que $\varphi(W)$ debe ser igual a un
subintervalo externo, es decir que $\varphi(W)=(a,r)$ con $a<r\leq b$
o que $\varphi(W)=(s,b)$ con $a\leq s <b$.

Esto concluye la demostraci\'{o}n del lema \ref{thm:essubintervaloexterno}
y se deduce que, como hay s\'{o}lo dos subintervalos externos en cada
intervalo, que $\varphi(W)$ (e, igualmente, $\psi(W)$) es uno de dos
subintervalos. En definitiva, $W$ es una de a lo sumo dos posibiles
componentes $U\cap V$. Concluimos, de esta manera, la demostraci\'{o}n
de \ref{thm:alosumodoscomponentes}.

Supongamos ahora que $(U,\varphi)$ y $(V,\psi)$ son cartas en $M$ tales
que $\varphi(U)=(a,b)$ y $\psi(V)=(c,d)$ con $a<b$ y $c<d$ n\'{u}meros
reales. Supongamos que la intersecci\'{o}n $U\cap V$ es no vac\'{\i}a y
tiene exactamente una componente conexa. Entonces, sin p\'{e}rdida de
generalidad, podemos asumir que $\varphi(U\cap V)=(c',b)$ y que
$\psi(U\cap V)=(c,b')$. En otro caso, como s\'{o}lo hay dos maneras de
orientar un intervalo, componiendo alguna de las dos cartas con una
inversi\'{o}n de los extremos (por ejemplo, la funci\'{o}n lineal
$(a,b)\rightarrow(a,b)$ tal que $a\mapsto b$ y $b\mapsto a$) y,
posiblemente, intercambiando los roles de $U$ y de $V$, volvemos a la
situaci\'{o}n del primer caso. Las funciones coordenadas $\varphi$ y
$\psi$ son homeomorfismos de $U$ y de $V$, respectivamente, con
intervalos reales.

\begin{propoExtenderHomeoConIntervalo}\label{thm:extenderhomeoconintervalo}
	Bajo las hip\'{o}tesis y con la notaci\'{o}n del p\'{a}rrafo
	anterior, existe un homeomorfismo $f:\,U\cup V\rightarrow (a,d')$
	con $d'>b$ tal que $f=\varphi$ en $U$. An\'{a}logamente,
	existe un homeomorfismo $g:\,U\cup V\rightarrow (a',d)$
	con $a'<c$ tal que $g=\psi$ en $V$.
\end{propoExtenderHomeoConIntervalo}

Es decir, dados dos homeomorfismos de abiertos de $M$ con intervalos reales,
si la intersecci\'{o}n de losabiertos es conexa y no vac\'{\i}a,
es decir, posee exactamente una componente conexa, entonces podemos extender
cualquiera de los dos homeomorfismos a un homeomorfismo en la uni\'{o}n.
En t\'{e}rminos de cartas, dadas dos cartas coordenadas para $M$ cuya
intersecci\'{o}n consiste en una \'{u}nica componente, se puede extender a
una carta en la uni\'{o}n de los dominios coordenados extendiende
efectivamente alguno de las dos funciones coordenadas.

Notemos que los cambios de cartas $\varphi\circ\psi^{-1}$ y
$\psi\circ\varphi^{-1}$ son homeomorfismos entre intervalos (abiertos)
de $\bb{R}$. En particular, $\varphi\circ\psi^{-1}:\,(c,b')\rightarrow (c',b)$
es mon\'{o}tona creciente (esto se debe a la hip\'{o}tesis acerca de la
forma en que se intersecan los intervalos) y tambi\'{e}n lo es su inversa
$\psi\circ\varphi^{-1}:\,(c',b)\rightarrow (c,b')$. Sea
$f:\,U\cup V\rightarrow\bb{R}$ la funci\'{o}n partida
\begin{align*}
	f(x) & \,=\,
		\begin{cases}
			\varphi(x) & \quad\text{si }x\in U \\
			\psi(x) - (b'-b) & \quad\text{si }x\in V\setmin U
		\end{cases}
	\text{ .}
\end{align*}
%
La imagen de $f$ es igual a
\begin{align*}
	f(U\cup V) & \,=\,f(U)\cup f(V\setmin U) \\
	& \,=\,(a,b)\cup \left[\psi(\psi^{-1}(b'))-(b'-b),d-(b'-b)\right)
		\,=\,(a,d')
\end{align*}
%
con $d'=d-(b'-b)$. Como $f|_{U}=\varphi$, la restricci\'{o}n de $f$ a $U$
es un homeomorfismo con su imagen.
Por otro lado, para ver que $f|_{V}$ tambi\'{e}n es un homeomorfismo
con su imagen, es suficiente mostrar que $f|_{V}\circ\psi^{-1}$ lo es.
Pero esta composici\'{o}n es continua, pues
\begin{align*}
	f|_{V}\circ\psi^{-}(u) & \,=\,
		\begin{cases}
			\varphi\circ\psi^{-1}(u) & \quad\text{si }
				u\in\psi(U\cap V)=(c,b') \\
			u-(b'-b) & \quad\text{si no}
		\end{cases} \\
	& \,=\,
		\begin{cases}
			\varphi\circ\psi^{-1}(u) & \quad\text{si } u<b' \\
			b & \quad\text{si } u=b' \\
			u-(b'-b) & \quad\text{si } u>b'
		\end{cases}
	\text{ .}
\end{align*}
%
El \'{u}nico problema de continuidad de esta funci\'{o}n est\'{a} en
$u=b'$, pero, por un lado,
$(\varphi\circ\psi^{-1})^{-1}((b-\epsilon,b))=(\text{algo},b)$, con lo
que la funci\'{o}n es continua inferiormente en $u=b'$ (por ejemplo, usando
sucesiones), y, por otro lado, $u-(b'-b)$ tiende a $b$, si $u\to b'$. Con
lo cual, la funci\'{o}n $f|_{V}\circ\psi^{-1}$ es continua. Adem\'{a}s,
esta funci\'{o}n es invertible: su inversa est\'{a} dada por
\begin{align*}
	v & \,\mapsto\,
		\begin{cases}
			\psi\circ\varphi^{-1}(v) & \quad\text{si } v<b \\
			b' & \quad\text{si } v=b \\
			v-(b-b') & \quad\text{si } v>b
		\end{cases}
	\text{ .}
\end{align*}
%
Vemos, entonces, que la inversa tiene la misma forma que
$f|_{V}\circ\psi^{-1}$. Argumentando de manera similar, se deduce que la
inversa tambi\'{e}n es continua y que, por lo tanto, $f|_{V}$ es
homeomorfismo con su imagen. En consecuencia, $f:\,U\cup V\rightarrow (a,d')$
es un homeomorfismo que coincide con $\varphi$ en $U$. La demostraci\'{o}n
de que existe una extensi\'{o}n $g:\,U\cup V\rightarrow (a',d)$ de $\psi$ es
an\'{a}loga.

Pasemos finalemente a la demostraci\'{o}n del teorema \ref{thm:clastop}.
Sea $U$ un elemento maximal de la familia $\cal{O}$. Recordemos
que esto quiere decir que $U$ es abierto y conexo en $M$ homeomorfo a
$\bb{R}$ y que no est\'{a} contenido propiamente en otro abierto as\'{\i}.
Supongamos que $U\not =M$. Como $M$ es una variedad conexa, debe valer
que $U\not =\clos{U}$. Si tomamos un punto $x\in\clos{U}\setmin U$ y un
entorno conexo $V$ de $x$, entonces $U\cap V$ es no vac\'{\i}a y, por
\ref{thm:alosumodoscomponentes}, tiene una o dos componentes. Si la
intersecci\'{o}n $U\cap V$ tiene una \'{u}nica componente, entonces,
la uni\'{o}n $U\cup V$ es abierta, conexa y, por
\ref{thm:extenderhomeoconintervalo}, admite un homeomorfismo con $\bb{R}$.
Pero esto est\'{a} en contradicci\'{o}n con la maximalidad de $U$.
Concluimos que si $x\in\clos{U}\setmin U$, todo entorno conexo de $x$
intersecado con $U$ posee dos componentes. Sea $(V,\psi)$ una carta en
$x$ con $V$ conexo. Sea $\{V_{n}\}_{n\geq 1}$ una base de entornos conexos
de $x$ contenida en $V$ y que decrece a $x$, es decir, $V_{n}\supset V_{m}$,
si $m\geq n$. Tomando coordenadas,
\begin{align*}
	\psi(V_{n}) & \,=\,(a_{n},b_{n})
\end{align*}
%
con $a_{n}$ una sucesi\'{o}n que crece a $x$ y $b_{n}$ una sucesi\'{o}n
que decrece a $x$. Como $U\cap V$ tiene dos componentes,
\begin{align*}
	\psi(U\cap V) & \,=\,(a,r)\cup (s,b)
\end{align*}
%
para ciertos reales $r,s$. Pero $U\cap V_{n}$ tambi\'{e}n tiene dos
componentes, con lo que $U\cap V_{n}$ debe, en particular, contener un
punto en $(a_{n},x)$ y un punto en $(x,b_{n})$. Pero como est es cierto
para todo $n$, las componentes conexas de $U\cap V$ deben ser iguales
exactamente a $(a,x)$ y a $(x,b)$.

Concluimos que, si $M\not =\bb{R}$, entonces $M\not =U$ y $\clos{U}\not =U$.
Repitiendo el argumento expuesto al comienzo tratando el caso
$M\setmin U=\{\infty\}$, se deduce que, para cada $x\in\clos{U}$, el
subconjunto $U\cup\{x\}$ es un subespacio homeomorfo a $\esfera{1}$.
En particular, $U\cup\{x\}$ es compacto y, por lo tanto, cerrado en $M$.
Pero entonces $\esfera{1}=U\cup\{x\}=\clos{U}$.

Resta verificar que $\clos{U}=M$. Si no fuese cierto e
$y\in M\setmin\clos{U}$, podemos, porque $M$ es arcoconexa, tomar un camino
de un punto $x\in U$ a $y$. Sea $\gamma:\,[0,1]\rightarrow M$ un camino
continuo con $\gamma(0)=x$ y $\gamma(1)=y$. Si $t\in (0,1)$ es tal que
$\gamma(t')\in U$ para todo $t'<t$, entonces $\gamma(t)\in\clos{U}$. Si
$\gamma(t)\not\in U$, entonces $\gamma(t'')\not\in U$ para $t''>t$, pues
$U$ y $\gamma([0,1])$ son conexos. En particular, tomando un entorno conexo
$V$ de $\gamma(t)$, $U\cap V$ tiene una o dos componentes. En el segundo
caso se deduce que $\gamma$ debe ser constante a partir de $t$ y que,
entonces $y=\gamma(t)\in\clos{U}$, lo cual es absurdo. En el primer caso,
podemos definir un homeomorfismo de $U\cup V$ en $\bb{R}$, contradiciendo
la maximalidad de $U$. Estas contradicciones provienen de asumir que
$\gamma(t)$ no pertenece a $U$. Consideremos, ahora, el supremo
\begin{align*}
	\sigma & \,=\,\sup\left\lbrace
		t\in[0,1]\,:\,\gamma(t')\in U\text{ para todo }t'<t
		\right\rbrace
	\text{ .}
\end{align*}
%
Como $U$ es abierto, $\sigma >0$ estrictamente. Por el argumento de
reci\'{e}n, debe valer $\sigma=1$. En definitva $y\in\clos{U}$,
lo cual es, nuevamente, una contradicci\'{o}n, proveniente de
suponer que $\clos{U}\not =M$. Esto concluye la demostraci\'{o}n de
\ref{thm:clastop}.

Habiendo clasificado las variedades sin borde conexas de dimensi\'{o}n uno,
pasamos a dar la clasificiaci\'{o}n para variedades con borde. Si $M$
es una variedad conexa con borde de dimensi\'{o}n uno, entonces
el interior (en tanto variedad) $\interior{M}$ es una variedad sin borde,
conexa y de dimensi\'{o}n uno. Por el teorema \ref{thm:clastop},
$\interior{M}=\esfera{1}$ o $\interior{M}=\bb{R}$. Por otro lado,
$\clos{\interior{M}}=M$. En el primer caso, $\interior{M}$ resulta ser una
variedad compacta y, en particular, un conjunto cerrado de $M$, pero
entonces $M=\interior{M}=\esfera{1}$. En el segundo caso, si suponemos que
$\interior{M}=(a,b)$, entonces las \'{u}nicas opciones para variedades
con borde cuyo interior es $(a,b)$ son $(a,b)$, $[a,b)$, $(a,b]$ y $[a,b]$.
En definitiva, hemos demostrado el siguiente resultado.

\begin{teoClasTopConBorde}\label{thm:clastopconborde}
	Sea $M$ una variedad topol\'{o}gica, posiblemente con borde. Si
	$M$ es conexa, entonces $M$ es homeomorfa a $\esfera{1}$ o a
	$\bb{R}$, si $\borde[M]=\varnothing$, a un intervalo abierto y
	cerrado o a un intervalo cerrado propio (compacto) de $\bb{R}$,
	si el borde es no vac\'{\i}o. En particular,
	$\#\borde[M]=0,1,2$. Si $M$ es compacta,
	$M$ es homeomorfa a $\esfera{1}$ o a un intervalo compacto. En
	tal caso $\#\borde[M]=0,2$.
\end{teoClasTopConBorde}

%
\section{El toro}
%

\begin{ejemplo}\nom{La exponencial}
	Sea $\varepsilon:\,\bb{R}\rightarrow S^{1}$ la funci\'{o}n
	\begin{align*}
		\varepsilon(t) & \,=\,\exp{2\pi i t}
		\text{ .}
	\end{align*}
	%
	Esta funci\'{o}n es suave. Una opci\'{o}n es pensar que $S^{1}$
	no es el grupo multiplicativo de los complejos de valor absoluto
	$1$, sino la esfera de dimensi\'{o}n $1$ contenida en el plano, es
	decir, como subariedad de $\bb{R}^{2}$.
	
	Sea $\inc[\esfera{1}]:\,\esfera{1}\rightarrow\bb{R}^{2}$ la
	inclusi\'{o}n de la esfera en el plano y sea
	$\widehat{\varepsilon}=\varepsilon\circ\inc[\esfera{1}]$.
	La funci\'{o}n $\widehat{\varepsilon}$ se obtiene, tambi\'{e}n,
	identificando $\bb{C}$ con el plano $\bb{R}^{2}$ --es decir, tomando
	las coordenadas usuales en $\bb{C}$-- e incluyendo $S^{1}$ en
	$\bb{C}^{\times}$ como subgrupo y \'{e}ste a su vez en $\bb{C}$ como
	abierto. Vemos, en todo caso, que
	\begin{align*}
		\widehat{\varepsilon}(t) & \,=\,(\cos 2\pi t,\,\sin 2\pi t)
		\text{ .}
	\end{align*}
	%
	Sean $U_{1}^{\pm}$ y $U_{2}^{\pm}$ los dominios de las cartas
	usuales para la esfera $\esfera{1}$. Con estas coordenadas en $S^{1}$
	y las coordenadas usuales en $\bb{R}$, vale, por ejemplo, que
	\begin{align*}
		\varphi_{1}^{+}\circ\varepsilon(t) & \,=\,\pi_{1}\big(
			\widehat{\varepsilon}(t)\big) \,=\,
			\cos 2\pi t
		\text{ ,}
	\end{align*}
	%
	donde $\pi_{1}:\,\bb{R}^{2}\rightarrow\bb{R}$ es la proyecci\'{o}n
	en la primer coordenada. An\'{a}logamente, $\pi_{2}$ es la
	proyecci\'{o}n en la segunda. Como sabemos que $\cos 2\pi t$
	y $\sin 2\pi t$ son funciones suaves en el sentido usual,
	concluimos que $\varepsilon$ es suave.

	Otra opci\'{o}n es usar las coordenadas en $S^{1}$ dadas por tomar
	\'{a}ngulo. En estas coordenadas, en un abierto suficientemente
	peque\~{n}o alrededor de $t$, $\widehat{\varepsilon}(t)=t+c$ para
	alguna constante real $c$. Deducimos entonces que $\varepsilon$ es
	suave.

	Un poco m\'{a}s en general definimos el \emph{toro de dimensi\'{o}n %
	$n$} como el producto de $n$ copias de $S^{1}$. Si
	$\varepsilon:\,\bb{R}^{n}\rightarrow\toro[n]$ es la funci\'{o}n
	\begin{align*}
		\varepsilon(\lista*{t}{n}) & \,=\,
		\big(\exp{2\pi it^{1}},\,\dots,\,\exp{2\pi it^{n}}\big)
	\end{align*}
	%
	es suave, por ser un producto finito de funciones suaves.
\end{ejemplo}

\begin{ejemplo}\nom{La funci\'{o}n \'{a}ngulo}
	Dado un abierto $U\subset S^{1}$, una funci\'{o}n
	\'{a}ngulo en $U$ es una funci\'{o}n continua
	$\theta:\,U\rightarrow\bb{R}$ tal que $e^{i\theta(z)}=z$ para todo
	$z\in U$, es decir, es una secci\'{o}n (local) continua de la
	exponencial $(\tau\mapsto e^{i\tau}):\,\bb{R}\rightarrow S^{1}$.
	Para ser consistentes con la definici\'{o}n de la exponencial en
	los otros ejemplos, llamamos \emph{funci\'{o}n \'{a}ngulo} a las
	secciones locales continuas de $t\mapsto\exp{2\pi it}$.

	Todo punto de $S^{1}$ es de la forma $z=e^{2\pi i\tau}$ para alg\'{u}n
	$\tau\in\bb{R}$. Si $z=e^{2\pi i\tau}=e^{2\pi i\tau'}$ entonces
	$\tau-\tau'=2\pi k$ para alg\'{u}n entero $k$.
	Sea $U\subset S^{1}$ es un subconjunto propio de la forma
	$U=S^{1}\setmin\{z_{0}\}$ con $z_{0}=e^{2\pi i\tau_{0}}$. En este
	subconjunto, todo punto se puede representar de manera \'{u}nica
	como $e^{2\pi i\tau}$ con $\tau\in(\tau_{0}-1,\tau_{0}+1)$.
	Para cada $n\geq 1$, definimos
	$K_{n}=[\tau_{0}-1+\frac{1}{n},\tau_{0}+1-\frac{1}{n}]$. Estos
	subconjuntos del intervalo abierto centrado en $\tau_{0}$ son
	compactos. Por otro lado, la funci\'{o}n
	$e:\,\tau\mapsto e^{2\pi i\tau}$ es continua: la noci\'{o}n de
	cercan\'{\i}a en $U$, como en $S^{1}$ es la determinada por el valor
	absoluto en $\bb{C}$ en tanto subespacios del plano complejo, entonces
	\begin{align*}
		|z-z'|^{2} & \,=\, |\cos(2\pi\tau)-\cos(2\pi\tau')|^{2}
			\,+\,|\sin(2\pi\tau)-\sin(2\pi\tau')|^{2} \\
		& \,\leq\, (2\pi)^{2} |\tau-\tau'|^{2}\quad\text{y} \\
		|z-z'| & \,\leq\, 2\pi |\tau-\tau'|
		\text{ .}
	\end{align*}
	%
	Como la exponencial $e$ es continua, cada $K_{n}$ es compacto y $U$
	es $T_{2}$, la funci\'{o}n $e$ restringida a $K_{n}$ es homeomorfismo,
	pues es una biyecci\'{o}n. Sea $\theta:\,U\rightarrow%
	(\tau_{0}-1,\tau_{0}+1)$ la inversa de la exponencial, es decir,
	$\theta(z)=\tau$, donde $\tau$ es el \'{u}nico valor real en el
	intervalo tal que $z=e(\tau)$. Veamos que $\theta$ es continua.
	Sea $z\in U$. Entonces existe $n\geq 1$ tal que $z$ pertenece a
	$U_{n}=e(\interior{K_{n}})$, que es abierto en $U$, por ser la
	imagen v\'{\i}a el homeomorfismo
	$e|_{K_{n}}:\,K_{n}\rightarrow e(K_{n})\subset U$ de un abierto
	contenido en el compacto. En $U_{n}$, la funci\'{o}n $\theta$
	coincide con $e|_{K_{n}}^{-1}$ que es continua. Entonces, como los
	abiertos $U_{n}$ cubren $U$, $\theta$ es continua en $U$.

	Si $U\subset S^{1}$ es un subconjunto propio arbitrario, est\'{a}
	incluido en alg\'{u}n subcojunto de la forma $S^{1}\setmin\{z_{0}\}$
	y all\'{\i} podemos definir una funci\'{o}n \'{a}ngulo como
	en el p\'{a}rrafo anterior. Como $U$ es un subespacio, la
	restricci\'{o}n de esta funci\'{o}n a $U$ es continua.
	En $S^{1}$ no puede existir una funci\'{o}n \'{a}ngulo global:
	si $\theta:\,S^{1}\rightarrow\bb{R}$ es continua, entonces, la
	imagen es conexa, si es una secci\'{o}n de $e$, entonces debe ser
	inyectiva, y, en particular, debe ser subespacio, con lo que
	la imagen est\'{a} forzada a ser un intervalo cerrado y $\theta$
	un homeomorfismo con su imagen, pero los intervalos son simplemente
	conexos y $S^{1}$ no\dots
\end{ejemplo}

\begin{ejemplo}\nom{Coordenadas angulares}
	Las funciones \'{a}ngulo definidas en el ejemplo anterior pueden
	usarse para definir cartas en $S^{1}$. Dado $U\subset S^{1}$
	un subconjunto propio, existe una funci\'{o}n \'{a}ngulo
	$\theta:\,U\rightarrow\bb{R}$ continua definida en $U$. Esta
	funci\'{o}n determina un homeomorfismo sobre su imagen, su inversa
	est\'{a} dada por la restricci\'{o}n de la exponencial. En definitiva,
	si $U\subset S^{1}$ es abierto y $\theta$ es una funci\'{o}n
	\'{a}ngulo definida en $U$, entonces el par $(U,\theta)$ es una
	carta para $S^{1}$. Como los dominios de estas cartas cubren a
	$S^{1}$, deducimos que el c\'{\i}rculo es una variedad topol\'{o}gica
	de dimensi\'{o}n $1$. Notemos que la topolog\'{\i}a en $S^{1}$
	como subespacio de $\bb{C}^{\times}$ coincide con la topolog\'{\i}a
	de la esfera $\esfera{1}$ de dimensi\'{o}n $1$ contenida en
	$\bb{R}^{2}$. Si bien sab\'{\i}amos que $S^{1}$ tiene una
	topolog\'{\i}a que la hace una variedad topol\'{o}gica, acabamos
	de demostrarlo usando \emph{otra} estructura euclidea local, otro
	cubrimiento por cartas, en principio, distinto del dado por
	lo que llamamos cartas usuales en la esfera, los abiertos
	$U_{i}^{\pm}$ y las proyecciones
	$\varphi_{i}^{\pm}:\,U_{i}^{\pm}\rightarrow\bola{1}{0}$.

	Vamos a demostrar que las coordenadas angulares, $(U,\theta)$ son
	compatibles entre s\'{\i} y, adem\'{a}s, compatibles con la
	estructura usual de la esfera $\esfera{1}$. Sean $U$ y $U'$ abiertos
	propiamente contenidos en $S^{1}$ y sean $\theta$ y $\theta'$
	funciones \'{a}ngulo, secciones de $\exp{2\pi it}$, definidas en
	$U$ y en $U'$, respectivamente. Sea $W\subset U\cap U'$ una
	componente conexa. Como $\theta$ y $\theta'$ son homeomorfismos,
	$\theta(W)$ es un subintervalo de $(\tau_{0}-1,\tau_{0}+1)=I$ y
	$\theta'(W)$ es un subintervalo de $(\tau_{0}'-1,\tau_{0}'+1)=I'$ para
	ciertos n\'{u}meros reales $\tau_{0}$ y $\tau_{0}'$. Si $w\in W$,
	$\theta(w)=\tau\in I$ y $\theta'(w)=\tau'\in I'$ y
	\begin{align*}
		w & \,=\, \exp{2\pi i\tau} \,=\,\exp{2\pi i\tau'}
		\text{ .}
	\end{align*}
	%
	En particular, $\tau=\tau'+ k(w)$ con $k(w)\in\bb{Z}$. Entonces
	tenemos una funci\'{o}n $k:\,W\rightarrow\bb{Z}$ dada por
	\begin{align*}
		k(w) & \,=\,\theta(w)-\theta'(w)
		\text{ .}
	\end{align*}
	%
	Por ser una diferencia de funciones continuas, $k$ es continua.
	Pero $k$ tiene imagen en un conjunto discreto, entonces debe
	ser localmente constante. Como la imagen $k(W)\subset\bb{Z}$ es
	conexa, debe ser $k(w)=k_{0}$ para todo $w\in W$. En definitiva,
	hemos demostrado que los cambios de carta $\theta\circ(\theta')^{-1}$
	est\'{a}n dados, en cada componente conexa de la intersecci\'{o}n
	de los dominios, por sumar una constante entera, lo cual muestra
	que los cambios de coordenada angular son tan regulares como
	pueden serlo.

	Veamos, finalmente, que las cartas $(U,\theta)$ son compatibles
	con las cartas usuales en $\esfera{1}$. Lo demostraremos
	en un caso particular, los otros casos son an\'{a}logos.
	Supongamos que $U$ es $S^{1}\setmin\{-1\}$ y que $V=U_{i}^{\pm}$ es
	$\{x>0\}$. Sea $\theta:\,U\rightarrow (-1,1)$ la
	funci\'{o}n \'{a}ngulo definida en $U$ y sea $\psi(x,y)=x$
	la coordenada correspondiente en $V$. Entonces
	\begin{align*}
		\psi\circ\theta^{-1}(\tau) & \,=\,
			\psi\big(\exp{2\pi i\tau}\big)\,=\,
			\cos 2\pi\tau\quad\text{y} \\
		\theta\circ\psi^{-1}(x) & \,=\,
			\theta\big(x,\sqrt{1-|x|^{2}}\big)\,=\,
			\frac{1}{2\pi}\mathsf{arc\,cos}\, x
		\text{ .}
	\end{align*}
	%
	Notemos que
	\begin{align*}
		(\mathsf{arc\,cos}\,x)' & \,=\,
			\frac{-1}{\sin(\mathsf{arc\,cos}\,x)} \,=\,
			\frac{-1}{\sqrt{1-|x|^{2}}}
	\end{align*}
	%
	que es suave en $|x|<1$.
\end{ejemplo}

\begin{ejemplo}\nom{La exponencial es un difeomorfismo local}
	Sea $\varepsilon:\,\bb{R}\rightarrow S^{1}$ la funci\'{o}n
	exponencial $\varepsilon(t)=\exp{2\pi it}$. Esta funci\'{o}n
	es un difeomorfismo local. En un entorno de cada punto, usando
	las coordenadas angulares, $\varepsilon$ tiene una expresi\'{o}n
	en coordenadas de la forma $t\mapsto 2\pi t+c$ para alguna constante
	$c\in\bb{R}$, que es un difeomorfismo. Lo mismo se puede decir
	de la exponencial en el toro $\toro[n]$ de dimensi\'{o}n $n$, ya
	que esta funci\'{o}n es un producto de difeomorfismos locales.
\end{ejemplo}


%

%--------

\chapter{Notas sueltas}

\section{Existencia de marcos continuos sobre curvas diferenciables}
%
Sea $M$ una variedad diferencial de dimensi\'{o}n $n$ y sea
$\gamma:\,I\rightarrow M$ una curva diferenciable definida en el intervalo
$I=[0,1]$. Un \emph{campo continuo sobre $\gamma$} es una aplicaci\'{o}n
continua $X:\,I\rightarrow\tangente{M}$ tal que $X=\pi\circ\gamma$, es decir,
$X(t)\in\tangente[\gamma(t)]{M}$ es un vector tangente a $M$ en el punto
$\gamma(t)$ para cada instante $t$. Un \emph{marco continuo sobre $\gamma$}
es un conjunto ordenado $\{\lista{E}{n}\}$ de campos continuos sobre $\gamma$
tal que, para cada $t\in I$, el conjunto ordenado
$\{E_{1}(t),\,\dots,\,E_{n}(t)\}$ es una base del espacio tangente
$\tangente[\gamma(t)]{M}$. En esta secci\'{o}n se demostrar\'{a} que siempre
existen marcos continuos sobre curvas diferenciales.

Sea entonces $M$ una variedad diferencial y sea $\gamma:\,I\rightarrow M$
una curva diferenciable. Dado $p\in\gamma(I)$, se puede elegir una carta
$(U,\varphi)$ centrada en $p$, compatible con la estructura diferencial
de $M$. Sea $\cal{U}=\left\lbrace (U_{t},\varphi_{t})\,:\,%
t\in I\right\rbrace$ la colecci\'{o}n dada por una posible elecci\'{o}n
de cartas compatibles tales que $(U_{t},\varphi_{t})$ est\'{e} centrada en
$\gamma(t)$. En particular, $\cal{U}$ es un cubrimiento de la imagen
$\gamma(I)$ por abiertos de $M$. Dado que $\gamma:\,I\rightarrow M$ es
continua, las preim\'{a}genes $\gamma^{-1}(U_{t})\subset I$ son abiertas
y $t\in\gamma^{-1}(U_{t})$ para cada instante $t\in I$. En particlar,
$\left\lbrace\gamma^{-1}(U_{t})\,:\,t\in I\right\rbrace$ es un cubrimiento
por abiertos del inervalo $I$. Dado que $I$ es un espacio m\'{e}trico
compacto, por el \emph{lema del n\'{u}mero de Lebesgue}, existen finitos
instantes $a_{0}=0<a_{1}<\,\cdots\,<a_{r}=1$ tales que, para
$i\in[\![0,r-1]\!]$, el subintervalo $[a_{i},a_{i+1}]$ est\'{e} contenido
en alg\'{u}n abierto $V_{i}=\gamma^{-1}(U_{t_{i}})$ del cubrimiento
del intervalo.

Sea $\gamma_{i}=\gamma|_{[a_{i},a_{i+1}]}$ la restricci\'{o}n de la curva
$\gamma$ a alguno de los subintervalos $[a_{i},a_{i+1}]$. Cada una de las
curvas $\gamma_{i}:\,[a_{i},a_{i+1}]\rightarrow M$ es diferenciable y
admite un marco continuo: sea $E^{i}_{k}:\,%
[a_{i},a_{i+1}]\rightarrow\tangente{M}$ la aplicaci\'{o}n
\begin{align*}
	E^{i}_{k}(t) & \,=\,\gancho[\gamma(t)]{x_{i}^{k}}
\end{align*}
%
donde $\gancho[\gamma(t)]{x_{i}^{k}}$ denota la derivaci\'{o}n en
$\tangente[\gamma(t)]{M}$ determinada por las funciones coordenadas
$\varphi_{t_{i}}=(x_{i}^{l})$ correspondiente a la funci\'{o}n
$x_{i}^{k}:\,U_{t_{i}}\rightarrow\bb{R}$. La cuesti\'{o}n es c\'{o}mo
pegar estos marcos de manera obtener un marco continuo sobre $\gamma$.
Para cada \'{\i}ndice $i\in[\![0,r-1]\!]$, los campos
$\Big\{\gancho{x_{i+1}^{k}}\Big\}_{k}$ se pueden escribir en t\'{e}rminos
de los campos $\Big\{\gancho{x_{i}^{k}}\Big\}_{k}$: para cada punto
$p\in U_{t_{i}}\cap U_{t_{i+1}}$, existen coeficientes
$A^{i,k}_{l}(p)\in\bb{R}$ tales que
\begin{align*}
	\gancho[p]{x_{i+1}^{l}} & \,=\, A^{i,k}_{l}(p)\,
		\gancho[p]{x_{i}^{k}}
\end{align*}
%
(sin sumar sobre $i$). De esta manera, dado que
$\Big\{\gancho[p]{x_{i+1}^{k}}\Big\}_{k}$ y que
$\Big\{\gancho[p]{x_{i}^{k}}\Big\}_{k}$ forman bases del espacio tangente
en $p$, se obtiene un elemento, una matriz, $A^{i}(p)\in\GL{n,\bb{R}}$,
cuyo coeficiente $(k,l)$ es $A^{i,k}_{l}(p)$. Dado que las cartas
$(U_{t_{i}},\varphi_{t_{i}})$ y $(U_{t_{i+1}},\varphi_{t_{i+1}})$ son
compatibles, las funciones
\begin{align*}
	A^{i,k}_{l} & \,:\,U_{t_{i}}\cap U_{t_{i+1}} \,\rightarrow\,\bb{R}
\end{align*}
%
son suaves y la aplicaci\'{o}n $A^{i}:\,%
U_{t_{i}}\cap U_{t_{i+1}}\rightarrow\GL{n,\bb{R}}$ es suave, tambi\'{e}n.
En t\'{e}rminos de los campos, vale que
\begin{align*}
	\gancho{x_{i+1}^{l}} & \,=\,A^{i,k}_{l}\,\gancho{x_{i}^{k}}
\end{align*}
%
en $U_{t_{i}}\cap U_{t_{i+1}}$. En particular, componiendo con la curva
$\gamma$, siempre que $\gamma(t)$ pertenezca a la intersecci\'{o}n,
se cumple que
\begin{align*}
	E^{i+1}_{l}(t) & \,=\,A^{i,k}_{l}(\gamma(t))\,E^{i}_{k}(t)
	\text{ .}
\end{align*}
%
Esta igualdad vale, por ejemplo, cuando $t=a_{i+1}$.

Para cada $i\in[\![0,r-1]\!]$, sea $p_{i}=\gamma(a_{i})$ y sea
$p_{r}=\gamma(a_{r})=\gamma(1)$. Cada punto de la forma $p_{i+1}$ de la
variedad pertenece a la intersecci\'{o}n $U_{t_{i}}\cap U_{t_{i+1}}$ de
abiertos coordenados. Sea $\epsilon_{i}>0$ tal que
$a_{i}<a_{i+1}-\epsilon_{i}$ y que
$\gamma([a_{i+1}-\epsilon_{i},a_{i+1}])\subset U_{t_{i}}\cap U_{t_{i+1}}$.
Sea $B(a_{i+1}-\epsilon)=I_{n}\in\GL{n,\bb{R}}$ la matriz identidad y sea
$B(a_{i+1})=A^{i}(p_{i+1})=A^{i}(\gamma(a_{i+1}))\in\GL{n,\bb{R}}$ la matriz
de coeficientes de los campos coordenados en $U_{t_{i+1}}$ en t\'{e}rminos
de los campos coordenados en $U_{t_{i}}$. Si existiese una curva
(suave) $B:\,[a_{i+1}-\epsilon_{i},a_{i+1}]\rightarrow\GL{n,\bb{R}}$
tal que $B(t)=I_{n}$ cerca de $t=a_{i+1}-\epsilon_{i}$ y tal que
$B(t)=A^{i}(p_{i+1})$ cerca de $t=a_{i+1}$, entonces se podr\'{\i}an definir
campos $F_{1},\,\dots,\,F_{n}:\,[a_{i+1}-\epsilon_{i},a_{i+1}]\rightarrow%
\tangente{M}$ dados por
\begin{align*}
	F_{l}(t) & \,=\, B(t)^{k}_{l}\,E^{i}_{k}(t) \,=\,
		B(t)^{k}_{l}\,\gancho[\gamma(t)]{x_{i}^{k}}
	\text{ .}
\end{align*}
%
Dado que $B(t)\in\GL{n,\bb{R}}$ para todo $t$ y que
$\{E^{i}_{1}(t),\,\dots,\,E^{i}_{n}(t)\}$ es base de $\tangente[\gamma(t)]{M}$,
el conjunto ordenado $\{F_{1}(t),\,\dots,\,F_{n}(t)\}$ tambi\'{e}n es base
del espacio tangente para cada instante $t$ en donde $B$ est\'{e} definida.
Adem\'{a}s, como cada una de las funciones $t\mapsto B(t)^{k}_{l}$ son
continuas (suaves), los campos $\lista{F}{n}$ sobre la restricci\'{o}n
$\gamma|_{[a_{i+1}-\epsilon_{i},a_{i+1}]}$ son continuos (suaves). En
definitiva, asumiendo que la curva $B$ existe, se obtiene un marco continuo
$\{\lista{F}{n}\}$ en $[a_{i+1}-\epsilon_{i},a_{i+1}]$ tal que
\begin{align*}
	F_{l}(a_{i+1}-\epsilon_{i}) & \,=\,
		(I_{n})^{k}_{l}\,E^{i}_{k}(a_{i+1}-\epsilon_{i}) \,=\,
		E^{i}_{l}(a_{i+1}-\epsilon_{i})
	\quad\text{y} \\
	F_{l}(a_{i+1}) & \,=\,
		A^{i,k}_{l}(\gamma(a_{i+1}))\,E^{i}_{k}(a_{i+1}) \,=\,
		E^{i+1}_{l}(a_{i+1})
\end{align*}
%
para cada $l\in[\![1,n]\!]$ (sin sumar sobre $i$). De esta manera,
concatenando con el marco $\{E^{i}_{1},\,\dots,\,E^{i}_{n}\}$ en
$[a_{i},a_{i+1}-\epsilon_{i}]$, se obtiene un marco continuo (?`suave?)
sobre $\gamma_{i}$. Pero, si $\{F^{i}_{1},\,\dots,\,F^{i}_{n}\}$ denota
el marco continuo definido en $[a_{i+1}-\epsilon_{i},a_{i+1}]$, entonces,
para cada $i$, este marco se pega bien con el marco
$\{E^{i+1}_{1},\,\dots,\,E^{i+1}_{n}\}$ en $[a_{i+1},a_{i+2}-\epsilon_{i+1}]$,
dando como resultado un marco continuo en
$[a_{i+1}-\epsilon_{i},a_{i+2}-\epsilon_{i+1}]$. Inductivamente, asumiendo
que las curvas $B:\,[a_{i+1}-\epsilon_{i},a_{i+1}]\rightarrow\GL{n,\bb{R}}$
existen para cada $i\in[\![0,r-1]\!]$, se deduce que existe un marco
continuo sobre $\gamma$.

Para concluir la demostraci\'{o}n de que existe un marco continuo sobrela curva
(diferenciable) $\gamma$, resta demostrar que existen las curvas continuas
(suaves) $B:\,[a_{i+1}-\epsilon_{i},a_{i+1}]\rightarrow\GL{n,\bb{R}}$ tales
que $B(a_{i+1}-\epsilon_{i})=I_{n}$ y $B(a_{i+1})=A^{i}(p_{i+1})$. Pero
una curva con estas propiedades existe, si y s\'{o}lo si $A^{i}(p_{i+1})$
pertenece a la componente conexa de la identidad $I_{n}$ en $\GL{n,\bb{R}}$,
es decir, si y s\'{o}lo si $\det\big(A^{i}(p_{i+1})\big)>0$. Pero esta
condici\'{o}n no se tiene por qu\'{e} cumplir, \textit{a priori}. Aun
as\'{\i}, esto se puede corregir progresivamente, a medida que se va
avanzando sobre la curva. Si $A^{0}(p_{1})$ tiene determinante negativo,
entonces, cambiando la coordenada
$\varphi_{t_{1}}=(x_{1}^{1},\,\dots,\,x_{1}^{n})$ por
$\tilde{\varphi}_{t_{1}}=(-x_{1}^{1},\,x_{1}^{2},\,\dots,\,x_{1}^{n})$,
los campos coordenados correspondientes en $U_{t_{1}}$ (el dominio no cambia)
est\'{a}n dados, en relaci\'{o}n con los anteriores, por
\begin{align*}
	\gancho{\tilde{x}_{1}^{1}} & \,=\,-\gancho{x_{1}^{1}}
	\quad\text{y} \\
	\gancho{\tilde{x}_{1}^{k}} & \,=\,\gancho{x_{1}^{k}}
	\text{ ,}
\end{align*}
%
si $k\geq 2$. Definiendo $E^{1}_{k}(t)=\gancho[\gamma(t)]{\tilde{x}^{k}_{1}}$
y $\tilde{A}^{0}:\,U_{t_{0}}\cap U_{t_{1}}\rightarrow\GL{n,\bb{R}}$ como
la aplicaci\'{o}n que a un punto $p$ le asigna la matriz cuyos coeficientes
est\'{a}n determinados por
\begin{align*}
	\gancho[p]{\tilde{x}^{l}_{1}} & \,=\,
		\tilde{A}^{0,k}_{l}(p)\,\gancho[p]{x_{0}^{k}}
	\text{ ,}
\end{align*}
%
se deduce que $\det\big(\tilde{A}^{0}(p_{1})\big)>0$. Si, ahora,
$\epsilon_{0}>0$ es tal que $a_{0}=0<a_{1}-\epsilon_{0}$ y que
$\gamma([a_{1}-\epsilon_{0},a_{1}])$, se puede elegir una
curva (posiblemente componiendo con alguno reparametrizaci\'{o}n)
$B:\,[a_{1}-\epsilon_{0},a_{1}]\rightarrow\GL{n,\bb{R}}$ tal que
$B(a_{1}-\epsilon_{0})=I_{n}$ y que $B(a_{1})=\tilde{A}^{0}(\gamma(a_{1}))$.
La existencia de esta curva permite concatenar el marco continuo
$\{E^{0}_{1},\,\dots,\,E^{0}_{n}\}$ en $[a_{0},a_{1}-\epsilon_{0}]$ con
un marco continuo $\{F^{0}_{1},\,\dots,\,F^{0}_{n}\}$ definido en
$[a_{1}-\epsilon_{0},a_{1}]$ tal que $F^{0}_{l}(a_{1}-\epsilon_{0})=E^{0}_{l}(a_{1}-\epsilon_{0})$ y que $F^{0}_{l}(a_{1})=E^{1}_{l}(a_{1})$. Concatenando
el marco $\{E^{0}_{l}\}_{l}$ en $[a_{0},a_{1}-\epsilon_{0}]$ con
$\{F^{0}_{l}\}_{l}$ definido en $[a_{1}-\epsilon_{0},a_{1}]$, seguido luego
del marco $\{E^{1}_{l}\}_{l}$ en $[a_{1},a_{2}]$, se obtiene un marco
continuo en $[a_{0},a_{2}]$. Inductivamente, si se tiene definido un
marco continuo sobre $\gamma$ en el subintervalo $[a_{0},a_{i}]$, para
extenderlo hasta $a_{i+1}$, se repite el procedimiento anterior, obteniendo
un marco continuo hasta $a_{i+1}-\epsilon_{i}$ usando el
marco dado por los campos coordenados $\gancho{x_{i+1}^{l}}$ (o bien por
los campos $\gancho{\tilde{x}_{i+1}^{l}}$ correspondientes a modificar
la funci\'{o}n coordenada $x_{i+1}^{1}$ por $\tilde{x}_{i+1}^{1}=-x_{i+1}^{1}$,
en caso de que $\det\big(A^{i}(p_{i+1})\big)<0$), seguido de un marco
de transici\'{o}n $\{F^{i+1}_{l}\}_{l}$ hasta $a_{i+2}$, si $i+1<r-1$, o
sin modificaciones, si $i+1=r-1$.


%
\section{Planos como puntos}
%\theoremstyle{plain}
\newtheorem{teoEquivalenciasTensoresElementalesRCuatro}{Teorema}[section]
\newtheorem{coroTensoresElementalesRTres}%
	[teoEquivalenciasTensoresElementalesRCuatro]{Corolario}

\theoremstyle{remark}

%-------------

Un tensor elemental en $\exterior[2]{\bb{R}^{4}}$ es, simplemente, un
elemento de la forma $f_{1}\wedge f_{2}$, donde $f_{1}$ y $f_{2}$ son
vectores en $\bb{R}^{4}$. Todo tensor elemental no nulo determina
un\'{\i}vocamente un plano en $\bb{R}^{4}$: si $f_{1}\wedge f_{2}\not =0$
entonces $\{f_{1},f_{2}\}$ es un conjunto linealmente independiente y
consituye una base de un subespacio de dimensi\'{o}n $2$ en $\bb{R}^{4}$,
es decir, un plano por el origen. Esto permitir\'{a} pensar al conjunto
de tales planos como una variedad y a cada plano por el origen como
un punto de dicha variedad. Pero para poder establecer esta correspondencia
entre planos y puntos es necesario, en primera instancia, dar una
caracterizaci\'{o}n de los tensores elementales en $\exterior[2]{\bb{R}^{4}}$.

\subsection{Tensores elementales en el producto exterior %
	$\exterior[2]{\bb{R}^{4}}$}
Sea $\phi\in\dual{(\bb{R}^{4})}$. Se define una transformaci\'{o}n
$\exterior[2]{\bb{R}^{4}}\rightarrow\exterior[1]{\bb{R}^{4}}$ que en
tensores elementales est\'{a} dada por
\begin{align*}
	\phi\convol (f_{1}\wedge f_{2}) & \,=\,
		\phi(f_{1})\,f_{2}-\phi(f_{2})\,f_{1}
\end{align*}
%
denominada \emph{convoluci\'{o}n por $\phi$}. Dados vectores $f_{1}$ y
$f_{2}$ en $\bb{R}^{4}$, o bien $\phi|_{\generado{f_{1},f_{2}}}=0$, o bien
$\phi(f_{1})\not =0$ o $\phi(f_{2})\not=0$. En todo caso,
\begin{align*}
	\big(\phi\convol(f_{1}\wedge f_{2})\big)\wedge (f_{1}\wedge f_{2}) &
		\,=\,\big(\phi(f_{1})\,f_{2}-\phi(f_{2})\,f_{1}\big)\wedge
			(f_{1}\wedge f_{2}) \,=\,0
\end{align*}
%
en $\exterior[2]{\bb{R}^{4}}$. Dicho de otra manera, el vector de $\bb{R}^{4}$
dado por $\phi\convol(f_{1}\wedge f_{2})$ pertenece al subespacio generado
por $f_{1}$ y por $f_{2}$.

Sea $x\in\exterior[2]{\bb{R}^{4}}$ un elemento arbitrario del producto
exterior. Si $\{e_{1},\,e_{2},\,e_{3},\,e_{4}\}$ es una base de
$\bb{R}^{4}$, entonces $\{e_{i}\wedge e_{j}\,:\,i<j\}$ es una base de
$\exterior[2]{\bb{R}^{4}}$. En t\'{e}rminos de esta base,
\begin{align*}
	x & \,=\,\sum_{i<j}\,a^{ij}\,e_{i}\wedge e_{j}
\end{align*}
%
para ciertos coeficientes $a^{ij}\in\bb{R}$. Si $\phi\in\dual{(\bb{R}^{4})}$,
entonces, en t\'{e}rminos de esta descomposici\'{o}n, vale que
\begin{align*}
	(\phi\convol x)\wedge x & \,=\,\Big(\sum_{i<j}\,a^{ij}\,
		\big(\phi(e_{i})\,e_{j}-\phi(e_{j})\,e_{i}\big)\Big)\wedge
		\Big(\sum_{k<l}\,a^{kl}\,e_{k}\wedge e_{l}\Big) \\
	& \,=\,\sum_{i<j}\,\sum_{k<l}\,a^{ij}a^{kl}\,
		\big(\phi(e_{i})\,e_{j}-\phi(e_{j})\,e_{i}\big)\wedge
			(e_{k}\wedge e_{l}) \\
	& \,=\,
	\sbox0{$\begin{smallmatrix} k<l \\k,l\not =j\end{smallmatrix}$}
		\sum_{i<j}\,\sum_{\usebox{0}}\,a^{ij}a^{kl}\phi(e_{i})\,
			e_{j}\wedge e_{k}\wedge e_{l} \\
	& \qquad\qquad \,-\,
	\sbox1{$\begin{smallmatrix} k<l \\k,l\not =i\end{smallmatrix}$}
		\sum_{i<j}\,\sum_{\usebox{1}}\,a^{ij}a^{kl}\phi(e_{j})\,
			e_{i}\wedge e_{k}\wedge e_{l}
\end{align*}
%
En particular, si $\{\varepsilon^{1},\,\varepsilon^{2},\,%
\varepsilon^{3},\,\varepsilon^{4}\}$ es la base de
$\dual{(\bb{R}^{4})}$ dual de $\{e_{1},\,e_{2},\,e_{3},\,e_{4}\}$,
entonces, tomando $\phi=\varepsilon^{1}$, por ejemplo, se tiene que
\begin{align*}
	(\varepsilon^{1}\convol x)\wedge x
	& \,=\,
	\sbox0{$\begin{smallmatrix} k<l \\k,l\not =j\end{smallmatrix}$}
		\sum_{1<j}\,\sum_{\usebox{0}}\,a^{1j}a^{kl}\cdot 1\,
			e_{j}\wedge e_{k}\wedge e_{l} \\
	& \qquad\qquad \,-\,
	\sbox1{$\begin{smallmatrix} k<l \\k,l\not =i\end{smallmatrix}$}
		\sum_{i<1}\,\sum_{\usebox{1}}\,a^{i1}a^{kl}\cdot 1\,
			e_{i}\wedge e_{k}\wedge e_{l}
	\text{ .}
\end{align*}
%
Las sumatorias que aparecen restando son vac\'{\i}as pues se suma sobre
$i<1$, con lo cual
\begin{align*}
	(\varepsilon^{1}\convol x)\wedge x & \,=\,
		a^{12}\,\big( a^{13}\,e_{2}\wedge e_{1}\wedge e_{3} +
			a^{14}\,e_{2}\wedge e_{1}\wedge e_{4} +
			a^{34}\,e_{2}\wedge e_{3}\wedge e_{4}\big) \\
	& \quad\,+\,a^{13}\,\big( a^{12}\,e_{3}\wedge e_{1}\wedge e_{2} +
			a^{14}\,e_{3}\wedge e_{1}\wedge e_{4} +
			a^{24}\,e_{3}\wedge e_{2}\wedge e_{4}\big) \\
	& \quad\,+\,a^{14}\,\big( a^{12}\,e_{4}\wedge e_{1}\wedge e_{2} +
			a^{13}\,e_{4}\wedge e_{1}\wedge e_{3} +
			a^{23}\,e_{4}\wedge e_{2}\wedge e_{3}\big) \\
	& \,=\, (-a^{12}a^{13}+a^{12}a^{13})\,e_{1}\wedge e_{2}\wedge e_{3} +
		(-a^{12}a^{14}+a^{12}a^{14})\,e_{1}\wedge e_{2}\wedge e_{4} \\
	& \quad\,+\,
		(-a^{13}a^{14}+a^{13}a^{14})\,e_{1}\wedge e_{3}\wedge e_{4} +
		(a^{12}a^{34}-a^{13}a^{24}+a^{14}a^{23})\,
			e_{2}\wedge e_{3}\wedge e_{4} \\
	& \,=\,	(a^{12}a^{34}-a^{13}a^{24}+a^{14}a^{23})\,
			e_{2}\wedge e_{3}\wedge e_{4}
\end{align*}
%
Se deduce de esto que, si $(\phi\convol x)\wedge x=0$ para toda funcional
$\phi\in\dual{(\bb{R}^{4})}$ y $x=\sum_{i<j}\,a^{ij}\,e_{i}\wedge e_{j}$,
entonces se debe cumplir que
\begin{equation}
	\label{eq:pluckerparaplanos}
	a^{12}a^{34}-a^{13}a^{24}+a^{14}a^{23} \,=\,0
	\text{ .}
\end{equation}
%
Las otras funcionales de la base, $\varepsilon^{2},\,\varepsilon^{3},\,%
\varepsilon^{4}$, no determinan nuevas relaciones entre los coeficientes
de $x$. De hecho, asumiendo se cumple la igualdad anterior, se
puede deducir que $x$ es un tensor elemental y que por lo tanto
$(\phi\convol x)\wedge x=0$ para toda funcional $\phi$.

Los tensores elementales $f_{1}\wedge f_{2}\in\exterior[2]{\bb{R}^{4}}$
tienen otra propiedad que no es cierta en general para un elemento
arbitrario del producto exterior:
\begin{align*}
	(f_{1}\wedge f_{2})\wedge (f_{1}\wedge f_{2}) & \,=\,0
	\text{ .}
\end{align*}
%
Si $x\in\exterior[2]{\bb{R}^{4}}$ es un elemento arbitrario y
$x=\sum_{i<j}\,a^{ij}\,e_{i}\wedge e_{j}$, entonces
\begin{align*}
	x\wedge x & \,=\,
		a^{12}a^{34}\,e_{1}\wedge e_{2}\wedge e_{3}\wedge e_{4} +
		a^{13}a^{24}\,e_{1}\wedge e_{3}\wedge e_{2}\wedge e_{4} \\
	& \quad\,+\,
		a^{14}a^{23}\,e_{1}\wedge e_{4}\wedge e_{2}\wedge e_{3} +
		a^{23}a^{14}\,e_{2}\wedge e_{3}\wedge e_{1}\wedge e_{4} \\
	& \quad\,+\,
		a^{24}a^{13}\,e_{2}\wedge e_{4}\wedge e_{1}\wedge e_{3} +
		a^{34}a^{12}\,e_{3}\wedge e_{4}\wedge e_{1}\wedge e_{2} \\
	& \,=\,2\cdot(a^{12}a^{34}-a^{13}a^{24}+a^{14}a^{23})\,
		e_{1}\wedge e_{2}\wedge e_{3}\wedge e_{4}
	\text{ .}
\end{align*}
%
De esto se deduce, como la caracter\'{\i}stica de $\bb{R}$ es $0$, que
$x\wedge x=0$, si y s\'{o}lo si se cumple \eqref{eq:pluckerparaplanos}.

Sea $x\in\exterior[2]{\bb{R}^{4}}$ con $x=\sum_{i<j}\,%
a^{ij}\,e_{i}\wedge e_{j}$ y $x\not =0$. Alguno de los coeficientes
$a^{ij}$ debe ser distinto de cero. Asumiendo que es $a^{12}\not =0$,
sean $f_{1},f_{2}\in\bb{R}^{4}$ los vectores dados por
\begin{align*}
	f_{1} & \,=\, e_{1}+p^{13}\,e_{3}+p^{14}\,e_{4} \\
	f_{2} & \,=\, e_{2}+p^{23}\,e_{3}+p^{24}\,e_{4}
	\text{ .}
\end{align*}
%
El producto de estos dos vectores est\'{a} dado por
\begin{align*}
	f_{1}\wedge f_{2} & \,=\, e_{1}\wedge e_{2}+p^{23}\,e_{1}\wedge e_{3}
		+p^{24}\,e_{1}\wedge e_{4}+p^{13}\,e_{3}\wedge e_{2} \\
	& \quad +p^{13}p^{24}\,e_{3}\wedge e_{4}+p^{14}\,e_{4}\wedge e_{2}
		+p^{14}p^{23}\,e_{4}\wedge e_{3}
\end{align*}
%
El elemento $x$ y $a^{12}\cdot f_{1}\wedge f_{2}$ son iguales en el
coeficiente de $e_{1}\wedge e_{2}$. La igualdad
\begin{align*}
	x & \,=\, a^{12}\cdot f_{1}\wedge f_{2}
\end{align*}
%
se cumple, si y s\'{o}lo si se verifica el siguiente sistema de ecuaciones:
\begin{equation}
	\label{eq:tensorelemental}
	\begin{aligned}
		a^{13} & \,=\, a^{12}p^{23} \\
		a^{14} & \,=\, a^{12}p^{24} \\
		a^{23} & \,=\, -a^{12}p^{13} \\
		a^{24} & \,=\, -a^{12}p^{14} \\
		a^{34} & \,=\,a^{12}\,(p^{13}p^{24}-p^{14}p^{23})
	\end{aligned}
\end{equation}
%
Las primeras cuatro ecuaciones determinan los valores de los coeficientes
de $f_{1}$ y de $f_{2}$:
\begin{align*}
	p^{23} \,=\,\frac{a^{13}}{a^{12}} & \quad\text{,}\quad
		p^{24} \,=\,\frac{a^{14}}{a^{12}} \text{ ,} \\
	p^{13} \,=\,\frac{-a^{23}}{a^{12}} & \quad\text{,}\quad
		p^{14} \,=\,\frac{-a^{24}}{a^{12}}
	\text{ .}
\end{align*}
%
Reemplazando en la \'{u}ltima ecuaci\'{o}n, se deduce,
asumiendo que $a^{12}\not =0$, que $x=a^{12}\cdot f_{1}\wedge f_{2}$, si y
s\'{o}lo si se verifica \eqref{eq:pluckerparaplanos}. Pero esto es equivalente
a $x\wedge x=0$. En definitiva, vale el siguiente resultado.

\begin{teoEquivalenciasTensoresElementalesRCuatro}%
	\label{thm:equivtensoreselementalesrcuatro}
	Sea $x\in\exterior[2]{\bb{R}^{4}}$ con $x=\sum_{i<j}\,
	a^{ij}\,e_{i}\wedge e_{j}$ y $x\not =0$. Entonces $x$ es un
	tensor elemental, si y s\'{o}lo si se cumple cualquiera de las
	siguientes condiciones equivalentes:
	\begin{itemize}
		\item[(a)] $x\wedge x=0$, o bien,
		\item[(b)] $(\phi\convol x)\wedge x=0$ para toda funcional
			$\phi\in\dual{(\bb{R}^{4})}$.
	\end{itemize}
	%
	En coordenadas, estas condiciones se traducen en
	\begin{align*}
		a^{12}a^{34}-a^{24}a^{13}+a^{14}a^{23} & \,=\,0
		\text{ .}
	\end{align*}
	%
	Tambi\'{e}n en coordenadas, si se verifica esta igualdad y
	$a^{12}\not =0$, entonces $x=a^{12}\cdot f_{1}\wedge f_{2}$, donde
	\begin{align*}
		f_{1} & \,=\,e_{1}+\frac{-a^{23}}{a^{12}}\,e_{3}+
			\frac{-a^{24}}{a^{12}}\,e_{4}
		\quad\text{y} \\
		f_{2} & \,=\,e_{2}+\frac{a^{13}}{a^{12}}\,e_{3}+
			\frac{a^{14}}{a^{12}}\,e_{4}
		\text{ .}
	\end{align*}
	%
\end{teoEquivalenciasTensoresElementalesRCuatro}

\begin{coroTensoresElementalesRTres}%
	\label{thm:tensoreselementalesrtres}
	En $\exterior[2]{\bb{R}^{3}}$, todo tensor es elemental.
\end{coroTensoresElementalesRTres}

\begin{proof}
	Sea $\{e_{1},\,e_{2},\,e_{3}\}$ la base can\'{o}nica de
	$\bb{R}^{3}$. La base inducida en $\exterior[2]{\bb{R}^{3}}$ es
	\begin{align*}
		 & \{e_{1}\wedge e_{2},
		 	\,e_{1}\wedge e_{3},\,e_{2}\wedge e_{3}\}
		\text{ .}
	\end{align*}
	%
	Sea $x=x^{12}\,e_{1}\wedge e_{2}+x^{13}\,e_{1}\wedge e_{3}+%
	x^{23}\,e_{2}\wedge e_{3}\in\exterior[2]{\bb{R}^{3}}$ un
	elemento arbitrario y sea
	$\{\varepsilon^{1},\,\varepsilon^{2},\,\varepsilon^{3}\}$ la base
	dual de $\bb{R}^{3}$ respecto de $\{e_{1},\,e_{2},\,e_{3}\}$.
	La convoluci\'{o}n de $x$ por cada una funcional $\phi$ es igual a
	\begin{align*}
		\phi\convol x & \,=\,
			x^{12}\,\big(\phi(e_{1})\,e_{2}-\phi(e_{2})\,e_{1}\big)
			+
			x^{13}\,\big(\phi(e_{1})\,e_{3}-\phi(e_{3})\,e_{1}\big)
			\\
		& \quad +
			x^{23}\,\big(\phi(e_{2})\,e_{3}-\phi(e_{3})\,e_{2}\big)
		\text{ .}
	\end{align*}
	%
	En particular, tomando como $\phi$ las funcionales de la base dual,
	\begin{align*}
		\varepsilon^{1}\convol x & \,=\,x^{12}\,e_{2}+x^{13}\,e_{3}
		\text{ ,} \\
		\varepsilon^{2}\convol x & \,=\,x^{12}\,e_{1}+x^{23}\,e_{3}
		\quad\text{y} \\
		\varepsilon^{3}\convol x & \,=\,x^{13}\,e_{1}+(-x^{23})\,e_{2}
		\text{ .}
	\end{align*}
	%
	Tomando el producto contra $x$, se deduce que
	\begin{align*}
		(\varepsilon^{1}\convol x)\wedge x & \,=\,
			(x^{12}\,e_{2}+x^{13}\,e_{3})\wedge
			\big(x^{12}\,e_{1}\wedge e_{2}+
			x^{13}\,e_{1}\wedge e_{3}+
			x^{23}\,e_{2}\wedge e_{3}\big) \\
		& \,=\,x^{12}x^{13}\,e_{2}\wedge e_{1}\wedge e_{3}+
			x^{13}x^{12}\,e_{3}\wedge e_{1}\wedge e_{2}
			\,=\,0
	\end{align*}
	%
	y, de manera an\'{a}loga, que
	\begin{align*}
		(\varepsilon^{2}\convol x)\wedge x & \,=\,0
		\quad\text{y} \\
		(\varepsilon^{3}\convol x)\wedge x & \,=\,0
		\text{ ,}
	\end{align*}
	%
	tambi\'{e}n. Esto implica que $(\phi\convol x)\wedge x=0$
	para toda funcional $\phi:\,\bb{R}^{3}\rightarrow\bb{R}$.

	Ahora bien, si $\bb{R}^{3}$ se incluye en $\bb{R}^{4}$
	v\'{\i}a $e_{1}\mapsto e_{1}$, $e_{2}\mapsto e_{2}$ y
	$e_{3}\mapsto e_{3}$, entonces $\exterior[2]{\bb{R}^{3}}$ tambi\'{e}n
	se incluye en $\exterior[2]{\bb{R}^{4}}$ como el subespacio
	\begin{align*}
		\exterior[2]{\bb{R}^{3}} & \,=\,\bigg\{
			\sum_{i<j}\,a^{ij}\,e_{i}\wedge e_{j}\,:\,
			a^{14}=a^{24}=a^{34}=0\bigg\}
		\text{ .}
	\end{align*}
	%
	Por otro lado, toda funcional $\phi:\,\bb{R}^{3}\rightarrow\bb{R}$ se
	extiende a $\bb{R}^{4}$ tomando el valor $0$ en el elemento $e_{4}$
	de la base, de manera que $(\varepsilon^{i}\convol x)\wedge x=0$ para
	$i\in\{1,2,3\}$, si $x\in\exterior[2]{\bb{R}^{3}}$. Pero tambi\'{e}n
	vale que $\varepsilon^{4}\convol x=0$ y, en particular,
	\begin{align*}
		(\varepsilon^{4}\convol x)\wedge x & \,=\,0
		\text{ .}
	\end{align*}
	%
	Por lo tanto, $(\phi\convol x)\wedge x=0$ para toda funcional de
	$\bb{R}^{4}$ y $x=f\wedge g$ para ciertos vectores $f,g\in\bb{R}^{4}$.
	Dado que $\varepsilon^{4}\convol x=0$, debe valer
	\begin{align*}
		0 & \,=\varepsilon^{4}\convol(f\wedge g) \,=\,
			\varepsilon^{4}(f)\,g-\varepsilon^{4}(g)\,f
		\text{ .}
	\end{align*}
	%
	Pero, entonces, si $\varepsilon^{4}(f)\not=0$ o
	$\varepsilon^{4}(g)\not =0$, los vectores $f$ y $g$ no son linealmente
	independientes y $x=f\wedge g=0$. Es decir, si $x\not =0$, entonces
	$\varepsilon^{4}(f)=\varepsilon^{4}(g)=0$ y $f$ y $g$ pertenecen
	a $\bb{R}^{3}$. Con lo cual, $x=f\wedge g$ en
	$\exterior[2]{\bb{R}^{3}}$.
\end{proof}

Una manera m\'{a}s simple de demostrar que todo elemento de
$\exterior[2]{\bb{R}^{3}}$ es elemental es notando que, si
$x=x^{12}\,e_{1}\wedge e_{2}+x^{13}\,e_{1}\wedge e_{3}+%
x^{23}\,e_{2}\wedge e_{3}$ y $x^{12}\not =0$, entonces, mediante un
argumento similar al del caso $\bb{R}^{4}$, tomando los vectores
\begin{align*}
	f_{1} & \,=\,e_{1}+p^{13}\,e_{3}\quad\text{y} \\
	f_{2} & \,=\,e_{2}+p^{23}\,e_{3}\text{ ,}
\end{align*}
%
entonces $x=x^{12}\cdot f_{1}\wedge f_{2}$, si y s\'{o}lo si
\begin{align*}
	p^{23} & \,=\,\frac{x^{13}}{x^{12}}\quad\text{y} \\
	p^{13} & \,=\,\frac{-x^{23}}{x^{12}}\text{ .}
\end{align*}
%
Si alg\'{u}n otro coeficiente es distinto de cero, se deber\'{a} modificar
la definici\'{o}n de estos vectores, pero tambi\'{e}n es posible escribir
a $x$ como tensor elemental.


\section{Suavidad de funciones, campos y formas}
\theoremstyle{plain}
\newtheorem{lemaNotasSueltasGermenes}{Lema}[section]
\newtheorem{propoSuavidadCampos}[lemaNotasSueltasGermenes]{Proposici\'{o}n}
\newtheorem{propoEquivalenciaSuavidadCampos}[lemaNotasSueltasGermenes]%
	{Proposici\'{o}n}
\newtheorem{propoSuavidadFormas}[lemaNotasSueltasGermenes]{Proposici\'{o}n}
\newtheorem{propoEquivalenciaSuavidadFormas}[lemaNotasSueltasGermenes]%
	{Proposici\'{o}n}

\theoremstyle{remark}
\newtheorem{obsBaseEnElCotangente}[lemaNotasSueltasGermenes]{Observaci\'{o}n}

%-------------

\begin{lemaNotasSueltasGermenes}\label{lema:notassueltasgermenes}
	Sea $M$ una variedad diferencial y sea $p\in M$. Sean
	$v\in\tangente[p]{M}$, $U\subset M$ un entorno de $p$ y
	$f,g\in C^\infty(U)$. Si $f|_V=g|_V$ para cierto abierto tal que
	$V\subset U$ y $p\in V$, entonces $v\,f=v\,g$.
\end{lemaNotasSueltasGermenes}

\begin{proof}
	Sea $h=f-g$ y sea $\psi:\,M\rightarrow\bb{R}$ una funci\'{o}n
	suave, $\psi\geq 0$, tal que existe una bola abierta $B$ que verifica
	$p\in B\subset\soporte{\psi}\subset V$ y $\psi|_B=1$. En particular,
	$\psi(p)=1$. Entonces $\psi\,h=0$ en $M$ y
	\begin{align*}
		0 & \,=\,v(\psi\,h)\,=\,(v\,\psi)\,h(p)\,+\,
			\psi(p)\,(v\,h)\,=\,v\,h
		\text{ ,}
	\end{align*}
	%
	pues $v(\psi\,h)=v(0)=0$ y $\psi(p)=1$ (no estamos usando que $\psi$
	es constante en un entorno de $p$). As\'{\i},
	\begin{align*}
		\psi\,h & \,=\,
			\begin{cases}
				0 & \text{ en } U\setmin\soporte{\psi} \\
				0 & \text{ en } V
			\end{cases}
		\text{ .}
	\end{align*}
	%
\end{proof}

\begin{propoSuavidadCampos}\label{propo:suavidadcampos}
	Sea $X:\,M\rightarrow\tangente{M}$ una secci\'{o}n del fibrado
	tangente. Entonces $X$ es suave (como secci\'{o}n), si y s\'{o}lo si,
	para todo punto $p\in M$, existe una carta $(U,\psi)$, compatible con
	la estructura de $M$, tal que $p\in U$ y con respecto a la cual las
	funciones $X^i:\,U\rightarrow\bb R$ determinadas por
	\begin{equation}\label{eq:suavidadcampos}
		X|_U \,=\,X^i\,\gancho{x^i}
	\end{equation}
	%
	son suaves.
\end{propoSuavidadCampos}

\begin{proof}
	En primer lugar, $X^i=X(x^i)$, evaluar el campo $X$ en la funci\'{o}n
	coordenada $x^i$. Que $X$ sea suave quiere decir que, para toda carta
	compatible $(U,\varphi)$, la expresi\'{o}n en coordenadas
	\begin{math}
		\encoordenadas{X}=
			\delfibrado{\varphi}\circ X\circ\varphi^{-1}:\,
			\varphi(U)\rightarrow\bb{R}^{2n}
	\end{math} --donde
	\begin{math}
		\delfibrado{\varphi}:\,\delfibrado{U}=\pi^{-1}(U)\rightarrow
			\delfibrado{\varphi}(\delfibrado{U})=U\times\bb{R}^{n}
	\end{math} es la carta correspondiente
	$(\delfibrado{U},\delfibrado{\varphi})$ en $\tangente{M}$-- es suave,
	en sentido usual. Pero $\encoordenadas{X}$ est\'{a} dada por
	\begin{align*}
		\encoordenadas{X}(x) & \,=\,\encoordenadas{X}(\lista{x}{n}) \\
		& \,=\,
			\big(X^1(\varphi^{-1}(x)),\,\dots,\,
				X^n(\varphi^{-1}(x)),\,
				x^1(\varphi^{-1}(x)),\,\dots,\,
				x^n(\varphi^{-1}(x))\big) \\
		& \,=\,\big(\encoordenadas{X}^1(x),\,\dots,\,
			\encoordenadas{X}^n(x),\lista{x}{n}\big)
		\text{ .}
	\end{align*}
	%
	Esta expresi\'{o}n es v\'{a}lida para toda carta $(U,\varphi)$ y todo
	campo $X:\,M\rightarrow\tangente{M}$, independientemente de su
	suavidad. En particular, $X$ es suave (continua) en $U$, si y s\'{o}lo
	si las funciones $\encoordenadas{X}^i=X^i\circ\varphi^{-1}$ los son en
	$\varphi(U)$. Como $\varphi:\,U\rightarrow\varphi(U)$ es difeomorfismo,
	esto equivale a que las funciones $X^i:\,U\rightarrow\bb R$ sean
	suaves.
\end{proof}

\begin{propoEquivalenciaSuavidadCampos}\label{propo:equivalenciasuavidadcampos}
	Sea $X:\,M\rightarrow\tangente{M}$ una secci\'{o}n. Las siguientes
	afirmaciones son equivalentes.
	\begin{enumerate}
		\item\label{propo:equivalenciasuavidadcampos:seccion}
			$X$ es suave;
		\item\label{propo:equivalenciasuavidadcampos:global}
			$Xf\in C^\infty(M)$ para toda $f\in C^\infty(M)$;
		\item\label{propo:equivalenciasuavidadcampos:entornos}
			$X|_Uf\in C^\infty(U)$ para toda $f\in C^\infty(U)$.
	\end{enumerate}
\end{propoEquivalenciaSuavidadCampos}

Antes de demostrar esto, vale la pena recordar que una funci\'{o}n en $M$ es
suave en un punto respecto de una carta compatible, si y s\'{o}lo si es suave,
en el mismo punto, respecto de cualquier otra carta compatible.

\begin{proof}
	Si $X\in\champs M$ y $f\in C^\infty(M)$, dados $p\in M$ y una carta
	$(U,\varphi)$ en $p$, en un entorno del punto (en $U$),
	\begin{align*}
		X\,f & \,=\,X^i\,\derivada{f}{x^i}
		\text{ ,}
	\end{align*}
	%
	que es suave. Expl\'{\i}citamente, en coordenadas, si $x\in\varphi(U)$,
	\begin{align*}
		\big(X\,f\big)(\varphi^{-1}(x)) & \,=\,
			X^i\circ\varphi^{-1}(x)\,
				\derivada{(f\circ\varphi^{-1})}{x^i}
					(\varphi(\varphi^{-1}(x))) \,=\,
			\encoordenadas{X}^i(x)\,
				\derivada{\encoordenadas{f}}{x^i}(x)
		\text{ ,}
	\end{align*}
	%
	donde $\encoordenadas{X}^i$ y $\encoordenadas{f}$ son las
	representaciones en coordenadas de las funciones componentes $X^i$ y
	$f$, definidas en $U$ (en el caso de $X^i$) y en $M$. Entonces $X\,f$
	es suave en un entorno de $p$. Como el punto fue elegido de forma
	arbitraria, $X\,f\in\suaves M$.

	Si vale \ref{propo:equivalenciasuavidadcampos:global}, y
	$f\in\suaves U$ para cierto abierto $U\subset M$, dado $p\in U$, existe
	una funci\'{o}n $\psi:\,M\rightarrow\bb R$ suave tal que $\psi\geq 0$ y
	un abierto $V$ tal que $p\in V$, $\clos{V}\subset U$, $\psi|_V=1$ y
	$\soporte{\psi}\subset U$. Sea $\tilde{f}:\,M\rightarrow\bb R$ la
	funci\'{o}n
	\begin{align*}
		\tilde{f} & \,:=\,
			\begin{cases}
				f\cdot\psi & \text{ en } U \\
				0 & \text{ en } M\setmin\soporte{\psi}
			\end{cases}
		\text{ .}
	\end{align*}
	%
	Como el soporte de la funci\'{o}n $\psi$ est\'{a} contenido en $U$,
	esta funci\'{o}n est\'{a} bien definida y es suave. Adem\'{a}s,
	\begin{align*}
		\big(\tilde f|_U\big)|_V & \,=\,\tilde f|_V \,=\,f|_V
		\text{ ,}
	\end{align*}
	%
	con lo cual, para $q\in V$,
	\begin{align*}
		\big(X|_Uf\big)(q) & \,=\,\big(X|_U\tilde f|_U\big)(g) \,=\,
			\big(X\,\tilde f\big)(q)
		\text{ .}
	\end{align*}
	%
	Pero, por hip\'{o}tesis, $X\,\tilde f$ es suave en $M$, con lo que
	$X|_Uf$ es suave en $V$. Como $p\in U$ fue elegido de manera
	arbitraria, $X|_Uf$ es suave en todo el abierto $U$.

	Finalmente, si se cumple
	\ref{propo:equivalenciasuavidadcampos:entornos} y $(U,\varphi)$ es una
	carta compatible, las componentes de $X$ en la carta est\'{a}n dadas
	por
	\begin{align*}
		X|_U & \,=\, X^i\,\gancho{x^i}
		\text{ ,}
	\end{align*}
	%
	donde $X^i=X|_U(x^i)$. Por hip\'{o}tesis, cada una de estas funciones
	es suave en $U$ y, por la Proposici\'{o}n~\ref{propo:suavidadcampos},
	$X\in\champs M$.
\end{proof}

\begin{propoSuavidadFormas}\label{propo:suavidadformas}
	Sea $\omega:\,M\rightarrow\tangente*{M}$ una secci\'{o}n del fibrado
	cotangente. Entonces $\omega$ es suave, si y s\'{o}lo si, para todo
	$p\in M$, existe una carta $(U,\varphi)$ compatible, cuyo dominio
	contiene a $p$ y tal que la representaci\'{o}n en coordenadas
	\begin{equation}\label{eq:suavidadformas}
		\omega|_U \,=\,\omega_i\,\lambda^i
	\end{equation}
	%
	sea suave, es decir, de forma que las funciones
	$\omega_i:\,U\rightarrow\bb R$ sean suaves, donde
	$\big\{\lambda^i|_p\big\}_i$ es la base dual de
	$\big\{\gancho[p]{x^i}\big\}_i$ en $\tangente*[p]{M}$ para $p\in U$.
\end{propoSuavidadFormas}

\begin{proof}
	Sea $\pi:\,\tangente*{M}\rightarrow M$ la proyecci\'{o}n can\'{o}nica
	del fibrado. Sea $(U,\varphi)$ una carta compatible con la estructura
	de $M$ y sean $\delfibrado{U}=\pi^{-1}(U)\subset\tangente*{M}$ y
	$\delfibrado{\varphi}:\,\delfibrado{U}\rightarrow U\times\bb R^n$ las
	coordenadas correspondientes. La representaci\'{o}n en coordenadas de
	la forma $\omega$ es la funci\'{o}n
	\begin{math}
		\encoordenadas{\omega}=
			\delfibrado{\varphi}\circ\omega\circ\varphi^{-1}:\,
			\varphi(U)\rightarrow\bb R^{2n}
	\end{math} dada por
	\begin{align*}
		\encoordenadas{\omega}(x) & \,=\,\big(
			\encoordenadas{\omega}_1(x),\,\dots,\,
			\encoordenadas{\omega}_n(x),\,\lista{x}{n}\big)
		\text{ ,}
	\end{align*}
	%
	donde $\encoordenadas{\omega}_i=\omega_i\circ\varphi^{-1}$ y cada
	funci\'{o}n $\omega_i:\,U\rightarrow\bb R$ est\'{a} determinada por la
	representaci\'{o}n \eqref{eq:suavidadformas}, es decir,
	$\omega_i=\omega|_U\big(\gancho{x^i}\big)$. As\'{\i},
	$\omega:\,M\rightarrow\tangente*{M}$ es, por definici\'{o}n, suave
	(continua) en $U$, si y s\'{o}lo si las funciones
	$\omega_i:\,U\rightarrow\bb R$ lo son.
\end{proof}

\begin{obsBaseEnElCotangente}\label{obs:baseenelcotangente}
	Sea $(U,\varphi)$ una carta, sea $\big\{\gancho[p]{x^i}\big\}$ la base
	dada por los campos coordenados en $\tangente[p]{M}$ y sea
	$\big\{\lambda^i|_p\big\}_i$ la base dual. Sean $\{x^i\}_i$ las
	funciones coordenadas en $U$ y sean $\big\{\de{x^i}\big\}_i$ las
	$1$-formas $(\de{x^i})\,X=X(x^i)$. Entonces, si $p\in U$ y
	$v\in\tangente[p]{M}$,
	\begin{align*}
		\de[p]{x^i}(v) & \,=\,v(x^i)
		\text{ .}
	\end{align*}
	%
	Pero $v=v^i\,\gancho[p]{x^i}$, donde $v^i=v(x^i)$. Como
	$\big\{\lambda^i|_p\big\}_i$ es la base dual a la base can\'{o}nica,
	\begin{align*}
		\lambda^i|_p\,v & \,=\,v^i
		\text{ ,}
	\end{align*}
	%
	es decir, para cada \'{\i}ndice $i$,
	\begin{equation}\label{eq:baseenelcotangente}
		\lambda^i|_p \,=\,\de[p]{x^i}
		\text{ .}
	\end{equation}
	%
\end{obsBaseEnElCotangente}

\begin{propoEquivalenciaSuavidadFormas}\label{propo:equivalenciasuavidadformas}
	Sea $\omega:\,M\rightarrow\tangente*{M}$ una secci\'{o}n. Las
	siguientes afirmaciones son equivalentes.
	\begin{enumerate}
		\item\label{propo:equivalenciasuavidadformas:seccion}
			$\omega$ es suave;
		\item\label{propo:equivalenciasuavidadformas:global}
			$\omega\,X\in\suaves M$ para todo $X\in\champs M$;
		\item\label{propo:equivalenciasuavidadformas:entornos}
			$\omega|_UX\in\suaves U$ para todo $X\in\champs U$.
	\end{enumerate}
\end{propoEquivalenciaSuavidadFormas}

\begin{proof}
	Asumiendo que $\omega:\,M\rightarrow\tangente*{M}$ es suave y que
	$X\in\champs M$, para verificar la suavidad de
	\begin{math}
		\omega\,X:\,p\mapsto\omega_p\,X_p
	\end{math}, verificamos que sea suave en coordenadas: sea $(U,\varphi)$
	una carta compatible. La expresi\'{o}n de la funci\'{o}n en coordenadas
	est\'{a} determinada por
	\begin{equation}\label{eq:formacampoencoordenadas}
		\omega\,X \,=\,\omega|_U\Big(X^i\,\gancho{x^i}\Big) \,=\,
			X^i\Big(\omega|_U\gancho{x^i}\Big) \,=\,
			X^i\,\omega_i
		\text{ .}
	\end{equation}
	%
	Por un lado, como $X$ es suave en $M$, las funciones $X^i$ son suaves
	en $U$. Por otro, como $\omega\in\formes[1]{M}$, vale que
	$\omega_i\in\suaves U$. En definitiva, la funci\'{o}n determinada por
	evaluar $\omega\,X$ es suave en $U$. Como la carta era arbitraria, la
	funci\'{o}n es suave en $M$.

	Asumiendo \ref{propo:equivalenciasuavidadformas:global}, sea
	$U\subset M$ un abierto y sea $X\in\champs U$. Sea $p\in U$ y sea
	$\psi:\,M\rightarrow\bb R$ una funci\'{o}n suave tal que
	$\psi\geq 0$, $\psi|_V=1$ en cierto abierto $V\subset U$,
	$p\in V$, $\clos V\subset U$ y $\soporte{\psi}\subset U$. Sea
	$\tilde X:\,M\rightarrow\tangente M$ el \emph{campo} dado por
	\begin{align*}
		\tilde X & \,:=\,
			\begin{cases}
				\psi\cdot X & \text{ en } U \\
				0 & \text{ en } M\setmin\soporte{\psi}
			\end{cases}
		\text{ .}
	\end{align*}
	%
	Como el soporte de la funci\'{o}n chich\'{o}n $\psi$ est\'{a} contenido
	en $U$, $\tilde X$ est\'{a} bien definido y es suave en $M$. Pero,
	adem\'{a}s, $\tilde X|_V=X|_V$. Por un lado,
	$\omega\,\tilde X\in\suaves M$, por hip\'{o}tesis. Por otro lado, si
	$q\in V$,
	\begin{align*}
		\big(\omega\,\tilde X\big)(q) & \,=\,
			\omega_q\,\tilde X|_q \,=\,
			\omega_q\,X_q \,=\,\big(\omega|_UX\big)(q)
		\text{ .}
	\end{align*}
	%
	Entonces $\omega|_UX$ es suave en $V$. Como $p\in U$ fue elegido de
	manera arbitraria, $\omega|_UX\in\suaves U$.

	Finalmente, asumiendo que vale
	\ref{propo:equivalenciasuavidadformas:entornos} y que $(U,\varphi)$ es
	una carta compatible, por hip\'{o}tesis, las componentes
	$\omega_i=\omega|_U\gancho{x^i}$ de $\omega$ en $U$ son suave.
	As\'{\i}, por la Proposici\'{o}n~\ref{propo:suavidadformas}, se ve que
	$\omega$ es suave como secci\'{o}n, ya que la carta $(U,\varphi)$
	hab\'{\i}a sido elegida de manera arbitraria.
\end{proof}

\section{Particiones de la unidad}
\theoremstyle{plain}
\newtheorem{teoParticiones}{Teorema}[section]
\newtheorem{lemaBaseDeBolasPrecompactas}[teoParticiones]{Lema}
\newtheorem{lemaConjuntoLocalmenteFinito}[teoParticiones]{Lema}
\newtheorem{teoRefinamiento}[teoParticiones]{Teorema}
\newtheorem{propoVariedadEsSigmaCompacta}[teoParticiones]{Proposici\'{o}n}
\newtheorem{lemaParticiones}[teoParticiones]{Lema}

\theoremstyle{remark}
\newtheorem{obsBaseDeBolasRegulares}[teoParticiones]{Observaci\'{o}n}

%-------------

\begin{lemaBaseDeBolasPrecompactas}\label{lema:basedebolasprecompactas}
	Toda variedad topol\'{o}gica admite una base de bolas coordenadas
	precompactas.
\end{lemaBaseDeBolasPrecompactas}

\begin{proof}
	Si $M=U$ es el dominio de una \'{u}nica carta,
	$\varphi:\,U\rightarrow\encoordenadas{U}$ es un homeomorfismo con
	alg\'{u}n abierto de $\bb R^n$. Dado un abierto arbitrario $W$, diremos
	que una bola $B=\bola{r}{x}\subset W$ es una \emph{bola premium} (o
	\emph{especial}, \emph{seleccionada}, \emph{de calidad}, etc.)
	\emph{en/para el abierto $W$}, si existe $r'>r$ tal que
	$\clos{\bola{r}{x}}\subset\bola{r'}{x}\subset W$. Consideramos el
	conjunto
	\begin{align*}
		\widetilde{\cal B} & \,:=\,
			\Big\{\bola{r}{x} \text{ premium en }
				\encoordenadas{U}\,:\,x
			\text{ tiene coordenadas racionales y } r>0
			\text{ es racional }\Big\}
		\text{ .}
	\end{align*}
	%
	Entonces $\widetilde{\cal B}$ tiene cardinal a lo sumo numerable y
	constituye una base para la topolog\'{\i}a de $\encoordenadas{U}$. Como
	$\varphi:\,U\rightarrow\encoordenadas{U}$ es un homeomorfismo, el
	conjunto
	\begin{math}
		\cal B:=\big\{ B=\varphi^{-1}(\tilde B)\,:\,
			\tilde B\in\widetilde{\cal B}\big\}
	\end{math} es una base para la topolog\'{\i}a de $U$ y es numerable.
	Por definici\'{o}n $\cal B$ est\'{a} compuesto por bolas coordenadas
	precompactas.

	En general, una variedad topol\'{o}gica arbitraria, $M$, se puede
	cubrir por una familia numerable de abiertos coordenados:
	\begin{math}
		M=\bigcup_{n\geq 1}\,U_n
	\end{math}. Si
	\begin{math}
		\cal B:=\bigcup_{n\geq 1}\,\cal B_n
	\end{math}, donde $\cal B_n$ es una base numerable de bolas
	precompactas para $U_n$, entonces $\cal B$ es numerable y constituye
	una base para la topolog\'{\i}a de $M$. Lo \'{u}nico que falta
	demostrar es que las bolas $B\in\cal B$ son precompactas \emph{en $M$}.

	Sea $B\in\cal B_n$ un elemento de la base. Por definici\'{o}n,
	$\clos B^{U_n}$ --la clausura \emph{en $U_n$} de la bola-- es compacta.
	Pero $\clos B^M$ es cerrada en $M$ y contiene a $B$, de lo que se
	deduce que $\clos B^M\cap U_n$ es cerrado en $U_n$ y contiene a $B$.
	As\'{\i},
	\begin{align*}
		\clos B^M & \,\supset\,\clos B^M\,\cap\,U_n\,\supset\,
			\clos B^{U_n}
		\text{ .}
	\end{align*}
	%
	Por otro lado, como $\clos B^{U_n}$ es compacta y ``subespacio de
	subespacio es subespacio'', $\clos B^{U_n}$ es compacta como subespacio
	de $M$. Como $M$ es $T_2$, este conjunto es cerrado en $M$. Pero
	tambi\'{e}n contiene a $B$. En consecuencia,
	$\clos B^M\subset\clos B^{U_n}$. En definitiva, ambas clausuras
	coinciden, de lo que se deduce que $B$ es precompacta.
\end{proof}

\begin{obsBaseDeBolasRegulares}\label{obs:basedebolasregulares}
	La afirmaci\'{o}n del Lema~\ref{lema:basedebolasprecompactas} sigue
	siendo v\'{a}lida si se reemplaza ``bolas coordenadas precompactas''
	por ``bolas coordenadas regulares''. Una bola coordenada regular es un
	abierto $B\subset M$ con las siguientes caracter\'{\i}sticas:
	existe una carta $(B',\varphi)$ y $r'>r>0$ tales que
	$\clos B\subset B'$ y
	\begin{align*}
		\varphi(B) \,=\,\bola{r}{x} & \quad\text{,}\quad
			\varphi(\clos B) \,=\,\clos{\bola{r}{x}}
			\quad\text{y}\quad
			\varphi(B') \,=\,\bola{r'}{x}
	\end{align*}
	%	
	(notemos que no hemos introducido la noci\'{o}n de estructura suave).
	La \'{u}nica parte del argumento que se debe modificar es
	el p\'{a}rrafo final; el resto es v\'{a}lido luego de hacer el
	reemplazo textual. Supongamos que $B\in\cal B_n$ y que
	$\varphi_n:\,U_n\rightarrow\encoordenadas U_n$ es el homeo de la carta
	compatible correspondiente. Por definici\'{o}n,
	\begin{math}
		\varphi_n(B)=\bola{r}{x}
	\end{math} para cierto $r>0$ racional y $x\in\encoordenadas U_n$ con
	coordenadas racionales y existe $r'>r$ tal que
	\begin{align*}
		\clos{\bola{r}{x}} & \,\subset\,\bola{r'}{x}\,\subset\,
			\encoordenadas U_n
		\text{ .}
	\end{align*}
	%
	Si definimos $B'=\varphi_n^{-1}(\bola{r'}{x})$, entonces,
	\emph{en $U_n$},
	\begin{align*}
		\clos B^{U_n} & \,=\,\varphi_n^{-1}\big(\clos{\bola{r}{x}}\big)
			\,\subset\,\varphi_n^{-1}\big(\bola{r'}{x}\big)
			\,=\,B'
		\text{ .}
	\end{align*}
	%
	Pero $\clos B^M=\clos B^{U_n}$, porque $B$ es precompacta en $U_n$
	(regular implica precompacta), seg\'{u}n el Lema~%
	\ref{lema:basedebolasprecompactas}. Esto quiere decir que la
	expresi\'{o}n anterior es v\'{a}lida tomando clausura en $M$, en lugar
	de clausura en $U_n$:
	\begin{align*}
		\clos B^M & \,=\,\clos B^{U_n}\,\subset\,B'
		\text{ .}
	\end{align*}
	%
\end{obsBaseDeBolasRegulares}

Si asumimos que $M$ es una variedad diferencial, podemos preguntarnos si la
estructura adicional nos permite concluir algo m\'{a}s fuerte, o si, dicho de
otra manera, este resultado tiene una versi\'{o}n ``compatible'' con la
estructura diferencial.

\begin{lemaBaseDeBolasPrecompactas}[Porisma]%
	\label{lema:basedebolasprecompactas:porisma}
	Toda variedad diferencial admite una base de bolas coordenadas (suaves)
	regulares.
\end{lemaBaseDeBolasPrecompactas}

\begin{proof}
	Reemplazar ``homeo'' por ``difeo'' en la demostraci\'{o}n anterior.
\end{proof}

\begin{lemaConjuntoLocalmenteFinito}\label{lema:conjuntolocalmentefinito}
	Sea $\cal X$ un conjunto localmente finito de subconjuntos de $M$ (un
	espacio topol\'{o}gico). Entonces
	\begin{enumerate}
		\item\label{lema:conjuntolcalmentefinito:clausuras}
			\begin{math}
				\widetilde{\cal X}:=
					\big\{\clos X\,:\,X\in\cal X\big\}
			\end{math} es localmente finito;
		\item\label{lema:conjuntolocalmentefinito:union}
			\begin{math}
				\clos{\bigcup\,\cal X}=
					\bigcup\,\widetilde{\cal X}
			\end{math}.
	\end{enumerate}
	%
\end{lemaConjuntoLocalmenteFinito}

\begin{proof}
	Sea $x\in M$. Sea $U\subset M$ un abierto al cual $x$ pertenece. Si
	$U\cap\clos X\not=\varnothing$ e $y$ es un elemento de este conjunto,
	entonces $U$ es un entorno de $y$ y, por lo tanto,
	$U\cap X\not=\varnothing$. En definitiva, para todo abierto
	$V\subset M$,
	\begin{align*}
		V\,\cap\,\clos X\,\not=\,\varnothing &
			\quad\Leftrightarrow\quad
		V\,\cap\,X\,\not=\,\varnothing
		\text{ .}
	\end{align*}
	%
	As\'{\i} $\cal X$ es localmente finito, si y s\'{o}lo si
	$\widetilde{\cal X}$ lo es. Esto demuestra
	\ref{lema:conjuntolocalmentefinito:clausuras}. En cuanto a
	\ref{lema:conjuntolocalmentefinito:union}, el conjunto
	$\clos{\bigcup\,\cal X}$ es cerrado y contiene a todo elemento de
	$\cal X$. Es decir,
	\begin{align*}
		\clos X & \,\subset\,\clos{\bigcup\,\cal X}
		\text{ ,}
	\end{align*}
	%
	si $X\in\cal X$, lo que implica
	\begin{math}
		\bigcup\,\widetilde{\cal X}\subset
			\clos{\bigcup\,\cal X}
	\end{math}. Por otro lado, si $x\in\clos{\bigcup\,\cal X}$, existe un
	entorno $U_0\subset M$ de $x$, tal que $U_0\cap X=\varnothing$ para
	casi todo $X\in\cal X$ (todos salvo finitos). Si definimos
	\begin{align*}
		\cal A_{x,0} & \,:=\,\Big\{X\in\cal X\,:\,
			U_0\cap X\not=\varnothing\Big\}
		\text{ ,}
	\end{align*}
	%
	entonces $\cal A_{x,0}$ es finito. Pero tambi\'{e}n sabemos que debe
	existir $y\in U_0\cap \big(\bigcup\,\cal X\big)$, pues $U_0$ es entorno
	de $x$. Entonces $y\in U_0$ e $y\in X$ para alg\'{u}n $X\in\cal X$.
	Este conjunto $X$ verifica $U_0\cap X\not=\varnothing$ y
	$X\in\cal A_{x,0}$.
	Llamemos $A=\bigcup\,\cal A_{x,0}$ (la uni\'{o}n de los elementos de
	esta familia (finita) de conjuntos $X\in\cal X$). Entonces
	$y\in U_0\cap A$. Eso se puede expresar de la siguiente manera:
	\begin{align*}
		U_0\,\cap\,\big(\bigcup\,\cal x\big) & \,=\,
			U_0\,\cap\,A
		\text{ .}
	\end{align*}
	%
	Si $V\subset M$ es una abierto tal que $x\in V$, $V\cap U_0$ es
	abierto, est\'{a} contenido en $U_0$ y contiene a $x$. Nuevamente,
	podemos afirmar que existe
	$y\in (V\cap U_0)\cap\big(\bigcup\,\cal X\big)$. As\'{\i},
	\begin{align*}
		(V\cap U_0)\,\cap\,\big(\bigcup\,\cal X\big) & \,\not=\,
			\varnothing \quad\text{implica} \\
		(V\cap U_0)\,\cap\,A & \,\not=\,\varnothing
			\quad\text{implica} \\
		V\,\cap\,A & \,\not=\,\varnothing
		\text{ .}
	\end{align*}
	%
	Si $x\in\clos{\bigcup\,\cal X}$, por ser $A=\bigcup\,\cal A_{x,0}$ una
	uni\'{o}n finita,
	\begin{align*}
		x & \,\in\,\clos{\bigcup\,\cal A_{x,0}} \,=\,
			\bigcup\,\big\{\clos X\,:\,X\in\cal A_{x,0}\big\}
			\,\subset\,\bigcup\,\widetilde{\cal X}
		\text{ .}
	\end{align*}
	%
\end{proof}

\begin{teoRefinamiento}\label{thm:refinamiento}
	Sea $M$ una variedad topol\'{o}gica y sea $\cal U$ un cubrimiento de
	$M$ por abiertos. Sea $\cal B$ una base para la topolg\'{\i}a de $M$.
	Existe un refinamiento numerable y localmente finito de $\cal U$
	compuesto por elementos de $\cal B$.
\end{teoRefinamiento}

\begin{propoVariedadEsSigmaCompacta}\label{propo:variedadessigmacompacta}
	Dada una variedad topol\'{o}gica $M$, existe una familia numerable de
	compactos $\{K_n\}_{n\geq 1}$ tal que
	\begin{align*}
		\interior{K_n}\,\not=\,\varnothing & \quad\text{,}\quad
			K_n\,\subset\,\interior{K_{n+1}}
			\quad\text{y}\quad
			M \,=\,\bigcup_{n\geq 1}\,K_n
		\text{ .}
	\end{align*}
	%
\end{propoVariedadEsSigmaCompacta}

\begin{proof}
	La variedad $M$ se puede cubrir por bolas precompactas. Como $M$ es
	$N_2$, existe un subcubrimiento numerable. Ordenando este
	subcubrimiento y tomando la clausura de las bolas, se obtiene la
	familia de compactos: por ejemplo, $K_1=\clos{B_1}$ se cubre por
	$B_1$ y el resto de las bolas, tomando un subcubrimiento finito, debe
	haber al menos una $B_2$ para cubrir $K_1$; numeramos $\lista[2]{B}{n}$
	y definimos $K_2$ como la uni\'{o}n de $\clos{B_i}$, $i\leq n$\dots
\end{proof}

\begin{proof}[Demostraci\'{o}n de \ref{thm:refinamiento}]
	Sea $\{K_j\}_{j\geq 1}$ una familia numerable de compactos como
	en el enunciado de la Proposici\'{o}n~%
	\ref{propo:variedadessigmacompacta}. Sean
	\begin{align*}
		F_j & \,:=\,K_{j+1}\setmin\interior{K_j} \quad\text{y} \\
		W_j & \,:=\,\interior{K_{j+2}}\setmin K_{j-1}
		\text{ .}
	\end{align*}
	%
	Si $j=1$, $K_0:=\varnothing$. Cada $F_j$ es cerrado y est\'{a}
	contenido en el compacto $K_{j+1}$ y es, por lo tanto, compacto. Para
	cada $x\in F_j$, existe $U_x\in\cal U$ tal que $x\in U_x$; elegimos,
	tambi\'{e}n, $B_x^j\in\cal B$ tal que
	\begin{align*}
		x & \,\in\,B_x^j\subset U_x\cap W_j
		\text{ .}
	\end{align*}
	%
	Como $F_j$ es compacto y $\big\{B_x^j\,:\,x\in F_j\big\}$ es un
	cubrimiento por abiertos, podemos extraer un subcubrimiento finito:
	existen $\lista{x}{n_j}\in F_j$ tales que
	\begin{align*}
		F_j & \,\subset\, B_{x_1}^j\,\cup\,\cdots\,\cup\,
			B_{x_{n_j}}^j
		\text{ .}
	\end{align*}
	%
	Definimos
	\begin{math}
		\cal V:=\big\{B_{x_i}^j\,:\,j\geq 1,\,n_j\geq i\geq 1\big\}
	\end{math}. Entonces $\cal V$ es numerable y la inclusi\'{o}n
	$B_{x_i}^j\subset U_{x_i}$ implica que es un refinamiento de $\cal U$.
	Este refinamiento est\'{a} conformado por elementos de la base
	$\cal B$, con lo que lo \'{u}nico que resta demostrar es que $\cal V$
	es localmente finito.

	Para demostrar esto, observamos que
	\begin{align*}
		|j-j'| \,\geq\,3 & \quad\Rightarrow\quad
			W_j\,\cap\,W_{j'} \,=\,\varnothing
		\text{ .}
	\end{align*}
	%
	Como $B_{x_i}^j\subset W_j$, una intersecci\'{o}n de la forma
	$B_{x_k}^j\cap B_{x_l}^{j'}$ es vac\'{\i}a salvo, posiblemente, para
	finitos valores de $l$ y $j'$. En definitiva, si $x\in M$, basta tomar
	un elemento del mismo cubrimiento $\cal V$ que lo contenga, para
	verificar que $\cal V$ es localmente finito; cualquier elemento de
	$\cal V$ que contega a $x$ es un abierto que demuestra la finitud local
	del cubrimiento en el punto.
\end{proof}

\begin{lemaParticiones}\label{lema:particiones:pendientesuave}
	La funci\'{o}n
	\begin{align*}
		f(t) & \,:=\,
			\begin{cases}
				e^{-1/t} & t>0 \\
				0 & t \leq 0
			\end{cases}
	\end{align*}
	%
	es suave.
\end{lemaParticiones}

\begin{lemaParticiones}\label{lema:particiones:escalonsuave}
	Si $r_1<r_2$ son n\'{u}meros reales, la funci\'{o}n
	\begin{align*}
		h(t) & \,:=\,\frac{f(r_2-t)}{f(r_2-t)+f(t-r_1)}
	\end{align*}
	%
	verifica: ser suave, $h=1$ en $t\leq r_1$ y $h=0$ en $r_2\leq t$.
	Adem\'{a}s, $0<h(t)<1$, si $r_1<t<r_2$.
\end{lemaParticiones}

\begin{lemaParticiones}\label{lema:particiones:chichonsuave}
	Si $0<r_1<r_2$, la funci\'{o}n
	\begin{align*}
		H(x) & \,:=\,h(|x|)
	\end{align*}
	%
	verifica: ser suave, $H=1$ en $\clos{\bola{r_1}{0}}$ y $H=0$ en
	$\bb R^n\setmin\bola{r_2}{0}$. Adem\'{a}s, $0<H(x)<1$ en
	$\bola{r_2}{0}\setmin\clos{\bola{r_1}{0}}$.
\end{lemaParticiones}

\begin{teoParticiones}[Existencia de particiones suaves de la unidad]%
	\label{thm:particiones}
	Sea $M$ una variedad (diferencial). Sea $\cal U=\{U_\alpha\}_\alpha$ un
	cubrimiento de $M$ por abiertos. Existe una partici\'{o}n de la unidad
	para $M$, subordinada a $\cal U$ y compuesta por funciones suaves.
\end{teoParticiones}

\begin{proof}
	Sean $\cal B_\alpha$ bases numerables conformadas por bolas regulares
	para cada uno de los abiertos $U_\alpha$ y sea
	$\cal B=\bigcup_\alpha\,\cal B_\alpha$. Por el Teorema del refinamiento
	(Teorema~\ref{thm:refinamiento}), existe un refinamiento de $\cal U$,
	localmente finito, numerable y compuesto por elementos de la base
	$\cal B$. Sea $\cal V=\{B_i\}_{i\geq 1}$ dicho refinamiento.

	Para cada $i$, $B_i$ es una bola coordenada regular, por lo tanto,
	existen $B_i'\supset\clos{B_i}$ bolas coordenadas y cartas
	$\varphi_i:\,B_i'\rightarrow\bola{r_i'}{0}$ tales que
	$\varphi_i(B_i)=\bola{r_i}{0}$ para cierto $r_i<r_i'$ y
	$\varphi_i(\clos{B_i})=\clos{\bola{r_i}{0}}$. Dentro de la bola
	$\bola{r_i'}{0}$, podemos definir una funci\'{o}n suave $H_i$ tal que
	$H_i=1$ en $\clos{\bola{s_i}{0}}$, $H=0$ en el complemento de
	$\bola{t_i}{0}$ y tome un valor intermedio en el anillo
	$\bola{t_i}{0}\setmin\clos{\bola{s_i}{0}}$ (eligiendo
	$0<s_i<t_i\leq r_i<r_i'$). Luego definimos $f_i:\,M\rightarrow\bb R$
	por
	\begin{align*}
		f_i & \,:=\,
			\begin{cases}
				H_i\circ\varphi_i & \text{ en } B_i' \\
				0 & \text{ en } M\setmin\clos{B_i}
			\end{cases}
		\text{ .}
	\end{align*}
	%
	Entonces $f_i$ est\'{a} bien definida y es suave. Adem\'{a}s,
	$\soporte{f_i}\subset\clos{B_i}$ (son iguales, si $t_i=r_i$). Por otro
	lado, como $\{B_i\}_i$ es localmente finita, la familia de clausuras
	$\{\clos{B_i}\}_i$ tambi\'{e}n lo es y, en particular,
	$\{\soporte{f_i}\}_i$ es localmente finita. En consecuencia, la
	expresi\'{o}n $f=\sum_i\,f_i$ est\'{a} bien definida y es suave. Esta
	funci\'{o}n verifica $f>0$: en general, $f_i\geq 0$ en su dominio de
	definici\'{o}n, todo punto $x$ pertenece a alg\'{u}n abierto $B_i$ y,
	si $t_i=r_i$, $f_i>0$ en dicho abierto.

	Finalmente, definimos $g_i=f_i/f$. Estas funciones son suaves y suman
	$1$ en $M$. Adem\'{a}s, $\soporte{g_i}=\soporte{f_i}=\clos{B_i}$ es
	compacto. Para obtener una partici\'{o}n subordinada a $\cal U$, usamos
	el Axioma de elecci\'{o}n\dots Cada conjunto $B_i$ est\'{a} incluido en
	alg\'{u}n abierto $U_\alpha$. Denotamos por $\alpha=a(i)$ alguno de
	estos \'{\i}ndices y definimos
	\begin{align*}
		\psi_\alpha & \,:=\,\sum_{i\,:\,a(i)=\alpha}\,g_i
		\text{ .}
	\end{align*}
	%
	Algunas de estas sumatorias podr\'{\i}an ser vac\'{\i}as. La finitud
	local de los soportes de las $g_i$ garantizan que estas nuevas
	funciones est\'{e}n bien definidas y sean suaves. Las funciones
	$\psi_\alpha$ tambi\'{e}n cumplen que $0\leq\psi_\alpha\leq 1$ y
	$\sum_\alpha\,\psi_\alpha=\sum_i\,g_i=1$ en $M$. Por \'{u}ltimo,
	como las $g_i$ son no negativas y estrictamente positivas en el
	correspondiente abierto $B_i$,
	\begin{align*}
		\soporte{\psi_\alpha} & \,=\,\clos{%
				\bigcup\,\{B_i\,:\,a(i)=\alpha\}} \,=\,
			\bigcup\,\{\clos{B_i}\,:\,a(i)=\alpha\} \\
		& \,\subset\,\bigcup\,\{B_i'\,:\,a(i)=\alpha\}\,\subset\,
			U_\alpha
		\text{ .}
	\end{align*}
	%
\end{proof}

\section{Inmersiones y embeddings}
\theoremstyle{plain}
\newtheorem{propoEntornoTajadaEmbedding}{Proposici\'{o}n}[section]
\newtheorem{propoInmersionEsLocalmenteEmbedding}[propoEntornoTajadaEmbedding]%
	{Proposici\'{o}n}
\newtheorem{propoEmbeddingEntornoTajada}[propoEntornoTajadaEmbedding]%
	{Proposici\'{o}n}

\theoremstyle{remark}

%-------------

\begin{propoInmersionEsLocalmenteEmbedding}%
	\label{propo:inmersioneslocalmenteembedding}
	Toda inmersi\'{o}n es localmente embedding.
\end{propoInmersionEsLocalmenteEmbedding}

\begin{proof}
	Si $f:\,N\rightarrow M$ es una inmersi\'{o}n y $p\in N$, existen cartas
	en $p$ y en $q=f(p)$ tales que
	$\encoordenadas{f}=(\lista*{x}{n},0,\,\dots,\,0)$, donde $n=\dim\,N$.
	En particular, $f$ es inyectiva en un entorno $U_1$ de $p$ en $N$. Sea
	$U\subset U_1$ otro entorno del punto tal que $\clos U\subset U_1$ y
	$\clos U$ sea compacta. Entonces
	$f|_{\clos U}:\,\clos U\rightarrow f(\clos U)$ es biyectiva (y
	continua) entre compactos (Hausdorff) y, por lo tanto, subespacio.
	Restringiendo $f$ al entorno $U$, $f|_U:\,U\rightarrow M$ es embedding.
\end{proof}

\begin{proof}[Otra demostraci\'{o}n]
	Como $f$ tiene rango constante, si $p\in N$, existen cartas
	$(U,\varphi)$ y $(V,\psi)$ para $N$ en $p$ y para $M$ en $f(p)$,
	respectivamente, tales que $f(U)\subset V$ y
	\begin{align*}
		\encoordenadas{f}(\lista*{x}{n}) & \,=\,
			(\lista*{x}{k},0,\,\dots,\,0)
		\text{ .}
	\end{align*}
	%
	Sea $\epsilon'>0$ tal que $\cubo[m]{\epsilon'}{0}\subset\psi(V)$
	($\psi(f(p))=0$) y sea
	$V_0=\psi^{-1}(\cubo[m]{\epsilon'}{0})\subset V$. Como $V_0$ es
	abierto, $f|_U^{-1}(V_0)\subset U$ es abierto, por continuidad. Sea
	$W=U\cap f^{-1}(V_0)=f|_U^{-1}(V_0)$. Notamos que $p\in W$, por
	definici\'{o}n. Sea $\epsilon>0$ tal que
	$\cubo[n]{\epsilon}{0}\subset\varphi(W)$ ($\varphi(p)=0$) y sea
	$U_0=\varphi^{-1}(\cubo[n]{\epsilon}{0})\subset W$. Entonces
	\begin{align*}
		\psi\circ f\circ\varphi^{-1}(\cubo[n]{\epsilon}{0}) & \,=\,
			\psi\circ f(U_0) \,\subset\,\psi\circ f(W)\,\subset\,
			\psi(V_0)\,=\,\cubo[m]{\epsilon'}{0}
		\text{ .}
	\end{align*}
	%
	Con respecto a la expresi\'{o}n en coordenadas $\encoordenadas{f}$, se
	deduce que $\epsilon\leq\epsilon'$ y que
	\begin{align*}
		\encoordenadas{f}(\cubo[n]{\epsilon}{0}) & \,=\,
			\big\{(\lista*{x}{m})\,:\,
				|x^1|,\,\dots,\,|x^k|<\epsilon,\,
				x^{k+1}=\cdots=x^m=0
				\big\}
		\text{ .}
	\end{align*}
	%
	Expresado de otra manera,
	\begin{align*}
		f(U_0) & \,=\,V_1\cap \big\{x^{k+1}=\cdots=x^m=0\big\}
		\text{ ,}
	\end{align*}
	%
	donde $V_1=\psi^{-1}(\cubo[m]{\epsilon}{0})$. Vale observar que no
	es necesariamente cierto que $U_0$ sea exactamente igual a la preimagen
	de esta intersecci\'{o}n por la transformaci\'{o}n $f$.

	En definitiva, si $f$ tiene rango constante $k$ (cerca de $p$), existen
	cartas $(U_0,\varphi_0)$ y $(V_0,\psi_0)$ centradas en $p$ y en $f(p)$,
	respectivamente, tales que
	\begin{align*}
		\varphi_0(U_0) & \,=\,\cubo[n]{\epsilon}{0} \text{ ,} \\
		\psi_0(V_0) & \,=\,\cubo[m]{\epsilon}{0}
			\quad\text{(mismo radio),} \\
		f(U_0) & \,=\,V_0\cap\big\{\psi^{k+1}=\cdots=\psi^m=0\big\}
		\text{ .}
	\end{align*}
	%
\end{proof}

\begin{propoEmbeddingEntornoTajada}\label{propo:embeddingentornotajada}
	Dado un embedding $f:\,N\rightarrow M$ y un punto $p\in N$, existe un
	entorno $V_1\subset M$ de $q=f(p)$ tal que $f(N)\cap V_1$ sea una $n$-%
	tajada de $V_1$.
\end{propoEmbeddingEntornoTajada}

\begin{proof}
	Sean $(U_0,\varphi_0)$ y $(V_0,\psi_0)$ como en la demostraci\'{o}n
	anterior. Dado que $f$ es subespacio, existe un abierto $W\subset M$
	tal que $f(U_0)=f(N)\cap W$. Tomamos $V_1:=V_0\cap W$ y
	$\psi_1=\psi_0|_{V_1}$. Entonces $(V_1,\psi_1)$ es una carta para $M$
	centrada en $q$ y vale que $f(U_0)\subset V_1$. Adem\'{a}s,
	\begin{align*}
		\psi_1\circ f\circ\varphi_0^{-1}(\lista*{x}{n}) & \,=\,
			(\lista*{x}{n},\,0,\,\dots,\,0)
	\end{align*}
	%
	($k=n$). Pero, entonces,
	\begin{align*}
		V_1\,\cap\,\big\{\psi^{n+1}=\cdots=\psi^m=0\big\} & \,=\,
			f(U_0) \,=\,f(N)\,\cap\,V_1
		\text{ ,}
	\end{align*}
	%
	lo que quiere decir que $f(N)\cap V_1$ es una $n$-tajada de $V_1$.
\end{proof}

\begin{propoEntornoTajadaEmbedding}\label{propo:entornotajadaembedding}
	Si $f:\,N\rightarrow M$ verifica que, para todo punto $p\in N$, existe
	una carta $(V,\psi)$ para $M$ en $f(p)$ tal que $f(N)\cap V$ sea una
	$n$-tajada de $V$, entonces $f(N)$ tiene estructura de subvariedad
	regular.
\end{propoEntornoTajadaEmbedding}

\begin{proof}
	A $S=f(N)\subset M$ le damos la topolog\'{\i}a subespacio. Con respecto
	a esta topolog\'{\i}a, $S$ es $T_2$ y $N_2$. Sea
	\begin{align*}
		\cal A & \,:=\,\big\{(V,\psi)\text{ carta para } M\,:\,
			U\cap S\text{ es } n\text{-tajada }\big\}
		\text{ .}
	\end{align*}
	%
	Es decir, si $(V,\psi)\in\cal A$, entonces
	\begin{align*}
		\psi(V)\,=\,\cubo[m]{\epsilon}{0} & \quad\text{y}\quad
			\psi(V\cap S) \,=\,\cubo[m]{\epsilon}{0}\,\cap\,
				\big\{x^{n+1}=\cdots=x^m\big\}
		\text{ .}
	\end{align*}
	%
	Sea $\pi:\,\psi(V)\rightarrow\bb R^n$ la proyecci\'{o}n en las
	primeras $n$ coordenadas y sea $j:\,\bb R^n\rightarrow\bb R^m$ la
	inserci\'{o}n en las primeras $n$ coordenadas. En t\'{e}rminos de esta
	notaci\'{o}n, observamos que $V\cap S$ es abierto en $S$, la
	restricci\'{o}n
	\begin{align*}
		\pi_V|_{\psi(V\cap S)} & \,:\,\psi(V\cap S)\,\rightarrow\,
			\cubo[n]{\epsilon}{0}
	\end{align*}
	%
	es un homeo con inversa la restricci\'{o}n correstricci\'{o}n
	\begin{math}
		j|:\,\cubo[n]{\epsilon}{0}\rightarrow\psi(V\cap S)
	\end{math} y el par $(V\cap S,\pi_V\circ\psi)$ es una carta para $S$.

	Ahora bien, como todo punto $p\in S$ pertenece al dominio de alguno de
	las cartas en $\cal A$, la familia
	\begin{align*}
		\cal A_S & \,:=\,\big\{(V\cap S,\pi_V\circ\psi)\,:\,
			(V,\psi)\in\cal A\big\}
	\end{align*}
	%
	es un atlas para $S$. En definitiva, $S$ es localmente euclidea de
	dimensi\'{o}n $n$ y, por lo tanto, una variedad topol\'{o}gica.

	Para darle estructura diferencial, corroboramos que el atlas $\cal A_S$
	sea un atlas suavemente compatible. Sean $(V,\psi)$ y $(V',\psi')$
	cartas en $\cal A$. El cambio de coordenadas
	\begin{align*}
		(\pi_V\circ\psi)\circ(\pi_{V'}\circ\psi')^{-1} & \,:\,
		\pi_{V'}\circ\psi'\big((V\cap S)\cap(V'\cap S)\big)
			\,\rightarrow\,
		\pi_V\circ\psi\big((V\cap S)\cap(V'\cap S)\big)
	\end{align*}
	%
	est\'{a} dado por
	\begin{align*}
		(\pi_V\circ\psi)\circ(\pi_{V'}\circ\psi')^{-1}
			(\lista*{x}{n}) & \,=\,
			\pi_V\circ\psi\big(
				{\psi'}^{-1}(\lista*{x}{n},\,0,\,\dots,\,0)
				\big) \\
		& \,=\,\pi_V(\lista*{y}{m}) \,=\,(\lista*{y}{n})
		\text{ .}
	\end{align*}
	%
	Como ${\psi'}^{-1}(\lista*{x}{n},\,0,\,\dots,\,0)$ est\'{a} en $S$,
	debe ser $y^{n+1}(=\psi^{n+1})=\cdots=y^m=0$. De todos modos, como la
	composici\'{o}n $\psi\circ{\psi'}^{-1}$ es suave y $\pi_V$ y $j$
	tambi\'{e}n lo son, las cartas son compatibles. En definitiva, la
	variedad $S$ admite una estructura de variedad diferencial de
	dimensi\'{o}n $n$, sobre la topolog\'{\i}a de subespacio de $M$.

	La inclusi\'{o}n $\inc[S]:\,S\hookrightarrow M$ es un embedding: en
	coordenadas,
	\begin{align*}
		\psi\circ\inc[S]\circ(\pi_V\circ\psi)^{-1}(\lista*{x}{n})
			& \,=\,(\lista*{x}{n},\,0,\,\dots,\,0)
		\text{ .}
	\end{align*}
	%
	En particular, $\inc[S]$ es suave y tiene rango constante $n=\dim\,S$.

	Si consideramos ahora $f|:\,N\rightarrow S$, dado $p\in N$ existen
	cartas preferenciales $(U,\varphi)$ para $N$ en $p$ y $(V,\psi)$ para
	$M$ en $q=f(p)$. Como $f(N)=S$ tiene la propiedad de $n$-subvariedad,
	vale que
	\begin{math}
		f(N)\cap V=
			\cubo[m]{\epsilon}{0}\cap\{\psi^{n+1}=\cdots\psi^m=0\}
	\end{math}, con lo cual, $(V,\psi)\in\cal A$ y
	$(V\cap S,\pi_V\circ\psi)\in\cal A_S$. Con respecto a estas cartas,
	\begin{align*}
		(\pi_V\circ\psi)\circ f\circ\varphi^{-1}(\lista*{x}{n}) & \,=\,
			\pi_V\circ\encoordenadas{f}(\lista*{x}{n}) \,=\,
			\pi_V(\lista*{x}{n},\,0,\,\dots,\,0) \\
		& \,=\, (\lista*{x}{n})
		\text{ .}
	\end{align*}
	%
	Es decir, la representaci\'{o}n en coordenadas es
	\begin{math}
		(\pi_V\circ\psi)\circ f\circ\varphi^{-1}=
			\id[{\cubo[n]{\epsilon}{0}}]
	\end{math}, suave y, m\'{a}s aun, un difeomorfismo. Como
	$f|:\,N\rightarrow S$ es biyectiva y difeomorfismo local, debe ser
	difeomorfismo. En particular, $f:\,N\rightarrow M$ tiene rango
	constante (el rango de $\inc[S]$) y es subespacio.
\end{proof}

\section{Tangente en el borde}
\theoremstyle{plain}
\newtheorem{propoTangenteEnElBorde}{Proposici\'{o}n}[section]

\theoremstyle{remark}

%-------------

\begin{propoTangenteEnElBorde}\label{propo:tangenteenelborde}
	La inclusi\'{o}n $\inc:\,\hemi[n]\rightarrow\bb R^n$ induce un iso
	\begin{math}
		\diferencial[x]{\inc}:\,\tangente[x]{\hemi[n]}\rightarrow
			\tangente[x]{\bb R^n}
	\end{math} para todo $x\in\hemi[n]$
\end{propoTangenteEnElBorde}

\begin{proof}
	Sea $x\in\hemi[n]$ y sea $v\in\tangente[x]{\hemi[n]}$ una
	derivaci\'{o}n en $\suaves{x,\hemi[n]}$. Si $f\in\suaves{\hemi[n]}$,
	existe un abierto $U\subset\bb R^n$ que contiene a $x$ y una
	funci\'{o}n suave $\tilde f:\,U\rightarrow\bb R$ tales que
	\begin{align*}
		\tilde f|_{U\cap\hemi[n]} & \,=\,f|_{U\cap\hemi[n]}
		\text{ .}
	\end{align*}
	%
	En particular, $\tilde f|_{U\cap\hemi[n]}$ y $f$ tiene el mismo germen
	en $x$. Entonces, llamando $j$ a la inclusi\'{o}n
	$U\cap\hemi[n]\hookrightarrow\hemi[n]$, se verifica que
	\begin{align*}
		v\,f & \,=\,v(\tilde f\circ j)
		\text{ .}
	\end{align*}
	%
	Por definici\'{o}n de la diferencial,
	\begin{align*}
		v\,f & \,=\,\big(
			\diferencial[x]{(\inc\circ j)}\,v\big)\,\tilde f \,=\,
			\big((\diferencial[x]{\inc})\cdot
				(\diferencial[x]{j})\,v\big)\,\tilde f
		\text{ .}
	\end{align*}
	%
	Como $j$ es la inclusi\'{o}n de un abierto,
	$\diferencial[x]{j}=\id[{\tangente[x]{\hemi[n]}}]$, identificando
	naturalmente $\tangente[x]{(U\cap\hemi[n])}$ con
	$\tangente[x]{\hemi[n]}$, v\'{\i}a
	$\suaves{x,U\cap\hemi[n]}=\suaves{x,\hemi[n]}$. En definitiva, si
	$(\diferencial[x]{\inc})\,v=0$, debe ser $v\,f=0$ para toda
	$f\in\suaves{x,\hemi[n]}$. Esto demuestra que $\diferencial[x]{\inc}$
	es monomorfismo.

	Para ver que es epimorfismo, sea $w\in\tangente[x]{\bb R^n}$ y sea
	\begin{align*}
		w & \,=\,w^i\,\gancho[x]{x^i}
	\end{align*}
	%
	su escritura en la base can\'{o}nica. Supongamos, m\'{a}s aun, que
	$x\in\borde{\hemi[n]}$. Sean $g_1:\,U_1\rightarrow\bb R$ y
	$g_2:\,U_2\rightarrow\bb R$ dos funciones suaves definidas en
	abiertos $U_1,U_2\subset\bb R^n$ que contienen a $x$. Si
	$g_1|_{U_1\cap\hemi[n]}=g_2|_{U_2\cap\hemi[n]}$ --si coinciden en
	el semiespacio superior-- entonces
	\begin{align*}
		\derivada{g_1}{x^i}(x) & \,=\,\lim_{t\to0}\,
			\frac{g_1(x+e_it)-g_1(x)}{t} \,=\,
			\lim_{t\to0^+}\,\frac{g_1(x+e_it)-g_1(x)}{t} \\
		& \,=\,\lim_{t\to0^+}\,\frac{g_2(x+e_it)-g_2(x)}{t} \,=\,
			\lim_{t\to0}\,\frac{g_2(x+e_it)-g_2(x)}{t} \\
		& \,=\,\derivada{g_2}{x^i}(x)
		\text{ .}
	\end{align*}
	%
	Es decir, toda derivaci\'{o}n $w\in\tangente[x]{\bb R^n}$ queda
	determinada por los valores en $\hemi[n]$:
	\begin{align*}
		w\,g_1 & \,=\,w^i\,\derivada{g_1}{x^i}(x)\,=\,
			w^i\,\derivada{g_2}{x^i}(x) \,=\,w\,g_2
		\text{ .}
	\end{align*}
	%
	Si llamamos $v$ a la derivaci\'{o}n $v\in\tangente[x]{\hemi[n]}$ dada
	por $v\,f=w\,\tilde f$, donde $\tilde f$ es alguna extensi\'{o}n de $f$
	a un entrno de $x$ en $\bb R^n$, por lo anterior, $v$ queda bien
	definida y es derivaci\'{o}n, porque $w$ lo es. Si ahora
	$g\in\suaves{x,\bb R^n}$, entonces
	\begin{align*}
		\big(\diferencial[x]{\inc}\,v\big)\,g & \,=\,
			v\big(g|_{\hemi[n]}\big) \,=\,w\,g
	\end{align*}
	%
	y $\diferencial[x]{\inc}\,v=w$. En conclusi\'{o}n,
	$\diferencial[x]{\inc}$ es epimorfismo, tambi\'{e}n.
\end{proof}

\section{Otra demostraci\'{o}n de la existencia de particiones}
\theoremstyle{plain}

\theoremstyle{remark}

%-------------

Dado un cubrimiento por abiertos $\cal U$ de una variedad $M$, existen
funciones $\{f_U\}_{U\in\cal U}$ tales que $f_U:\,M\rightarrow\bb R$,
$f_U\geq 0$, $\sum_U\,f_U=1$, la colecci\'{o}n de soportes
$\{\soporte{f_U}\}_U$ es localmente finita y $\soporte{f_U}\subset U$ para todo
$U\in\cal U$. Se dice que la familia de funciones $\{f_U\}_U$ es una
partici\'{o}n de la unidad subordinada al cubrimiento $\cal U$.

Una bola regular en una variedad $M$ es un subconjunto abierto $B\subset M$ tal
que existe una carta (compatible) $(B',\varphi)$ donde $B'$ es una bola
coordenada (es decir, $\varphi(B')=\bola{r'}{z_0}$) y, adem\'{a}s,
$\varphi(B)=\bola{r}{z_0}$ para cierto $r<r'$ y
$\varphi(\clos B)=\clos{\bola{r}{z_0}}\subset\bola{r'}{z_0}$. Componiendo con
una traslaci\'{o}n, se puede suponer que $z_0=0$ en $\bb R^m$. Cada dominio
coordenado, dominio de una carta (compatible con la estructura de $M$) admite
una base numerable compuesta de bolas regulares para su topolog\'{\i}a: como
todo dominio coordenado es homeomorfo (difeomorfo) a un abierto de $\bb R^m$,
basta con considerar las bolas $\bola{r}{z_0}$ con centro $z_0\in\bb R^m$ de
coordenadas racionales y radio $r>0$ tambi\'{e}n racional tales que existe
$r'>r$ con $\bola{r'}{z_0}\subset\varphi(U)$. Por otro lado, si $U\subset M$ es
un abierto arbitrario y $B\subset U$ es una bola regular \emph{de $U$},
entonces $B$ tambi\'{e}n es una bola regular de $M$, es decir, existe
$(B',\varphi)$ carta compatible para $M$ con $\varphi(B')=\bola{r'}{z_0}$,
$\varphi(\clos B^M)=\clos{\bola{r}{z_0}}$ y $\varphi(B)=\bola{r}{z_0}$.

Demostremos esta \'{u}ltima afirmaci\'{o}n. Por hip\'{o}tesis, existe una carta
$(B',\varphi)$ para $U$ con la propiedad de que $\varphi(B')=\bola{r'}{z_0}$,
$\varphi(\clos B^U)=\clos{\bola{r}{z_0}}$ y $\varphi(B)=\bola{r}{z_0}$. Como
$U\subset M$ es abierto, $(B',\varphi)$ es una bola coordenada compatible para
$M$. Por razones conjunt\'{\i}sticas, $\varphi(B)=\bola{r}{z_0}$ y
$\varphi(B')=\bola{r'}{z_0}$, considerando $(B',\varphi)$ como carta para $M$ y
todo lo que resta ver es que $\varphi(\clos B^M)=\clos{\bola{r}{z_0}}$. Por un
lado, $\clos B^U$ es compacta, pues es homeomorfa (difeomorfa) a
$\clos{\bola{r}{z_0}}$. En particular, $\clos B^U$ es compacto como subespacio
de $M$ y, por lo tanto, ($M$ es $T_2$) cerrado. As\'{\i},
$\clos B^M\subset\clos B^U$. Pero$\clos B^M$ es cerrada, por definici\'{o}n, y,
como $U$ es subespacio, $\clos B^M\cap U$ es cerrado en $U$. Como $B\subset U$
y $B\subset\clos B^M$, vale que $\clos B^U\subset\clos B^M\cap U$. En
definitiva, $\clos B^M=\clos B^U$ y $B$ es regular de $M$.

La demostraci\'{o}n de la existencia de particiones de la unidad subordinadas a
un cubrimiento se basa en:
\begin{enumerate}
	\item dadas $r<r'$, $\bola{r}{0}\subset\bola{r'}{0}$, existe
		$H:\,\bb R^m\rightarrow\bb R$ suave tal que $H=1$ en
		$\clos{\bola{r}{0}}$, $H=0$ en $M\setmin\bola{r'}{0}$ y
		$0\leq H\leq 1$ en $\bb R^m$ (se puede ser m\'{a}s preciso);
	\item toda variedad admite una base de bolas regulares; y
	\item\label{existenciaderefinamientoregular}
		todo cubrimiento por abiertos de una variedad admite un
		\emph{refinamiento regular}.
\end{enumerate}
%
Un cubrimiento regular es un cubrimiento $\cal V'$ numerable y localmente
finito cuyos elementos son bolas coordenadas $B'$ tales que, si
$\varphi(B')=\bola{r'}{z_0}$, eligiendo $B=\varphi^{-1}(\bola{r}{z_0})$ para
cierto $r<r'$, la colecci\'{o}n $\cal V=\big\{B\,:\,B'\in\cal V'\big\}$ sigue
siendo un cubrimiento. Alternativamente, podemos hablar de un cubrimiento
$\cal V$ por bolas regulares $B$, de manera que podemos elegir, para cada $B$,
$(B',\varphi)$, bola coordenada que garantiza que $B$ sea regular y
$\cal V'=\big\{B'\,:\,B\in\cal V\big\}$ siga siendo localmente finito.

Veamos primero el \'{\i}tem \ref{existenciaderefinamientoregular} y, luego,
c\'{o}mo esto permite demostrar la existencia de particiones de la unidad. Los
argumentos son muy similares a los utilizados en las otras demostraciones

Sea $\cal U$ un cubrmiento por abiertos de una variedad $M$ y sea $\cal B$ una
base de bolas coordenadas (no necesariamente regulares) para $M$. Sea
$\{K_n\}_{n\geq 1}$ una sucesi\'{o}n exhaustiva de compactos y sean $F_j$ y
$W_j$ definidos de la manera usual\dots Para cada $x\in F_j$ podemos elegir
$B_x'\in\cal B$ de modo tal que $B_x'\subset W_j\cap U_x$, donde $U_x$ (elegido
de antemano) es tal que $x\in U_x$ (cualquiera de todos los posibles abiertos).
Aplicando una traslaci\'{o}n y una homotecia, podemos asumir que
$\varphi(B_x')=\bola{3}{0}$ (esto no quiere decir que $\varphi(x)=0$).
Definiendo $B_x:=\varphi^{-1}(\bola{1}{0})$, obtenemos una bola regular
(realizada por $B_x'$). Adem\'{a}s,
\begin{align*}
	x & \,\in\,B_x\,\subset\,\clos{B_x}\,\subset\,B_x'\,\subset\,
		U_x\cap W_j
	\text{ .}
\end{align*}
%
La colecci\'{o}n $\{B_x\,:\,x\in F_j\}$ es un cubrimiento de $F_j$, del cual
podemos extraer un subcubrimiento finito $\{B_{j,1},\,\dots,\,B_{j,k_j}\}$
($F_j\cap F_{j+1}\not=\varnothing$, tal vez, por lo que $B_x$ y $B_x'$ dependen
de $j$, tambi\'{e}n). Sea $B_{j,t}'$ la bola correspondiente a $B_{j,t}$ y sean
\begin{align*}
	\cal V & \,:=\,\bigcup_{j\geq 0}\,\big\{B_{j,t}\,:\,
		1\leq t\leq k_j\big\} \quad\text{y} \\
	\cal V' & \,:=\,\bigcup_{j\geq 0}\,\big\{B_{j,t}'\,:\,
		1\leq t\leq k_j\big\}
	\text{ .}
\end{align*}
%
Tanto $\cal V$ como $\cal V'$ son familias numerables de abiertos que cubren
$M$ y, para cada par $j,t$, existe $U\in\cal U$ tal que
$B_{j,t}\subset B_{j,t}'\subset U$, con lo que son refinamientos del
cubrimiento original. Como siempre, $|j-j'|\geq 3$ implica que $W_j\cap W_{j'}$
es vac\'{\i}a y, por lo tanto, cada bola $B_{j,t}'\subset V_j$ interseca, a lo
sumo, finitas otras bolas $B_{i,s}'$. Esto quiere decir que $\cal V'$ (y
\emph{a fortiori} $\cal V$) es localmente finito.

El refinamiento $\cal V'$ definido en el p\'{a}rrafo anterior tiene la
propiedad de que cada elemento de $\cal V'$ es una bola $(B',\varphi)$ tal que
$\varphi(B')=\bola{3}{0}$ y, adem\'{a}s, la colecci\'{o}n $\cal V$
compuesta por las bolas $(B,\varphi|)$, con $B=\varphi^{-1}(\bola{1}{0})$,
tambi\'{e}n es un cubrimiento de $M$. En conclusi\'{o}n, $\cal V'$ (o $\cal V$)
es un refinamiento regular (y numerable y localmente finito, por si hace falta
la aclaraci\'{o}n) de $M$.

La demostraci\'{o}n de la existencia de particiones de la unidad es
esencialmente la misma, usando $\cal V$ en lugar del cubrimiento $\cal V$ de
bolas regulares como en la demostraci\'{o}n usual (pero deber\'{\i}a ser
indistinto). Para $B'\in\cal V'$ tomamos la bola interior regular
correspondiente $B\in\cal V$ y definimos una funci\'{o}n chich\'{o}n de manera
un poco m\'{a}s intr\'{\i}nseca:
\begin{align*}
	f_{B'} & \,:=\,
		\begin{cases}
			H\circ\varphi & \text{ en }B' \\
			0 & \text{ en } M\setmin
				\clos{\varphi^{-1}(\bola{2}{0})}
		\end{cases}
	\text{ ,}
\end{align*}
%
donde $H:\,\bb R^m\rightarrow [0,1]$ es una funci\'{o}n suave que cumple
$H=0$ en el complemento de $\bola{2}{0}$, $H=1$ en $\clos{\bola{1}{0}}$ y
$0<H<1$ en otro caso. La ventaja de este argumento es que podemos ser un poco
m\'{a}s precisos en cuanto al lugar en donde $f_{B'}$ toma el valor $1$. Como
$\cal V$ es un cubrimiento, para cada $x\in M$ existe $B$ ($B'$) tal que
$f_{B'}(x)=1$. El resto de la demostraci\'{o}n es igual. Lo importante para
rescatar de este comentario es la noci\'{o}n de refinamiento regular (y su
existencia).


\begin{thebibliography}{9}
%\input{biblio.tex}
\end{thebibliography}

\end{document}
