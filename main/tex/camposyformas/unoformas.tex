\theoremstyle{plain}
\newtheorem{propoUnoFormasSuaves}{Proposici\'{o}n}[section]
\newtheorem{lemaElPullbackEnUnoFormas}[propoUnoFormasSuaves]{Lema}
\newtheorem{propoElPullbackDeUnaSuave}[propoUnoFormasSuaves]{Proposici\'{o}n}
\newtheorem{teoUnoFormasSuavesComoMorfismos}[propoUnoFormasSuaves]{Teorema}
\newtheorem{propoIntegrarElPullDeUnaCurva}%
	[propoUnoFormasSuaves]{Proposici\'{o}n}
\newtheorem{propoIntegrarConcretamenteSobreUnaCurva}%
	[propoUnoFormasSuaves]{Proposici\'{o}n}
\newtheorem{teoUnoFormaConservativaSiiExacta}[propoUnoFormasSuaves]{Teorema}

\theoremstyle{remark}
\newtheorem{obsDiferencialDerivacion}{Observaci\'{o}n}[section]
\newtheorem{obsUnoFormasYCampos}[obsDiferencialDerivacion]{Observaci\'{o}n}
\newtheorem{obsVariedadSuaveATrozosConexa}%
	[obsDiferencialDerivacion]{Observaci\'{o}n}
\newtheorem{obsIntegralEsLinealEnFormas}%
	[obsDiferencialDerivacion]{Observaci\'{o}n}
\newtheorem{obsIntegralEnCaminoInverso}%
	[obsDiferencialDerivacion]{Observaci\'{o}n}
\newtheorem{obsIntegralEsLinealEnCaminos}%
	[obsDiferencialDerivacion]{Observaci\'{o}n}
\newtheorem{obsIntegrarFormasExactas}%
	[obsDiferencialDerivacion]{Observaci\'{o}n}
\newtheorem{obsDiferencialCeroConstante}%
	[obsDiferencialDerivacion]{Observaci\'{o}n}
\newtheorem{obsIntegrarExactasEnCaminosCerrados}%
	[obsDiferencialDerivacion]{Observaci\'{o}n}
\newtheorem{obsConservativaSiiIntegralIndependienteDelCamino}%
	[obsDiferencialDerivacion]{Observaci\'{o}n}

%-------------

Un campo vectorial se define como una secci\'{o}n local del fibrado tangente.
Las secciones locales del fibrado cotangente reciben el nombre de
\emph{$1$-formas}.

Sea $(U,\varphi)$ una carta en $M$. Sean $\varphi=(x^{i})$ las funciones
coordenadas, $\Big\{\gancho[p]{x^{i}}\Big\}_{i}$ la base del espacio tangente
a $M$ en un punto $p\in U$ y sea $\big\{\de[p]{x^{i}}\big\}_{i}$ la base
dual en el espacio cotangente, la cual est\'{a} compuesta por los
diferenciales de las funciones coordenadas $x^{i}$ en $p$. Sea
$\omega:\,U\rightarrow\tangente*{M}$ una secci\'{o}n de la proyecci\'{o}n
can\'{o}nica $\pi:\,\tangente*{M}\rightarrow M$ definida, localmente,
en el abierto $U$. En particular, $\pi\circ\omega=\id[U]$. Si $p\in U$,
denotamos al elemento en el espacio $\tangente*[p]{M}$ dado por la
secci\'{o}n $\omega$ en $p$ por $\omega_{p}$, o bien $\omega|_{p}$.
Es decir, $\omega:\,p\in U\mapsto\omega_{p}\in\tangente*{M}$ es una
apliaci\'{o}n que a cada elemento $p$ en el abierto coordenado $U$ le
asigna una funcional en el espacio cotangente $\tangente*[p]{M}$.
En coordenadas, esta aplicaci\'{o}n est\'{a} dada de la siguiente manera:
cada covector $\omega_{p}\in\tangente*[p]{M}$ se escribe de manera \'{u}nica
como combinaci\'{o}n lineal de los elementos de la base
$\big\{\de[p]{x^{i}}\big\}_{i}$ dada por los diferenciales de las funciones
coordenadas, es decir, existen n\'{u}meros reales
$\omega_{i}(p)\in\bb{R}$ \'{u}nicos tales que
\begin{equation}
	\label{eq:covectencoordes}
	\omega_{p} \,=\,\omega_{i}(p)\,\de[p]{x^{i}}
	\text{ .}
\end{equation}
%
Cada uno de los coeficientes $\omega_{i}(p)$ en la escritura anterior
se denomina \emph{componente de $\omega$ en $p$} y su valor est\'{a} dado,
recurriendo a las funciones coordenadas, por
\begin{align*}
	\omega_{i}(p) & \,=\,\omega_{p}\Big(\gancho[p]{x^{i}}\Big)
	\text{ .}
\end{align*}
%
Esto determina un\'{\i}vocamente funciones $\omega_{i}:\,U\rightarrow\bb{R}$
tales que vale \eqref{eq:covectencoordes} para todo punto $p\in U$. Estas
funciones se denominan \emph{componentes de $\omega$} con respecto a
las coordenadas $\varphi=(x^{i})$ en el abierto coordenado $U$.

\subsection{$1$-formas suaves}
Recordando que todo abierto $U\subset M$ tiene estructura de variedad
diferencial en tanto subespacio abierto de la variedad $M$ y que a
$\tangente*{M}$ tambi\'{e}n se le dio una estructura de variedad diferencial,
diremos que una $1$-forma $\omega:\,U\rightarrow\tangente*{M}$ definida
en un abierto $U$ de $M$ es \emph{suave}, si es suave en tanto
transformaci\'{o}n entre las variedades $U$ y $\tangente*{M}$.
La siguiente proposici\'{o}n da una forma alternativa de caracterizar la
suavidad de una $1$-forma.

\begin{propoUnoFormasSuaves}\label{thm:unoformassuavescomponentes}
	Sea $M$ una variedad diferencial y sea $\omega:\,%
	M\rightarrow\tangente*{M}$ una $1$-forma. Las siguientes afirmaciones
	son equivalentes.
	\begin{itemize}
		\item[(\i)] $\omega$ es suave;
		\item[(\i\i)] si $(U,\varphi)$ es una carta compatible
			con la estructura de $M$, entonces las componentes
			de $\omega$ con respecto a $\varphi$ son funciones
			suaves en $U$;
		\item[(\i\i\i)] todo punto $p\in M$ est\'{a} contenido en
			el dominio de alguna carta compatible
			$(U,\varphi)$ tal que las componentes de $\omega$
			en $U$ son funciones suaves.
	\end{itemize}
	%
\end{propoUnoFormasSuaves}

Dada una carta $(U,\varphi)$, los diferenciales $\de[p]{x^{i}}$ dan
lugar a $1$-formas definidas en el abierto $U$. Dado que
los covectores $\de[p]{x^{i}}$ constituyen una base de los espacios
cotangentes en cada punto $p$ de $U$, toda $1$-forma definida en (alg\'{u}n
lugar de) $U$ se puede escribir como combinaci\'{o}n lineal de las
$1$-formas $\de{x^{i}}:\,p\mapsto\de[p]{x^{i}}$ por ciertas funciones
definidas en $U$ a valores reales. Es decir, la igualdad
\eqref{eq:covectoresencoordes} v\'{a}lida para todo $p\in U$ se puede
escribir como una igualdad entre $1$-formas: si
$\omega:\,U\rightarrow\tangente*{M}$ es una $1$-forma, entonces
existen funciones $\omega_{i}:\,U\rightarrow\bb{R}$ \'{u}nicas tales que
\begin{align*}
	\omega & \,=\,\omega_{i}\de{x^{i}}
	\text{ .}
\end{align*}
%
En particular, si $f:\,U\rightarrow\bb{R}$ es una funci\'{o}n suave,
su \emph{diferencial (derivada)} es la $1$-forma dada por
\begin{align*}
	\derext{f} & \,=\,\derivada{f}{x^{i}}\,\de{x^{i}}
	\text{ .}
\end{align*}
%

Sea $\omega$ una $1$-forma y sea $X$ un campo tangente, es decir, secciones
localmente definidas de los fibrados cotantegente y tangente, respectivamente.
Sean o no suaves o continuas, estas aplicaciones determinan, all\'{\i}
donde ambas est\'{a}n definidas, una funci\'{o}n a valores reales:
\begin{align*}
	\big(\omega X\big)(p) & \,=\,\omega_{p}\big(X_{p}\big)
\end{align*}
%
dada por evaluar el covector $\omega_{p}\in\tangente*[p]{M}$ en el vector
tangente $X_{p}\in\tangente[p]{M}$. En coordenadas, si $(U,\varphi)$ es una
carta compatible tal que ambos $\omega$ y $X$ est\'{a}n definidos en $U$,
entonces existen funciones (no necesariamente continuas) $X^{i}$ y
$\omega_{i}$ definidas en el abierto $U$ tales que
\begin{align*}
	X \,=\,X^{i}\gancho{x^{i}} & \quad\text{, }\quad
	\omega \,=\,\omega_{i}\de{x^{i}}\quad\text{y} \\
	\omega X & \,=\,\omega_{i}X^{i}
	\text{ .}
\end{align*}
%

\begin{propoUnoFormasSuaves}\label{thm:unoformassuavescampos}
	Sea $M$ una variedad diferencial y sea $\omega$ una $1$-forma
	definida en $M$. Entonces que $\omega$ sea suave equivale a que
	se cumpla cualquiera de las siguientes afirmaciones.
	\begin{itemize}
		\item[(\j)] Si $X\in\champs{M}$ es un campo (global) suave,
			entonces la funci\'{o}n $\omega X:%
			\,M\rightarrow\bb{R}$ es suave;
		\item[(\j\j)] si $U\subset M$ es abierto y $X\in\champs{U}$
			es un campo suave definido en $U$, entonces
			la funci\'{o}n $\omega X:\,U\rightarrow\bb{R}$ es
			suave.
	\end{itemize}
	%
\end{propoUnoFormasSuaves}

\begin{proof}
	Si $p\in M$ es un punto arbitrario de la variedad, entonces,
	tomando coordenadas $(U,\varphi)$ en $p$, vale que la funci\'{o}n
	$\omega X$ es igual a $\omega_{i}X^{i}$ en $U$. Si $X$ y $\omega$
	son suaves, entonces, por el \'{\i}tem \textit{(\i\i)} de
	la proposici\'{o}n \ref{thm:unoformassuavescomponentes}, las funciones
	componentes $\omega_{i}$ son suaves y, como las componentes
	$X^{i}$ tambi\'{e}n son suaves, $\omega_{i}X^{i}$ es suave en $U$
	y, por lo tanto, $\omega X$ es suave en $U$. Como $p$ es arbitrario,
	se deduce que $\omega X\in C^{\infty}(M)$.

	Supongamos que vale \textit{(\j)} y que $X:\,U\rightarrow\tangente{M}$
	es un campo suave definido en un abierto $U\subset M$. Sea $p\in U$
	un punto arbitrario del abierto. Es necesario definir un campo global
	suave que coincida con $X$ cerca de $p$. Sea
	$\psi:\,M\rightarrow\bb{R}$ una funci\'{o}n chich\'{o}n en $p$ y sea
	$V\subset U$ un entorno de $p$ contenido en $U$ tal que
	$\clos{V}\subset U$ y $\psi=1$ en $V$ y $\soporte{\psi}\subset U$.
	Sea $\tilde{X}:\,M\rightarrow\tangente{M}$ el campo global dado
	por $\psi X$ en $U$ y por $0$ en $M\setmin\soporte{\psi}$. Como
	$\psi$ es una funci\'{o}n chich\'{o}n suave, $\tilde{X}\in\champs{M}$
	y $\tilde{X}|_{V}=X|_{V}$. Entonces, por \textit{(\j)}, la funci\'{o}n
	$\omega\tilde{X}:\,M\rightarrow\bb{R}$ es suave y, adem\'{a}s, por
	construcci\'{o}n,
	\begin{align*}
		\omega\tilde{X}|_{V} & \,=\,\omega X|_{V}
		\text{ .}
	\end{align*}
	%
	Pero, entonces, $\omega X$ coincide en el abierto $V$ con una
	funci\'{o}n suave. Como $p\in U$ es arbitrario, se deduce que
	$\omega X:\,U\rightarrow\bb{R}$ es suave en $U$.

	Supongamos ahora que la afirmaci\'{o}n \textit{(\j\j)} es cierta
	para $\omega$. Sea $p\in M$ y sea $(U,\varphi)$ una carta compatible
	en $p$. Los campos locales $\gancho{x^{i}}:\,U\rightarrow\tangente{M}$
	son suaves (son constantes). Por hip\'{o}tesis, evaluar $\omega$ en
	cada uno de estos campos define una funci\'{o}n suave
	$\omega\gancho{x^{i}}$ en $U$. Pero
	\begin{align*}
		\omega\gancho{x^{i}} & \,=\,\omega_{i}
		\text{ ,}
	\end{align*}
	%
	donde $\omega_{i}$ es la componente correspondiente a
	$\de{x^{i}}$. En definitiva, todo punto de la variedad admite una
	carta en donde la forma $\omega$ tiene componentes suaves.
	Con lo cual, por \textit{(\i\i\i)}, se concluye que $\omega$ es una
	$1$-forma suave en $M$.
\end{proof}

Usaremos la notaci\'{o}n $\formes[1]{M}$ para denotar el conjunto de
$1$-formas suaves en una variedad diferencial $M$. Definiendo la suma y el
producto por escalares puntualmente:
\begin{align*}
	(a\omega+b\eta)|_{p} & \,=\,a\omega_{p}+b\eta_{p}
	\text{ ,}
\end{align*}
%
se le da una estructura de espacio vectorial real al conjunto de $1$-formas.
M\'{a}s aun, al igual que con los campos vectoriales, dada una
funci\'{o}n $f$ y una $1$-forma $\omega$ (no necesariamente continuas),
el \emph{producto} de $f$ por $\omega$,
\begin{align*}
	\big(f\omega\big)|_{p} & \,=\,f(p)\omega_{p}
	\text{ ,}
\end{align*}
%
define una nueva $1$-forma en donde ambas aplicaciones est\'{e}n definidas.
En particular, si $f\in C^{\infty}(M)$ y $\omega\in\formes[1]{M}$, se
deduce que $f\omega\in\formes[1]{M}$, tambi\'{e}n. Es decir, el espacio
vectorial $\formes[1]{M}$ tiene estructura de $C^{\infty}(M)$-m\'{o}dulo.

\begin{obsDiferencialDerivacion}\label{obs:diferencialderivacion}
	Dada una variedad $M$ y funciones suaves $f,g$, la aplicaci\'{o}n
	$f\mapsto\derext{f}\in\formes[1]{M}$ dada por tomar diferencial
	es lineal:
	\begin{align*}
		\derext{(af+bg)} & \,=\,a\derext{f}+b\derext{g}
	\end{align*}
	%
	y se comporta, tambi\'{e}n, como una derivaci\'{o}n:
	\begin{align*}
		\derext{(fg)} & \,=\,f\derext{g} + g\derext{f}
		\text{ .}
	\end{align*}
	%
\end{obsDiferencialDerivacion}

\begin{obsUnoFormasYCampos}\label{obs:unoformasycampos}
	Sea $M$ una variedad diferencial y sea
	$\omega:\,M\rightarrow\tangente*{M}$ una $1$-forma. Como se
	mencion\'{o} anteriormente, dado un campo tangente
	$X:\,M\rightarrow\tangente{M}$, se obtiene una funci\'{o}n
	$\omega X:\,M\rightarrow\bb{R}$ dada por $p\mapsto\omega_{p}X_{p}$.
	Adem\'{a}s, si $X\in\champs{M}$ y $\omega\in\formes[1]{M}$ son
	suaves, entonces $\omega X\in C^{\infty}(M)$. Por otro lado,
	$C^{\infty}(M)$ es una $\bb{R}$-\'{a}lgebra y $\champs{M}$
	y $\formes[1]{M}$ son $C^{\infty}(M)$-m\'{o}dulos. Cada $1$-forma
	suave $\omega$ define una aplicaci\'{o}n
	\begin{align*}
		\omega & \,:\,\champs{M}\,\rightarrow\,C^{\infty}(M)
		\text{ .}
	\end{align*}
	%
	Si $f\in C^{\infty}(M)$ es una funci\'{o}n suave,
	\begin{align*}
		\omega\big( fX\big) & \,:\,p\,\mapsto\,
			\omega_{p}\big(f(p)X_{p}\big) \,=\,
			f(p)\omega_{p}X_{p}
		\text{ .}
	\end{align*}
	%
	En definitiva, $\omega\big(fX\big)=f\omega X$, es decir, $\omega$
	respeta la estrctura de $C^{\infty}(M)$-m\'{o}dulos.

	Rec\'{\i}procamente, si $T:\,\champs{M}\rightarrow C^{\infty}(M)$
	es morfismo de $C^{\infty}(M)$-m\'{o}dulos, entonces se puede
	definir una $1$-forma que da lugar a morfismo $T$. En primer lugar,
	la aplicaci\'{o}n $p\mapsto (TX)(p)$, es decir, la
	funci\'{o}n suave $TX:\,M\rightarrow\bb{R}$ depende de $X$
	\'{u}nicamente en su valor en cada punto. Dicho de otra manera,
	entodo punto $p\in M$, el valor $TX(p)$ depende \'{u}nicamente del
	vector tangente $X_{p}$ y no del campo $X$. Si $Y\in\champs{M}$ es
	otro campo tal que $Y_{p}=X_{p}$ entonces $TY(p)=TX(p)$. Para
	demostrar esta afirmaci\'{o}n es suficiente demostrarlo que,
	si $X_{p}=0$, entonces $TX(p)=0$, pues $TX-TY=T(X-Y)$ para todo
	par de campos suaves $X$ e $Y$, dado que $T$ es, en particular,
	$\bb{R}$-lineal. En segundo lugar, una $1$-forma es, puntualmente,
	un covector y un covector en un punto $p$ es una funcional lineal
	en el espacio tangente a $M$ en $p$. As\'{\i}, dado que $TX$ depende
	\emph{puntualmente} de $X$, es posible definir una $1$-forma
	por $\omega_{p}v=TX(p)$, donde $X$ es cualquier extensi\'{o}n
	suave a $M$ del vector tangente $v\in\tangente[p]{M}$.

	En cuanto a la primera afirmaci\'{o}n, sea $p\in M$ un punto
	de la variedad y sea $X\in\champs{M}$ un campo suave tal que $X_{p}$
	es el vector cero. Si, en coordenadas, $X = X^{i}\gancho{x^{i}}$,
	cerca de $p$, en $p$ vale que $X_{p}=0$ y que, por independencia,
	$X^{i}(p)=0$. Sea $U$ el entorno coordenado de $p$ en donde las
	funciones $X^{i}$ y los campos $\gancho{x^{i}}$ est\'{a}n definidos.
	Sea $\psi:\,M\rightarrow\bb{R}$ una funci\'{o}n chich\'{o}n en $p$
	con soporte contenido en $U$. Para demostrar que $TX(p)=0$, ser\'{a}
	necesario poder realizar una c\'{a}lculo puntual en $p$. Para
	ello se extender\'{a}n los campos coordenados y las funciones
	componentes de $X$. Para cada \'{\i}ndice $i$, sea $\tilde{X}^{i}$
	la funci\'{o}n que toma el valor $\psi\cdot X^{i}$ en $U$ y el valor
	$0$ fuera del soporte de $\psi$. Sean, tambi\'{e}n $Y_{i}$ los campos
	dados por $\psi\cdot\gancho{x^{i}}$ en $U$ y por $0$ fuera del
	soporte de $\psi$. Entonces, en toda $M$,
	\begin{align*}
		X & \,=\, \tilde{X}^{i}Y_{i} + (1-(\psi)^{2})X
		\text{ .}
	\end{align*}
	%
	Aplicando $T$,
	\begin{align*}
		TX & \,=\, \tilde{X}^{i}TY_{i} + (1-(\psi)^{2})TX
		\text{ .}
	\end{align*}
	%
	En el punto $p$, la funci\'{o}n $\psi$ vale $\psi(p)=1$, por lo que
	\begin{align*}
		TX(p) & \,=\,\tilde{X}^{i}(p)TY_{i}(p) +
			(1-\psi(p)^{2})TX(p) \,=\,
			X^{i}(p)TY_{i}(p)
		\text{ .}
	\end{align*}
	%
	Entonces, asumiendo que $X_{p}=0$, las funciones componentes
	$X^{i}$ toman el valor $0$ en $p$ y $TX(p)=0$.

	Definir una $1$-forma a partir del morfismo $T$, implica dar, para
	cada $p\in M$, una funcional $\omega_{p}\in\tangente*[p]{M}$.
	Sea $p\in M$ y sea $v_{p}\in\tangente[p]{M}$ un vector tangente
	en $p$. Sea $X\in\champs{M}$ un campo suave tal que $X_{p}=v_{p}$.
	Un campo con esta propiedad se puede definir usando una funci\'{o}n
	chich\'{o}n en $p$. Sea $\omega_{p}(v_{p})=TX(p)$. El valor
	de $\omega_{p}$ en el vector $v_{p}$ est\'{a} bien definido pues
	el valor de la funci\'{o}n $TX$ en $p$ no depende del campo $X$
	particular elegido como extensi\'{o}n de $v_{p}$. Adem\'{a}s,
	$\omega_{p}:\,\tangente[p]{M}\rightarrow\bb{R}$ es lineal, ya que,
	si $v_{p},v'_{p}\in\tangente[p]{M}$ y $a\in\bb{R}$, y si
	$X,X'\in\champs{M}$ son campos suaves tales que $X_{p}=v_{p}$ y
	$X'_{p}=v'_{p}$, entonces $X+aX'\in\champs{M}$ es un campo suave
	tal que $(X+aX')|_{p}=v_{p}+av'_{p}$. En particular, vale que
	\begin{align*}
		\omega_{p}(v_{p}+av'_{p}) & \,=\,T(X+aX')(p) \,=\,
			TX(p)+aTX'(p) \,=\,
			\omega_{p}(v_{p})+a\omega_{p}(v'_{p})
		\text{ .}
	\end{align*}
	%
	As\'{\i}, se deduce que $\omega:\,M\rightarrow\tangente*{M}$ es una
	$1$-forma. Por otro lado, dado un campo suave $X\in\champs{M}$,
	\begin{align*}
		\omega X (p) & \,=\,\omega_{p}X_{p} \,=\, TX(p)
		\text{ .}
	\end{align*}
	%
	Esta igualdad muestra dos cosas: en primer lugar, muestra que
	$\omega X=TX$ para todo campo suave $X$ y, en segundo lugar, que
	$\omega X=TX\in C^{\infty}(M)$ para todo campo suave
	$X\in\champs{M}$. Por la proposici\'{o}n
	\ref{thm:unoformassuavescampos}, la forma $\omega$ es suave.
\end{obsUnoFormasYCampos}

De la misma manera que el teorema \ref{thm:derivacionesycampos} dice que
los campos suaves en una variedad $M$ se corresponden con las derivaciones de
$C^{\infty}(M)$, la observaci\'{o}n anterior muestra que las $1$-formas
suaves se pueden caracterizar de manera similar.

\begin{teoUnoFormasSuavesComoMorfismos}%
	\label{thm:unoformassuavescomomorfismos}
	Sea $M$ una variedad diferencial. Toda $1$-forma suave
	$\omega\in\formes[1]{M}$ define un morfismo de
	$C^{\infty}(M)$-m\'{o}dulos
	\begin{align*}
		\omega & \,:\,\champs{M}\,\rightarrow\,C^{\infty}(M)
	\end{align*}
	%
	por $\omega:\,X\mapsto (\omega X:\,p\mapsto\omega_{p}X_{p})$.
	Rec\'{\i}procamente, todo morfismo $T:\,%
	\champs{M}\rightarrow C^{\infty}(M)$ de $C^{\infty}(M)$-m\'{o}dulos
	determina una \'{u}nica $1$-forma suave $\omega\in\formes[1]{M}$
	tal que
	\begin{align*}
		\omega X & \,=\, TX
	\end{align*}
	%
	para todo campo suave $X\in\champs{M}$.
\end{teoUnoFormasSuavesComoMorfismos}

\subsection{El \emph{pullback} de una $1$-forma}
Sean $M$ y $N$ variedades diferenciales y sea $F:\,M\rightarrow N$ una
transformaci\'{o}n suave. Sea $\omega:\,N\rightarrow\tangente*{N}$ una
$1$-forma (no necesariamente continua) en $N$. Puntualmente, dado $p\in M$,
se defin\'{\i}a el pullback de $\omega_{F(p)}$ por $F$ en $p$ como el
covector en $\tangente*[p]{M}$ dado por
\begin{align*}
	\diferencial*[p]{F}(\omega_{F(p)})(v_{p}) & \,=\,
		\omega_{F(p)}(\diferencial[p]{F}(v_{p}))
	\text{ .}
\end{align*}
%
Esta construcci\'{o}n da lugar a una $1$-forma $\pull{F}\omega$ definida
en $M$ (si $\omega$ est\'{a} definida en un abierto $V\subset N$, entonces
$\pull{F}\omega$ queda definida en el abierto $F^{-1}(V)\subset M$):
dado un punto $p\in M$, $(\pull{F}\omega)|_{p}\in\tangente*[p]{M}$ es el
covector dado por
\begin{align*}
	(\pull{F}\omega)|_{p}(v_{p}) & \,=\,
		\omega_{F(p)}(\diferencial[p]{F}(v_{p})) \,=\,
	\text{ ,}
\end{align*}
%
es decir,
\begin{align*}
	(\pull{F}\omega)|_{p} & \,=\,\diferencial*[p]{F}(\omega_{F(p)})
	\text{ .}
\end{align*}
%

En el caso de un campo, no es posible definir el \emph{pushforward} del mismo,
salvo casos muy particulares. En cuanto a $1$-formas, no hay inconveniente
en definir el \emph{pullback} de una $1$-forma por una transformaci\'{o}n
suave. En definitiva, dada una transformaci\'{o}n suave
$F:\,M\rightarrow N$, queda definida una transformaci\'{o}n
$\pull{F}$ que toma una $1$-forma en $N$ y devuelve una $1$-forma en $M$
(en el sentido opuesto). Esta aplicaci\'{o}n es una transformaci\'{o}n lineal
ya que, bajando a un punto $p\in M$, se ve que
\begin{align*}
	(\pull{F}(a\omega+b\eta))|_{p} & \,=\,
		\diferencial*[p]{F}((a\omega+b\eta)|_{F(p)}) \,=\,
		a\diferencial*[p]{F}(\omega_{F(p)}) +
			b\diferencial*[p]{F}(\eta_{F(p)}) \,=\,
		a(\pull{F}\omega)|_{p} +b(\pull{F}\eta)|_{p}
	\text{ .}
\end{align*}
%
En definitiva,
\begin{align*}
	\pull{F}(a\omega+b\eta)=a\pull{F}\omega+b\pull{F}\eta
\end{align*}
%
para todo par de $1$-formas $\omega$ y $\eta$ y n\'{u}meros reales
$a$ y $b$.

\begin{lemaElPullbackEnUnoFormas}\label{thm:elpullbackenunoformas}
	Sea $F:\,M\rightarrow N$ una transformaci\'{o}n suave.
	Sea $u:\,N\rightarrow\bb{R}$ una funci\'{o}n y sea
	$\omega:\,N\rightarrow\tangente*{N}$ una $1$-forma en $N$.
	Entonces
	\begin{align*}
		\pull{F}(u\omega) & \,=\,(u\circ F)\,\pull{F}\omega
		\text{ .}
	\end{align*}
	%
	Si $u\in C^{\infty}(N)$, entonces
	\begin{align*}
		\pull{F}\derext{u} & \,=\,\derext{(u\circ F)}
		\text{ .}
	\end{align*}
	%
\end{lemaElPullbackEnUnoFormas}

\begin{proof}
	En las hip\'{o}tesis del enunciado del lema, puntualmente,
	\begin{align*}
		\big(\pull{F}(u\omega)\big)|_{p} & \,=\,
			u(F(p))\diferencial*[p]{F}(\omega_{F(p)}) \,=\,
			\big((u\circ F)\,\pull{F}\omega\big)|_{p}
		\text{ .}
	\end{align*}
	%
	Asumiendo que $u$ es suave, su diferencial est\'{a} bien definido y,
	dado $v\in\tangente[p]{M}$, vale que
	\begin{align*}
		(\pull{F}\derext{u})|_{p} & \,=\,
			\derext[F(p)]{u}\big(\diferencial[p]{F}(v)\big) \,=\
			\diferencial[p]{F}(v)u \\
		& \,=\,v(u\circ F) \,=\,\derext[p]{(u\circ F)}(v)
		\text{ .}
	\end{align*}
	%
\end{proof}

\begin{propoElPullbackDeUnaSuave}\label{thm:elpullbackdeunasuave}
	Sea $F:\,M\rightarrow N$ una transformaci\'{o}n suave y
	sea $\omega:\,N\rightarrow\tangente*{N}$ una $1$-forma.
	Entonces, si $\omega$ es continua, $\pull{F}\omega$ es una
	$1$-forma continua en $M$. Si $\omega$ es suave, entonces
	$\pull{F}\omega$ es una $1$-forma suave en $M$.
\end{propoElPullbackDeUnaSuave}

\begin{proof}
	Sea $p\in M$ un punto arbitrario de la variedad. Sea $(V,\psi)$
	una carta en $F(p)$ compatible con la estructura de $N$ y sea
	$U=F^{-1}(V)$. Por continuidad de $F$, la preimagen $U$ es abierta
	en $M$ y contiene al punto $p$. En coordenadas, y usando
	el lema \ref{thm:elpullbackenunoformas}, se deduce que
	\begin{align*}
		\pull{F}\omega & \,=\,\pull{F}\big(\omega_{j}\de{y^{j}}\big)
			\,=\,(\omega_{j}\circ F)\,\derext{(y^{j}\circ F)}
			\text{ .}
	\end{align*}
	%
	Ahora bien, $\omega$ es continua, si y s\'{o}lo si las componentes
	$\omega_{i}$ son continuas. Por otro lado, las funciones
	$y^{j}\circ F:\,U\rightarrow\bb{R}$ son suaves, por que la carta
	$(V,\psi)$ es compatible y $F$ es suave. Esto implica que las
	$1$-formas $\derext{(y^{j}\circ F)}$ (definidas en $U$) son
	suaves. En coordenadas,
	\begin{align*}
		\derext{(y^{j}\circ F)} & \,=\,
			\derivada{(y^{j}\circ F)}{x^{i}}\de{x^{i}}
		\text{ .}
	\end{align*}
	%
	Pero las funciones $\derivada{(y^{j}\circ F)}{x^{i}}$ no son otra
	cosa que las componentes de la matriz jacobiana de $F$ con respecto
	a estas coordenadas (donde est\'{e}n definidas) y, por lo tanto,
	son funciones suaves.
	
	En definitiva, si $\omega$ es continua, las componentes $\omega_{i}$
	son continuas y la $1$-forma en $M$ dada por
	$\pull{F}\omega$ se expresa localmente como combinaci\'{o}n
	de formas suaves por funciones continuas, con lo que resulta ser
	continua. Y si $\omega$ es suave, las componentes son suaves y
	el pullback resulta ser suave, tambi\'{e}n.
\end{proof}

Sabiendo que $\pull{F}\omega$ es una $1$-forma continua, si $\omega$ lo
es y que es suave, si $\omega$ lo es, se deduce que el pullback de $F$
es una transformaci\'{o}n lineal $\pull{F}:\,%
\formes[1]{N}\rightarrow\formes[1]{M}$. Adem\'{a}s, por
\ref{thm:elpullbackenunoformas}, respeta en cierto sentido las estructuras
de m\'{o}dulos sobre $C^{\infty}(M)$ y sobre $C^{\infty}(N)$, asignando, a
una funci\'{o}n (suave, continua, nada) $g$ definida en $N$, la funci\'{o}n
(suave, continua, nada, respectivamente) $f=g\circ F$ en $M$
(el pullback de $g$ por $F$).

\subsection{Integrar sobre segmentos}
Sea $[a,b]\subset\bb{R}$ un intervalo real compacto ($a$ y $b$ finitos).
Sea $\omega\in\formes[1]{[a,b]}$ una $1$-forma suave. Definiremos lo que
significa integrar la $1$-forma $\omega$ en/sobre el intervalo $[a,b]$ --o,
mejor dicho, mostraremos que la noci\'{o}n intuitiva tiene sentido.

Antes de pasar a integrar sobre el intervalo, aplicamos los resultados
generales sobre $1$-formas al caso particular de formas sobre esta
variedad. El intervalo, como variedad con borde se puede cubrir usando
dos cartas: $(\left[a,b\right),r:\,x\mapsto x-a)$ y
$(\left(a,b\right],s:\,x\mapsto -(x-b))$. Estas dos cartas son compatibles.
El cambio de coordenadas en $(a,b)$ est\'{a}, en uno de los dos sentidos,
dado por
\begin{align*}
	& x\,\mapsto\, x+a\,\mapsto\,-(x+a-b)= (b-a) - x
	\text{ ,}
\end{align*}
%
es decir, por invertir el intervalo $(0,b-a)$. Sea
$\omega\in\formes[1]{[a,b]}$. En cada una de estas cartas, $\omega$ se puede
escribir como un m\'{u}ltiplo, por una funci\'{o}n suave, de los
diferenciales de las funciones coordenadas respectivas:
\begin{align*}
	\omega(r+a) & \,=\, f(r)\,\derext{r}\quad\text{, si }a\leq r<b
	\quad\text{y} \\
	\omega(b-s) & \,=\,g(s)\,\derext{s}\quad\text{, si }a<s\leq b
	\text{ .}
\end{align*}
%
El cambio de coordenadas $s\circ r^{-1}(x)=F(x)=(b-a)-x$ es la
representaci\'{o}n de la funci\'{o}n identidad en $(a,b)$. El pullback
de $\omega$ por la identidad expresa la relaci\'{o}n entre las dos
representaciones de $\omega$. Por un lado,
\begin{align*}
	\pull{\id[(a,b)]}(g\,\derext{s}) & \,=\,
		g\circ\id[(a,b)]\cdot\big(\pull{\id[(a,b)]}\derext{s}\big)
	\text{ .}
\end{align*}
%
Para expresar $\pull{\id[(a,b)]}\derext{s}$ con respecto a la
$1$-forma $\derext{r}$, recurrimos a la definici\'{o}n:
\begin{align*}
	(\pull{\id}\derext{s})|_{x}\gancho[x]{r} & \,=\,
		\derext[\id(x)]{s}\cdot
		\diferencial[x]{\id}\Big(\gancho[x]{r}\Big)\\
	& \,=\,\derext[\id(x)]{s}\Big(v\cdot\gancho[\id(x)]{s}\Big) \,=\,v
	\text{ ,}
\end{align*}
%
donde
\begin{align*}
	v & \,=\,\diferencial[x]{\id}\Big(\gancho[x]{r}\Big)(s) \,=\,
		\gancho[x]{r}(s\circ\id)\,=\,(s\circ\id\circ r^{-1})'(r(x)) \\
	& \,=\,F'(r(x)) \,=\,-1
	\text{ .}
\end{align*}
%
En definitiva,
\begin{align*}
	f\,\derext{r} \,=\,\omega & \,=\,g\,\derext{s} \,=\,
		\pull{\id[(a,b)]} \big(g\,\derext{s}\big) \\
	& \,=\,	g\circ\id[(a,b)]\cdot\big(\pull{\id[(a,b)]}\derext{s}\big)
		\,=\,-g\,\derext{r}
\end{align*}
%
de lo que se deduce que $f=-g$ 

Volviendo a la cuesti\'{o}n de la integral, sea $\omega=f\,\derext{r}$ la
expresi\'{o}n de una $1$-forma suave en el abierto coordenado $[a,b)$.
El punto $b$ en el borde del intervalo es un \emph{conjunto de medida nula}.
Si bien esta noci\'{o}n no fue definida, pero es una propiedad deseable.
Entonces, para integrar $\omega$ en $[a,b]$, deber\'{\i}a, intuitivamente,
ser suficiente considerar $\omega$ restringida a $[a,b)$, en donde
la forma se puede representar como una funci\'{o}n por la forma
can\'{o}nica $\derext{r}$ proveniente de la carta $([a,b),r)$. Se define
\emph{la integral de $\omega$ en el inervalo $[a,b]$} como
\begin{align*}
	\int_{[a,b]}\,\omega & \,=\,\int_{0}^{b-a}\,f\circ r^{-1}(x)\,dx
	\text{ .}
\end{align*}
%
La integral as\'{\i} definida viene con un \emph{sentido de integraci\'{o}n},
se integra sobre $[a,b]$ de $a$ hacia $b$. Tal vez de manera un poco m\'{a}s
precisa \'{e}sta es la integral sobre un \emph{intervalo orientado}.
La integral $\int_{[a,b]}\,\omega$ est\'{a} dada por la integral de Riemann
de la representaci\'{o}n en coordenadas $f\circ r^{-1}$ de la funci\'{o}n
$f$ que representa a la forma $\omega$ en t\'{e}rminos de la forma
$\derext{r}$. En particular, $f\circ r^{-1}$ est\'{a} definida en
el intervalo abierto cerrado $[0,b-a)$.

Una \emph{reparametrizaci\'{o}n de $[a,b]$} es un difeomorfismo
$\varphi:\,[c,d]\rightarrow [a,b]$. Usando la f\'{o}rmula del pullback,
\begin{align*}
	\pull{\varphi}\omega & \,=\,\pull{\varphi}\big(f\,\derext{r}\big)
		\,=\,(f\circ\varphi)\cdot\pull{\varphi}\derext{r}
	\text{ .}
\end{align*}
%
Si $\varphi(c)=a$, $\varphi(d)=b$ (es decir, $\varphi$ es creciente, es
una \emph{reparametrizaci\'{o}n}) y en $[c,d)$ la coordenada est\'{a}
dada por $s:\,[c,d)\rightarrow [0,d-c)$ con $s(x)=x-c$, entonces
\begin{align*}
	\pull{(\varphi\circ s^{-1})}\derext{r} & \,=\,
		(r\circ\varphi\circ s^{-1})'\cdot\derext{x} \,=\,
		(\varphi(x+c)-a)'(x)\,\derext{x} \\
	& \,=\,\varphi'(x+c)\,\derext{x} \,=\,\varphi'\circ s^{-1}\,\derext{x}
	\text{ .}
\end{align*}
%
Expresado de manera m\'{a}s concisa,
\begin{align*}
	\pull{\varphi}\derext{r} & \,=\,\varphi'\,\derext{s}
	\quad\text{y} \\
	\pull{\varphi}\omega & \,=\,\pull{\varphi}\big(f\,\derext{r}\big) \,=\,
		(f\circ\varphi)\cdot\varphi'\,\derext{s}
	\text{ .}
\end{align*}
%
En particular, en $[c,d)$, la integral del pullback es igual a
\begin{align*}
	\int_{[c,d]}\,\pull{\varphi}\omega & \,=\,
		\int_{[c,d]}\,(f\circ\varphi)\cdot\varphi'\,\derext{s} \,=\,
		\int_{0}^{d-c}\,(f\circ\varphi\circ s^{-1})(x)\,
			(\varphi'\circ s^{-1})(x)\,dx
		\text{ .}
\end{align*}
%
Dado que $\frac{\varphi'\circ s^{-1}}{|\varphi'\circ s^{-1}|}=+1$ en
$[c,d]$, se deduce que
\begin{align*}
	\int_{[c,d]}\,\pull{\varphi}\omega & \,=\,
		\int_{0}^{b-a}\,(f\circ r^{-1})(y)\,dy \,=\,
		\int_{[a,b]}\,\omega
	\text{ .}
\end{align*}
%
Si, en cambio, $\varphi(c)=b$ y $\varphi(d)=a$ ($\varphi$ es decreciente),
entonces
\begin{align*}
	\int_{[c,d]}\,\pull{\varphi}\omega & \,=\,
		\int_{0}^{d-c}\,(f\circ\varphi\circ s^{-1})(x)\,
			(\varphi'\circ s^{-1})(x)\,dx \\
	& \,=\,-\int_{0}^{b-a}\,(f\circ r^{-1})(y)\,dy \,=\,
		-\int_{[a,b]}\,\omega
	\text{ ,}
\end{align*}
%
pues $\frac{\varphi'\circ s^{-1}}{|\varphi'\circ s^{-1}|}=-1$ en $[c,d]$,
en ese caso.

Todo esto es para dar sentido a una identificaci\'{o}n entre
$1$-formas diferenciables en un intervalo compacto $[a,b]\subset\bb{R}$ y
funciones de manera de que se pueda hablar de la integral de una forma
en t\'{e}rminos de la noci\'{o}n conocida de integral de una funci\'{o}n.
Tal vez haya una manera m\'{a}s obvia de hacer esta identificaci\'{o}n o
alguna manera natural.

Una vez definida la noci\'{o}n de integral sobre un intervalo, integrar en
variedades es algo m\'{a}s o menos directo.

\begin{obsVariedadSuaveATrozosConexa}\label{obs:variedadsuaveatrozosconexa}
	En una variedad conexa, todo par de puntos se puede conectar por
	una curva $C^{1}$ (suave) a trozos.
\end{obsVariedadSuaveATrozosConexa}

Sea $M$ una varieda diferencial y sea $\gamma:[a,b]\rightarrow M$ una
curva suave ($C^{1}$), es decir, una transformaci\'{o}n suave entre
variedades, o bien una funci\'{o}n que se extiende a un entorno del
intervalo en $\bb{R}$ de manera que se obtenga una transformaci\'{o}n
suave. Dada $\omega\in\formes[1]{M}$, la \emph{integral de $\omega$ sobre %
$\gamma$} se define como
\begin{align*}
	\int_{\gamma}\,\omega & \,\equiv\,\int_{[a,b]}\,\pull{\gamma}\omega
	\text{ .}
\end{align*}
%
Si $\gamma$ es suave a trozos, de manera que $\gamma$ sea suave en ciertos
subintervalos $[a_{i},a_{i+1}]$, entonces se define
\begin{align*}
	\int_{\gamma}\,\omega & \,\equiv\,\sum_{i}\,
		\int_{[a_{i},a_{i+1}]}\,\pull{\gamma_{i}}\omega
	\text{ ,}
\end{align*}
%
donde $\gamma_{i}=\gamma|_{[a_{i},a_{i+1}]}$.

\begin{obsIntegralEsLinealEnFormas}\label{obs:integraleslinealenformas}
	Dada $\gamma:\,[a,b]\rightarrow M$ suave a trozos, la integral
	sobre $\gamma$ define una funci\'{o}n lineal
	\begin{align*}
		\int_{\gamma} & \,:\,\formes[1]{M}\,\rightarrow\,\bb{R}
	\end{align*}
	%
	e $\int_{\gamma}=0$, si $\gamma$ es un camino constante.
\end{obsIntegralEsLinealEnFormas}

\begin{obsIntegralEnCaminoInverso}\label{obs:integralencaminoinverso}
	La relaci\'{o}n con reparametrizaciones se mantiene en este
	contexto, tambi\'{e}n: si $\tilde{\gamma}=\gamma\circ\varphi$
	es una reparametrizaci\'{o}n de $\gamma$, entonces
	\begin{align*}
		\int_{\tilde{\gamma}}\,\omega & \,=\,
			+\int_{\gamma}\,\omega\quad\text{o}\quad
			-\int_{\gamma}\,\omega
		\text{ ,}
	\end{align*}
	%
	de acuerdo a si $\tilde{\gamma}$ es una \emph{reparametrizaci\'{o}n %
	positiva}, es decir, si $\varphi$ es creciente, o si,
	respectivamente, $\tilde{\gamma}$ es una \emph{reparametrizaci\'{o}n %
	negativa}, es decir, $\varphi$ es decreciente.
\end{obsIntegralEnCaminoInverso}

\begin{obsIntegralEsLinealEnCaminos}\label{obs:integraleslinealencaminos}
	Extendiendo por linealidad, la aplicaci\'{o}n
	$\gamma\mapsto\int_{\gamma}\,\omega$ dada en curvas suaves a
	trozos, fijada la forma $\omega\in\formes[1]{M}$, determina una
	funcional lineal
	\begin{align*}
		\omega & \,:\,\cadenas[1]{M}\,\rightarrow\,\bb{R}
	\end{align*}
	%
	en el espacio de \emph{$1$-cadenas suaves}. M\'{a}s all\'{a} de esta
	extensi\'{o}n formal, integrar una forma respeta subdivisiones del
	intervalo: si $\gamma:\,[a,b]\rightarrow M$ y
	$\gamma_{1}=\gamma|_{[a,c]}$ y $\gamma_{2}=\gamma|_{[c,b]}$, entonces
	\begin{align*}
		\int_{\gamma}\,\omega & \,=\,\int_{\gamma_{1}}\,\omega
			+\int_{\gamma_{2}}\,\omega
		\text{ .}
	\end{align*}
	%
\end{obsIntegralEsLinealEnCaminos}

\begin{propoIntegrarElPullDeUnaCurva}\label{thm:integrarenelpulldeunacurva}
	Si $F:\,M\rightarrow N$ es suave y $\eta\in\formes[1]{N}$, dada
	$\gamma:\,[a,b]\rightarrow M$,
	\begin{align*}
		\int_{\gamma}\,\pull{F}\eta & \,=\,\int_{F\circ\gamma}\,\eta
		\text{ .}
	\end{align*}
	%
\end{propoIntegrarElPullDeUnaCurva}

\begin{proof}
	\begin{align*}
		\int_{\gamma}\,\pull{F}\eta & \,=\,
			\int_{[a,b]}\,\pull{\gamma}(\pull{F}\eta) \,=\,
			\int_{[a,b]}\,\pull{(F\circ\gamma)}\eta \,=\,
			\int_{F\circ\gamma}\,\eta
		\text{ .}
	\end{align*}
	%
\end{proof}

\begin{propoIntegrarConcretamenteSobreUnaCurva}%
	\label{thm:integrarconcretamentesobreunacurva}
	Si $\gamma:\,[a,b]\rightarrow M$ es una curva suave a trozos y
	$\omega\in\formes[1]{M}$,
	\begin{align*}
		\int_{\gamma}\,\omega & \,=\,
			\int_{a}^{b}\,\omega_{\gamma(t)}
				\big(\dot{\gamma}(t)\big)\,dt
	\end{align*}
	%
\end{propoIntegrarConcretamenteSobreUnaCurva}

\begin{proof}
	Por definici\'{o}n,
	\begin{align*}
		\int_{\gamma}\,\omega & \,=\,\int_{[a,b]}\,\pull{\gamma}\,=\,
			\int_{0}^{b-a}\,f\circ r^{-1}(x)\,dx
		\text{ ,}
	\end{align*}
	%
	donde $r:\,[a,b)\rightarrow\bb{R}$ es la funci\'{o}n coordenada
	$y\mapsto y-a$ y la integral es la integral de Riemann.
	Componiendo con la traslaci\'{o}n, $x\mapsto x+a$, se obtiene
	la igualdad
	\begin{align*}
		\int_{0}^{b-a}\,f\circ r^{-1}(x)\,dx & \,=\,
			\int_{a}^{b}\,f(t)\,dt
		\text{ .}
	\end{align*}
	%
	Esto corresponde a tomar la ``carta de borde''
	$\id:\,[a,b)\rightarrow[a,b)$, en lugar de
	$r:\,[a,b)\rightarrow[0,b-a)$ que, si bien es lo correcto por
	definici\'{o}n, no es muy intuitivo.

	Ahora bien, como no hay, propiamente, una carta global en
	$[a,b]$ --aunque $\gamma$ se podr\'{\i}a extender a un intervalo
	abierto, de forma que s\'{\i} la haya-- no es posible argumentar de
	la siguiente manera:
	\begin{align*}
		(\pull{\gamma}\omega)_{t}\gancho[t]{t} & \,=\,
			\omega_{\gamma(t)}\Big(\diferencial[t]{\gamma}
				\gancho[t]{t}\Big) \,=\,
			\omega_{\gamma(t)}\dot{\gamma}(t)
		\text{ .}
	\end{align*}
	%
	De todas maneras, tomando coordenadas en $M$,
	\begin{align*}
		\pull{\gamma}\omega & \,=\,(\omega_{j}\circ\gamma)\cdot
			\derext{\gamma^{j}} \,=\,\omega_{j}\circ\gamma\cdot
			(\gamma^{j})'\,\derext{t} \\
		& \,=\,\omega_{\gamma(t)}\,\dot{\gamma}(t)\,\derext{t}
		\text{ .}
	\end{align*}
	%
	Por otro lado, porque $[a,b]$ es compacto, se puede subdividir el
	intervalo de manera que cada subintervalo de la divisi\'{o}n est\'{e}
	contenido en la preimagen del dominio de una carta. Aplicando el
	argumento anterior a cada subintervalo y usando la propiedad
	de que $\omega$ respeta subdivisiones,
	\begin{align*}
		\int_{\gamma}\,\omega & \,=\,\sum_{i}\,
			\int_{\gamma_{i}}\,\omega \,=\,\sum_{i}\,
			\int_{a_{i}}^{a_{i+1}}\,\omega_{\gamma_{i}(t)}\,
				\dot{\gamma_{i}}(t)\,dt
		\text{ .}
	\end{align*}
	%
	Si $\gamma$ es suave a trozos, se aplica el argumento en cada
	subintervalo en donde $\gamma$ es suave y se suma como antes.
\end{proof}

\begin{obsIntegrarFormasExactas}\label{obs:integrarformasexactas}
	Si $\omega$ es el diferencial de una funci\'{o}n suave,
	$\omega=\derext{f}$, entonces
	\begin{align*}
		\int_{\gamma}\,\derext{f} & \,=\,\int_{[a,b]}\,
			\pull{\gamma}(\derext{f}) \,=\,
			\int_{a}^{b}\,\dot{\gamma}(t)\,f\,dt \\
		& \,=\,\int_{a}^{b}\,(f\circ\gamma)'(t)\,dt \,=\,
			\sum_{i}\,f(\gamma(a_{i+1}))-f(\gamma(a_{i})) \\
		& \,=\,f(\gamma(b))-f(\gamma(a))
		\text{ .}
	\end{align*}
	%
\end{obsIntegrarFormasExactas}

\begin{obsDiferencialCeroConstante}\label{obs:diferencialceroconstante}
	Si $M$ es una variedad conexa y $f\in C^{\infty}(M)$,
	entonces $\derext{f}=0$ implica que $f$ es constante.
\end{obsDiferencialCeroConstante}

Una $1$-forma $\omega\in\formes[1]{M}$ se dice \emph{exacta}, si existe
$f\in C^{\infty}(M)$ tal que $\omega=\derext{f}$. Si $f,g\in C^{\infty}(M)$
son funciones tales que $\derext{f}=\derext{g}$, entonces $\derext{(f-g)}=0$,
por linealidad, y $f-g$ es constante en cada componente conexa de la variedad.

\begin{obsIntegrarExactasEnCaminosCerrados}%
	\label{obs:integrarexactasencaminoscerrados}
	Dada una funci\'{o}n suave $f\in C^{\infty}(M)$, la integral
	$\int_{\gamma}\,\derext{f}$ es cero para toda curva suave a trozos
	$\gamma$ tal que $\gamma(a)=\gamma(b)$. Tales curvas se denominan
	\emph{cerradas}.
\end{obsIntegrarExactasEnCaminosCerrados}

Una $1$-forma se dice \emph{conservativa}, si $\int_{\gamma}\,\omega=0$ para
todo camino suave a trozos cerrado $\gamma$. Un camino cerrado suave a trozos,
no necesariamente es un ``loop suave'' (una transformaci\'{o}n suave
$\esfera[1]\rightarrow M$).

\begin{obsConservativaSiiIntegralIndependienteDelCamino}%
	\label{obs:conservativasiiintegralindependientedelcamino}
	Una $1$-forma $\omega\in\formes[1]{M}$ es conservativa, si y
	s\'{o}lo si $\int_{\gamma}\,\omega=\int_{\delta}\,\omega$ para todo
	par de curvas suaves a trozos tales que empiecen en el mismo
	punto y terminen en el mismo punto en $M$.
\end{obsConservativaSiiIntegralIndependienteDelCamino}

\begin{teoUnoFormaConservativaSiiExacta}%
	\label{thm:unoformaconservativasiiexacta}
	Sea $M$ una variedad diferencial. Sea $\omega\in\formes[1]{M}$ una
	$1$-forma suave. Entonces $\omega$ es conservativa, si y s\'{o}lo
	si es exacta.
\end{teoUnoFormaConservativaSiiExacta}

