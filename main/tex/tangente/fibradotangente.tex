\theoremstyle{plain}
\newtheorem{propoFibradoTangente}{Proposici\'{o}n}[section]
\newtheorem{propoElDiferencialFuntorial}[propoFibradoTangente]{Proposici\'{o}n}

\theoremstyle{remark}

%-------------


El \emph{fibrado tangente} es la construcci\'{o}n que permite estudiar de
manera coherente los espacios tangentes a una variedad. Si $M$ es una
variedad diferencial, como conjunto, el fibrado tangente a/de $M$ es la
uni\'{o}n disjunta de todos los espacios tangentes:
\begin{align*}
	TM & \,=\,\sqcup_{p\in M}\,T_{p}M
	\text{ .}
\end{align*}
%
A los elementos de $TM$ los denotamos $(p,v)$, donde $p\in M$ y $v\in T_{p}M$.
El fibrado tangente tiene asociada una proyecci\'{o}n
$\pi:\,TM\rightarrow M$ que a un par $(p,v)$ le asigna $\pi(p,v)=p$. La
coherencia a la que se hizo alusi\'{o}n se refiere a la posibilidad de dar a
$TM$ una estructura natural de variedad diferencial.

\begin{propoFibradoTangente}\label{thm:fibradotangente}
	Si $M$ es una variedad diferencial de dimensi\'{o}n $n$, el fibrado
	tangente $TM$ tiene una topolog\'{\i}a y una estructura diferencial
	naturales de manera que, con ellas, $TM$ sea una variedad diferencial
	de dimensi\'{o}n $2\cdot n$ y respecto a las cuales
	$\pi:\,TM\rightarrow M$ sea diferenciable.
\end{propoFibradoTangente}

\begin{proof}
	Para demostrar esta proposici\'{o}n, haremos uso del lema de las
	cartas (lema \ref{thm:delascartas}). Sea $(U,\varphi)$ una carta
	para $M$. Su preimagen v\'{\i}a $\pi$ es el conjunto
	$\pi^{-1}(U)$ de pares $(p,v_{p})$ con $p\in U$ y $v_{p}\in T_{p}M$.
	Es decir, $\pi^{-1}(U)$ consiste en los vectores tangentes a $M$ en
	puntos de $U$. En $\pi^{-1}(U)$, definimos una aplicaci\'{o}n
	$\widetilde{\varphi}:\,\pi^{-1}(U)\rightarrow\bb{R}^{2n}$ por
	\begin{align*}
		\widetilde{\varphi}(p,v_{p}) & \,=\,
			(\lista*{x}{n},\,\lista*{v}{n})
		\text{ ,}
	\end{align*}
	%
	donde $\varphi(p)=(\lista*{x}{n})\in\varphi(U)\subset\bb{R}^{n}$ y
	$v_{p}=v^{i}\gancho[p]{x^{i}}\in T_{p}M$ es la escritura de $v_{p}$
	en la base can\'{o}nica $\gancho[p]{x^{1}},\,\dots,\,\gancho[p]{x^{n}}$
	(o, expl\'{\i}citamente, $\gancho[p]{\varphi^{i}}$) del espacio
	tangente a $M$ en $p$. La imagen $\widetilde{\varphi}(\pi^{-1}(U))=%
	\varphi(U)\times\bb{R}^{n}$ es un abierto de
	$\bb{R}^{n}\times\bb{R}^{n}$ en biyecci\'{o}n con $\pi^{-1}(U)$
	v\'{\i}a $\widetilde{\varphi}$: su inversa est\'{a} dada por
	\begin{align*}
		\widetilde{\varphi}^{-1}(\lista*{x}{n},\,\lista*{v}{n}) &
		\,=\,\Big(
		\varphi^{-1}(x),\,v^{i}\gancho[\varphi^{-1}(x)]{x^{i}}\Big)
		\text{ .}
	\end{align*}
	%
	Dadas dos cartas $(U,\varphi)$ y $(V,\psi)$ con intersecci\'{o}n
	no vac\'{\i}a, se cumple que $\pi^{-1}(U)\cap\pi^{-1}(V)\not =%
	\varnothing$ y
	\begin{align*}
		\widetilde{\varphi}(\pi^{-1}(U)\cap\pi^{-1}(V)) & \,=\,
			\varphi(U\cap V)\times\bb{R}^{n}\quad\text{y} \\
		\widetilde{\psi}(\pi^{-1}(U)\cap\pi^{-1}(V)) & \,=\,
			\psi(U\cap V)\times\bb{R}^{n}
		\text{ .}
	\end{align*}
	%
	Estos conjuntos son abiertos en $\bb{R}^{2n}$ y las funciones de
	transici\'{o}n $\widetilde{\varphi}\circ\widetilde{\psi}^{-1}$ y
	$\widetilde{\psi}\circ\widetilde{\varphi}^{-1}$ son suaves, pues:
	\begin{align*}
		\widetilde{\psi}\circ\widetilde{\varphi}^{-1}
			(\lista*{x}{n},\,\lista*{v}{n}) & \,=\,
			\Big(
			\widetilde{x}^{1}(x),\,\dots,\,\widetilde{x}^{n}(x),\,
			\derivada{\widetilde{x}^{1}}{x^{i}}(x)v^{i},\,\dots,\,
			\derivada{\widetilde{x}^{n}}{x^{i}}(x)v^{i}
			\Big)
		\text{ .}
	\end{align*}
	%
	(Notemos que, si $M$ es $C^{1}$, entonces $TM$ no llegar\'{a} a ser
	una variedad $C^{1}$, pues las funciones
	$\derivada{\widetilde{x}^{j}}{x^{i}}$ son meramente continuas).
	Por otro lado, si $\{U_{i}\}_{i}$ es un cubrimiento (numerable) de $M$
	por cartas compatibles, entonces $\{\pi^{-1}(U_{i})\}_{i}$ ser\'{a}
	un cubrimiento (numerable) de $TM$. Finalmente, para verificar la
	hip\'{o}tesis de separabilidad ($T_{2}$) del lema, si $(p,v_{p})$ y
	$(q,w_{q})$ son dos puntos distintos en $TM$ con $p\not =q$,
	entonces, eligiendo cartas con dominios disjuntos que separen a
	$p$ y a $q$ obtenemos, tomando preimagen por $\pi$, conjuntos
	disjuntos en $TM$ de la forma $\pi^{-1}(U)$, donde $U$ es el dominio
	de una carta, que separan a $(p,v_{p})$ y a $(q,w_{q})$. Si, por
	otro lado, $p=q$, entonces, tomando cualquier carta $U$ que
	contenga a $p$, el conjunto $pi^{-1}(U)$ contiene a ambos puntos
	del fibrado. Con esto se terminan de verificar las hip\'{o}tesis
	del lema \ref{thm:delascartas}.

	Para ver que $\pi:\,TM\rightarrow M$ es suave, basta tomar una
	carta $(U,\varphi)$ para $M$ y la carta correspondiente
	$(\pi^{-1}(U),\widetilde{\varphi})$ para $TM$. Con respecto a estas
	coordenadas,
	\begin{align*}
		\varphi\circ\pi\widetilde{\varphi}^{-1}
			(\lista*{x}{n},\,\lista*{v}{n}) & \,=\,(\lista*{x}{n})
		\text{ ,}
	\end{align*}
	%
	es decir, $\widehat{\pi}$ es, tambi\'{e}n, la proyecci\'{o}n en las
	primeras coordenadas. En el caso de variedades con borde, podemos
	tomar la expresi\'{o}n $(v,p)$ para que, al tomar coordenadas,
	$\widetilde{\varphi}(v_{p},p)=(\lista*{v}{n},\,\lista*{x}{n})\in %
	\bb{R}^{n}\times\hemi[n]=\hemi[2n]$.
\end{proof}

Una funci\'{o}n suave $F:\,M\rightarrow N$ determina una funci\'{o}n suave a
nivel de los fibrados tangentes. Esta funci\'{o}n es el \emph{diferencial}
(o \emph{diferencial global} o \emph{diferencial total}) de $F$,
$\diferencial{F}:\,TM\rightarrow TN$, dado por
\begin{align*}
	\diferencial{F}(v_{p},p) & \,=\,\diferencial[p]{F}(v_{p})\,\in\,
		T_{F(p)}N\quad\text{, o bien} \\
	\diferencial{F}(v_{p},p) & \,=\,(\diferencial[p]{F}(v_{p}),F(p))
\end{align*}
%
En coordenadas,
\begin{align*}
	\widehat{\diferencial{F}}(\lista*{v}{n},\,\lista*{x}{n}) & \,=\,
		\Big(
		\derivada{\widehat{F}^{1}}{x^{i}}(x)v^{i},\,\dots,\,
		\derivada{\widehat{F}^{m}}{x^{i}}(x)v^{i},\,
		\widehat{F}^{1}(x),\,\dots,\,\widehat{F}^{m}(x)
		\Big)
		\text{ .}
\end{align*}
%
Al igual que en el caso de la diferencial en un punto, $\diferencial{F}$
posee las siguientes propiedades funtoriales:

\begin{propoElDiferencialFuntorial}\label{thm:eldiferencialfuntorial}
	\emph{(a)} Si $F:\,M\rightarrow N$ y $G:\,N\rightarrow\tilde{N}$ son
	transformaciones suaves, la composici\'{o}n
	$G\circ F:\,M\rightarrow\tilde{N}$ es suave y
	$\diferencial{(G\circ F)}=\diferencial{G}\cdots\diferencial{F}$;
	\emph{(b)} la identidad $\id[M]:\,M\rightarrow M$ es suave y su
	diferencial est\'{a} dado por $\diferencial{\id[M]}=\id[TM]$;
	\emph{(c)} si $F:\,M\rightarrow N$ es un difeomorfismo, entonces
	$\diferencial{F}:\,TM\rightarrow TN$ es un difeomorfismo y
	$(\diferencial{F})^{-1}=\diferencial{(F^{-1})}$.
\end{propoElDiferencialFuntorial}
